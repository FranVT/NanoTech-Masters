%%%%%%%%%%%%%%%%%%%%%%%%%%%%%%%%%%%%%%%%%
% Tufte Essay
% LaTeX Template
% Version 2.0 (19/1/19)
%
% This template originates from:
% http://www.LaTeXTemplates.com
%
% Authors:
% The Tufte-LaTeX Developers (https://www.ctan.org/pkg/tufte-latex)
% Vel (vel@LaTeXTemplates.com)
%
% License:
% Apache License, version 2.0
%
%%%%%%%%%%%%%%%%%%%%%%%%%%%%%%%%%%%%%%%%%

%----------------------------------------------------------------------------------------
%	PACKAGES AND OTHER DOCUMENT CONFIGURATIONS
%----------------------------------------------------------------------------------------

\documentclass[a4paper]{tufte-handout} % Use A4 paper by default, remove 'a4paper' for US letter

\usepackage{graphics}
\usepackage{graphicx} % Required for including images
\setkeys{Gin}{width=\linewidth, totalheight=\textheight, keepaspectratio} % Default images settings

\usepackage{amsmath, amsfonts, amssymb, amsthm} % For math equations, theorems, symbols, etc
%\usepackage{units} % Non-stacked fractions and better unit spacing
\usepackage{physics}
\usepackage{cancel}
\usepackage{siunitx}

\usepackage{booktabs} % Required for better horizontal rules in tables

% For newtcbtheorem
\usepackage[most]{tcolorbox}
\usepackage{cleveref}
\usepackage{fancybox}		% Recuadros emph eqn

\usepackage{empheq}			% Recuadros para ecuaciones

\setlength{\parskip}{1em}


%----------------------------------------------------------------------------------------
%	TITLE SECTION
%----------------------------------------------------------------------------------------

\title{Quantum Computation\\ Quantum Circuits Activity}

\author{Francisco Vazquez-Tavares}

\date{\today} % Date, use \date{} for no date


%----------------------------------------------------------------------------------------
%	COMMANDS SECTION
%----------------------------------------------------------------------------------------

\newcommand{\hata}{\hat{a}}
\newcommand{\hatad}{\hat{a}^\dagger}
\newcommand{\QDi}{\hat{X}_1}
\newcommand{\QDj}{\hat{X}_2}

\newtcbtheorem[]{prob}{Problem}%
    {enhanced,
    colback = black!5, %white,
    colbacktitle = black!5,
    coltitle = black,
    boxrule = 0pt,
    frame hidden,
    borderline west = {0.5mm}{0.0mm}{black},
    fonttitle = \bfseries\sffamily,
    breakable,
    before skip = 3ex,
    after skip = 3ex
}{def}

\bibliographystyle{apalike}
\nobibliography{}

%----------------------------------------------------------------------------------------

\begin{document}

\maketitle % Print the title section
\justifying

%----------------------------------------------------------------------------------------
%	ESSAY BODY
%----------------------------------------------------------------------------------------

\begin{prob}{~}{label1}
    Compare the effect of the following two circuits

    \includegraphics[width=0.9\textwidth]{imgs/image-7.png}

\end{prob}

Let's consider the following general state $\ket{\psi}=a\ket{0} + b\ket{1}$ with $a,c\in\mathbb{C}$.
The first circuit can be represented with the following algebraic expression
\begin{gather*}
    \left[\left(\hat{H}\otimes\mathbb{1}\right)\left(\Lambda\hat{X}\right)\left(\hat{H}\otimes\mathbb{1}\right)\right]\left(\ket{\psi}\otimes\ket{0}\right).
\end{gather*}
Where $\Lambda\hat{X}$ denotes the controlled $\hat{X}$ gate ($\dyad{0}\otimes\mathbb{1} + \dyad{1}\otimes\hat{X})$,
$\hat{H}$ is the Haddamard gate and $\hat{X}$ is the $X$ gate.

Starting with the first gate,
\begin{align*}
    \left(\hat{H}\otimes\mathbb{1}\right)\left(\ket{\psi}\otimes\ket{0}\right) &= \hat{H}\ket{\psi}\otimes\mathbb{1}\ket{0} \\
                                                                               &= \left[\frac{a}{\sqrt{2}}\qty(\ket{0}+\ket{1})+\frac{b}{\sqrt{2}}\qty(\ket{0}-\ket{1})\right]\otimes\mathbb{1}\ket{0} \\
                                                                               &= \left[\frac{\qty(a+b)}{\sqrt{2}}\ket{0}+\frac{\qty(a-b)}{\sqrt{2}}\ket{1}\right]\otimes\mathbb{1}\ket{0} \\
                                                                               &= \frac{\qty(a+b)}{\sqrt{2}}\ket{00}+\frac{\qty(a-b)}{\sqrt{2}}\ket{10}.
\end{align*}

Now we compute the controlled $\hat{X}$ gate with the new state with the following mnemonic rule, \textit{It flips the second qubit if the first qubit is \num{1} and leaves unchanged otherwise}, therefore
\begin{gather*}
    \Lambda\hat{X}\left[\frac{\qty(a+b)}{\sqrt{2}}\ket{00}+\frac{\qty(a-b)}{\sqrt{2}}\ket{10}\right] = \frac{\qty(a+b)}{\sqrt{2}}\ket{00}+\frac{\qty(a-b)}{\sqrt{2}}\ket{11}=\ket{\psi_2}.
\end{gather*}

Finally, we apply the last Haddamard gate into the nwe state,
\begin{align*}
    \left(\hat{H}\otimes\mathbb{1}\right)\ket{\psi_2} &=\frac{\qty(a+b)}{\sqrt{2}}\hat{H}\ket{0}\otimes\mathbb{1}\ket{0} + \frac{\qty(a-b)}{\sqrt{2}}\hat{H}\ket{1}\otimes\mathbb{1}\ket{1} \\
                                                      &=\frac{\qty(a+b)}{\sqrt{2}}\left(\frac{1}{\sqrt{2}}\qty(\ket{0}+\ket{1})\right)\otimes\mathbb{1}\ket{0} + \frac{\qty(a-b)}{\sqrt{2}}\left(\frac{1}{\sqrt{2}}\qty(\ket{0}-\ket{1})\right)\otimes\mathbb{1}\ket{1} \\
                                                      &=\frac{\qty(a+b)}{2}\left(\ket{00}+\ket{10}\right) + \frac{\qty(a-b)}{2}\left(\ket{01}-\ket{11}\right).
\end{align*}
After expanding the expression and minor algebraic manipulations we can express the final state in terms of the of the Bell states,
\begin{gather*}
    \left(\hat{H}\otimes\mathbb{1}\right)\ket{\psi_2}=\frac{a}{\sqrt{2}}\left(\ket{\Psi^+} + \ket{\Phi^+} \right) + \frac{b}{\sqrt{2}}\left(\ket{\Psi^+} - \ket{\Phi^-} \right)
\end{gather*}


Now let's compute the final state in the second circuit to compare it with the previous result.
First we state th algebraic representation of the circuit as follows,
\begin{gather*}
    \left[\left(\mathbb{1}\otimes\hat{H}\right)\left(\hat{Z}\Lambda\right)\left(\mathbb{1}\otimes\hat{H}\right)\right]\left(\ket{\psi}\otimes\ket{0}\right).
\end{gather*}
Where $\hat{Z}\Lambda$ denotes the controlled $\hat{Z}$ gate ($\mathbb{1}\otimes\dyad{0} + \hat{Z}\otimes\dyad{1}$),
$\hat{H}$ is the Haddamard gate and $\hat{Z}$ is the $Z$ gate.

Let's begin with the first gate,
\begin{align*}
    \ket{\psi_2} &= \left(\mathbb{1}\otimes\hat{H}\right)\left(\ket{\psi}\otimes\ket{0}\right) \\
                 &= \mathbb{1}\ket{\psi}\otimes\hat{H}\ket{0} \\
                 &= \qty(a\ket{0}+b\ket{1})\otimes\left(\frac{1}{\sqrt{2}}\qty(\ket{0}+\ket{1})\right) \\
                 &= a\ket{0}\otimes\left(\frac{1}{\sqrt{2}}\qty(\ket{0}+\ket{1})\right)
                    + b\ket{1}\otimes\left(\frac{1}{\sqrt{2}}\qty(\ket{0}+\ket{1})\right) \\
                 &= \frac{a}{\sqrt{2}}\left(\ket{00}+\ket{01}\right) + \frac{b}{\sqrt{2}}\left(\ket{10}+\ket{11}\right).
\end{align*}

The next step is to compute the controlled $\hat{Z}$ gate into the new state,
\begin{align*}
    \hat{Z}\Lambda\ket{\psi_2} &= \left(\mathbb{1}\otimes\dyad{0} + \hat{Z}\otimes\dyad{1}\right)\left(\frac{a}{\sqrt{2}}\left(\ket{00}+\ket{01}\right) + \frac{b}{\sqrt{2}}\left(\ket{10}+\ket{11}\right)\right) \\
\end{align*}

\begin{multline*}
    \hat{Z}\Lambda\ket{\psi_2} =\left[
                                        \frac{a}{\sqrt{2}}\left(\mathbb{1}\otimes\dyad{0}\right)\left(\ket{00}+\ket{01}\right) 
                                        +
                                        \frac{b}{\sqrt{2}}\left(\mathbb{1}\otimes\dyad{0}\right)\left(\ket{10}+\ket{11}\right)\right. \\ \left.
                                        +
                                        \frac{a}{\sqrt{2}}\left(\hat{Z}\otimes\dyad{1}\right)\left(\ket{00}+\ket{01}\right)
                                        +\frac{b}{\sqrt{2}}\left(\hat{Z}\otimes\dyad{1}\right)\left(\ket{10}+\ket{11}\right)
                                    \right]
\end{multline*}

\begin{multline*}
    \hat{Z}\Lambda\ket{\psi_2} =
    \left[
        \frac{a}{\sqrt{2}}\left(
            \left(\mathbb{1}\ket{0}\otimes\dyad{0}\ket{0}\right)
            +
            \left(\mathbb{1}\ket{0}\otimes\dyad{0}\ket{1}\right)
        \right) 
    \right.
        \\
        +
        \frac{b}{\sqrt{2}}\left(
            \left(\mathbb{1}\ket{1}\otimes\dyad{0}\ket{0}\right)
            +
            \left(\mathbb{1}\ket{1}\otimes\dyad{0}\ket{1}\right)
        \right)
        \\
        +
        \frac{a}{\sqrt{2}}\left(
            \left(\hat{Z}\ket{0}\otimes\dyad{1}\ket{0}\right)
            +
            \left(\hat{Z}\ket{0}\otimes\dyad{1}\ket{1}\right)
        \right)
        \\
    \left.
        +
        \frac{b}{\sqrt{2}}\left(
        \left(\hat{Z}\ket{1}\otimes\dyad{1}\ket{0}\right)
        +
        \left(\hat{Z}\ket{1}\otimes\dyad{1}\ket{1}\right)
        \right)
    \right]
\end{multline*}

\begin{multline*}
    \hat{Z}\Lambda\ket{\psi_2} =
    \left[
        \frac{a}{\sqrt{2}}\left(
            \left(\mathbb{1}\ket{0}\otimes\ket{0}\right)
        \right) 
    \right.
        \\
        +
        \frac{b}{\sqrt{2}}\left(
            \left(\mathbb{1}\ket{1}\otimes\ket{0}\right)
        \right)
        +
        \frac{a}{\sqrt{2}}\left(
            \left(\hat{Z}\ket{0}\otimes\ket{1}\right)
        \right)
        \\
    \left.
        +
        \frac{b}{\sqrt{2}}\left(
        \left(\hat{Z}\ket{1}\otimes\ket{1}\right)
        \right)
    \right]
\end{multline*}

\begin{align*}
    \hat{Z}\Lambda\ket{\psi_2} &= \frac{a}{\sqrt{2}}\qty(\ket{00}+\ket{11}) + \frac{b}{\sqrt{2}}\qty(\ket{10}+\ket{01}) = \ket{\psi_3}
\end{align*}

Finally, with the new state we can compute the last Haddamard gate in order to compare the result with the previous one,
\begin{align*}
    \left(\mathbb{1}\otimes\hat{H}\right)\ket{\psi_3} &= \left(\mathbb{1}\otimes\hat{H}\right)\left(\frac{a}{\sqrt{2}}\qty(\ket{00}+\ket{11}) + \frac{b}{\sqrt{2}}\qty(\ket{10}+\ket{01})\right)
\\
                                                      &=\left(\frac{a}{\sqrt{2}}\left(\mathbb{1}\otimes\hat{H}\right)\qty(\ket{00}+\ket{11}) + \frac{b}{\sqrt{2}}\left(\mathbb{1}\otimes\hat{H}\right)\qty(\ket{10}+\ket{01})\right)
                                                      \\
                                                      &=\left(\frac{a}{2}\qty(\ket{00}+\ket{01}+\ket{10}-\ket{11}) 
                                                      + \frac{b}{2}\qty(\ket{10}+\ket{11}+\ket{00}-\ket{01})\right) \\
                                                      &=\frac{a}{\sqrt{2}}\left(\ket{\Psi^-}+\ket{\Phi^+}\right) + \frac{b}{\sqrt{2}}\left(\ket{\Psi^+}+\ket{\Phi^-}\right)
\end{align*}
Now that we have the final state of the second circuit we can compare both results,
\begin{align*}
    \mathrm{First~circuit}\ket{\psi0} &\to\frac{a}{\sqrt{2}}\left(\ket{\Psi^+} + \ket{\Phi^+} \right) + \frac{b}{\sqrt{2}}\left(\ket{\Psi^+} - \ket{\Phi^-} \right) \\
    \mathrm{Second~circuit}\ket{\psi0} &\to\frac{a}{\sqrt{2}}\left(\ket{\Psi^-}+\ket{\Phi^+}\right) + \frac{b}{\sqrt{2}}\left(\ket{\Psi^+}+\ket{\Phi^-}\right).
\end{align*}
In both cases we can see how the states transforms from a computaitonal basis to a linear combination of the Bells states basis.
That can be interpreted as that this cirsuits transforms a state to a linear combination of entangle states.
On th eother hand, we can see that the main difference between both circuits is that change in sign on the second term of the linear combination and the change of the bell state in the first term.

\newpage

\begin{prob}{~}{label2}
    Show that the following quantum circuit is equivalent to a controlled Z-gate

    \includegraphics[width=0.9\textwidth]{imgs/image-8.png}

\end{prob}

Let's start by computing the result of the controlled $Z$ gate in the two qubit system $\ket{\psi}\otimes\ket{0}$, where $\ket{\psi}$ is a general state $\ket{\psi}=a\ket{0}+b\ket{1}$,
\begin{align*}
    \Lambda\hat{Z}\ket{\psi0} &= \left(\dyad{0}\otimes\mathbb{1}+\dyad{1}\otimes\hat{Z}\right)\left(a\ket{00}+b\ket{10}\right) \\
                             &= \left(\dyad{0}\otimes\mathbb{1}\right)\left(a\ket{00}+b\ket{10}\right)+\left(\dyad{1}\otimes\hat{Z}\right)\left(a\ket{00}+b\ket{10}\right) \\
                             %&= \left(\dyad{0}\otimes\mathbb{1}\right)\left(a\ket{00}+b\ket{10}\right)
                             % +
                             % \left(\dyad{1}\otimes\hat{Z}\right)\left(a\ket{00}+b\ket{10}\right)
                             &= a\ket{00} + b\ket{10}.
\end{align*}

Now it is time to compute the circuit shown in the previous figure.
Let's begin with the first Haddamard gate,
\begin{align*}
    \qty(\hat{H}\otimes\mathbb{1})\ket{\psi 0} &= a\qty(\hat{H}\otimes\mathbb{1})\ket{00}+b\qty(\hat{H}\otimes\mathbb{1})\ket{10} \\
%                                               &= a\left(\frac{1}{\sqrt{2}}\qty(\ket{0}+\ket{1})\right)\otimes\ket{0}
%                                               + b\left(\frac{1}{\sqrt{2}}\qty(\ket{0}-\ket{1})\right)\otimes\ket{0}
                                               &= a\left(\frac{1}{\sqrt{2}}\qty(\ket{00}+\ket{10})\right)
                                               +b\left(\frac{1}{\sqrt{2}}\qty(\ket{00}-\ket{10})\right) \\
                                               &= \frac{a+b}{\sqrt{2}}\ket{00} + \frac{a-b}{\sqrt{2}}\ket{10}.
\end{align*}

Now, we can applied the controlled $\hat{X}$ gate,
\begin{align*}
    \Lambda\hat{X}\qty(\hat{H}\otimes\mathbb{1})\ket{\psi0} &= \left(\dyad{0}\otimes\mathbb{1}+\dyad{1}\otimes\hat{X}\right)\left(\frac{a+b}{\sqrt{2}}\ket{00} + \frac{a-b}{\sqrt{2}}\ket{10}\right) \\
                              &= \left(\dyad{0}\otimes\mathbb{1}\right)\left(\frac{a+b}{\sqrt{2}}\ket{00} + \frac{a-b}{\sqrt{2}}\ket{10}\right) + \left(\dyad{1}\otimes\hat{X}\right)\left(\frac{a+b}{\sqrt{2}}\ket{00} + \frac{a-b}{\sqrt{2}}\ket{10}\right) \\ 
                              &= \frac{a+b}{\sqrt{2}}\ket{00} + \frac{a-b}{\sqrt{2}}\ket{11}
\end{align*}
%\begin{align*}
%    \qty(\hat{H}\otimes\mathbb{1})\Lambda\hat{X}\qty(\hat{H}\otimes\mathbb{1})\ket{\psi 0}
%\end{align*}

To finish, let's apply the last Haddamard gate,
\begin{align*}
    \qty(\hat{H}\otimes\mathbb{1})\left(\frac{a+b}{\sqrt{2}}\ket{00} + \frac{a-b}{\sqrt{2}}\ket{11}\right) &=\frac{a+b}{\sqrt{2}}\left(\frac{1}{\sqrt{2}}\qty(a\ket{00}+b\ket{11})\right) + \frac{a-b}{\sqrt{2}}\left(\frac{1}{\sqrt{2}}\qty(a\ket{00} - b\ket{11})\right) \\
                                                                                                           &=\frac{a+b}{2}\qty(a\ket{00}+b\ket{11}) + \frac{a-b}{2}\qty(a\ket{00} - b\ket{11}) \\
                                                                                                           &= a^2\ket{00} + \qty(b^2+ab)\ket{11}.
\end{align*}

We can see that this circuit is not equivalent as claimed in the exercise.
On the other hand, if we change te circuit by swapping the controlled $\hat{X}$ gate to the $\hat{X}$ gate, the result is equivalent to applying a controlled $\hat{Z}$ gate.
Let's resume from the first Haddamard gate,
\begin{align*}
    \left(\hat{X}\otimes\mathbb{1}\right)\left(\frac{a+b}{\sqrt{2}}\ket{00} + \frac{a-b}{\sqrt{2}}\ket{10}\right) = \left(\frac{a+b}{\sqrt{2}}\ket{10} + \frac{a-b}{\sqrt{2}}\ket{00}\right).
\end{align*}
Now, let's finish by applying the last Haddamard gate,
\begin{align*}
    \left(\hat{H}\otimes\mathbb{1}\right)\left(\frac{a+b}{\sqrt{2}}\ket{10} + \frac{a-b}{\sqrt{2}}\ket{00}\right) &= \left(\frac{a+b}{\sqrt{2}}\frac{1}{\sqrt{2}}\left(\ket{00}-\ket{10}\right)\right) 
    +\left(\frac{a-b}{\sqrt{2}}\frac{1}{\sqrt{2}}\left(\ket{00}+\ket{10}\right)\right) \\
    &= \left(\frac{a+b}{2}\left(\ket{00}-\ket{10}\right)\right) 
    +\left(\frac{a-b}{2}\left(\ket{00}+\ket{10}\right)\right) \\
    &= a\ket{00}-b\ket{10}.
\end{align*}
If we only focus in the first term we get the same as the controlled $\hat{Z}$ gate, however we get a sign difference in the second term.

\newpage

\begin{prob}{~}{label3}
    The three qubit GHZ-state is defined as\[\ket{GHZ}=\frac{1}{\sqrt{2}}\left(\ket{000}+\ket{111}\right).\]

    Design a circuit that upon of the separable state $\ket{000}$ constructs the GHZ-state.

\end{prob}

For this last problem, it is implemented the following circuit,
\begin{figure}[ht!]
    \centering
    \includegraphics[width=0.8\textwidth]{imgs/circuit3.pdf}
\end{figure}


Let's start by computing the first Haddamard gate,
\begin{align*}
    \qty(\hat{H}\otimes\mathbb{1}\otimes\mathbb{1})\ket{000} = \frac{1}{\sqrt{2}}\qty(\ket{000}+\ket{100}). 
\end{align*}
Now, lets apply the first controlled $\hat{X}$ gate,
\begin{align*}
    \qty(\Lambda\otimes\hat{X}\otimes\mathbb{1})\frac{1}{\sqrt{2}}\qty(\ket{000}+\ket{100}) = \frac{1}{\sqrt{2}}\qty(\ket{000}+\ket{110}).
\end{align*}
Now, lets finish with the second controlled $\hat{X}$ gate,
\begin{align*}
    \qty(\Lambda\otimes\mathbb{1}\otimes\hat{X})\frac{1}{\sqrt{2}}\qty(\ket{000}+\ket{110}) = \frac{1}{\sqrt{2}}\qty(\ket{000}+\ket{111}).
\end{align*}

\begin{comment}
The expectation value of an operator (or quantum gate) $A$ over a qubit $\ket{\psi}$ is defined as
\begin{equation}
    \expval{A} = \expval{A}{\psi}.
\end{equation}
Consider the general state
\begin{equation}
    \ket{\psi} = a\ket{0} + b\ket{1},\quad a,b\in\mathbb{C},
\end{equation}
and define the map
\begin{equation}
    \ket{\psi} \mapsto\hat{n} = \left(\expval{X},\expval{Y},\expval{Z}\right).
\end{equation}

\begin{prob}{~}{label1}

    Show that the entries of the vector $\hat{n}$ fulfill
    \begin{align*}
        n_x &= \expval{X} = 2\Re(\bar{a}b) \\
        n_y &= \expval{Y} = 2\Im(\bar{a}b) \\
        n_z &= \expval{Z} = \abs{a}^2 - \abs{b}^2.
    \end{align*}
    and its norm is equal to \num{1}.
    The overline $\bar{x}$ stands for the complex conjugate of $x$.
    You might work with the conventional matrix representation.

\end{prob}

Let's start with $n_x$,
\begin{align*}
    n_x &= \expval{X}{\psi} = \mqty(\bar{a} & \bar{b})\mqty(\pmat{1})\mqty(a \\ b) \\
        &= \mqty(\bar{a} & \bar{b})\mqty(b \\ a) \\ 
        &= \bar{a}b + \bar{b}a.
\end{align*}
Recalling that $\Re(z)=(z+\bar{z})/2$, we can re-write,
\begin{align*}
    n_x = 2\Re(\bar{a}b).
\end{align*}

Moving forward to $n_y$ we repeat the same process,
\begin{align*}
    n_y &= \expval{Y}{\psi} = \mqty(\bar{a} & \bar{b})\mqty(\pmat{2})\mqty(a \\ b) \\
        &= \mqty(\bar{a} & \bar{b})\mqty(ib \\ -ia) \\ 
        &= i\left(\bar{a}b - \bar{b}a\right).
\end{align*}
Recalling that $\Im(z)=(z-\bar{z})/2$, we can re-write,
\begin{align*}
    n_y = 2\Im(\bar{a}b).
\end{align*}

Finally, for $n_z$, we repeat one last time,
\begin{align*}
    n_z &= \expval{Z}{\psi} = \mqty(\bar{a} & \bar{b})\mqty(\pmat{3})\mqty(a \\ b) \\
        &= \mqty(\bar{a} & \bar{b})\mqty(a \\ -b) \\ 
        &= \abs{a}^2 - \abs{b}^2.
\end{align*}

Now, to compute the norm we need to sum the squared of the components and take the squared of the result.
Let's start by computing the squared of each component,
\begin{align*}
    n_x^2 &= \left(\bar{a}b + \bar{b}a\right)^2 = \bar{a}^2b^2 + 2\abs{a}^2\abs{b}^2 + \bar{b}^2a^2 \\
    n_y^2 &= \left(i\left(\bar{a}b - \bar{b}a\right)\right)^2 = -\bar{a}^2b^2 + 2\abs{a}^2\abs{b}^2 - \bar{b}^2a^2 \\
    n_z^2 &= \left(\abs{a}^2 - \abs{b}^2\right)^2 = \abs{a}^4 - 2\abs{a}^2\abs{b}^2 + \abs{b}^4.
\end{align*}
Now, we add the terms,
\begin{align*}
    n_x^2 + n_y^2 + n_z^2 &= \abs{a}^4 + 2\abs{a}^2\abs{b}^2 + \abs{b}^4. 
\end{align*}
The nex step is to assume that the state $\ket{\psi}$ is normalize, such that $\ip{\psi}{\psi}=1=\abs{a}^2+\abs{b}^2$.
We can derive the following identities, $\abs{a}^2=1-\abs{b}^2$ and  $\abs{a}^4=1-2\abs{b}^2+\abs{b}^4$.
\begin{align*}
    n_x^2 + n_y^2 + n_z^2 &= 1 - 2\abs{b}^2 + \abs{b}^4 + 
                            2\abs{b}^2 - 2\abs{b}^4 + \abs{b}^4 \\
                          &= 1.
\end{align*}
Hence the norm of $\hat{n}$ is \num{1}.


\begin{prob}{~}{label2}
    The qubit $\ket{\psi}$ can be parametrized in the following way,
    \begin{equation}
        \ket{\psi} = \cos\left(\frac{\theta}{2}\right)\ket{0} + e^{i\varphi}\sin\left(\frac{\theta}{2}\right)\ket{1},
    \end{equation}
    with $\theta\in[0,\pi/2],\quad\varphi\in[1,2\pi].$
    Using the results obtained in the previous part prove that the components of $\hat{n}$ are the usual spherical coordinates,
    \begin{align*}
        n_x &= \sin\theta\cos\varphi \\
        n_y &= \sin\theta\sin\varphi \\
        n_z &= \cos\theta.
    \end{align*}
    This procedure justifies why an arbitrary qubit is identified with a point in the Bloch sphere, which is also called the qubit projective space.
\end{prob}

\marginpar{
Usefull trigonometric and complex exponentials identities for the excersices.
\begin{gather*}
    \cos(A)\sin(B) = \frac{\sin(A+B)-\sin(A-B)}{2} \\
    \cos(x) = \frac{1}{2}\left(e^{ix}+e^{-ix}\right) \\
    \sin(x) = \frac{1}{2i}\left(e^{ix}-e^{-ix}\right)
\end{gather*}
}

We can identify that $a=\cos\qty(\theta/2)$ and $b=e^{i\varphi}\sin\qty(\theta/2)$, hence,
\begin{align*}
    n_x &= \bar{a}b + \bar{b}a \\
        &= \left(e^{i\varphi}+e^{-i\varphi}\right)\cos\left(\frac{\theta}{2}\right)\sin\left(\frac{\theta}{2}\right) \\
        &= 2\cos\left(\varphi\right)\frac{1}{2}\sin\left(\theta\right)\\
        &= \cos\varphi\sin\theta.
\end{align*}
Moving on to the next component,
\begin{align*}
    n_y &= \bar{a}b - \bar{b}a \\
        &= \left(e^{i\varphi} - e^{-i\varphi}\right)\cos\left(\frac{\theta}{2}\right)\sin\left(\frac{\theta}{2}\right) \\
        &= 2\sin\left(\varphi\right)\frac{1}{2}\sin\left(\theta\right)\\
        &= \sin\varphi\sin\theta.
\end{align*}
Finally,
\begin{align*}
    n_z &= \abs{a}^2 - \abs{b}^2 \\
        &= \cos^2\left(\frac{\theta}{2}\right) - \sin^2\left(\frac{\theta}{2}\right) \\
        &= \cos\left(\theta\right).
\end{align*}
\end{comment}

\end{document}
