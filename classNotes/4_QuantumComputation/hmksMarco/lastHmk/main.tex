%%%%%%%%%%%%%%%%%%%%%%%%%%%%%%%%%%%%%%%%%
% Tufte Essay
% LaTeX Template
% Version 2.0 (19/1/19)
%
% This template originates from:
% http://www.LaTeXTemplates.com
%
% Authors:
% The Tufte-LaTeX Developers (https://www.ctan.org/pkg/tufte-latex)
% Vel (vel@LaTeXTemplates.com)
%
% License:
% Apache License, version 2.0
%
%%%%%%%%%%%%%%%%%%%%%%%%%%%%%%%%%%%%%%%%%

%----------------------------------------------------------------------------------------
%	PACKAGES AND OTHER DOCUMENT CONFIGURATIONS
%----------------------------------------------------------------------------------------

\documentclass[a4paper]{tufte-handout} % Use A4 paper by default, remove 'a4paper' for US letter

\usepackage{graphics}
\usepackage{graphicx} % Required for including images
\setkeys{Gin}{width=\linewidth, totalheight=\textheight, keepaspectratio} % Default images settings

\usepackage{amsmath, amsfonts, amssymb, amsthm} % For math equations, theorems, symbols, etc
%\usepackage{units} % Non-stacked fractions and better unit spacing
\usepackage{physics}
\usepackage{cancel}
\usepackage{siunitx}

\usepackage{booktabs} % Required for better horizontal rules in tables

% For newtcbtheorem
\usepackage[most]{tcolorbox}
\usepackage{cleveref}
\usepackage{fancybox}		% Recuadros emph eqn

\usepackage{empheq}			% Recuadros para ecuaciones

\setlength{\parskip}{1em}


%----------------------------------------------------------------------------------------
%	TITLE SECTION
%----------------------------------------------------------------------------------------

\title{Quantum Computation\\ Quantum Circuits Activity}

\author{Francisco Vazquez-Tavares}

\date{\today} % Date, use \date{} for no date


%----------------------------------------------------------------------------------------
%	COMMANDS SECTION
%----------------------------------------------------------------------------------------

\newcommand{\hata}{\hat{a}}
\newcommand{\hatad}{\hat{a}^\dagger}
\newcommand{\QDi}{\hat{X}_1}
\newcommand{\QDj}{\hat{X}_2}

\newtcbtheorem[]{prob}{Problem}%
    {enhanced,
    colback = black!5, %white,
    colbacktitle = black!5,
    coltitle = black,
    boxrule = 0pt,
    frame hidden,
    borderline west = {0.5mm}{0.0mm}{black},
    fonttitle = \bfseries\sffamily,
    breakable,
    before skip = 3ex,
    after skip = 3ex
}{def}


%----------------------------------------------------------------------------------------

\begin{document}

\maketitle % Print the title section
\justifying

%----------------------------------------------------------------------------------------
%	ESSAY BODY
%----------------------------------------------------------------------------------------

\section{Mandatory exercises}

\paragraph{5} Assume that we start with a fully separable three-qubit states.
First, qubits 1 and 2 become maximally entangled through an appropiate quantum operation.
Your task is to design a quantum circuit that transfers this entanglement to qubits (2,3).
In other words, at the end of the circuit, qubits 2 and 3 should be maximally entangled, while qubit 1 should be disentangled from the rest.
You are allowed to use elementary gates alone.

We start with a fully separable three-qubit states $\ket{000}$, which can be expressed as $\ket{0}\otimes\ket{0}\otimes\ket{0}$.
In order to get a maximally entangled state for qubits 1 and 2 we apply a Haddamard gate in qubit 1,
\begin{gather}
    \left(\hat{H}\otimes\hat{I}\otimes\hat{I}\right)\ket{000} = \frac{1}{\sqrt{2}}\left(\ket{000}+\ket{100}\right),
\end{gather}
and a controlled X gate with qubit 1 as the control,
\begin{gather}
    \left(\hat{\Lambda}\otimes\hat{X}\otimes\hat{I}\right)\frac{1}{\sqrt{2}}\left(\ket{000}+\ket{100}\right)=\frac{1}{\sqrt{2}}\left(\ket{000}+\ket{110}\right).
\end{gather}
This final state can be expressed in terms of the Bells states for qubits 1 and 2,
\begin{align}
    \frac{1}{\sqrt{2}}\left(\ket{000}+\ket{110}\right) &= \frac{1}{\sqrt{2}}\left(\ket{00}\otimes\ket{0}+\ket{11}\otimes\ket{0}\right) \\
                                                       &= \ket{\Psi^{+}}\otimes\ket{0}\label{eqn:psi0}.
\end{align}
Now that we achive the maximally entangled for qubit 1 and 2, let's show con to transfer this entanglement to qubits 2 and 3.
First, let's apply a controlled X gate into qubits 2 and 3 to create a maximally entangled 3 qubit state\footnote{From previous homework, this is the GHZ-state.},
\begin{gather}
    \left(\hat{I}\otimes\hat{\Lambda}\otimes\hat{X}\right)\ket{\Psi^{+}}\otimes\ket{0}=\frac{1}{\sqrt{2}}\left(\ket{000}+\ket{111}\right).
\end{gather}
At this point we can interpert that applying the controlled X gate after the Haddamard gate, it allow us to entangle the control with the target qubit.
Hence, in order to disentangle the first qubit we apply a controlled qubit with the second qubit as controlled and the first qubit as the target,
\begin{align}
    \left(\hat{X}\otimes\hat{\Lambda}\otimes\hat{I}\right)\frac{1}{\sqrt{2}}\left(\ket{000}+\ket{111}\right)&=\frac{1}{\sqrt{2}}\left(\ket{000}+\ket{111}\right)&\\
    \ket{0}\otimes\ket{\Psi^{+}}\label{eqn:0psi}.
\end{align}
By comparing equations~\eqref{eqn:psi0} with~\eqref{eqn:0psi}, we can see how the maximal entanglement has been transferred from qubits 1-2 to qubits 2-3, achieving the desired quantum circuit.
The graphical representation is shown in figure~\ref{fig:qc5}.

\begin{figure}[ht!]
    \centering
    \includegraphics[width=0.9\textwidth]{imgs/qc5.pdf}
    \caption{Transfer of entangled states cricuit.}\label{fig:qc5}
\end{figure}

\paragraph{7} A boolean function $f:\{0,1\}^n \mapsto\{0,1\}$ is said to be constant of $f(x)$ has the same vaue for all $2^n$ inputs and balanced if $f(x)$ returns \num{0} for exactly half of all inputs and $1$ for the other half,
\begin{itemize}
    \item Consider a generalization of the Deuthsch's algorithm having two registers ($n=2$).
        The correpondent circuit is essentially the same as in the one register case.
        Discuss the conditions that would determine if a function is wether balanced or constant.
    \item Analyze the case when the function $f$ is neither constant or nor balanced.
\end{itemize}

For this problem we are going to use the following equations,
\begin{align}
    \ket{\pm} &=\frac{1}{\sqrt{2}}\left(\ket{0}\pm\ket{1}\right)\label{eqn:hpm} \\
    \hat{H}\ket{0} &= \ket{+}\label{eqn:hp} \\
    \hat{H}\ket{1} &= \ket{-}\label{eqn:hm} \\
    \hat{H}\ket{+} &= \ket{0}\label{eqn:h0} \\
    \hat{H}\ket{-} &= \ket{1}\label{eqn:h1} \\
    \hat{U}_f\ket{x}\ket{y} &= \ket{x}\ket{y\oplus f(x)}\label{eqn:oracle}.
\end{align}

Now, let's compute $\ket{\psi_1}$, which corresponds to the state after the Haddamard gates,
\begin{align}
    \ket{\psi_1} &= \left(\hat{H}\otimes\hat{H}\otimes\hat{H}\right)\ket{0}\otimes\ket{0}\otimes\ket{0} \\
                 &= \ket{+}\ket{+}\ket{-}.
\end{align}

Now, let's compute the state after the oracle $\hat{U}_f$ $\ket{\psi_2}$,
\begin{align}
    \ket{\psi_2} &= \hat{U}_f\ket{\psi_1}\\
                 &= \ket{+}\ket{+}\ket{-\oplus f(x)}\\
                 &= \ket{+}\left(\frac{1}{\sqrt{2}}\left(\ket{0}\ket{-\oplus f(0)} + \ket{1}\ket{-\oplus f(1)}\right)\right).
\end{align}

To clarify the procedure, we are going to compute $\hat{U}_f\ket{0}\ket{-}$ and $\hat{U}_f\ket{1}\ket{-}$ for a balanced and constant functions.
Here are the results for a balanced functions $f(0)=0\wedge f(1)=1$ ($\mathrm{Ba}$), and $f(0)=1\wedge f(1)=0$ ($\mathrm{Bb}$),
\begin{align}
    \ket{\mathrm{Ba}0} = \hat{U}_f\ket{0}\ket{-} &= \ket{0}\ket{-} \\
    \ket{\mathrm{Ba}1} = \hat{U}_f\ket{1}\ket{-} &= -\ket{1}\ket{-} \\
    \ket{\mathrm{Bb}0} = \hat{U}_f\ket{0}\ket{-} &= \ket{0}\ket{-} \\
    \ket{\mathrm{Bb}1} = \hat{U}_f\ket{1}\ket{-} &= -\ket{1}\ket{-}. 
\end{align}
Now, for the constant functions $f(0)=0\wedge f(1)=0$ ($\mathrm{Ca}$), and $f(0)=1\wedge f(1)=1$ ($\mathrm{Cb}$),
\begin{align}
    \ket{\mathrm{Ca}0} = \hat{U}_f\ket{0}\ket{-} &= \ket{0}\ket{-} \\
    \ket{\mathrm{Ca}1} = \hat{U}_f\ket{1}\ket{-} &= \ket{1}\ket{-} \\
    \ket{\mathrm{Cb}0} = \hat{U}_f\ket{0}\ket{-} &= -\ket{0}\ket{-} \\
    \ket{\mathrm{Cb}1} = \hat{U}_f\ket{1}\ket{-} &= -\ket{1}\ket{-}. 
\end{align}

Now that we compute this results, we only need to replace those results into the state $\ket{\psi_2}$.
For the case when the function is balanced we get,
\begin{align}
    \hat{U}_f\ket{+}\ket{-}&\to\frac{1}{\sqrt{2}}\left(\ket{0}\ket{-} - \ket{1}\ket{-}\right) = \ket{-}\ket{-} \\
    \hat{U}_f\ket{+}\ket{-}&\to\frac{1}{\sqrt{2}}\left(\ket{0}\ket{-} + \ket{1}\ket{-}\right) = -\ket{-}\ket{-}.
\end{align}
Therefore, when $f(x)$ is balanced,
\begin{align}
    \ket{\psi_2} &= \ket{+}\ket{+}\ket{-\oplus f(x)}\\
                 &=\pm\ket{+}\ket{-}\ket{-}.
\end{align}

For the case of a constant function we get,
\begin{align}
    \hat{U}_f\ket{+}\ket{-}&\to\frac{1}{\sqrt{2}}\left(\ket{0}\ket{-} + \ket{1}\ket{-}\right) = \ket{+}\ket{-} \\
    \hat{U}_f\ket{+}\ket{-}&\to\frac{1}{\sqrt{2}}\left(-\ket{0}\ket{-} - \ket{1}\ket{-}\right) = -\ket{+}\ket{-}.
\end{align}
Therefore, when $f(x)$ is constant,
\begin{align}
    \ket{\psi_2} &= \ket{+}\ket{+}\ket{-\oplus f(x)}\\
                 &=\pm\ket{+}\ket{+}\ket{-}.
\end{align}

Hence,
\begin{align}
    \ket{\psi_2} = \left\{
        \begin{array}{ll}
            \pm\ket{+}\ket{-}\ket{-} &\quad f(x)\text{ is balanced} \\
             \pm\ket{+}\ket{+}\ket{-} &\quad f(x)\text{ is constant}
        \end{array}
    \right.
\end{align}

To finish the gate, we apply the haddamard to the first two qubits,
\begin{align}
    \ket{\psi_3} &= \left(\hat{H}\otimes\hat{H}\otimes\mathbb{I}\right)\ket{\psi_2} \\
                 &= \left\{
        \begin{array}{ll}
            \pm\ket{0}\ket{1}\ket{-} &\quad f(x)\text{ is balanced} \\
             \pm\ket{0}\ket{0}\ket{-} &\quad f(x)\text{ is constant}
        \end{array}
    \right.
\end{align}

\section{Optional exercises}

Due to time restrictions\footnote{I am a graduate candidate and was preparing for my thesis defense.} I couldn't include a more detail procedures/algebraic comprobation of the results shown in this part of the homework.
However I search the solutions for the quantum circuits related problems and I found the webpage \href{http://twoqubits.wikidot.com/}{http://twoqubits.wikidot.com/}.
In this webpage I found most of the solutions and explanations.
Then I use Mathematica software to proof the answers and learn what type of gates they used.
Since I used Mathematica, I used constantly the gate $U_3\qty(\theta,\phi,\lambda)$, which is defined as follows,
\begin{gather*}
    U_{3}\qty(\theta,\phi,\lambda) = 
    \begin{pmatrix}
        \cos\qty(\frac{\theta}{2}) & -e^{i\lambda} \\
        e^{i\phi}\sin\qty(\frac{\theta}{2}) & e^{i\qty(\lambda+\phi)}\cos\qty(\frac{\theta}{2})
    \end{pmatrix}.
\end{gather*}
Then I declare linear equations to find the values of the angles that match allows to match the values of operators shown at the webpage.
Also, another gate that it was used is,
\begin{gather*}
    T = 
    \begin{pmatrix}
        1 & 0 \\
        0 & e^{i\pi/4}
    \end{pmatrix}
\end{gather*}

An apology for delivering this section incomplete, but I had a hard time with this section and could not dedicate much time to it.

\paragraph{1} Describe the action of the phase shift gate $p(\gamma)=\dyad{0}+e^{i\gamma}\dyad{1}$ on a qubit from the geometrical point of view.

Recalling that we can express a qubti as $\ket{\psi}=\cos\left(\frac{\theta}{2}\right)\ket{0} + e^{i\varphi}\sin\left(\frac{\theta}{2}\right)\ket{1}$.
This expression allow us to create a geometrical interpretation as a point in an unit sphere.
Where $\theta$ represent the angle between the $\hat{x}$ and $\hat{y}$ axis, and the angle $\varphi$ is the angle between the $\hat{x}$ or $\hat{y}$ with the $\hat{z}$ axis.
With this in mind, let's compute the resulting state from the given gate,
\begin{align*}
    p(\gamma)\ket{\psi} &= \dyad{0}\left(\cos\left(\frac{\theta}{2}\right)\ket{0} + e^{i\varphi}\sin\left(\frac{\theta}{2}\right)\ket{1}\right)+e^{i\gamma}\dyad{1}\left(\cos\left(\frac{\theta}{2}\right)\ket{0} + e^{i\varphi}\sin\left(\frac{\theta}{2}\right)\ket{1}\right) \\
                                                        &= \cos\left(\frac{\theta}{2}\right)\ket{0} + e^{i\varphi+\gamma}\sin\left(\frac{\theta}{2}\right)\ket{1}.
\end{align*}

We can se that the $p(\gamma)$ gate introduces a phase shift in the angle related with the $\hat{x}-\hat{z}$ or $\hat{y}-\hat{z}$ planes.
That is that introduces a displacement along the latitude of the unit sphere.


\paragraph{2} The 4-qubit W-state is defined as, \[\ket{W_4} = \frac{1}{2}\left(\ket{1000}+\ket{0100}+\ket{0010}+\ket{0001}\right).\]
Design a quantum circuit that upon the initial state $\ket{0000}$ constructs $\ket{W_4}$.

\begin{figure}[ht!]
    \centering
    \includegraphics[width=0.9\textwidth]{imgs/qc2.png}
    \caption{Quantum circuit to construct the 4-qubit W-state.}\label{fig:qc2}
\end{figure}



\paragraph{3} Design a circuir constructing the Hardy state, \[\ket{H} = \frac{1}{\sqrt{12}}\left(3\ket{00}+\ket{01}+\ket{10}+\ket{11}\right).\]

\begin{figure}[ht!]
    \centering
    \includegraphics[width=0.9\textwidth]{imgs/qc3.png}
    \caption{Quantum circuit to construct the Hardy state.}\label{fig:qc3}
\end{figure}


\paragraph{4} Show how to implement the Toffoli gate in terms of single-qubit and controlled-NOT gates.

\begin{figure}[ht!]
    \centering
    \includegraphics[width=0.9\textwidth]{imgs/qc4.pdf}
    \caption{Toffoli gate in terms of single-qubit and controlled-NOT gates. Obtain from https://arxiv.org/pdf/0803.2316}\label{fig:qc4}
\end{figure}


\paragraph{6} In the BB84 protocol, Alice create an 8-qubit string (in the conventional $X$ and $Z$ basis): \[\ket{+}\ket{1}\ket{+}\ket{-}\ket{0}\ket{-}\ket{+}\ket{-}.\]
Use a coin to randomly determine what basis Bob uses to measure each bit, and describe the resulting bit string that Alice and Bob keep.


\end{document}
