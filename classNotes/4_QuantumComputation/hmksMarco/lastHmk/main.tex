%%%%%%%%%%%%%%%%%%%%%%%%%%%%%%%%%%%%%%%%%
% Tufte Essay
% LaTeX Template
% Version 2.0 (19/1/19)
%
% This template originates from:
% http://www.LaTeXTemplates.com
%
% Authors:
% The Tufte-LaTeX Developers (https://www.ctan.org/pkg/tufte-latex)
% Vel (vel@LaTeXTemplates.com)
%
% License:
% Apache License, version 2.0
%
%%%%%%%%%%%%%%%%%%%%%%%%%%%%%%%%%%%%%%%%%

%----------------------------------------------------------------------------------------
%	PACKAGES AND OTHER DOCUMENT CONFIGURATIONS
%----------------------------------------------------------------------------------------

\documentclass[a4paper]{tufte-handout} % Use A4 paper by default, remove 'a4paper' for US letter

\usepackage{graphics}
\usepackage{graphicx} % Required for including images
\setkeys{Gin}{width=\linewidth, totalheight=\textheight, keepaspectratio} % Default images settings

\usepackage{amsmath, amsfonts, amssymb, amsthm} % For math equations, theorems, symbols, etc
%\usepackage{units} % Non-stacked fractions and better unit spacing
\usepackage{physics}
\usepackage{cancel}
\usepackage{siunitx}

\usepackage{booktabs} % Required for better horizontal rules in tables

% For newtcbtheorem
\usepackage[most]{tcolorbox}
\usepackage{cleveref}
\usepackage{fancybox}		% Recuadros emph eqn

\usepackage{empheq}			% Recuadros para ecuaciones

\setlength{\parskip}{1em}


%----------------------------------------------------------------------------------------
%	TITLE SECTION
%----------------------------------------------------------------------------------------

\title{Quantum Computation\\ Quantum Circuits Activity}

\author{Francisco Vazquez-Tavares}

\date{\today} % Date, use \date{} for no date


%----------------------------------------------------------------------------------------
%	COMMANDS SECTION
%----------------------------------------------------------------------------------------

\newcommand{\hata}{\hat{a}}
\newcommand{\hatad}{\hat{a}^\dagger}
\newcommand{\QDi}{\hat{X}_1}
\newcommand{\QDj}{\hat{X}_2}

\newtcbtheorem[]{prob}{Problem}%
    {enhanced,
    colback = black!5, %white,
    colbacktitle = black!5,
    coltitle = black,
    boxrule = 0pt,
    frame hidden,
    borderline west = {0.5mm}{0.0mm}{black},
    fonttitle = \bfseries\sffamily,
    breakable,
    before skip = 3ex,
    after skip = 3ex
}{def}

\bibliographystyle{apalike}
\nobibliography{}

%----------------------------------------------------------------------------------------

\begin{document}

\maketitle % Print the title section
\justifying

%----------------------------------------------------------------------------------------
%	ESSAY BODY
%----------------------------------------------------------------------------------------

\paragraph{1} Describe the action of the phase shift gate $p(\gamma)=\dyad{0}+e^{i\gamma}\dyad{1}$ on a qubit from the geometrical point of view.

\paragraph{2} Te 4-qubit W-state is defined as, \[\ket{W_4} = \frac{1}{2}\left(\ket{1000}+\ket{0100}+\ket{0010}+\ket{0001}\right).\]
Design a quantum circuit that upon the initial state $\ket{000}$ constructs $\ket{W_4}$.

\paragraph{3} Design a circuir constructing the Hardy state, \[\ket{H} = \frac{1}{12}\left(3\ket{00}+\ket{01}+\ket{10}+\ket{11}\right).\]

\paragraph{4} Show how to implement the Toffoli gate in terms of single-qubit and controlled-NOT gates.

\paragraph{5} Assume that we start with a fully separable three-qubit states.
First, qubits 1 and 2 become maximally entangled through an appropiate quantum operation.
Your task is to design a quantum circuit that transfers this entanglement to qubits (2,3).
In other words, at the end of te circuit, qubits 2 and 3 should be maximally entangled, while qubit 1 should be disentangled from the rest.
You are allowed to use elementary gates alone.

\paragraph{6} In the BB84 protocol, Alice create an 8-qubit string (in the conventional $X$ and $Z$ basis): \[\ket{+}\ket{1}\ket{+}\ket{-}\ket{0}\ket{-}\ket{+}\ket{-}.\]
Use a coin to randomly determine what basis Bob uses to measure each bit, and describe the resulting bit string that Alice and Bob keep.

\paragraph{7} a boolean function $f:\{0,1\}^n \mapsto\{0,1\}$ is said to be constant of $f(x)$ has the same vaue for all $2^n$ inputs and balanced if $f(x)$ returns 0 for exactly half of all inputs and $1$ for the other half,
\begin{itemize}
    \item Consider a generalization of the Deuthsch's algorithm having two registers ($n=2$).
        The correpondent circuit is essentially the same as in the one register case.
        Discuss the conditions that would determine if a function is wether balanced or constant.
    \item Analyze the case when the function $f$ is neither constant or nor balanced.
\end{itemize}

\end{document}
