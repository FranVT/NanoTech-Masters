
%%%%%%%%%%%%%%%%%%%%%%%%%%%%%%%%%%%%%%%%%
% Tufte Essay
% LaTeX Template
% Version 2.0 (19/1/19)
%
% This template originates from:
% http://www.LaTeXTemplates.com
%
% Authors:
% The Tufte-LaTeX Developers (https://www.ctan.org/pkg/tufte-latex)
% Vel (vel@LaTeXTemplates.com)
%
% License:
% Apache License, version 2.0
%
%%%%%%%%%%%%%%%%%%%%%%%%%%%%%%%%%%%%%%%%%

%----------------------------------------------------------------------------------------
%	PACKAGES AND OTHER DOCUMENT CONFIGURATIONS
%----------------------------------------------------------------------------------------

\documentclass[a4paper]{tufte-handout} % Use A4 paper by default, remove 'a4paper' for US letter

\usepackage{graphics}
\usepackage{graphicx} % Required for including images
\setkeys{Gin}{width=\linewidth, totalheight=\textheight, keepaspectratio} % Default images settings

\usepackage{amsmath, amsfonts, amssymb, amsthm} % For math equations, theorems, symbols, etc
%\usepackage{units} % Non-stacked fractions and better unit spacing
\usepackage{physics}
\usepackage{cancel}
\usepackage{siunitx}

\usepackage{booktabs} % Required for better horizontal rules in tables

% For newtcbtheorem
\usepackage[most]{tcolorbox}
\usepackage{cleveref}
\usepackage{fancybox}		% Recuadros emph eqn

\usepackage{empheq}			% Recuadros para ecuaciones

\setlength{\parskip}{1em}


%----------------------------------------------------------------------------------------
%	TITLE SECTION
%----------------------------------------------------------------------------------------

\title{Smart Materials\\ Conceptos básicos de polímeros}

\author{Francisco Vazquez-Tavares}

\date{\today} % Date, use \date{} for no date


%----------------------------------------------------------------------------------------
%	COMMANDS SECTION
%----------------------------------------------------------------------------------------

\newcommand{\hata}{\hat{a}}
\newcommand{\hatad}{\hat{a}^\dagger}
\newcommand{\QDi}{\hat{X}_1}
\newcommand{\QDj}{\hat{X}_2}

\newtcbtheorem[]{prob}{Problem}%
    {enhanced,
    colback = black!5, %white,
    colbacktitle = black!5,
    coltitle = black,
    boxrule = 0pt,
    frame hidden,
    borderline west = {0.5mm}{0.0mm}{black},
    fonttitle = \bfseries\sffamily,
    breakable,
    before skip = 3ex,
    after skip = 3ex
}{def}

%----------------------------------------------------------------------------------------

\begin{document}

\maketitle % Print the title section
\justifying

%----------------------------------------------------------------------------------------
%	ESSAY BODY
%----------------------------------------------------------------------------------------

\section*{Concpetos}

\paragraph{Temperatura de servicio}
Es el rango de temperaturas en el cual un polímero puede funcionar de forma continua y fiable, manteniendo sus propiedades mecánicas y dimensionales.

\paragraph{Temperatura VICAT}
Es una temperatura a la cual un polímero se ablanda de manera medible, determinada bajo una condición de carga y tasa de calentamiento específicas.

\paragraph{Temperatura de transición vítrea}
Es el rango de temperaturas en el cual un polímero amorfo o las regiones amorfas de un polímero semicristalino pasan de un estado rígido y vítreo a un estado flexible y gomoso.

\paragraph{Índice de fluidez}
Es una medida de la facilidad con la que un polímero fundido (termoplástico) puede fluir. Se expresa como la masa de polímero (en gramos) que extrusiona a través de un dado de dimensiones específicas en un tiempo de 10 minutos.

\paragraph{Número de CAS}
Es un identificador numérico único asignado a cada sustancia química (elementos, compuestos, polímeros) registrada en la base de datos de Chemical Abstracts Service (CAS).

\newpage

\section*{Tabla}

\begin{table}[ht!]
\begin{tabular}{|c|c|c|c|}
\hline
\textbf{Parámetro}                           & \textbf{PP}     & \textbf{HDPE}   & \textbf{PS}     \\ \hline
\textbf{Temp. de fusión (°C)}                & 160-170         & 130-137         & 180-240         \\ \hline
\textbf{Temp. de servicio (°C)}              & 0-90            & \textless{}90   & Hasta 70-90     \\ \hline
\textbf{Temp. VICAT (°C)}                    & 150             & 121–131         & 90–100          \\ \hline
\textbf{Índice de fluidez (g/10 min, 210°C)} & 2-18            & 0.35–20         & 2-5             \\ \hline
\textbf{Temp. transición vítrea (°C)}        & -10             & -120            & 90–100          \\ \hline
\textbf{Densidad (g/cm³)}                    & 0.90–0.91       & 0.94–0.97       & 1.03–1.06       \\ \hline
\textbf{Peso molecular (g/mol)}              & 100,000–600,000 & 200,000–500,000 & 100,000–400,000 \\ \hline
\textbf{Número CAS}                          & 9003-07-0       & 9002-88-4       & 9003-53-6       \\ \hline
\end{tabular}
\end{table}


\begin{table}[ht!]
\begin{tabular}{|c|c|c|}
\hline
\textbf{Parámetro}                           & \textbf{PLA}   & \textbf{PHB}      \\ \hline
\textbf{Temp. de fusión (°C)}                & 150-160        & 170-180           \\ \hline
\textbf{Temp. de servicio (°C)}              & Hasta 50-60    & Hasta 60          \\ \hline
\textbf{Temp. VICAT (°C)}                    & 62.9           & 150               \\ \hline
\textbf{Índice de fluidez (g/10 min, 210°C)} & 7-10           & 17-20             \\ \hline
\textbf{Temp. transición vítrea (°C)}        & 55-65          & 2-7               \\ \hline
\textbf{Densidad (g/cm³)}                    & 1.17–1.25      & 1.20–1.25         \\ \hline
\textbf{Peso molecular (g/mol)}              & 60,000–200,000 & 500,000–1,000,000 \\ \hline
\textbf{Número CAS}                          & 26100-51-6     & 26063-88-5        \\ \hline
\end{tabular}
\end{table}

\end{document}
