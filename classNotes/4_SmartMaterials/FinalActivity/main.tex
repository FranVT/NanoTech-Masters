
%%%%%%%%%%%%%%%%%%%%%%%%%%%%%%%%%%%%%%%%%
% Tufte Essay
% LaTeX Template
% Version 2.0 (19/1/19)
%
% This template originates from:
% http://www.LaTeXTemplates.com
%
% Authors:
% The Tufte-LaTeX Developers (https://www.ctan.org/pkg/tufte-latex)
% Vel (vel@LaTeXTemplates.com)
%
% License:
% Apache License, version 2.0
%
%%%%%%%%%%%%%%%%%%%%%%%%%%%%%%%%%%%%%%%%%

%----------------------------------------------------------------------------------------
%	PACKAGES AND OTHER DOCUMENT CONFIGURATIONS
%----------------------------------------------------------------------------------------

\documentclass[a4paper]{tufte-handout} % Use A4 paper by default, remove 'a4paper' for US letter

\usepackage{graphics}
\usepackage{graphicx} % Required for including images
\setkeys{Gin}{width=\linewidth, totalheight=\textheight, keepaspectratio} % Default images settings

\usepackage{amsmath, amsfonts, amssymb, amsthm} % For math equations, theorems, symbols, etc
%\usepackage{units} % Non-stacked fractions and better unit spacing
\usepackage{physics}
\usepackage{cancel}
\usepackage{siunitx}

\usepackage{booktabs} % Required for better horizontal rules in tables

% For newtcbtheorem
\usepackage[most]{tcolorbox}
\usepackage{cleveref}
\usepackage{fancybox}		% Recuadros emph eqn

\usepackage{empheq}			% Recuadros para ecuaciones



\setlength{\parskip}{1em}
\bibliographystyle{apalike}


%----------------------------------------------------------------------------------------
%	TITLE SECTION
%----------------------------------------------------------------------------------------

\title{A tittle}

\author{Francisco Vazquez-Tavares}

\date{\today} % Date, use \date{} for no date


%----------------------------------------------------------------------------------------
%	COMMANDS SECTION
%----------------------------------------------------------------------------------------

%----------------------------------------------------------------------------------------

\begin{document}

\maketitle % Print the title section
\justifying

%----------------------------------------------------------------------------------------
%	ESSAY BODY
%----------------------------------------------------------------------------------------

In this activity we use a simulator of a TEM microscope to visualize some nanoparticles, a mineral, a metal, and a zebra fish.
All of the visualization images are generated by transmitting a beam of electrons through an ultra-thin sample.
Areas of the sample that are denser or thicker scatter more electrons (appear darker), while thinner or less dense regions allow more electrons to pass through (appear brighter). 
This results in a high-resolution, black-and-white amplitude contrast image that can reveal details down to the atomic level, such as crystal lattices and individual atoms.

After that, the diffractiion pattern for the nanoparticles, minerals and metals samples are shown.
On the other hand, the zebra in the simulator we can not get the diffraction pattern or more detail visualizations.
This might because biological samples are largely amorphous and lack log-range, highly ordered atomic arregenment.

The diffraction pattern of the rest of the samples resembles a diffraction pattern of a cristaline lattice structure.
After that we can see a high resolution of the surface of the samples.
Then the bright and dark field of the STEAM visualization mode are shown.
A bright-field (BF) image is created by gathering direct, unscattered, or weakly scattered beams.  
Dark areas in a BF-STEM image indicate that many electrons were scattered away (e.g., by thicker or heavier atoms), resulting in less signal reaching the detector.
A dark-field (DF) image, specifically the typical high-angle annular dark-field (HAADF) mode, is created by collecting electrons scattered at high angles.
These high-angle scatterings are dependent on the atomic number (Z).
Hence, this visualizations tell us the density of the sample.
Finally, the atomic structures are shown for the 3 samples\citep{gnanSilicoSynthesisMicrogel2017}.


\bibliography{sample}
%\section{Nanoparticles}
\end{document}
