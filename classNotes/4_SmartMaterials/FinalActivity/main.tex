
%%%%%%%%%%%%%%%%%%%%%%%%%%%%%%%%%%%%%%%%%
% Tufte Essay
% LaTeX Template
% Version 2.0 (19/1/19)
%
% This template originates from:
% http://www.LaTeXTemplates.com
%
% Authors:
% The Tufte-LaTeX Developers (https://www.ctan.org/pkg/tufte-latex)
% Vel (vel@LaTeXTemplates.com)
%
% License:
% Apache License, version 2.0
%
%%%%%%%%%%%%%%%%%%%%%%%%%%%%%%%%%%%%%%%%%

%----------------------------------------------------------------------------------------
%	PACKAGES AND OTHER DOCUMENT CONFIGURATIONS
%----------------------------------------------------------------------------------------

\documentclass[a4paper]{tufte-handout} % Use A4 paper by default, remove 'a4paper' for US letter

\usepackage{graphics}
\usepackage{graphicx} % Required for including images
\setkeys{Gin}{width=\linewidth, totalheight=\textheight, keepaspectratio} % Default images settings

\usepackage{amsmath, amsfonts, amssymb, amsthm} % For math equations, theorems, symbols, etc
%\usepackage{units} % Non-stacked fractions and better unit spacing
\usepackage{physics}
\usepackage{cancel}
\usepackage{siunitx}

\usepackage{booktabs} % Required for better horizontal rules in tables

% For newtcbtheorem
\usepackage[most]{tcolorbox}
\usepackage{cleveref}
\usepackage{fancybox}		% Recuadros emph eqn

\usepackage{empheq}			% Recuadros para ecuaciones



\setlength{\parskip}{1em}
\bibliographystyle{apalike}


%----------------------------------------------------------------------------------------
%	TITLE SECTION
%----------------------------------------------------------------------------------------

\title{A tittle}

\author{Francisco Vazquez-Tavares \\ Tecnológico de Monterrey, Escuela de Ingeniería y Ciencias \\ A00827546@tec.mx\\Thesis title: Exploration of mechanical response of polymeric gels via molecular dynamics.}
\date{\today} % Date, use \date{} for no date


%----------------------------------------------------------------------------------------
%	COMMANDS SECTION
%----------------------------------------------------------------------------------------

%----------------------------------------------------------------------------------------

\begin{document}

\maketitle % Print the title section

\begin{abstract}
\noindent Here it goes the abstract. 250 word

Keywords: smart polymers, nanomaterials, responsive materials, 
\end{abstract}

\justifying

%----------------------------------------------------------------------------------------
%	ESSAY BODY
%----------------------------------------------------------------------------------------

\section{Introduction}

\paragraph{Problem statement} In real context 

\paragraph{State ofr the art} A brief introduction

\paragraph{Justification of the proposal.}

\section{Proposed material}

a. Polímero(s) base
b. Nanomaterial(es) (aditivo(s) funcional(es))
c. Mecanismo o mecanismo funcional (si aplica)
d. Comportamiento inteligente esperado (si aplica, por ej.: sensible a estímulos, autorreparable,
piezoeléctrico, biodegradable, memoria de forma, electroactivo)
e. Explique la base científica del material y su propósito.
Cuando corresponda incluya:
• Modelo conceptual
• Esquema del mecanismo
• Principio de activación físico o químico
• Figuras adaptadas o correctamente citadas

\section{Fabrication technique}

Describa la ruta prevista de manufactura, por ejemplo:
• Electrospinning
• Extrusión
• Impresión 3D
• Moldeo por solvente (solvent casting)
• Modificación superficial
• Polimerización in situ
• Otra (especificar)
• Incluya una breve justificación de la técnica seleccionada.

\section{Characterization}

Identifique las evaluaciones necesarias para confirmar la funcionalidad:
• Propiedades mecánicas
• Estabilidad o transiciones térmicas
• Caracterización morfológica
• Propiedades eléctricas/ópticas (si aplica)
• Biodegradación o estudios de envejecimiento
• Validación de respuesta a estímulos

\section{Validation of vialidad}

Discuta:
• Viabilidad económica
• Escalabilidad y capacidad de manufactura
• Consideraciones toxicológicas o de bioseguridad
• Restricciones ambientales o regulatorias
• Potenciales riesgos y debilidades

\section{Conclusion}

samples\citep{gnanSilicoSynthesisMicrogel2017}.


\bibliography{sample}
%\section{Nanoparticles}
\end{document}
