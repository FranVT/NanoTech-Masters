
%%%%%%%%%%%%%%%%%%%%%%%%%%%%%%%%%%%%%%%%%
% Tufte Essay
% LaTeX Template
% Version 2.0 (19/1/19)
%
% This template originates from:
% http://www.LaTeXTemplates.com
%
% Authors:
% The Tufte-LaTeX Developers (https://www.ctan.org/pkg/tufte-latex)
% Vel (vel@LaTeXTemplates.com)
%
% License:
% Apache License, version 2.0
%
%%%%%%%%%%%%%%%%%%%%%%%%%%%%%%%%%%%%%%%%%

%----------------------------------------------------------------------------------------
%	PACKAGES AND OTHER DOCUMENT CONFIGURATIONS
%----------------------------------------------------------------------------------------

\documentclass[a4paper]{tufte-handout} % Use A4 paper by default, remove 'a4paper' for US letter

\usepackage{graphics}
\usepackage{graphicx} % Required for including images
\setkeys{Gin}{width=\linewidth, totalheight=\textheight, keepaspectratio} % Default images settings

\usepackage{amsmath, amsfonts, amssymb, amsthm} % For math equations, theorems, symbols, etc
%\usepackage{units} % Non-stacked fractions and better unit spacing
\usepackage{physics}
\usepackage{cancel}
\usepackage{siunitx}

\usepackage{booktabs} % Required for better horizontal rules in tables

% For newtcbtheorem
\usepackage[most]{tcolorbox}
\usepackage{cleveref}
\usepackage{fancybox}		% Recuadros emph eqn

\usepackage{empheq}			% Recuadros para ecuaciones

\setlength{\parskip}{1em}


%----------------------------------------------------------------------------------------
%	TITLE SECTION
%----------------------------------------------------------------------------------------

\title{Smart Materials\\ Problem set \#3: Shape memory effect}

\author{Francisco Vazquez-Tavares}

\date{\today} % Date, use \date{} for no date


%----------------------------------------------------------------------------------------
%	COMMANDS SECTION
%----------------------------------------------------------------------------------------

\newcommand{\hata}{\hat{a}}
\newcommand{\hatad}{\hat{a}^\dagger}
\newcommand{\QDi}{\hat{X}_1}
\newcommand{\QDj}{\hat{X}_2}

\newtcbtheorem[]{prob}{Problem}%
    {enhanced,
    colback = black!5, %white,
    colbacktitle = black!5,
    coltitle = black,
    boxrule = 0pt,
    frame hidden,
    borderline west = {0.5mm}{0.0mm}{black},
    fonttitle = \bfseries\sffamily,
    breakable,
    before skip = 3ex,
    after skip = 3ex
}{def}

\bibliographystyle{apalike}
\nobibliography{}

%----------------------------------------------------------------------------------------

\begin{document}

\maketitle % Print the title section
\justifying

%----------------------------------------------------------------------------------------
%	ESSAY BODY
%----------------------------------------------------------------------------------------

\begin{prob}{~}{label1}
Assuming that the stress required to initiate martensitic transformation in a Nitinol wire increases linearly by 5 MPa for every 1°C increase in temperature, calculate the stress required to initiate martensitic transformation at 50°C. 
It is given that Nitinol undergoes martensitic transformation at a stress of 300 MPa at 30°C.
\end{prob}

Following the assumption of linear relation between the starting martensitic transformation stress with temperature\footnote{It is important to say that this linear relation is about the starting value of stress to \textit{initiate} the martensitic transformation, we are not analizing the inial and final points of the transfromation.}, we can use the following mathematical expression,
\begin{gather*}
    T(\theta) = C_{M}\theta + T_o.
\end{gather*}
Where $T_o$ is the stress required to start martensitic transformation at \SI{0}{\degreeCelsius}.
We can compute that value by applying the given data, $C_{M}=\SI{5}{\mega\pascal\per\degreeCelsius}$ and $T(\SI{30}{\degreeCelsius})=\SI{300}{\mega\pascal}$,
\begin{align*}
    T(\theta) &= C_{M}\theta + T_o \\
    T_o &= T(\theta) - C_{M}\theta \\
        &= \SI{300}{\mega\pascal} - \SI{5}{\mega\pascal\per\degreeCelsius}\SI{30}{\degreeCelsius} \\
        &= \SI{300}{\mega\pascal} - \SI{150}{\mega\pascal} \\
        &= \SI{150}{\mega\pascal}.
\end{align*}
That that we know this value, we can compute the stress required to initiate martensitic transformation at \SI{50}{\degreeCelsius} as follows:
\begin{align*}
    T(\theta) &= C_{M}\theta + T_o \\
    T(\SI{50}{\degreeCelsius}) &= \SI{5}{\mega\pascal\per\degreeCelsius}\SI{50}{\degreeCelsius} + \SI{150}{\mega\pascal} \\
                               &= \SI{250}{\mega\pascal} + \SI{150}{\mega\pascal} \\
                               &= \SI{400}{\mega\pascal}.
\end{align*}

Hence, martensitic transformation starts by applying \SI{400}{\mega\pascal} of stress at \SI{50}{\degreeCelsius}.
\begin{empheq}[box=\shadowbox]{equation*}
    T(\SI{50}{\degreeCelsius})= \SI{400}{\mega\pascal}
\end{empheq}

\begin{prob}{~}{label2}
A SMA wire shows a stress plateau during martensitic transformation from 350 MPa to 400 MPa. 
The strain increases from 0.02 to 0.05 during this plateau. 
Calculate the approximate work done per unit volume during the martensitic transformation.
\end{prob}

\begin{marginfigure} 
    \includegraphics{imgs/problem2.jpg}
    \caption{Graphical representation of the plateau.}\label{fig:prob2}
\end{marginfigure}

Recalling that the units of pressure are \SI{}{\pascal}$\equiv$\SI{}{\newton\per\meter\squared}=\SI{}{\newton\meter\per\meter\tothe{3}}=\SI{}{\joule\per\meter\tothe{3}} and that the strain is an adimensional measure, we need to compute the area under the given plateau.
For that, we can asume a linear relation and separate the area into two elements, as shown in figure~\ref{fig:prob2}.

Replacing the numeric values,
\begin{align*}
    \SI{}{\newton\meter\per\meter\tothe{3}} &= \SI{0.03}{}\cdot\SI{350}{\mega\pascal} + \frac{1}{2}\SI{0.03}{}\cdot\SI{50}{\mega\pascal} \\
                                            &= \SI{10}{\mega\pascal} + \SI{1.5}{\mega\pascal} \\
                                            &= \SI{11.5}{\mega\pascal}
\end{align*}

\begin{empheq}[box=\shadowbox]{equation*}
    \SI[per-mode=fraction]{}{\newton\meter\per\meter\tothe{3}} = \SI{11.5}{\mega\pascal}
\end{empheq}



\begin{prob}{~}{label3}
A Nitinol wire has a stress-induced martensitic transformation starting at 200 MPa at 30 °C and ending at 500 MPa. 
The transformation stress increases by 10 MPa/°C. 
If the wire is heated to 80 °C, determine the new stress required to start and complete the transformation. 
Calculate the stress required to start and complete the transformation at 80 °C.
\end{prob}

From the first sentence, it is understood that the martensitic transfromation goes from $\xi=0$ at \SI{30}{\degreeCelsius} with \SI{200}{\mega\pascal} with stress applied to $\xi=1$ at \SI{30}{\degreeCelsius} with \SI{500}{\mega\pascal}.
Assuming once more that the stress-induced martensitic transformation is linearly related to temperature, we can use the following mathematical expression:
\begin{align*}
    M_s(T) = M_s + \frac{T}{C_M},   
\end{align*}
where $M_s$ is the martensitic transformation starting temperature at stress $0$ and $C_M$ is the rate of change of the starting point with stress.
With the information given in the first two sentences, we can compute $M_s$,
\begin{align*}
    M_s(T) &= M_s + \frac{T}{C_M} \\
    M_s &= M_s(T) - \frac{T}{C_M} \\
    M_s &= \SI{30}{\degreeCelsius} - \frac{\SI{200}{\mega\pascal}}{\SI{10}{\mega\pascal\per\degreeCelsius}} \\
        &= \SI{30}{\degreeCelsius} - \SI{20}{\degreeCelsius} \\
        &= \SI{10}{\degreeCelsius}.
\end{align*}
Now that we now this value we can compute the starting stress-induced martensitic transformation at \SI{80}{\degreeCelsius} as follows:
\begin{align*}
    M_s(T) &= M_s + \frac{T}{C_M} \\
    T &= C_{M}(M_s(T) - M_s) \\
      &= \SI{10}{\mega\pascal\per\degreeCelsius}\left(\SI{80}{\degreeCelsius}-\SI{10}{\degreeCelsius}\right) \\
      &= \SI{700}{\mega\pascal}.
\end{align*}

Therefore, to start stress-induced martensitic transformation at \SI{80}{\degreeCelsius} we required \SI{700}{\mega\pascal} of stress.
\begin{empheq}[box=\shadowbox]{equation*}
    T(\SI{80}{\degreeCelsius})= \SI{700}{\mega\pascal}
\end{empheq}

\begin{marginfigure} 
    \includegraphics{imgs/problem3.jpg}
    \caption{Linear relations between the initial and final temperature-stress of martensitic transformations}\label{fig:prob3}
\end{marginfigure}

Now, with this information and assuming the linear relation (figure~\ref{fig:prob3}), we can compute the finish stress-induced martensitic transformation at \SI{80}{\degreeCelsius} as follows:
\begin{align*}
    T(\theta_2) - T(\theta_1) &= C_{M}(\theta_2 - \theta_1).
\end{align*}
In this case $\theta_2=\SI{80}{\degreeCelsius}$ and $\theta_1=\SI{30}{\degreeCelsius}$ and $T(\theta_1)=\SI{500}{\mega\pascal}$.
Solving for $T(\theta_2)$ and replacing the numeric values,

\begin{align*}
    T(\SI{80}{\degreeCelsius}) &= \SI{10}{\mega\pascal\per\degreeCelsius}(\SI{80}{\degreeCelsius} - \SI{30}{\degreeCelsius}) + \SI{500}{\mega\pascal} \\
                              &= \SI{10}{\mega\pascal\per\degreeCelsius}\SI{50}{\degreeCelsius} + \SI{500}{\degreeCelsius} \\
                              &= \SI{500}{\mega\pascal} + \SI{500}{\mega\pascal} \\
                              &= \SI{1000}{\mega\pascal} 
\end{align*}

Therefore, to finish the stress-induced martensitic transformation at \SI{80}{\degreeCelsius} we required \SI{1000}{\mega\pascal} of stress.
\begin{empheq}[box=\shadowbox]{equation*}
    T(\SI{80}{\degreeCelsius})= \SI{1000}{\mega\pascal}
\end{empheq}


\begin{prob}{~}{label4}
The martensite start and finish temperatures (Ms and Mf), the austenite start and finish temperatures (As and Af) and the slopes of the variation of Ms and Mf with stress (T) i.e., CM and the slope of variation of As and Af with T, i.e., CA, of a shape memory material are as follows: 
    Ms = 25 ◦C, 
    Mf = 5 ◦C, 
    As = 29 ◦C, 
    Af = 51 ◦C, 
    CA = 4.5 MPa/◦C, 
    CM = 11.3 MPa/◦C. 
The elastic modulus of the material is 15 GPa and it has a recovery strain of 8\%. 

For this material compute the following: 
\begin{itemize}
    \item Calculate the martensitic fraction when the material is cooled to 20 °C from 25 °C in a zero-stress state.
    \item If the temperature is maintained at 25 °C, determine the martensitic fraction when a tensile stress of 100 MPa is applied to the wire.
    \item The wire is heated above Af = 51 ◦C, and then cooled under zero stress to 30 °C. Subsequently, it is subjected to loading to activate the shape memory effect, leading to complete phase transformation, starting from point a to point e (as illustrated in the Figure below).
{\centering
\includegraphics[width=0.9\textwidth]{imgs/SMAcurve.jpg}
}
\end{itemize}
Calculate the stress and strain values at points a, b, c, and d on the stress–strain diagram.
\end{prob}

\paragraph{Compute Martensitic fraction at zero stress}~\\
First we recall that $M_f$ and $M_s$ are the temperatures of start and finish of martensitic transfromation with $T=0$.
With this in mind, we infer that we are going from an Austenitic phase to Martensitic phase, hence we can compute the value using the following equation,
\begin{gather}
    \xi_{A\to M} = \frac{1}{2}\left\{\cos\left[\frac{\pi}{M_s-M_f}\qty(\theta-M_f) - \frac{\pi}{C_M(M_s-M_f)}T \right]+1\right\}\label{eqn:martFraction},
\end{gather}
replacing the numeric values,
\begin{align*}
    \xi_{A\to M}\qty(\SI{20}{\degreeCelsius}) &= \frac{1}{2}\left\{\cos\left[\frac{\pi}{\SI{25}{\degreeCelsius}-\SI{5}{\degreeCelsius}}\qty(\SI{20}{\degreeCelsius}-\SI{5}{\degreeCelsius}) - \cancelto{0}{\frac{\pi}{C_M(M_f-M_s)}T} \right]+1\right\} \\ 
                 &= \frac{1}{2}\left\{\cos\left[\frac{\SI{15}{\degreeCelsius}}{\SI{20}{\degreeCelsius}}\pi\right]+1\right\} \\
                 &= \frac{1}{2}\left\{\frac{2-\sqrt{2}}{2}\right\} \\
                 &\approx 0.1464.
\end{align*}

The martensitic fraction increases to 0.1464.
\begin{empheq}[box=\shadowbox]{equation*}
    \xi_{A\to M}\qty(\SI{20}{\degreeCelsius}) = \frac{2-\sqrt{2}}{4}\approx0.1464
\end{empheq}

\paragraph{Martensitic fraction at constant temperature}~\\
Using the same equation~\eqref{eqn:martFraction}, with values of $T=\SI{25}{\degreeCelsius}$ and $T=\SI{100}{\mega\pascal}$ we get the following value,
\begin{align*}
    \xi_{A\to M} &= \frac{1}{2}\left\{\cos\left[\pi\cancelto{1}{\frac{\SI{25}{\degreeCelsius}-\SI{5}{\degreeCelsius}}{\SI{25}{\degreeCelsius}-\SI{5}{\degreeCelsius}}} - \pi\frac{\SI{100}{\mega\pascal}}{\SI{11.3}{\mega\pascal\per\degreeCelsius}(\SI{25}{\degreeCelsius}-\SI{5}{\degreeCelsius})} \right]+1\right\} \\
                 &= \frac{1}{2}\left\{\cos\left[\pi - \pi\frac{50}{113} \right]+1\right\} \\
                 &\approx \frac{1}{2}\left\{0.82027\right\} \\
                 &\approx 0.41013
\end{align*}

The martensitic fraction increases to 0.41013.
\begin{empheq}[box=\shadowbox]{equation*}
    \xi_{A\to M}\approx0.41013
\end{empheq}

\paragraph{Stress and strain values}~\\
Since the wire is heated above Austenitic final temperature, the alloy is at $\xi=0$.
Then, it is colled to \SI{30}{\degreeCelsius} with $T=0$.
Recalling that the martensitic start temperature is $M_s=\SI{25}{\degreeCelsius}$, we known that no martensite has formed, hence $\xi=0$.

Now, in order to compute the stress and strain in each point we are going to apply the constitutive equation of one dimension for shape memory alloys,
\begin{gather}
    T - T^o = Y(S-S^o)-YS_L(\xi-\xi^o)\label{eqn:constEqn}.
\end{gather}

\paragraph{a)} Applying the values of the martensitic values ($\xi=\xi^o=0$) and that this was perform under zero stress $T^o=S^o=0$, get that,
\begin{gather*}
    T = YS
\end{gather*}
and we known that at point $a$ $T=0$, then $S=0$.
\begin{empheq}[box=\shadowbox]{equation*}
    T^a=0,\quad S^a=0.    
\end{empheq}

\paragraph{b)} The strain-stress relation to point b) can still be computed by,
\begin{gather*}
    T = YS,
\end{gather*}
because no martensitic transformation has started.
However, we known that at b), we reach the critical stress at which the martensitic transformation starts.
The value of the stress at that point can be computed with the following linear relation, 
\begin{gather*}
    T^b = C_M\qty(\theta_o - M_s),
\end{gather*}
where $\theta_o = \SI{30}{\degreeCelsius}$, hence
\begin{align*}
    T^b &= \SI{11.3}{\mega\pascal\per\degreeCelsius}\qty(\SI{30}{\degreeCelsius} - \SI{25}{\degreeCelsius}) \\
        &= \SI{11.3}{\mega\pascal\per\degreeCelsius}\SI{5}{\degreeCelsius} \\
        &= \SI{56.5}{\mega\pascal}.
\end{align*}
Now that we known the stress we can compute the strain,
\begin{align*}
    T^b &= YS^b \\
    S^b &= \frac{T^b}{Y} \\
        &= \frac{\SI{56.5}{\mega\pascal}}{\SI{15}{\giga\pascal}} \\
        &= \SI{3.76d-3}{}.
\end{align*}

\begin{empheq}[box=\shadowbox]{equation*}
    T^b=\SI{56.5}{\mega\pascal},\quad S^b=\SI{3.76d-3}{}.    
\end{empheq}

\paragraph{c)} Now, from b) to c) we fullfill the martensitic transformation $\xi^c=1$ and $\xi^b=0$.
Also, from the figure we known that we reach the maximum strain $S_L$ which is the recovery strain 8\%.
With this value the constitutive equation\eqref{eqn:constEqn} for this process is,
\begin{align*}
    T^c - T^b &= Y(S^c-S^b)-YS_L(\xi^c-\xi^b) \\
    T^c - T^b &= Y(S^c-S^b)-YS_L(1-0) \\
    T^c - T^b &= Y(S^c-S^b)-YS_L.
\end{align*}
And since $T^b = YS^b$, the relation simplifies to,
\begin{align*}
    T^c - T^b &= Y(S^c-S^b)-YS_L \\
          T^c &= YS^c-YS_L.
\end{align*}
Finally, we recall the linear relation $T = C_m\qty(\theta_o-M_f)$ and recalling that this and isothermal process, then $\theta_o=\SI{30}{\degreeCelsius}$, we can compute the stress at c) as follows,
\begin{align*}
    T^c &= \SI{11.3}{\mega\pascal\per\degreeCelsius}\qty(\SI{30}{\degreeCelsius}-\SI{5}{\degreeCelsius}) \\
        &= \SI{282.5}{\mega\pascal}.
\end{align*}
Naturally, for the strain it follows,
\begin{align*}
    T^c &= YS^c-YS_L \\
    S^c &= \frac{T^c}{Y}+S_L \\
        &= \frac{\SI{282.5}{\mega\pascal}}{\SI{15}{\giga\pascal}} + 0.08 \\
        &= 0.09883
\end{align*}

\begin{empheq}[box=\shadowbox]{equation*}
    T^c=\SI{282.5}{\mega\pascal},\quad S^c=\SI{0.09883}{}.    
\end{empheq}

\paragraph{d} Finally, the unloading process $c\to d$ we know by looking the figure that $T=0$ and $S=0.08$.

\begin{empheq}[box=\shadowbox]{equation*}
    T^d=\SI{0}{\mega\pascal},\quad S^d=\SI{0.08}{}.    
\end{empheq}



\end{document}
