%%%%%%%%%%%%%%%%%%%%%%%%%%%%%%%%%%%%%%%%%
% Tufte Essay
% LaTeX Template
% Version 2.0 (19/1/19)
%
% This template originates from:
% http://www.LaTeXTemplates.com
%
% Authors:
% The Tufte-LaTeX Developers (https://www.ctan.org/pkg/tufte-latex)
% Vel (vel@LaTeXTemplates.com)
%
% License:
% Apache License, version 2.0
%
%%%%%%%%%%%%%%%%%%%%%%%%%%%%%%%%%%%%%%%%%

%----------------------------------------------------------------------------------------
%	PACKAGES AND OTHER DOCUMENT CONFIGURATIONS
%----------------------------------------------------------------------------------------

\documentclass[a4paper]{tufte-handout} % Use A4 paper by default, remove 'a4paper' for US letter

\usepackage{graphicx} % Required for including images
\setkeys{Gin}{width=\linewidth, totalheight=\textheight, keepaspectratio} % Default images settings
\graphicspath{{Figures/}{./}} % Specifies where to look for included images (trailing slash required)

\usepackage{amsmath, amsfonts, amssymb, amsthm} % For math equations, theorems, symbols, etc
\usepackage{units} % Non-stacked fractions and better unit spacing
\usepackage{physics}

\usepackage{booktabs} % Required for better horizontal rules in tables

%----------------------------------------------------------------------------------------
%	TITLE SECTION
%----------------------------------------------------------------------------------------

\title{ Quantumm Optics Class-Notes and others }

\author{Francisco Vazquez-Tavares}

\date{\today} % Date, use \date{} for no date


%----------------------------------------------------------------------------------------
%	COMMANDS SECTION
%----------------------------------------------------------------------------------------

\newcommand{\hata}{\hat{a}}
\newcommand{\hatad}{\hat{a}^\dagger}
\newcommand{\QDi}{\hat{X}_1}
\newcommand{\QDj}{\hat{X}_2}


%----------------------------------------------------------------------------------------

\begin{document}

\maketitle % Print the title section

%----------------------------------------------------------------------------------------
%	ABSTRACT/SUMMARY
%----------------------------------------------------------------------------------------

\begin{abstract}
	\textbf{Summary}
        Class notes, post-class notes and others for the course of Quantum Optics.
        Semester February-June 2025
\end{abstract}

%----------------------------------------------------------------------------------------
%	ESSAY BODY
%----------------------------------------------------------------------------------------

\section{Harmonic oscilator}\label{sec:Feb12}
February 12


\section{First Quantization}\label{sec:Feb17}
February 17

\section{Properties of Quantumm electric field}\label{sec:Feb19}
Februrary 19

\subsection{Fock States}
February 19

\section{Coherent states}\label{sec:Feb24}
February 24

\marginpar{
Reminders of some properties.
\begin{align*}
    \qty[\hata,\hatad] &= 1 \\
    \hat{X}_1 &= \frac{\hata+\hatad}{2} \\
    \hat{X}_2 &= \frac{\hata-\hatad}{2i} \\
    \hata &= \hat{X}_1 + i\hat{X}_2 \\
    \hatad &= \hat{X}_1 -i\hat{X}_2 \\
    \qty[\hat{X}_1,\hat{X}_2] &= \frac{i}{2} \\
    \Delta\hat{X}_1\Delta\hat{X}_2 &\geq \frac{1}{4}
\end{align*}

}

According to the professor the sections \ref{sec:Feb12,sec:Feb17,sec:Feb19,sec:Feb24}, where the hard part of the course.
The next sections we are going to describe different states and analize there properties.

Lets remember the operator of an enlectric field with $x$ component and one mode,
\begin{gather*}
    E_x = i\left(\frac{\hbar\omega}{2\epsilon_o V}\right)^{1/2}\left(\hata\exp\qty[-i\omega t] - \hatad\exp\qty[-i\omega t]\right),
\end{gather*}
which can be expressed in terms of quadratures as,
\begin{gather*}
    E_x = 2\left(\frac{\hbar\omega}{2\epsilon_o V}\right)^{1/2}\qty( \hat{X}_1\sin\qty(\omega t) +  \hat{X}_2\cos\qty(\omega t))
\end{gather*}

So, when we want to get the expeted value of the electric field of a Fock state $\ket{n}$ we get $\expval{E}{n}=0$.
So we need other states to model a laser.
A useful observation is that $\expval{\QDi^2}{n}=1/4(2n+1)$ and $\Delta\QDi\Delta\QDj=1/4(2n+1)$.

We are going to study the ``\textit{Gleuber states}''.
Which are the states that can describe the laser.
For that we have 4 definitions,
\paragraph{Definition 1}
Eigenstates of $\hata$ \[\hata\ket{\alpha}=\alpha\ket{\alpha},\quad\alpha\in\mathbb{C}.\]

\paragraph{Definition 2}
Displaced vacuum\footnote{
    It is important to use the following definition of $e^x$, becuase the argument are matrices and vectors,
\[\exp\qty(x)=\sum\frac{x^n}{n!}.\]
}.

\begin{gather*}
    \hat{D}\qty(\alpha) = \exp\qty[\alpha\hatad-\alpha^*\hata],\quad\ket{\alpha}=\hat{D}(\alpha)\ket{0}
\end{gather*}

\paragraph{Definition 3}
Fock States
\begin{gather*}
    \ket{\alpha} = \exp\qty[-\frac{\abs{\alpha}^2}{2}]\sum\frac{\alpha^n}{\sqrt{n!}}\ket{n}
\end{gather*}

\paragraph{Definition 4}
\[\Delta\QDi\Delta\QDj=\frac{1}{2}\]

Then we start to analyse a coherent state and a Fock state.
\begin{align*}
    \braket{\alpha}&=\exp\qty[-\abs{\alpha}^2]\sum_{n=0}\sum_{m=0}\frac{\alpha^{*m}\alpha^n}{\sqrt{m!}\sqrt{n!}}\braket{m}{n} \\
                   &=\exp\qty[-\abs{\alpha}^2]\sum\frac{\abs{\alpha}^{2n}}{n!} \\
                   &=1
\end{align*}

Later we analyse the creation and anhilihation operator.
\marginpar{
Usefull properties,
\begin{gather*}
    \hata\hatad-\hatad\hata=1\\
    \expval{\hata\hatad}{\alpha}=\abs{\alpha}^2+1 \\
    \hata\ket{n}=\sqrt{n}\ket{n-1}
\end{gather*}
}
Knowing that $\hata\ket{\alpha}=\alpha\ket{\alpha}$ and when we compute its adjoint $\bra{\alpha}\hatad=\alpha^*\bra{\alpha}$, we get different eigenvalues for each operator.
However, if we compute the expected value of both operators in a state we get,
\begin{align*}
    \expval{\hatad\hata}{\alpha} &=\braket{\alpha}{\hatad\hata\alpha} \\
                                 &=\bra{\alpha}\hatad\alpha \\
                                 &=\alpha^*\alpha \\
                                 &=\abs{\alpha}^2
\end{align*}

Now we apply the $\hata$ operator in a Fock state,
\begin{align*}
    \hata\ket{\alpha} &=\hata\left( \exp\qty[-\frac{\abs{\alpha}^2}{2}]\sum_{n=0}\frac{\alpha^n}{\sqrt{n!}}\ket{\alpha} \right) \\
                      &=\exp\qty[-\frac{\abs{\alpha}^2}{2}]\sum_{n=0}\frac{\alpha^n}{\sqrt{n!}}\hata\ket{\alpha} \\
                      &=\exp\qty[-\frac{\abs{\alpha}^2}{2}]\sum_{n=0}\frac{\alpha^n}{\sqrt{n!}}\qty(\sqrt{n}\ket{\alpha-1}),
\end{align*}
When $n=0$ is like taking photons to the vaccum, which does not make sense, hence we can translate the sum as follows,
\begin{align*}
    \hata\ket{\alpha} &=\exp\qty[-\frac{\abs{\alpha}^2}{2}]\sum_{n=1}\frac{\alpha^n}{\sqrt{(n-1)!}}\ket{n-1}
\end{align*}

What is the probability of detecting $n$ photons\footnote{Is the Poisson distribution.}?
\begin{gather*}
    \abs{\braket{n}{\alpha}}^2=\exp\qty[-\frac{\abs{\alpha}^2}{2}]\frac{\abs{\alpha}^2}{n!}
\end{gather*}

Now wetake the expected value of $\ket{\alpha}$ with $\hata,\hatad,\QDi,\QDj$,
\begin{align*}
    \expval{\hata}{\alpha} &= \alpha \\
    \expval{\hatad}{\alpha} &= \alpha^* \\
    \expval{\QDi}{\alpha} &= \frac{1}{2}(\alpha+\alpha^*) \\
    \expval{\QDj}{\alpha} &= \frac{1}{2}(\alpha-\alpha^*)
\end{align*}

Then we checked the Qadrature noise,
\begin{align*}
    \expval{\QDi^2}{\alpha} &=\expval{\frac{1}{4}(\hata+\hatad)^2}{\alpha}\\
                            &=\frac{1}{4}\qty(\alpha^2+\alpha^{*2}+2\abs{\alpha}^2+1)
\end{align*}
therefore,
\begin{gather*}
    \braket{\QDi^2} - \braket{\QDi}^2=\frac{1}{4}
\end{gather*}
\marginpar{As homework, compute $\expval{\QDj^2}{\alpha}$}

\subsection{Displacement Operator}

\section{Squeezed States}
February 26

\section{Beam Splitters}
March 10

In this session we start the second module of the course with analyzing the beam splitter in a quantum framework.

In the classical framework we can model the beam splitter using the Fresnel Coefficients.
Assuming with that we have an power input we analyze the reflection power and a transmision coefficient.
It is important to acknoledge that the reflective coefficient is complex.
This can be relatead to a phase shift during the reflection.
Also, we set that $\abs{r}^2 + \abs{t}^2 = 1$.

Now lets go Quantum, that is, that instead of analyzing the power input and output in the beam splitter we are going to explore the operators for each light beam.
Lets assume that the operator $\hata_1$ represent the input light beam, the operator $\hata_2=r\hata_1$ represent the light beam that is reflected and finally, an operator $\hata_3=r\hata_1$ that takes into account the transmitted light beam.
All of these operators needs to be indepenedent, because are representend 3 different modes.
Also, we kwnow that the creation and anhilihation operators follows these commutation rules,
\begin{gather*}
    \commutator{\hata_i}{\hatad_j}=\delta_{ij},\quad
    \commutator{\hatad_i}{\hatad_j}=0,\quad
    \commutator{\hata_i}{\hata_j}=0.
\end{gather*}


From those relations we can see that,
\begin{align*}
    \commutator{\hata_2}{\hatad_1} &= \delta_{}


\end{document}
