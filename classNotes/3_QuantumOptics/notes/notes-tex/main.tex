%%%%%%%%%%%%%%%%%%%%%%%%%%%%%%%%%%%%%%%%%
% Tufte Essay
% LaTeX Template
% Version 2.0 (19/1/19)
%
% This template originates from:
% http://www.LaTeXTemplates.com
%
% Authors:
% The Tufte-LaTeX Developers (https://www.ctan.org/pkg/tufte-latex)
% Vel (vel@LaTeXTemplates.com)
%
% License:
% Apache License, version 2.0
%
%%%%%%%%%%%%%%%%%%%%%%%%%%%%%%%%%%%%%%%%%

%----------------------------------------------------------------------------------------
%	PACKAGES AND OTHER DOCUMENT CONFIGURATIONS
%----------------------------------------------------------------------------------------

\documentclass[a4paper]{tufte-handout} % Use A4 paper by default, remove 'a4paper' for US letter

\usepackage{graphicx} % Required for including images
\setkeys{Gin}{width=\linewidth, totalheight=\textheight, keepaspectratio} % Default images settings
\graphicspath{{Figures/}{./}} % Specifies where to look for included images (trailing slash required)

\usepackage{amsmath, amsfonts, amssymb, amsthm} % For math equations, theorems, symbols, etc
\usepackage{units} % Non-stacked fractions and better unit spacing
\usepackage{physics}

\usepackage{booktabs} % Required for better horizontal rules in tables

%----------------------------------------------------------------------------------------
%	TITLE SECTION
%----------------------------------------------------------------------------------------

\title{ Quantumm Optics Class-Notes and others }

\author{Francisco Vazquez-Tavares}

\date{\today} % Date, use \date{} for no date


%----------------------------------------------------------------------------------------
%	COMMANDS SECTION
%----------------------------------------------------------------------------------------

\newcommand{\hata}{\hat{a}}
\newcommand{\hatad}{\hat{a}^\dagger}
\newcommand{\QDi}{\hat{X}_1}
\newcommand{\QDj}{\hat{X}_2}


%----------------------------------------------------------------------------------------

\begin{document}

\maketitle % Print the title section

%----------------------------------------------------------------------------------------
%	ABSTRACT/SUMMARY
%----------------------------------------------------------------------------------------

\begin{abstract}
	\textbf{Summary}
        Class notes, post-class notes and others for the course of Quantum Optics.
        Semester February-June 2025
\end{abstract}

%----------------------------------------------------------------------------------------
%	ESSAY BODY
%----------------------------------------------------------------------------------------

\section{Harmonic oscilator}\label{sec:Feb12}
February 12


\section{First Quantization}\label{sec:Feb17}
February 17

\section{Properties of Quantumm electric field}\label{sec:Feb19}
Februrary 19

\subsection{Fock States}
February 19

\section{Coherent states}\label{sec:Feb24}
February 24

\marginpar{
Reminders of some properties.
\begin{align*}
    \qty[\hata,\hatad] &= 1 \\
    \hat{X}_1 &= \frac{\hata+\hatad}{2} \\
    \hat{X}_2 &= \frac{\hata-\hatad}{2i} \\
    \hata &= \hat{X}_1 + i\hat{X}_2 \\
    \hatad &= \hat{X}_1 -i\hat{X}_2 \\
    \qty[\hat{X}_1,\hat{X}_2] &= \frac{i}{2} \\
    \Delta\hat{X}_1\Delta\hat{X}_2 &\geq \frac{1}{4}
\end{align*}

}

According to the professor the sections \ref{sec:Feb12,sec:Feb17,sec:Feb19,sec:Feb24}, where the hard part of the course.
The next sections we are going to describe different states and analize there properties.

Lets remember the operator of an enlectric field with $x$ component and one mode,
\begin{gather*}
    E_x = i\left(\frac{\hbar\omega}{2\epsilon_o V}\right)^{1/2}\left(\hata\exp\qty[-i\omega t] - \hatad\exp\qty[-i\omega t]\right),
\end{gather*}
which can be expressed in terms of quadratures as,
\begin{gather*}
    E_x = 2\left(\frac{\hbar\omega}{2\epsilon_o V}\right)^{1/2}\qty( \hat{X}_1\sin\qty(\omega t) +  \hat{X}_2\cos\qty(\omega t))
\end{gather*}

So, when we want to get the expeted value of the electric field of a Fock state $\ket{n}$ we get $\expval{E}{n}=0$.
So we need other states to model a laser.
A useful observation is that $\expval{\QDi^2}{n}=1/4(2n+1)$ and $\Delta\QDi\Delta\QDj=1/4(2n+1)$.

We are going to study the ``\textit{Gleuber states}''.
Which are the states that can describe the laser.
For that we have 4 definitions,
\paragraph{Definition 1}
Eigenstates of $\hata$ \[\hata\ket{\alpha}=\alpha\ket{\alpha},\quad\alpha\in\mathbb{C}.\]

\paragraph{Definition 2}
Displaced vacuum\footnote{
    It is important to use the following definition of $e^x$, becuase the argument are matrices and vectors,
\[\exp\qty(x)=\sum\frac{x^n}{n!}.\]
}.

\begin{gather*}
    \hat{D}\qty(\alpha) = \exp\qty[\alpha\hatad-\alpha^*\hata],\quad\ket{\alpha}=\hat{D}(\alpha)\ket{0}
\end{gather*}

\paragraph{Definition 3}
Fock States
\begin{gather*}
    \ket{\alpha} = \exp\qty[-\frac{\abs{\alpha}^2}{2}]\sum\frac{\alpha^n}{\sqrt{n!}}\ket{n}
\end{gather*}

\paragraph{Definition 4}
\[\Delta\QDi\Delta\QDj=\frac{1}{2}\]

\section{Squeezed States}
February 26



\end{document}
