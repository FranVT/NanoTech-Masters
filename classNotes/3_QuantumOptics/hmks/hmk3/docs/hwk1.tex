\documentclass[../main.tex]{subfiles}

\begin{document}

\section{Problem 4.2}

Use separation of variable in \textit{cartesian} coordinates to solve the infinite \textit{cubical} well (or particle in a box):
\begin{gather*}
    V(x,y,z) = 
    \left\{
        \begin{array}{cc}
            0, & \forall x,y,z \in \qty[0,a] \\
            \infty, & \forall x,y,z \notin \qty[0,a] 
        \end{array}
    \right.
\end{gather*}

\begin{enumerate}
    \item Find the stationary states, and the corresponding energies.
    \item Call the distinct energies $E_1,E_2,\dots$ in order of increasing energy.
        Find $E_1,E_2,E_3,E_4,E_5$ and $E_6$.
        Determine their degeneracies (that is, the number of different states that share the same energy).
    \item What is the degeneracy of $E_{14}$, and why is this case interesting?
\end{enumerate}

\begin{sol}{Stationary states}{label-1a}
    To find the stationary states of the infinite cubical well, we are going to solve the time independent Schrödinger equation,
    \begin{gather*}
        -\frac{\hbar^2}{2m}\nabla\psi=E\psi,~\forall x,y,z \in \qty[0,a],
    \end{gather*}
    with the following boundary conditions $\psi(0,0,0)=\psi(a,a,a)=0$.
    To solve the equation we are going to use the method of separation of variables, that is, that we assume that the solution of the differential equation has the following form $\psi(x,y,z)=X(x)Y(y)Z(z)$.
    Substituting this solution to the differential equation, we can perfom some algebraic manipulation,
    \begin{gather*}
        -\frac{\hbar^2}{2m}\nabla\psi =E\psi \\
        \qty(\pdv[2]{x}+\pdv[2]{y}+\pdv[2]{z})X(x)Y(y)Z(z) = -\frac{2m}{\hbar^2}EX(x)Y(y)Z(z) \\
        Y(y)Z(z)\pdv[2]{x}X(x)+X(x)Z(z)\pdv[2]{y}Y(y)+X(x)Y(y)\pdv[2]{z}Z(z) = -\frac{2m}{\hbar^2}EX(x)Y(y)Z(z) \\
        \frac{1}{X(x)}\pdv[2]{x}X(x)+\frac{1}{Y(y)}\pdv[2]{y}Y(y)+\frac{1}{Z(z)}\pdv[2]{z}Z(z)= -\frac{2m}{\hbar^2}E.
    \end{gather*}
    Now we can re-write this partial differential equation into three diferential equations assumming that $E=\frac{\hbar^2}{2m}\qty(k_x^2+k_y^2+k_z^2)$,
    \begin{align*}
        \dv[2]{X(x)}{x} = -k_x^2X(x) &\rightarrow X(x)=A_x\sin\qty[k_x x]+B_x\cos\qty[k_x x],\\
        \dv[2]{Y(y)}{y} = -k_y^2Y(y) &\rightarrow Y(y)=A_y\sin\qty[k_y y]+B_y\cos\qty[k_y y],\\
        \dv[2]{Z(z)}{z} = -k_z^2Z(z) &\rightarrow Z(z)=A_z\sin\qty[k_z z]+B_z\cos\qty[k_z z].
    \end{align*}
    In order to find the expression for the coefficients $A_n,~B_n$ and $k_n$, we start by applying the boundary conditions.
    Since $\sin$ and $\cos$ are periodic functions, they satisfy $f(0)=f(a)$, however only the $\sin$ function satisfy the condition $f(0)=f(a)=0$, hence, we set $B_x=B_y=B_z=0$ leading to,
     \begin{gather*}
        X(x)=A_x\sin\qty[k_x x],\quad Y(y)=A_y\sin\qty[k_y y],\quad Z(z)=A_z\sin\qty[k_z z].
    \end{gather*}
    Now we recall the fact that $x,y$ and $z$ have units of distance and that the argument of the $\sin$ function must be dimensonless, combining this restriction with the property of periodicity we can define the constants $k_n$ as, $k_x=n_x\pi/a,~k_y=n_y\pi/a,~k_z=n_z\pi/a$, where $\qty(n_x,n_y,n_z)\in\mathbb{Z}^+$.
    With this information we can re-write the solution as,
    \begin{gather*}
        \psi\qty(x,y,z)=A_x A_y A_z\sin\qty[\frac{n_x\pi}{a} x]\sin\qty[\frac{n_y\pi}{a} y]\sin\qty[\frac{n_z\pi}{a} z],
    \end{gather*} 
    with
    \begin{gather*}
        E=\frac{\pi^2\hbar^2}{2ma^2}\qty(n_x^2+n_y^2+n_z^2), \quad \qty(n_x,n_y,n_z)\in\mathbb{Z}^+.
    \end{gather*}
    Finally, in order to get the expression for $A_x,A_y$ and $A_z$ we apply the normalization restiction to each spatial dimension,
    \begin{align*}
        \int_0^a A_l^2\sin^2\qty[\frac{n_l\pi}{a} s] ds &= A^2_l\frac{a}{4}\qty(2-\frac{1}{\pi n}\sin\qty[2\pi n]) = 1,
    \end{align*}
    since $n\in\mathbb{Z}^+$ we get that $A_l=\sqrt{2/a}$, therefore,
    \begin{empheq}[box=\shadowbox]{equation*}
        \psi\qty(x,y,z)=\sqrt{\frac{8}{a^3}}\sin\qty[\frac{n_x\pi}{a} x]\sin\qty[\frac{n_y\pi}{a} y]\sin\qty[\frac{n_z\pi}{a} z], \quad \qty(n_x,n_y,n_z)\in\mathbb{Z}^+
    \end{empheq}
\end{sol}


\begin{sol}{Energy analysis}{label-1b}
    %In order to compute the degeneracy we need to get the amount of combinations of $(n_x,n_y,n_z)$ that get the same same energy.
    %This can be done using 

\end{sol}

\begin{sol}{Energy 14}{label-1c}
    %In order to compute the degeneracy we need to get the amount of combinations of $(n_x,n_y,n_z)$ that get the same same energy.
    %This can be done using 

\end{sol}


\section{Problem 4.3}

Use 
\begin{align*}
    P^m_l(x)&\equiv\qty(1-x^2)^{\abs{m}/2}\qty(\dv{x})^{\abs{m}}P_l(x) \\
    P_l\qty(x) &\equiv\frac{1}{2^l l!}\qty(\dv{x})^l\qty(x^2-1)^l \\
    Y^m_l\qty(\theta,\phi) &=\qty(-1)^{m}\sqrt{\frac{\qty(2l+1)}{4\pi}\frac{\qty(l-\abs{m})!}{\qty(l+\abs{m})!}}e^{im\phi}P^m_l\qty(\cos\qty[\theta]) 
\end{align*}
to construct $Y^0_0$ and $Y^1_2$.
Check that they are normalized and orthogonal.

\begin{sol}{Spherical harmonic}{label-2}
    We start with $Y^0_0\qty(\theta,\phi)$, $m=l=0$ subsistuting those values into the associate Legendre prolynomials,
    \begin{align*}
        P_0\qty(x) &\equiv\frac{1}{2^0 0!}\qty(\dv{x})^0\qty(x^2-1)^0 = 1,
    \end{align*}
   and 
    \begin{align*}
        P^0_0(x)&\equiv\qty(1-x^2)^{\abs{0}/2}\qty(\dv{x})^{\abs{0}}P_0(x) = 1,
    \end{align*}
     hence,
    \begin{align*}
        Y^0_0\qty(\theta,\phi) &=\frac{1}{\sqrt{4\pi}}.
    \end{align*}
    Now, to check if it is normalize we integrate in spherical coordinates from $\theta\in\qty(0,\pi)$ and $\phi\in\qty(0,2\pi)$,
    \begin{align*}
        \int_0^\pi\int_0^{2\pi}\abs{Y^0_0\qty(\theta,\phi)}^2\sin\theta d\theta d\phi &= \int_0^\pi\int_0^{2\pi}\frac{1}{4\pi}\sin\theta d\theta d\phi \\
                                                                                      &= \frac{1}{4\pi}\qty[\int_0^\pi\sin\theta d\theta]\qty[\int_0^{2\pi} d\phi] \\
                                                                                      &= \frac{1}{4\pi}\qty[2]\qty[2\pi] \\
                                                                                      &= 1 
    \end{align*}

    \begin{empheq}[box=\shadowbox]{equation*}
        \int_0^\pi\int_0^{2\pi}\abs{Y^0_0\qty(\theta,\phi)}^2\sin\theta d\theta d\phi = 1
    \end{empheq}

    Now we do the same procedure with $Y^1_2$, $m=1$ and $l=2$, which gives that $P^1_2(x)=\sqrt{1-x^2}\dv{x}P_2(x)$ and $P_2(x)=1/2\qty(3x^2-1)$, hence,
    \begin{align*}
        P_2\qty(x) &\equiv\frac{1}{2^2 2!}\qty(\dv{x})^2\qty(x^2-1)^2 = \frac{1}{2}\qty(3x^2-1)
    \end{align*}
    and
    \begin{align*}
        P^1_2(x)&\equiv\qty(1-x^2)^{\abs{1}/2}\qty(\dv{x})^{\abs{1}}P^1_2(x)=3x\sqrt{1-x^2}
    \end{align*}

    \begin{align*}
        Y^1_2\qty(\theta,\phi) &=-\sqrt{\frac{\qty(2(2)+1)}{4\pi}\frac{\qty(2-\abs{1})!}{\qty(2+\abs{1})!}}e^{im\phi}P^1_2\qty(\cos\qty[\theta]) \\ 
                               &=-\sqrt{\frac{5}{4\pi}\frac{1}{6}}e^{i\phi}3\cos\qty[\theta]\sqrt{1-\cos^2\qty[\theta]} =-\sqrt{\frac{5}{24\pi}}e^{i\phi}\sqrt{9}\cos\qty[\theta]\sin\qty[\theta] \\
                               &=-\sqrt{\frac{15}{8\pi}}e^{i\phi}\cos\qty[\theta]\sin\qty[\theta]
    \end{align*}
    Now we check if the function is normalize,
    \begin{align*}
        \int_0^\pi\int_0^{2\pi}\abs{Y^1_2\qty(\theta,\phi)}^2\sin\theta d\theta d\phi &= \int_0^\pi\int_0^{2\pi}\frac{15}{8\pi}\cos^2\qty[\theta]\sin^2\qty[\theta] \sin\theta d\theta d\phi \\
                                                                                      &= \frac{15}{8\pi}\qty[\int_0^\pi \cos^2\qty[\theta]\sin^2\qty[\theta] \sin\theta d\theta]\qty[\int_0^{2\pi} d\phi] \\
                                                                                      &= \frac{15}{8\pi}\qty[\frac{4}{15}]\qty[2\pi] \\
                                                                                      &= 1 
    \end{align*}

    \begin{empheq}[box=\shadowbox]{equation*}
        \int_0^\pi\int_0^{2\pi}\abs{Y^1_2\qty(\theta,\phi)}^2\sin\theta d\theta d\phi = 1
    \end{empheq}

    Finally, to check orthogonality we perform th following procedure,
    \begin{align*}
        \int_0^\pi\int_0^{2\pi}\qty[Y^0_0\qty(\theta,\phi)]^* Y^1_2\qty(\theta,\phi)\sin\theta d\theta d\phi &= \int_0^\pi\int_0^{2\pi}\qty(\frac{1}{\sqrt{4\pi}})^*\qty(-\sqrt{\frac{15}{8\pi}}e^{i\phi}\cos\qty[\theta]\sin\qty[\theta]) \sin\theta d\theta d\phi \\
                                                                                                             &= -\frac{1}{\sqrt{4\pi}}\sqrt{\frac{15}{8\pi}}\qty[\int_0^\pi\cos\qty[\theta]\sin\qty[\theta] \sin\theta d\theta]\qty[\int_0^{2\pi}e^{i\phi}d\phi] \\
                                                                                                             &=-\sqrt{\frac{15}{32\pi^2}}\qty[0]\qty[0] \\
                                                                                                             &=0
    \end{align*}

    \begin{empheq}[box=\shadowbox]{equation*}
        \int_0^\pi\int_0^{2\pi}\qty[Y^0_0\qty(\theta,\phi)]^* Y^1_2\qty(\theta,\phi)\sin\theta d\theta d\phi = 0
    \end{empheq}

\end{sol}


\section{Problem 4.13}

\begin{itemize}
    \item Find $\expval{r}$ and $\expval{r^2}$ for an electron in the ground state of hydrogen.
        Express your answers in terms of the Bohr radius ($\rho$).
    \item Find $\expval{x}$ and $\expval{x^2}$ for an electron in the ground state of hydrogen.
        \textit{Hint:} this requires no new integration-note that $r^2=x^2+y^2+z^2$, and exploit the symmetry of the ground state.
    \item Find $\expval{x^2}$ in the state $n=2,l=1,m=1$. \textit{Warning:} This state is not symmetrical in $x,y,z$.
        Use $x=r\sin\theta\cos\phi$.
\end{itemize}

\begin{sol}{Expected value of position.}{label-3a}
    By solving the Schrodinger equation in spherical coordinates with the Coulomb's law as the potential energy, the stationary states are in terms of the Bohr's radius,
    \begin{gather*}
        \psi_{\qty(n,m,l)}\qty(r,\theta,\phi) = \frac{1}{r}\rho^{l+1}e^{-\rho}\upsilon\qty(\rho)Y^m_l\qty(\theta,\phi),
    \end{gather*}
    with $\upsilon\qty(\rho)$ being a polynimial of degree $j_{\max}=n-l-1$ with coefficients,
    \begin{gather*}
        c_{j+1} = \frac{2\qty(j+l+1-n)}{\qty(j+1)\qty(j+2l+2)}c_j.
    \end{gather*}
    For the ground state of the Hydrogen atom we set the parameters to $\qty(n=1,l=0,m=0)$, which gives,
    \begin{gather*}
        \psi_{1,0,0}\qty(r,\theta,\phi) = \frac{1}{\sqrt{\pi a^3}}e^{-r/a}.
    \end{gather*}

    To compute the expected value we perform the following operation,
    \begin{align*}
        \expval{r} &= \int_V r\abs{\psi_{1,0,0}\qty(r,\theta,\phi)}^2 dV \\
                   &= \int_{0}^{\infty}\int_0^\pi\int_0^{2\pi} r \frac{1}{\pi a^3}e^{-2r/a} r^2\sin\theta d\theta d\phi dr \\
                   &= \frac{1}{\pi a^3}\int_{0}^{\infty} e^{-2r/a} r^3 dr\int_0^\pi\sin\theta d\theta\int_0^{2\pi} d\phi \\ 
                   &= \frac{1}{\pi a^3}\qty(\frac{3}{8}a^4)\qty(2)\qty(2\pi) \\ 
                   &= \frac{3}{2}a.
    \end{align*}
    Now, for $\expval{r^2}$,
    \begin{align*}
        \expval{r^2} &= \int_V r^2 \abs{\psi_{1,0,0}\qty(r,\theta,\phi)}^2 dV \\
                     &= \int_{0}^{\infty}\int_0^\pi\int_0^{2\pi} r^2 \frac{1}{\pi a^3}e^{-2r/a} r^2\sin\theta d\theta d\phi dr \\
                     &= \frac{1}{\pi a^3}\int_{0}^{\infty} e^{-2r/a} r^4 dr\int_0^\pi\sin\theta d\theta\int_0^{2\pi} d\phi \\ 
                     &= \frac{1}{\pi a^3}\qty(\frac{3}{4}a^5)\qty(2)\qty(2\pi) \\ 
                     &= 3a^2.
    \end{align*}
    Therefore,
    \begin{empheq}[box=\shadowbox]{equation*}
        \expval{r} = \frac{3}{2}a,\quad \expval{r^2} = 3a^2
    \end{empheq}
\end{sol}

\begin{sol}{Expected values and standard deviation of the $x$ component.}{label-3b}
    Recalling the hint, $r^2=x^2+y^2+z^2$ and using the symmetry of the ground state we can conclude that,
    \begin{align*}
        \expval{x^2} &= \frac{1}{3}\expval{r^2} \\
                     &= a^2.
    \end{align*}
    On the other hand, for $\expval{x}$ we can write the integrals considering that $x=r\sin\theta\cos\phi$,
    \begin{align*}
        \expval{x} &= \int_{0}^{\infty}\int_0^\pi\int_0^{2\pi} \qty(r\sin\theta\cos\phi) \frac{1}{\pi a^3}e^{-2r/a} r^2\sin\theta d\theta d\phi dr \\
                  &= \frac{1}{\pi a^3}\int_{0}^{\infty}e^{-2r/a} r^3 dr\int_0^\pi\sin^2\theta d\theta\int_0^{2\pi}\cos\phid\phi  \\
                  &= \frac{1}{\pi a^3}\qty(\frac{3}{8}a^4)\qty(\frac{\pi}{2})\qty(0) \\
                  &= 0
    \end{align*}

    Hence,
    \begin{empheq}[box=\shadowbox]{equation*}
        \expval{x} = 0,\quad \expval{x^2} = a^2
    \end{empheq}

\end{sol}

\begin{sol}{Stationary state $(n=2,l=1,m=1)$}{label-3c}
    For this case we recall the stationay state of the hydrogen atom,
    \begin{gather*}
        \psi_{\qty(n,m,l)}\qty(r,\theta,\phi) = \frac{1}{r}\rho^{l+1}e^{-\rho}\upsilon\qty(\rho)Y^m_l\qty(\theta,\phi),
    \end{gather*}
    with $\upsilon\qty(\rho)$ being a polynimial of degree $j_{\max}=n-l-1$ with coefficients,
    \begin{gather*}
        c_{j+1} = \frac{2\qty(j+l+1-n)}{\qty(j+1)\qty(j+2l+2)}c_j.
    \end{gather*}
    Applying the values of the parameters,
    \begin{gather*}
        \psi_{\qty(2,1,1)}\qty(r,\theta,\phi) = -\frac{1}{\sqrt{\pi a}}\frac{1}{8a^2}re^{-r/2a}e^{i\phi}\sin\theta. 
    \end{gather*}

    Now we can perform the previous procedures to compute $\expval{x^2}$,
    \begin{align*}
        \expval{x^2} &= \int_V r^2 \abs{\psi_{1,0,0}\qty(r,\theta,\phi)}^2 dV \\
                     &= \int_{0}^{\infty}\int_0^\pi\int_0^{2\pi} \qty(r\sin\theta\cos\phi)^2 \frac{1}{64\pi a^5}r^2e^{-r/a}\sin^2\theta r^2\sin\theta d\theta d\phi dr \\
                     &= \frac{1}{64\pi a^5}\int_{0}^{\infty}r^6e^{-r/a}dr\int_0^\pi\sin^5\theta d\theta\int_0^{2\pi}\cos^2\phi d\phi \\
                     &= \frac{1}{64\pi a^5}\qty(720a^7)\qty(\frac{16}{15})\qty(\pi) \\
                     &= 12a^2
    \end{align*}

    Therefore,
    \begin{empheq}[box=\shadowbox]{equation*}
        \expval{x^2} = 12a^2
    \end{empheq}

\end{sol}


\section{Problem 4.14}

What is the \textit{most probable} value of $r$, in the ground state of hydrogen?
(The answer is not zero!)
\textit{Hint:} First you must figure out the probability that the electron would be found between $r$ and $r+dr$.

\begin{sol}{Most probable value of the position in the ground state.}{label-4}
    To compute the most probable value of $r$ we need to compute the maximum value of the probability density function associate with the ground state.
    In order to get the probability density function we formulate the following integral,
    \begin{gather*}
        P = \int_{0}^{r}\int_0^\pi\int_0^{2\pi}\abs{\psi_{1,0,0}\qty(r,\theta,\phi)}^2 r^2\sin\theta d\theta d\phi dr,
    \end{gather*}
    since the angular symmetry is a guaranteed feature of the ground state, we can simplify the expression to,
    \begin{gather*}
        P = \int_{0}^{r}4\pi\abs{\psi_{1,0,0}\qty(r,\theta,\phi)}^2 r^2 dr.
    \end{gather*}
    Now, from the context of probability, we known that the probability density function is the integran,
    \begin{gather*}
        p(r) = 4\pi r^2\abs{\psi_{1,0,0}\qty(r,\theta,\phi)}^2.
    \end{gather*}
    Now we can get the most probable value of $r$,
    \begin{align*}
        \dv{p(r)}{x} &= \dv{x}4\pi r^2\abs{\psi_{1,0,0}\qty(r,\theta,\phi)}^2 \\
                     &= \dv{x}4\pi r^2\frac{1}{\pi a^3}e^{-2r/a} \\
                     &= \frac{1}{a^3}\qty(\frac{2}{a}\qty(a-r)re^{-2r/a}),
    \end{align*}
    finally, $\dv{x}p(r)=0$,
    \begin{align*}
        \frac{1}{a^3}\qty(\frac{2}{a}\qty(a-r)re^{-2r/a}) &= 0 \\
        are^{-2r/a} &= r^2e^{-2r/a} \\
        a &= r.
    \end{align*}

    \begin{empheq}[box=\shadowbox]{equation*}
        \dv{p(r)}{x} = 0\rightarrow r=a
    \end{empheq}
\end{sol}



\section{Problem 4.23}

In problem 4.3 you showed that 
\begin{gather*}
    Y^l_2\qty(\theta,\phi) = -\sqrt{\frac{15}{8\pi}}\sin\theta\cos\theta e^{i\phi}.
\end{gather*}
Apply the raising operator to find $Y^2_2\qty(\theta,\phi)$.
Use equation $ A^m_l =\hbar\sqrt{l\qty(l+1)-m\qty(m\pm1)}=\hbar\sqrt{\qty(l\mp m)\qty(l\pm m+1)} $ to get the normalization.

\end{document}
