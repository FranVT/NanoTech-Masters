\documentclass[../main.tex]{subfiles}

\begin{document}

\section{Problem 4.2}

Use separation of variable in \textit{cartesian} coordinates to solve the infinite \textit{cubical} well (or particle in a box):
\begin{gather*}
    V(x,y,z) = 
    \left\{
        \begin{array}{cc}
            0, & \forall x,y,z \in \qty[0,a] \\
            \infty, & \forall x,y,z \notin \qty[0,a] 
        \end{array}
    \right.
\end{gather*}

\begin{itemize}
    \item Find the stationary states, and the corresponding energies.
    \item Call the distinct energies $E_1,E_2,\dots$ in order of increasing energy.
        Find $E_1,E_2,E_3,E_4,E_5$ and $E_6$.
        Determine their degeneracies (that is, the number of different states that share the same energy).
    \item What is the degeneracy of $E_{14}$, and why is this case interesting?
\end{itemize}

\section{Problem 4.3}

Use 
\begin{align*}
    P^m_l(x)&\equiv\qty(1-x^2)^{\abs{m}/2}\qty(\dv{x})^{\abs{m}}P_l(x) \\
    P_l\qty(x) &\equiv\frac{1}{2 l!}\qty(\dv{x})^l\qty(x^2-1)^l \\
    Y^m_l\qty(\theta,\phi) &=\epsilon\sqrt{\frac{\qty(2l+1)}{4\pi}\frac{\qty(l-\abs{m})!}{\qty(l+\abs{m})!}}e^{im\phi}P^m_l\qty(\cos\qty[\theta]) 
\end{align*}
to construct $Y^0_0$ and $Y^l_2$.
Check that they are normalized and orthogonal.

\section{Problem 4.13}

\begin{itemize}
    \item Find $\expval{r}$ and $\expval{r^2}$ for an electron in the ground state of hydrogen.
        Express your answers in terms of the Bohr radius.
    \item Find $\expval{x}$ and $\expval{x^2}$ for an electron in the ground stat of hydrogen.
        \textit{Hint:} this requires no noew integration-note that $r^2=x^2+y^2+z^2$, and explot the symmetry of the ground state.
    \item Find $\expval{x^2}$ in the state $n=2,l=1,m=1$. \textit{Warning:} This state is not symmetrical in $x,y,z$.
        Use $x=r\sin\theta\cos\phi$.
\end{itemize}

\section{Problem 4.14}

What is the \textit{most probable} value of $r$, in the ground state of hydrogen?
(The answer is not zero!)
\textit{Hint:} First ypu must figure out the probability that the electron would be found between $r$ and $r+dr$.

\section{Problem 4.23}

In problem 4.3 you showed that 
\begin{gather*}
    Y^l_2\qty(\theta,\phi) = -\sqrt{\frac{15}{8\pi}}\sin\theta\cos\theta e^{i\phi}.
\end{gather*}
Apply the raising operator to find $Y^2_2\qty(\theta,\phi)$.
Use equation $ A^m_l =\hbar\sqrt{l\qty(l+1)-m\qty(m\pm1)}=\hbar\sqrt{\qty(l\mp m)\qty(l\pm m+1)} $ to get the normalization.

\end{document}
