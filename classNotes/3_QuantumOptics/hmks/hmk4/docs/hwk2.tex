\documentclass[../main.tex]{subfiles}

\begin{document}

\section{Problem 4.25}

If the electron were a classical solid sphere, with radius,
\begin{gather*}
    r_c = \frac{e^2}{4\pi\epsilon_omc^2}
\end{gather*}
(the so-called classical electron radius, obtained by assuming the electron's mass is attributable to energy sotred in its electric field, via the Einstein formula $E=mc^2$),
and its angular momentum is $\hbar/2$, then how fast (in $m/s$) would a point on the "equator" be moving?
Does this model make sense?
(Actually, the radius of the electrin is known experimentally to be much less than $r_c$ but this only makes matters worse.)

\begin{sol}{Stationary states}{label-1a}
    \begin{empheq}[box=\shadowbox]{equation*}
        \text{Sol}
    \end{empheq}
\end{sol}



\section{Problem 4.26}

\begin{itemize}
    \item Check that the spin matrices 4,145 and 4,147 obey the fundamental commutation relations for angular momentum eqn 4.134
    \item Show that the Pauli spin matrices 4.148 satisfy the product rule 
        \begin{gather*}
            \sigma_j\sigma_k=\delta_{jk}+i\sum_l\epsilon_{jkl}\sigma_l,
        \end{gather*}
        where the indices stand for $x,y,z$ and $\epsilon_{jkl}$ is the Levi-Civita symbol.
\end{itemize}

\section{Problem 4.27}

An electron is in the spin state,
\begin{gather*}
    \Xi=A\begin{pmatrix}3i\\4\end{pmatrix}
\end{gather*}

\begin{itemize}
    \item Determine the normalization constant $A$.
    \item Find the expectation values of $S_x, S_y$ and $S_z$.
    \item Find the ``uncertanties'' $\sigma_{S_x},\sigma_{S_y}$ and $\sigma_{S_z}$. (Note: These sigmas are standard deviations, not Pauli matrices!)
    \item Confirm that your results are consistent with all three uncertanty principles 4.100 and its cyclic permutations-only with $S$ in place of $L$, of course.)
\end{itemize}

\section{Problem 4.32 a)}

If ypu measure the component of spin angular momentum along the $x$ direction, at time $t$, what is the probability that ypu would get $+\hbar/2$?

\end{document}
