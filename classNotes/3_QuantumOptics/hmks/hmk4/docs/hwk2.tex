\documentclass[../main.tex]{subfiles}

\begin{document}

\section{Problem 4.25}

If the electron were a classical solid sphere, with radius,
\begin{gather*}
    r_c = \frac{e^2}{4\pi\epsilon_o mc^2}
\end{gather*}
(the so-called classical electron radius, obtained by assuming the electron's mass is attributable to energy sotred in its electric field, via the Einstein formula $E=mc^2$),
and its angular momentum is $\hbar/2$, then how fast (in $m/s$) would a point on the ``equator'' be moving?
Does this model make sense?
(Actually, the radius of the electrin is known experimentally to be much less than $r_c$ but this only makes matters worse.)

\begin{sol}{Classical spinning}{label-1a}
    From the classical framework the angular momentum is modeled with the following relation,
    \begin{gather*}
        L = I\omega,
    \end{gather*}
    where $I$ is the moment of interia, which in this case is $I=2/5mr^2$ and $w$ is the angular frequency, that can be express as $\omega=v/r$.
    Replacing this equivalences into the angular momentum equation we can get the following expression for $v$,
    \begin{gather*}
        v = \frac{5}{2}\frac{L}{mr_c},
    \end{gather*}
    substituting the values of $L$ and $r_c$,
    \begin{gather*}
        v = \frac{5\pi\hbar\epsilon_o}{e^2}c^2.
    \end{gather*}
    Recalling the order of magnitud of the constants, $e\approx\num{d-19}$, $\hbar\approx\num{d-34}$, $\epsilon_o\approx\num{d-12}$ and $c\approx\num{d8}$, we get that,
    \begin{gather*}
        \frac{5\pi\hbar\epsilon_o}{e^2}c\approx 90,
    \end{gather*}
    which tells us that the velocity at the ecuator is $90$ times the velocity of light, which does not make sense.

    \begin{empheq}[box=\shadowbox]{equation*}
        v\approx90c.
    \end{empheq}
\end{sol}



\section{Problem 4.26}

\begin{itemize}
    \item Check that the spin matrices\eqref{eqn:spin-matrices} obey the fundamental commutation relations for angular momentum\eqref{eqn:comm-rela}.
        \begin{gather}
            \hat{S}_x = \frac{\hbar}{2}\begin{pmatrix}0&1\\1&0\end{pmatrix},
            \quad
            \hat{S}_y = \frac{\hbar}{2}\begin{pmatrix}0&-i\\i&0\end{pmatrix},
            \quad
            \hat{S}_z = \frac{\hbar}{2}\begin{pmatrix}1&0\\0&-1\end{pmatrix}
            \label{eqn:spin-matrices} 
            \\
            \comm{\hat{S}_x}{\hat{S}_y}=i\hbar\hat{S}_z,
            \quad
            \comm{\hat{S}_y}{\hat{S}_z}=i\hbar\hat{S}_x,
            \quad
            \comm{\hat{S}_z}{\hat{S}_x}=i\hbar\hat{S}_y
            \label{eqn:comm-rela} 
        \end{gather}
    \item Show that the Pauli spin matrices\eqref{eqn:pauli-matrices} satisfy the product rule\eqref{eqn:prod_rule}, where the indices stand for $x,y,z$ and $\epsilon_{jkl}$ is the Levi-Civita symbol.
        \begin{gather}
            \sigma_x = \frac{\hbar}{2}\begin{pmatrix}0&1\\1&0\end{pmatrix},
            \quad
            \sigma_y = \frac{\hbar}{2}\begin{pmatrix}0&-i\\i&0\end{pmatrix},
            \quad
            \sigma_z = \frac{\hbar}{2}\begin{pmatrix}1&0\\0&-1\end{pmatrix}
            \label{eqn:pauli-matrices} 
            \\
            \sigma_j\sigma_k=\delta_{jk}+i\sum_l\epsilon_{jkl}\sigma_l\label{eqn:prod_rule}.
        \end{gather}
        
\end{itemize}

\begin{sol}{Commutation relations}{label-2a}
    We start computing $\comm{\hat{S}_x}{\hat{S}_y}$,
    \begin{align*}
        \comm{\hat{S}_x}{\hat{S}_y} &=
            \frac{\hbar^2}{4}\left[
                \begin{pmatrix}i&0\\0&-i\end{pmatrix}
                -
                \begin{pmatrix}-i&0\\0&i\end{pmatrix}
            \right]
            \\
                                 &=i\hbar\frac{\hbar}{2}\begin{pmatrix}1&0\\0&-1\end{pmatrix}
                                 \\
                                 &=i\hbar\hat{S}_z.
    \end{align*}

    Now $\comm{\hat{S}_y}{\hat{S}_z}$,
    \begin{align*}
        \comm{\hat{S}_y}{\hat{S}_z} &=
            \frac{\hbar^2}{4}\left[
                \begin{pmatrix}0&i\\i&0\end{pmatrix}
                -
                \begin{pmatrix}0&-i\\-i&0\end{pmatrix}
            \right]
            \\
                                 &=i\hbar\frac{\hbar}{2}\begin{pmatrix}0&1\\1&0\end{pmatrix}
                                 \\
                                 &=i\hbar\hat{S}_x.
    \end{align*}

    Finally $\comm{\hat{S}_z}{\hat{S}_x}$,
    \begin{align*}
        \comm{\hat{S}_z}{\hat{S}_x} &=
            \frac{\hbar^2}{4}\left[
        \begin{pmatrix}0&1\\-1&0\end{pmatrix}
                -
                \begin{pmatrix}0&-1\\1&0\end{pmatrix}
            \right]
            \\
                                 &=\qty(-ii)\frac{\hbar}{2}\begin{pmatrix}0&1\\-1&0\end{pmatrix}
                                 \\
                                 &=i\hbar\hat{S}_y.
    \end{align*}

    Hence, it is proof that,
    \begin{empheq}[box=\shadowbox]{equation*}
        \comm{\hat{S}_x}{\hat{S}_y}=i\hbar\hat{S}_z,
            \quad
            \comm{\hat{S}_y}{\hat{S}_z}=i\hbar\hat{S}_x,
            \quad
            \comm{\hat{S}_z}{\hat{S}_x}=i\hbar\hat{S}_y    
    \end{empheq}
\end{sol}

\begin{sol}{Pauli spin matrices properties}{label-2b}
    To proof that the product rule is satified by the Pauli matrices, we compute all the products.
    Strating with $\sigma_x\sigma_x$, $\sigma_y\sigma_y$, $\sigma_z\sigma_z$,
    \begin{align*}
        \sigma_x\sigma_x  &= \mqty(\pmat{x})\mqty(\pmat{x}) = \mqty(\imat{2}), \\
        \sigma_y\sigma_y  &= \mqty(\pmat{y})\mqty(\pmat{y}) = \mqty(\imat{2}), \\
        \sigma_z\sigma_z  &= \mqty(\pmat{z})\mqty(\pmat{z}) = \mqty(\imat{2}).
    \end{align*}

    Now, for the cross terms,
    \begin{align*}
        \sigma_x\sigma_y  &= \mqty(\pmat{x})\mqty(\pmat{y}) = i\mqty(\pmat{z}), \\
        \sigma_y\sigma_z  &= \mqty(\pmat{y})\mqty(\pmat{z}) = i\mqty(\pmat{x}), \\
        \sigma_x\sigma_z  &= \mqty(\pmat{x})\mqty(\pmat{z}) = i\mqty(\pmat{y}).
    \end{align*}

    Which confirms that Pauli's matrices satifies the product relation.

\end{sol}

\section{Problem 4.27}

An electron is in the spin state,
\begin{gather*}
    \chi=A\begin{pmatrix}3i\\4\end{pmatrix}
\end{gather*}

\begin{enumerate}
    \item Determine the normalization constant $A$.
    \item Find the expectation values of $S_x, S_y$ and $S_z$.
    \item Find the ``uncertanties'' $\sigma_{S_x},\sigma_{S_y}$ and $\sigma_{S_z}$. (Note: These sigmas are standard deviations, not Pauli matrices!)
    \item Confirm that your results are consistent with all three uncertanty principles 4.100 and its cyclic permutations-only with $S$ in place of $L$, of course.
\end{enumerate}

\begin{sol}{Normalization constant}{label-3a}
    To find the normalization constant we need to get the squared modulus of the state and solve for $A$,
    \begin{align*}
        \abs{\chi}^2 &= A\begin{pmatrix}-3i&4\end{pmatrix}A\begin{pmatrix}3i\\4\end{pmatrix} \\
                     &= 25A^2,
    \end{align*}
    tacking into account that $\abs{\chi}^2=1$ we get that $A=1/5$.

    \begin{empheq}[box=\shadowbox]{equation*}
        A=\frac{1}{5}
    \end{empheq}
\end{sol}

\begin{sol}{Expected values of spin}{label-3b}
    To find $\expval{\hat{S}_x},\expval{\hat{S}_y}$ and $\expval{\hat{S}_z}$ we compute, $\expval{\hat{S}_n}{\chi}$,
    \begin{align*}
        \expval{\hat{S}_x}{\chi} &=\frac{\hbar}{50}\mqty(-3i&4)\mqty(\pmat{x})\mqty(-3i\\4)=\frac{\hbar}{50}\qty(-12i+12i)=0, \\
        \expval{\hat{S}_y}{\chi} &=\frac{\hbar}{50}\mqty(-3i&4)\mqty(\pmat{y})\mqty(-3i\\4)=\frac{\hbar}{50}\qty(-12-12)=-\frac{12}{25}\hbar, \\
        \expval{\hat{S}_z}{\chi} &=\frac{\hbar}{50}\mqty(-3i&4)\mqty(\pmat{z})\mqty(-3i\\4)=\frac{\hbar}{50}\qty(9-16)=-\frac{7}{50}\hbar. 
    \end{align*}

    \begin{empheq}[box=\shadowbox]{equation*}
        \expval{\hat{S}_x}=0,\quad 
        \expval{\hat{S}_y}=-\frac{12}{25}\hbar,\quad
        \expval{\hat{S}_z}=-\frac{7}{50}\hbar.
    \end{empheq}
\end{sol}

\begin{sol}{Uncertanties}{label-3c}
    To compute the uncertanty we recall that $\sigma=\sqrt{\expval{\hat{S}^2} - \expval{\hat{S}}^2}$.
    From the previous task we already known $\expval{\hat{S}}$, hence we only need to compute $\expval{\hat{S}^2}$.
    \begin{align*}
        \expval{\hat{S}_x^2}{\chi} &=\frac{\hbar^2}{100}\mqty(-3i&4)\mqty(\pmat{x})\mqty(\pmat{x})\mqty(-3i\\4)=\frac{\hbar}{50}\qty(9+16)=\frac{\hbar^2}{4}, \\
        \expval{\hat{S}_y^2}{\chi} &=\frac{\hbar^2}{100}\mqty(-3i&4)\mqty(\pmat{y})\mqty(\pmat{y})\mqty(-3i\\4)=\frac{\hbar}{50}\qty(9+16)=\frac{\hbar^2}{4}, \\
        \expval{\hat{S}_z^2}{\chi} &=\frac{\hbar^2}{100}\mqty(-3i&4)\mqty(\pmat{z})\mqty(\pmat{z})\mqty(-3i\\4)=\frac{\hbar}{50}\qty(9+16)=\frac{\hbar^2}{4}. 
    \end{align*}

    Now we can compute the uncertanties,
    \begin{align*}
        \sigma_x&=\sqrt{\frac{\hbar^2}{4} - 0}=\frac{\hbar}{2}, \\
        \sigma_y&=\sqrt{\frac{\hbar^2}{4} - \frac{144}{625}\hbar^2}=\frac{7}{50}\hbar, \\
        \sigma_z&=\sqrt{\frac{\hbar^2}{4} - \frac{49}{2500}\hbar^2}=\frac{12}{25}\hbar. 
    \end{align*}


    \begin{empheq}[box=\shadowbox]{equation*}
        \sigma_x=\frac{\hbar}{2},\quad
        \sigma_y=\frac{7}{50}\hbar,\quad
        \sigma_z=\frac{12}{25}\hbar.
    \end{empheq}
\end{sol}

\begin{sol}{Checking consistency with the uncertanty principle.}{label-3d}
    With the uncertanty relation, $\sigma_A\sigma_B\geq1/2\abs{\expval{\comm{\hat{A}}{\hat{B}}}}$, we can test the previous results,

    \begin{align*}
        \sigma_{\hat{S}_x}\sigma_{\hat{S}_y}&\geq\frac{1}{2}\abs{\expval{\hat{S}_z}}=\frac{1}{2}\frac{7}{50}\hbar^2
    \end{align*}

    \begin{empheq}[box=\shadowbox]{equation*}
        \text{Sol}
    \end{empheq}
\end{sol}


\section{Problem 4.32 a}

If you measure the component of spin angular momentum along the $x$ direction, at time $t$, what is the probability that you would get $+\hbar/2$?

\begin{sol}{Practice spin state}{label-4}
    \begin{empheq}[box=\shadowbox]{equation*}
        \text{Sol}
    \end{empheq}
\end{sol}



\end{document}
