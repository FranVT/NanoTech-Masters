\documentclass[../main.tex]{subfiles}

\begin{document}

\section{Problem 4.25}

If the electron were a classical solid sphere, with radius,
\begin{gather*}
    r_c = \frac{e^2}{4\pi\epsilon_o mc^2}
\end{gather*}
(the so-called classical electron radius, obtained by assuming the electron's mass is attributable to energy sotred in its electric field, via the Einstein formula $E=mc^2$),
and its angular momentum is $\hbar/2$, then how fast (in $m/s$) would a point on the ``equator'' be moving?
Does this model make sense?
(Actually, the radius of the electrin is known experimentally to be much less than $r_c$ but this only makes matters worse.)

\begin{sol}{Classical spinning}{label-1a}
    From the classical framework the angular momentum is modeled with the following relation,
    \begin{gather*}
        L = I\omega,
    \end{gather*}
    where $I$ is the moment of interia, which in this case is $I=2/5mr^2$ and $w$ is the angular frequency, that can be express as $\omega=v/r$.
    Replacing this equivalences into the angular momentum equation we can get the following expression for $v$,
    \begin{gather*}
        v = \frac{5}{2}\frac{L}{mr_c},
    \end{gather*}
    substituting the values of $L$ and $r_c$,
    \begin{gather*}
        v = \frac{5\pi\hbar\epsilon_o}{e^2}c^2.
    \end{gather*}
    Recalling the order of magnitud of the constants, $e\approx\num{d-19}$, $\hbar\approx\num{d-34}$, $\epsilon_o\approx\num{d-12}$ and $c\approx\num{d8}$, we get that,
    \begin{gather*}
        \frac{5\pi\hbar\epsilon_o}{e^2}c\approx 90,
    \end{gather*}
    which tells us that the velocity at the ecuator is $90$ times the velocity of light, which does not make sense.

    \begin{empheq}[box=\shadowbox]{equation*}
        v\approx90c.
    \end{empheq}
\end{sol}



\section{Problem 4.26}

\begin{itemize}
    \item Check that the spin matrices 4,145 and 4,147 obey the fundamental commutation relations for angular momentum eqn 4.134
    \item Show that the Pauli spin matrices 4.148 satisfy the product rule 
        \begin{gather*}
            \sigma_j\sigma_k=\delta_{jk}+i\sum_l\epsilon_{jkl}\sigma_l,
        \end{gather*}
        where the indices stand for $x,y,z$ and $\epsilon_{jkl}$ is the Levi-Civita symbol.
\end{itemize}

\begin{sol}{Mathematical spin properties}{label-2a}
    \begin{empheq}[box=\shadowbox]{equation*}
        \text{Sol}
    \end{empheq}
\end{sol}


\section{Problem 4.27}

An electron is in the spin state,
\begin{gather*}
    \Xi=A\begin{pmatrix}3i\\4\end{pmatrix}
\end{gather*}

\begin{itemize}
    \item Determine the normalization constant $A$.
    \item Find the expectation values of $S_x, S_y$ and $S_z$.
    \item Find the ``uncertanties'' $\sigma_{S_x},\sigma_{S_y}$ and $\sigma_{S_z}$. (Note: These sigmas are standard deviations, not Pauli matrices!)
    \item Confirm that your results are consistent with all three uncertanty principles 4.100 and its cyclic permutations-only with $S$ in place of $L$, of course.
\end{itemize}

\begin{sol}{Practice spin state}{label-3a}
    \begin{empheq}[box=\shadowbox]{equation*}
        \text{Sol}
    \end{empheq}
\end{sol}


\section{Problem 4.32 a}

If you measure the component of spin angular momentum along the $x$ direction, at time $t$, what is the probability that you would get $+\hbar/2$?

\begin{sol}{Practice spin state}{label-4}
    \begin{empheq}[box=\shadowbox]{equation*}
        \text{Sol}
    \end{empheq}
\end{sol}



\end{document}
