\documentclass[../main.tex]{subfiles}

\begin{document}

\section{Problem 4.34}

\begin{itemize}
    \item Aplly $\hat{S}_{-}$ to $\ket{10}$ and confirm that ypu get $\sqrt{2}\hbar\ket{1-1}$
    \item Apply $\hat{S}_{+}$ to $\ket{00}$ and confirm that you get zero.
    \item Show that $ket{11}$ and $ket{1-1}$ are eqigenstates of $hat{S}^2$, with the appropriate eigenvalue.
\end{itemize}

\begin{align*}
    \ket{11} &= \uparrow\uparrow \\
    \ket{10} &= \frac{1}{\sqrt{2}}\qty(\uparrow\downarrow+\downarrow\uparrow),\quad s=1\mathrm{triplet} \\
    \ket{1-1} &= \downarrow\downarrow \\
\end{align*}

\begin{align*}
    \ket{00} &= \frac{1}{\sqrt{2}}\qty(\uparrow\downarrow-\downarrow\uparrow) \quad s=0\mathrm{singlet}
\end{align*}


\section{Problem 4.35}

Quarks carry spin $1/2$.
Three quarks bind together to make a make baryon (such as a proton or neutron): two quarks (or more precisely a quark and an antiquark) bind together to make a meson (such as the pion or the kaon).
Assume the quarks are in the ground (so the orbital angular momentum is zero).

\begin{itemize}
    \item What spins are possible for baryons?
    \item What spins are possible for mesons?
\end{itemize}

\section{Problem 5.4}

\begin{itemize}
    \item If $\psi_a$ are orthogonal, and both normalized, what is the constant $A$ in 5.10?
    \item If $\psi_a=\psi_b$ (and it is normalized), what is $A$? (This case, of course, occurs only for bosons.)
\end{itemize}

\section{Problem 5.5}

\begin{itemize}
    \item Write down the Hamiltonian for two noninteracting identical particles in the infinite square well.
        Verify that the fermion ground state given in Example 5.1 is an eienfunction of $H$, with the appropriate eigenvalue.
    \item Find the next two excited states (beyond the ones in Example 5.1)-wave functions and energies-for each of the three cases (distinguishable, identical bosons, identical fermions).
\end{itemize}

\end{document}
