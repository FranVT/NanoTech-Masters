\documentclass[../main.tex]{subfiles}

\begin{document}

\section{Problem 4.34}

\begin{enumerate}
    \item Aplly $\hat{S}_{-}$ to $\ket{10}$ and confirm that you get $\sqrt{2}\hbar\ket{1-1}$
    \item Apply $\hat{S}_{+}$ to $\ket{00}$ and confirm that you get zero.
    \item Show that $\ket{11}$ and $\ket{1-1}$ are eqigenstates of $\hat{S}^2$, with the appropriate eigenvalue.
\end{enumerate}

\begin{sol}{$\hat{S}_{-}\ket{10}$}{label-1a}
    Recalling that $\ket{10}=1/\sqrt{2}(\ket{\upa\doa}+\ket{\doa\upa})$ we can compute as follows,
    \begin{align*}
        \Sm^{(T)}\ket{10} &= \qty(\Sm^{(1)}\oplus\Sm^{(2)})\frac{1}{\sqrt{2}}\qty(\ket{\upa\doa}+\ket{\doa\upa}) \\
                          &= \frac{1}{\sqrt{2}}\left[
                                    \qty(\Sm^{(1)}\ket{\upa}_{(1)})\ket{\doa}_{(2)}
                                    \oplus
                                    \ket{\upa}_{(1)}\cancelto{0}{\qty(\Sm^{(2)}\ket{\doa}_{(2)})}
                                +
                                    \cancelto{0}{\qty(\Sm^{(1)}\ket{\doa}_{(1)})}\ket{\upa}_{(2)}
                                    \oplus
                                    \ket{\doa}_{(1)}\qty(\Sm^{(2)}\ket{\upa}_{(2)})
                                \right]\\
                          &= \frac{1}{\sqrt{2}}\left[\hbar\ket{\doa}_{(1)}\ket{\doa}_{(2)} + \hbar\ket{\doa}_{(1)}\ket{\doa}_{(2)}\right] \\
                          &= \frac{\hbar}{\sqrt{2}}\left(\ket{\doa\doa} + \ket{\doa\doa}\right)
    \end{align*}

    \begin{empheq}[box=\shadowbox]{equation*}
        \Sm^{(T)}\ket{10} = \sqrt{2}\hbar\ket{1-1}.
    \end{empheq}
\end{sol}

\begin{sol}{$\hat{S}_{+}\ket{00}$}{label-1b}
    Recalling that $\ket{00}=1/\sqrt{2}(\ket{\upa\doa}-\ket{\doa\upa})$ we can compute as follows,
    \begin{align*}
        \Sp^{(T)}\ket{10} &= \qty(\Sp^{(1)}\oplus\Sp^{(2)})\frac{1}{\sqrt{2}}\qty(\ket{\upa\doa}+\ket{\doa\upa}) \\
                          &= \frac{1}{\sqrt{2}}\left[
                                    \cancelto{0}{\qty(\Sp^{(1)}\ket{\upa}_{(1)})}\ket{\doa}_{(2)}
                                    \oplus
                                    \ket{\upa}_{(1)}\qty(\Sp^{(2)}\ket{\doa}_{(2)})
                                -
                                    \qty(\Sp^{(1)}\ket{\doa}_{(1)})\ket{\upa}_{(2)}
                                    \oplus
                                    \ket{\doa}_{(1)}\cancelto{0}{\qty(\Sp^{(2)}\ket{\upa}_{(2)})}
                                \right]\\
                          &= \frac{1}{\sqrt{2}}\left[\hbar\ket{\upa}_{(1)}\ket{\upa}_{(2)} - \hbar\ket{\upa}_{(1)}\ket{\upa}_{(2)}\right] \\
                          &= \frac{\hbar}{\sqrt{2}}\left(\ket{\upa\upa} - \ket{\upa\upa}\right)
    \end{align*}

    \begin{empheq}[box=\shadowbox]{equation*}
        \Sp^{(T)}\ket{00} = 0.
    \end{empheq}
\end{sol}

\begin{sol}{Eigenstates of $\hat{S}^{2}$}{label-1c}
    First we compute the expression of $\hat{S}^{2}$ for a system with two Hilbert spaces,
    \begin{align*}
        \qty(\Ss^{(T)})^2 &= \qty(\Ss^{(1)}\oplus\Ss^{(2)})\cdot\qty(\Ss^{(1)}\oplus\Ss^{(2)}) \\
                          &= \qty(\Ss^{(1)})^2 \oplus \qty(\Ss^{(2)})^2 \oplus 2 \vec{\Ss}^{(1)}\cdot\vec{\Ss}^{(2)}
    \end{align*}

    Starting with $\ket{11} = \ket{\uparrow\uparrow}$,
    \begin{align*}
        \qty(\Ss^{(T)})^2\ket{11} &= \qty(\qty(\Ss^{(1)})^2 \oplus \qty(\Ss^{(2)})^2 \oplus 2 \vec{\Ss}^{(1)}\cdot\vec{\Ss}^{(2)})\ket{\upa\upa} \\
    \end{align*}
    \begin{multline*}
        \qty(\Ss^{(T)})^2\ket{11} = \left[
                                        \qty(\Ss^{(1)})^{2}\ket{\upa}_{(1)}\ket{\upa}_{(2)}
                                     \right]
                                     \oplus
                                     \left[
                                        \ket{\upa}_{(1)}\qty(\Ss^{(2)})^{2}\ket{\upa}_{(2)}
                                     \right]
                                     \\
                                    \oplus
                                    2\left[
                                        \sx^{(1)}\ket{\upa}_{(1)}\sx^{(2)}\ket{\upa}_{(2)} 
                                        \oplus
                                        \sy^{(1)}\ket{\upa}_{(1)}\sy^{(2)}\ket{\upa}_{(2)}
                                        \oplus
                                        \sz^{(1)}\ket{\upa}_{(1)}\sz^{(2)}\ket{\upa}_{(2)}
                                    \right]
    \end{multline*}
    \begin{align*}
        \qty(\Ss^{(T)})^2\ket{11} &= \frac{3}{2}\hbar^2\ket{\upa\upa} \oplus 2\left[\frac{\hbs}{4}\ket{\doa\doa}\ominus\frac{\hbs}{4}\ket{\doa\doa}\oplus\frac{\hbs}{4}\ket{\upa\upa}\right] \\
                                  &= \frac{3}{2}\hbar^2\ket{\upa\upa} \oplus \frac{\hbs}{2}\ket{\upa\upa} \\
                                  &= 2\hbs\ket{\upa\upa}.
    \end{align*}

    Now, we apply the same procedure for $\ket{1-1}$,
\begin{align*}
        \qty(\Ss^{(T)})^2\ket{1-1} &= \qty(\qty(\Ss^{(1)})^2 \oplus \qty(\Ss^{(2)})^2 \oplus 2 \vec{\Ss}^{(1)}\cdot\vec{\Ss}^{(2)})\ket{\doa\doa} \\
    \end{align*}
    \begin{multline*}
        \qty(\Ss^{(T)})^2\ket{1-1} = \left[
                                        \qty(\Ss^{(1)})^{2}\ket{\doa}_{(1)}\ket{\doa}_{(2)}
                                     \right]
                                     \oplus
                                     \left[
                                        \ket{\doa}_{(1)}\qty(\Ss^{(2)})^{2}\ket{\doa}_{(2)}
                                     \right]
                                     \\
                                    \oplus
                                    2\left[
                                        \sx^{(1)}\ket{\doa}_{(1)}\sx^{(2)}\ket{\doa}_{(2)} 
                                        \oplus
                                        \sy^{(1)}\ket{\doa}_{(1)}\sy^{(2)}\ket{\doa}_{(2)}
                                        \oplus
                                        \sz^{(1)}\ket{\doa}_{(1)}\sz^{(2)}\ket{\doa}_{(2)}
                                    \right]
    \end{multline*}
    \begin{align*}
        \qty(\Ss^{(T)})^2\ket{1-1} &= \frac{3}{2}\hbar^2\ket{\doa\doa} \oplus 2\left[\frac{\hbs}{4}\ket{\upa\upa}\ominus\frac{\hbs}{4}\ket{\upa\upa}\oplus\frac{\hbs}{4}\ket{\doa\doa}\right] \\
                                  &= \frac{3}{2}\hbar^2\ket{\doa\doa} \oplus \frac{\hbs}{2}\ket{\doa\doa} \\
                                  &= 2\hbs\ket{\doa\doa}
    \end{align*}


    \begin{empheq}[box=\shadowbox]{equation*}
        \qty(\Ss^{(T)})^2\ket{11} = 2\hbs\ket{\upa\upa},\quad \qty(\Ss^{(T)})^2\ket{1-1} = 2\hbs\ket{\doa\doa}
    \end{empheq}
\end{sol}


\begin{comment}
\begin{align*}
    \ket{11} &= \uparrow\uparrow \\
    \ket{10} &= \frac{1}{\sqrt{2}}\qty(\uparrow\downarrow+\downarrow\uparrow),\quad s=1\mathrm{triplet} \\
    \ket{1-1} &= \downarrow\downarrow \\
\end{align*}

\begin{align*}
    \ket{00} &= \frac{1}{\sqrt{2}}\qty(\uparrow\downarrow-\downarrow\uparrow) \quad s=0\mathrm{singlet}
\end{align*}
\end{comment}

\section{Problem 4.35}

Quarks carry spin $1/2$.
Three quarks bind together to make a make baryon (such as a proton or neutron): two quarks (or more precisely a quark and an antiquark) bind together to make a meson (such as the pion or the kaon).
Assume the quarks are in the ground (so the orbital angular momentum is zero).

\begin{enumerate}
    \item What spins are possible for baryons?
    \item What spins are possible for mesons?
\end{enumerate}

\section{Problem 5.4}

\begin{enumerate}
    \item If $\psi_a$ are orthogonal, and both normalized, what is the constant $A$ in \[\Psi_{\pm}\qty(\vec{r}_1,\vec{r}_2) = A\left[\psi_a\qty(\vec{r}_1)\psi_b\qty(\vec{r}_2)\pm\psi_b\qty(\vec{r}_1)\psi_a\qty(\vec{r}_2)\right]?\]
    \item If $\psi_a=\psi_b$ (and it is normalized), what is $A$? (This case, of course, occurs only for bosons.)
\end{enumerate}

\begin{sol}{Constant of normalization $A$}{label-3a}
    First we can re-write the $\Psi$ as follows, $\Psi_{\pm} = A\left(\ket{\psi_a\psi_b}\pm\ket{\psi_b\psi_a}\left)$.
    Now we can compute the inner product of the state and apply the ortho-normal properties of $\ket{\psi_a}$ and $\ket{\psi_b}$,
    \begin{align*}
        \braket{\Psi} &= \abs{A}^2\left(\bra{\psi_a\psi_b}\pm\bra{\psi_b\psi_a}\left)\left(\ket{\psi_a\psi_b}\pm\ket{\psi_b\psi_a}\left) \\
                      &= \abs{A}^2\left(
                            \bra{\psi_a\psi_b}\ket{\psi_a\psi_b}
                            \pm
                            \bra{\psi_a\psi_b}\ket{\psi_b\psi_a}
                            \pm
                            \bra{\psi_b\psi_a}\ket{\psi_a\psi_b}
                            +
                            \bra{\psi_b\psi_a}\ket{\psi_b\psi_a}
                            \right) \\
                      &= 2\abs{A}^2,
    \end{align*}
    since, $\braket{\Psi}$ should be $1$, $A=1/\sqrt{2}$. 

    \begin{empheq}[box=\shadowbox]{equation*}
        A = \frac{1}{\sqrt{2}}.
    \end{empheq}
\end{sol}

\begin{sol}{Constant of normalization $A$ with $\psi_a=\psi_b$}{label-3b}
    From the previous procedure, we see that the second and third terms are equal to $1$, leading to hte following results,
    \begin{align*}
        \braket{\Psi} &= 4\abs{A}^2,
    \end{align*}
    since, $\braket{\Psi}$ should be $1$, $A=1/2$. 

    \begin{empheq}[box=\shadowbox]{equation*}
        A = \frac{1}{2}.
    \end{empheq}
\end{sol}


\section{Problem 5.5}

\begin{enumerate}
    \item Write down the Hamiltonian for two noninteracting identical particles in the infinite square well.
        Verify that the fermion ground state given in Example 5.1 is an eienfunction of $H$, with the appropriate eigenvalue.
    \item Find the next two excited states (beyond the ones in Example 5.1)-wave functions and energies-for each of the three cases (distinguishable, identical bosons, identical fermions).
\end{enumerate}

\end{document}
