\documentclass[../main.tex]{subfiles}

\begin{document}

\section{Problem 4.16}
A hydrogenic atom consists of a single electron orbiting a nucleus with $Z$ protons ($Z=1$ would be itself, $Z=2$ is ionized helium, $Z=3$ is doubly ionized lithium, and so on).
Determine
\begin{enumerate}
    \item Bohr energies $E_n(Z)$
    \item Binding energy $E_1(z)$
    \item Bohr radius $a(Z)$
    \item Rydberg constant $R(Z)$
\end{enumerate}
for a hydrogenic atom.
(Express your answers as appropriate multiples of the hydrogen values.)
Where in the electromagnetic spectrum would the Lyman series fall, for $Z=2$ and $Z=3$?
Hint: There's nothing much to calculate here-in the potential \[V(r)=-\frac{e^2}{4\pi\epsilon_0}\frac{1}{r}\]
$e^2\to Ze^2$, so all you have to do is make the same substitution in all the final results.

\begin{sol}{Bohr energies}{label-1a}
    Recalling that the energy of the Hydrogen atom is given by \[E_n=-\frac{m_e e^4}{2(4\pi\epsilon_o)^2\hbar^2}\frac{1}{n^2},\]
    and tacking into account the hint of introducing the following change of variable $e^2\to Ze^2$ we get the following expression,
    \begin{align*}
        E_n(Z) &= -\frac{m_e (Ze^2)^2}{2(4\pi\epsilon_o)^2\hbar^2}\frac{1}{n^2} \\
            &= -\frac{m_e e^4}{2(4\pi\epsilon_o)^2\hbar^2}\frac{1}{n^2}Z^2 \\
            &= E_n Z^2.
    \end{align*}

    Therefore, the Boher energies are
    \begin{empheq}[box=\shadowbox]{equation*}
        E_n (Z) = E_n Z^2
    \end{empheq}
\end{sol}

\begin{sol}{Binding energy $E_1(Z)$}{label-2a}
    From the previous result we can easily compute $E_1(Z)$ as follows,

    \begin{empheq}[box=\shadowbox]{equation*}
        E_1 (Z) = E_1 Z^2
    \end{empheq}
\end{sol}

\begin{sol}{Bohr radius $a(Z)$}{label-3a}
    With the same methodology of the first question, we start by recalling the Bohr's radius expression and introducing the suggested change of variable,
    \begin{align*}
        a(Z) &= \frac{4\pi\epsilon_o\hbar^2}{e^2 m_e} \\
          &= \frac{4\pi\epsilon_o\hbar^2}{Ze^2 m_e} \\
          &= \frac{4\pi\epsilon_o\hbar^2}{e^2 m_e}\frac{1}{Z} \\
          &= \frac{a}{Z}
    \end{align*}

    \begin{empheq}[box=\shadowbox]{equation*}
        a(Z) = \frac{a}{Z}
    \end{empheq}
\end{sol}

\begin{sol}{Rydberg constant $R(Z)$}{label-4a}
    Now, we apply the same procedure as before to compute the Rydberg constant,
    \begin{align*}
        R(Z) &= \frac{m_e e^4}{8\epsilon_o\hbar^3 c} \\
          &= \frac{m_e \qty(Ze^2)^2}{8\epsilon_o\hbar^3 c} \\
          &= \frac{m_e e^4}{8\epsilon_o\hbar^3 c}Z^2 \\
          &= R Z^2
    \end{align*}

    \begin{empheq}[box=\shadowbox]{equation*}
        R(Z) = R Z^2
    \end{empheq}
\end{sol}


\begin{sol}{Electromagnetic spectrum}{label-5a}
    To compute the Lyman lines we need to recall the following relation, \[\frac{1}{\lambda_2}=R\qty(1-\frac{1}{4})\implies\lambda_2=\frac{4}{3R}\] and \[\frac{1}{\lambda_1}=R\qty(1-\frac{1}{\infty})\implies\lambda_1=\frac{1}{R}.\]
    
    Now that we known the Ryberg constat in terms of $Z$ we compute the lines for $Z=2$ and $3$.
    For $Z=2$ we get that $\lambda_1=1/R 2^2$ and $\lambda_2=4/(3R 2^2)$.
    For $Z=3$ we get that $\lambda_1=1/R 3^2$ and $\lambda_2=4/(3R 3^2)$

    \begin{empheq}[box=\shadowbox]{equation*}
        Z=2 \rightarrow \lambda_{1}\SI{2.28d-8}{\meter},~\lambda_{2}\SI{3.04d-8}{\meter} \\
        Z=3 \rightarrow \lambda_{1}\SI{1.01d-8}{\meter},~\lambda_{2}\SI{1.35d-8}{\meter} \\
    \end{empheq}

\end{sol}


\section{Problem 5.6}

Imagine two noninteracting particles, each of mass $m$, in the infinite square well.
If one is in the state \[\psi_n(x)=\sqrt{\frac{2}{a}}\sin\qty(\frac{n\pi}{a}x).\], and the other in state $\psi_l$ ($l\neq n$), calculate $\expval{(x_1-x_2)^2}$, assuming
\begin{enumerate}
    \item they are distinguishable particles
    \item they are identical bosons
    \item they are identical fermions
\end{enumerate}

\begin{sol}{Distinguishable particles}{label-3a}
    From previous results in the chapter, the expectation value for distinguishable particles  we get that,
    \begin{gather*}
        \expval{(x_1-x_2)^2}_d = \expval{x^2}_a + \expval{x^2}_b - 2\expval{x}_a\expval{x}_b.
    \end{gather*}
    Also, from the state we get that,
    \begin{align*}
        \expval{x}_n = \frac{a}{2},\quad 
        \expval{x^2}_n = a^2\qty(\frac{1}{3}-\frac{1}{2(n\pi)^2}). 
    \end{align*}

    Subsistuing the values we get the following result,
    \begin{align*}
        \expval{(x_1-x_2)^2}_d &= a^2\qty(\frac{1}{3}-\frac{1}{2(n\pi)^2}) + a^2\qty(\frac{1}{3}-\frac{1}{2(n\pi)^2}) - 2\frac{a}{2}\frac{a}{2} \\
                               &= a^2\left[\frac{1}{6}-\frac{1}{2\pi^2}\qty(\frac{1}{n^2}+\frac{1}{m^2})\right].
    \end{align*}

    \begin{empheq}[box=\shadowbox]{equation*}
        \expval{(x_1-x_2)^2}_d =a^2\left[\frac{1}{6}-\frac{1}{2\pi^2}\qty(\frac{1}{n^2}+\frac{1}{m^2})\right]
    \end{empheq}
\end{sol}

\begin{sol}{Identical bosons}{label-3b}
    Now that we are going to analyze bosons, the state is represented by \[\Psi_+(x_1,x_2)=\frac{1}{\sqrt{2}}\left[\psi_a(x_1)\psi_b(x_2)+\psi_b(x_1)\psi_a(x_2)\right]\]

    \begin{empheq}[box=\shadowbox]{equation*}
        A = \frac{1}{2}.
    \end{empheq}
\end{sol}

\begin{sol}{Identical fermions}{label-3b}
    \begin{empheq}[box=\shadowbox]{equation*}
        A = \frac{1}{2}.
    \end{empheq}
\end{sol}


\section{Problem 5.9}

\begin{enumerate}
    \item Suppose you put both electrons in a helium atom into the $n=2$ state; what would the energy of the emitted electron be?
    \item Describe (quantitatively) the spectrum of the helium ion, $\mathrm{He}^+$.
\end{enumerate}

\begin{sol}{Hamiltonian of non-interacting identical particles}{label-4a}
    \begin{empheq}[box=\shadowbox]{equation*}
        \hat{H}\psi\qty(x_1,x_2)=5K\psi\qty(x_1,x_2),\quad K=\frac{\pi^2\hbs}{2a^2m}.
    \end{empheq}

\end{sol}

\begin{sol}{Energies and states of the next two excited states}{label-4b}

    \begin{empheq}[box=\shadowbox]{align*}
        \psi_{1,3} &= \frac{\sqrt{2}}{a}\left[\sin\qty(\pi\frac{x_1}{a})\sin\qty(3\pi\frac{x_2}{a}) - \sin\qty(3\pi\frac{x_2}{a})\sin\qty(\pi\frac{x_1}{a})\right],\quad E=10K \\
        \psi_{2,3} &= \frac{\sqrt{2}}{a}\left[\sin\qty(2\pi\frac{x_1}{a})\sin\qty(3\pi\frac{x_2}{a}) - \sin\qty(3\pi\frac{x_2}{a})\sin\qty(2\pi\frac{x_1}{a})\right],\quad E=13K 
    \end{empheq}



\end{sol}

\section{Problem 5.10}

Discuss (qualitatively) the energy level scheme for thelium if
\begin{enumerate}
    \item electrons were identical bosons
    \item if electrons were distinguishable particles (but with the same mass and charge).
        Pretend these "electrons" still have spin $1/2$, so the spin configuration are the singlet and the triplet.
\end{enumerate}

\section{Problem 5.12}

\begin{enumerate}
    \item Figure out the electron configurations (in the notation of eqn 5.33) for the first two rows of the periodic table (up to neon), and check your results against table 5.1
    \item Figure out the corresponding total angular momenta, in the notation of eqn 5.34, for the first four elements.
        List all possibilities for boron, carbon and nitrogen.
\end{enumerate}

\section{Problem 5.14}

The ground state of dysprosium (element 66, in the 6th row of the Periodic Table) is listed as $^5I_8$.
What are the total spin, total arbital and grand total angular momentum quantum numbers?
Suggest a likely electron configuration for dysprosium.

\end{document}
