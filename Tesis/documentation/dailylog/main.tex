% This a LaTeX template for a research journal, aimed at 
% being 
% 1. easy to use, so one can simply type the daily entries;
% 2. elegant.
% It was written by Níckolas Alves (alves-nickolas.github.io)

% THIS WAS ORIGINALLY COMPILED WITH LUALATEX, so I suggest going to the Menu (top-left of your screen, if you're on Overleaf) and selecting LuaLaTeX on Settings -> Compiler. I didn't test it on XeLaTeX, but I think it should work fine as well. While the document will still compile on pdfLaTeX, for example, it will not have access to the FiraMath font, and hence math text will look weird (it will be typeset in LaTeX's standard Computer Modern Math). If you don't plan on using mathematics at all, then there is a reasonable chance pdfLaTeX will do just fine

\documentclass[a4paper, 11pt, oneside]{researchjournal} % I wrote the design using a4paper, 11pt, oneside, but feel free to change

%\logo{} can be used to add a small decoration to the top of the cover page. My original idea was to put an \insergraphics command in it and load, e.g., the university logo or something

\author{Fco. Vázquez}

% colors are customizable using xcolor's (https://ctan.org/pkg/xcolor) \definecolor
\definecolor{ChapterBackground}{HTML}{2e3440} %colors to use on chapters
\definecolor{ChapterForeground}{HTML}{eceff4} %colors to use on chapters
\definecolor{DayColor}{HTML}{d08770} %colors to use on newdays and daybibs
\definecolor{CoverBackground}{HTML}{3b4252} %cover background
\definecolor{CoverForeground}{HTML}{e5e9f0} %cover letters
\definecolor{LinkColor}{HTML}{a3be8c} %color for links

\begin{document} % this will automatically generate a simple cover
\newday{2022-12-25} \verb|\newday| is the main command provided by the \verb|researchjournal| class. It receives a single argument: a date in the \verb|yyyy-mm-dd| format. Issuing it will automatically generate the necessary chapter and sections for the year and month, respectively. This happens in addition to issuing a subsection for the specific day. The command creates the new chapter and section by comparing the date it received to the previous date issued by the user. It also checks for chronological order and for repeated days, but it can't do much about it. At most, you will get a class warning to let you know your entries are out of order. The chapter and section creation facilities do assume you are typing your entries in chronological order, so you might not get a new chapter if you type a previous year. % newday is the main command provided by the class. It inserts a daily entry and automatically checks whether it is necessary to create a new chapter (due to a new year) or a new section (due to a new month). Its sole argument is a date in the format yyyy-mm-dd. Issue the command, then write as you will in front of it.

\newday{2023-01-27} Notice how writing a new day on a different year automatically creates a new chapter. Below this entry, I'm showing an example of the \verb|daybib| command. It simply prints ``References:'' in a cute manner, without any other fancy functionalities. I use it to list some references that I used throughout the day, but didn't want to list in the main paragraph (perhaps because I barely checked them, or just didn't have much to comment about them). I like to keep a comprehensive list of references so I can later check on them if I ever get a feeling like ``I once read a paper that discussed this, but what was it called again?'' % notice how a new year automatically creates a new chapter
\daybib\cite{weinberg1995Foundations,weinberg1996ModernApplications}. %daybib adds the text "References: " underneath the entry. It just prints text without doing anything fancy. I use it to list references that I used on some given day, but didn't make it to the main paragraph. Notice I manually added a period at the end of the line.

\newday{2023-01-28} Let us add a bit more text in here just to have a larger paragraph to work as an example. Most of my personal entries typically describe what I did in that particular day in general terms, such as ``I finished writing Section 2.3 of my thesis'', or ``Read the paper by \textcite{hawking1975ParticleCreationBlack} and really enjoyed it''. The idea is mostly to have a short and general description of how the day went so that I can recall it later when, for example, writing a report.

\newday{2023-02-12} New day, new text. Notice how the new month automatically generates a new section. 
\daybib\cite{wald1984GeneralRelativity}.

\newday*{2023-02-13} If you type \verb|\newday*{yyyy-mm-dd}| (with an asterisk), the output will also include a star. I typically use this feature to tag the days in which I had meetings with my advisor, but at the end of the day it is completely up to you. Maybe you use to tag happy days or something. 

\newday{2023-03-02} Notice how the headers work as a dictionary guide. The left header indicates the first entry on the page, while the right header indicates the last. 

\newday{2023-03-03} References are dealt with using \verb|biblatex|. You can add your own my modifying the file \verb|bib.bib|.

\newday{2024-07-16} About Voronoi tessellation and characterization techniques in softmatter.
In the article \cite{yangVoronoiTessellationPacking2002} 

They analyse a system with 5000 particles and vary the radius from $100$ to $0.1\mathrm{\mu m}$ and analyse the number of edges for each polyhedron face, number of faces for each polyhedron face, perimeter, area of a polyhedron face and perimeter, area and volume of a polyhedron.
In a nutshell, they said that the topological and metric properties of Voronoi polyhedra are quantified as a function of particles size and packing density.
Edges, vertices, volume, area in terms of the particle size and packing fraction.
According to the results of the article, the average face number decreases when the particle size and packing density decreases.
The distributions of face number and edges become broader and more asymmetric.
On the other hand, the average perimeter and area of polyhedra increses and the distributions of polyhedron surface area and volume become more flat and can be described by the log-normal distribution.
Finally, the average sphericity coefficient of Voronoi Polyhedra varies with packing density.
[The sphericity, $ \Psi $, of an object is the ratio of the surface area of a sphere with the same volume to the object's surface area $\Psi=\pi^{1/3}\left(6V_{p}\right)^{2/3}/A_{p}$].

They mention the following laws: Aboav-Weaire's law and Lewis's law.

% https://link.springer.com/content/pdf/10.1007/3-540-47789-6_10.pdf

\rule{\textwidth}{0.4pt}

In the article \cite{lazarVoronoiCellAnalysis2022} in the section of Stability they state that:
When geometric quantities are used for classification purposes, thresholds for the range of possible values must be chosen, on the other hand, topological ones can change abruptly under small perturbations of particle coordinates.



\rule{\textwidth}{0.4pt}


How we can relate the topological and metric properties with rehological properties?

Relation between the interatomic interaction with the geometric properties.
I thinkg that there is a relation between the superficial area of the voronoi faces with pressure between two particles.

Thinking about the Lennard-Jones potential in three dimensions, it creates an sphere of potential.
It can be computed the pressure in a plane that intersect that sphere.

I think that, that relation, between the potential and the intersecte plane, will help to create a conecction between rehological properties and structural stuff.

\begin{align*}
	\vec{P}_{VC} = -\dv{r}\mathrm{LJ}\qty(r)|_{r\in R^{2}}\frac{1}{A_{VC}}\hat{e}_{r},
\end{align*}
where $A_{VC}$ is the area of the shared plane between two voronoi cells and the force is evaluated at that plane.
The $VC$ indicates, "Voronoi Cell".

Now, this has the same units of the stress tensor.

To introduce the direction of the plane into the pressure, we take dot product with the spherical coodrinate system, an then multiply be the normal vector of the plane,
\begin{align*}
	\vec{P}_{plane} = \left[\vec{P}_{VC}\cdot\left(\hat{e}_r,\hat{e}_\phi,\hat{e}_\theta\right)\right]\hat{n}_{plane}
\end{align*}


Create an histogram of the normal vector.
Histogram of the dot product between the normal vector with the cartessian reference.

The better way to analyse the macro caracteristics is an histogram.


About packing density: Using the volumen of voronoi polyhedra.
It can be said that a system is jammed, when the mean of the volumens of the voronoi polyhedra tends to a minimum.

% https://www.nature.com/articles/nmat4674

\printbibliography

\end{document} % this will automatically generate a references chapter if you cited any references throughout the text
