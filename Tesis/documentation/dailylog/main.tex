% This a LaTeX template for a research journal, aimed at 
% being 
% 1. easy to use, so one can simply type the daily entries;
% 2. elegant.
% It was written by Níckolas Alves (alves-nickolas.github.io)

% THIS WAS ORIGINALLY COMPILED WITH LUALATEX, so I suggest going to the Menu (top-left of your screen, if you're on Overleaf) and selecting LuaLaTeX on Settings -> Compiler. I didn't test it on XeLaTeX, but I think it should work fine as well. While the document will still compile on pdfLaTeX, for example, it will not have access to the FiraMath font, and hence math text will look weird (it will be typeset in LaTeX's standard Computer Modern Math). If you don't plan on using mathematics at all, then there is a reasonable chance pdfLaTeX will do just fine

\documentclass[a4paper, 11pt, oneside]{researchjournal} % I wrote the design using a4paper, 11pt, oneside, but feel free to change

%\logo{} can be used to add a small decoration to the top of the cover page. My original idea was to put an \insergraphics command in it and load, e.g., the university logo or something

\author{Fco. Vázquez}

% colors are customizable using xcolor's (https://ctan.org/pkg/xcolor) \definecolor
\definecolor{ChapterBackground}{HTML}{2e3440} %colors to use on chapters
\definecolor{ChapterForeground}{HTML}{eceff4} %colors to use on chapters
\definecolor{DayColor}{HTML}{d08770} %colors to use on newdays and daybibs
\definecolor{CoverBackground}{HTML}{3b4252} %cover background
\definecolor{CoverForeground}{HTML}{e5e9f0} %cover letters
\definecolor{LinkColor}{HTML}{a3be8c} %color for links

\begin{document} % this will automatically generate a simple cover


\newday{2024-07-16} About Voronoi tessellation and characterization techniques in softmatter.
In the article \textcite{yangVoronoiTessellationPacking2002} 

They analyse a system with 5000 particles and vary the radius from $100$ to $0.1\mathrm{\mu m}$ and analyse the number of edges for each polyhedron face, number of faces for each polyhedron face, perimeter, area of a polyhedron face and perimeter, area and volume of a polyhedron.
In a nutshell, they said that the topological and metric properties of Voronoi polyhedra are quantified as a function of particles size and packing density.
Edges, vertices, volume, area in terms of the particle size and packing fraction.
According to the results of the article, the average face number decreases when the particle size and packing density decreases.
The distributions of face number and edges become broader and more asymmetric.
On the other hand, the average perimeter and area of polyhedra increses and the distributions of polyhedron surface area and volume become more flat and can be described by the log-normal distribution.
Finally, the average sphericity coefficient of Voronoi Polyhedra varies with packing density.
[The sphericity, $ \Psi $, of an object is the ratio of the surface area of a sphere with the same volume to the object's surface area $\Psi=\pi^{1/3}\left(6V_{p}\right)^{2/3}/A_{p}$].

They mention the following laws: Aboav-Weaire's law and Lewis's law.

% https://link.springer.com/content/pdf/10.1007/3-540-47789-6_10.pdf

\rule{\textwidth}{0.4pt}

In the article \cite{lazarVoronoiCellAnalysis2022} 
in the section of Stability they state that:
When geometric quantities are used for classification purposes, thresholds for the range of possible values must be chosen, on the other hand, topological ones can change abruptly under small perturbations of particle coordinates.


\rule{\textwidth}{0.4pt}

How we can relate the topological and metric properties with rehological properties?

Relation between the interatomic interaction with the geometric properties.
I thinkg that there is a relation between the superficial area of the voronoi faces with pressure between two particles.

Thinking about the Lennard-Jones potential in three dimensions, it creates an sphere of potential.
It can be computed the pressure in a plane that intersect that sphere.

I think that, that relation, between the potential and the intersecte plane, will help to create a conecction between rehological properties and structural stuff.

\begin{align*}
	\vec{P}_{VC} = -\dv{r}\mathrm{LJ}(r)|_{r\in R^{2}}\frac{1}{A_{VC}}\hat{e}_{r},
\end{align*}
where $A_{VC}$ is the area of the shared plane between two voronoi cells and the force is evaluated at that plane.
The $VC$ indicates, "Voronoi Cell".

Now, this has the same units of the stress tensor.

To introduce the direction of the plane into the pressure, we take dot product with the spherical coodrinate system, an then multiply be the normal vector of the plane,
\begin{align*}
	\vec{P}_{plane} = \left[\vec{P}_{VC}\cdot\left(\hat{e}_r,\hat{e}_\phi,\hat{e}_\theta\right)\right]\hat{n}_{plane}
\end{align*}


Create an histogram of the normal vector.
Histogram of the dot product between the normal vector with the cartessian reference.

The better way to analyse the macro caracteristics is an histogram.


About packing density: Using the volumen of voronoi polyhedra.
It can be said that a system is jammed, when the mean of the volumens of the voronoi polyhedra tends to a minimum.

% https://www.nature.com/articles/nmat4674
\rule{\textwidth}{0.4pt}

About the simulations.
50 simulations with 500 patchy particles took 24 hrs.
This is 500 Monomers and Crosslinker. 
All of them has 50 CL and 450 MO, hence, 200 patches for the 50 CL and 900 patches for MO, hence, 1600 particles in total.

I have no idea how to improve the time of the simulation.
Each simulation takes between 27-28 mins. 

\newday*{2024-07-17}
Key point of the reunion:
\begin{itemize}
	\item Start the script for shear deformation
	\item Keep in mind to vary the following parameters:
	\begin{enumerate}
		\item Cross-Linker concentration
		\item Size of the box (Packing density)
		\item The energy of the Patch-Patch interaction.
	\end{enumerate}
	\item Analyze the density distibution of the voronoi parameters
	\item Make and animation, where the color of the particle represent the volume of the voronoi cell and share the script.
	\item Hint: Directores
\end{itemize}

About the idea of the normla vector: They do not sees a great future.

\rule{\textwidth}{0.4pt}

The deformation script is done, basically I copy the code from Felipe.

\newday{2024-07-19}
Yesterday I finich the scripts for the shear deformation and investigate some papers that help me to understad what I'm trying to do for the thesis project.

I found the artilce by \cite{sheikoArchitecturalCodeRubber2019} and here are some ``main'' points:
\begin{itemize}
	\item Materials response to deformation is expressed by a stress-strain curve measured art controlled strain rate and temperature.
	\item A set of stres-strain curves constitue the ``mechanical phenotype''.
	\item The dependcy of those curves help us to understand the molecular relacation processes, such as conformational tranformation, bond scission and molecule displacement.
	\item In the article, they classify a material, based on their stress-strain curves, in three categories: Thermoplastics, Biological networks and  Synthetic network.
	\item Feature of interset for the article: Firmness is the stiffeness enhancement in response to deformationa and is one of nature's defense mechanisms preventig accidental organ rupture.
	\item Solif thermoplastics, the mechanical properties are predominantly controlled by molecular interactions.
	\item Elastomers, the mechanical properties are controlled by the changes in the system configuration,by unravelling network strands at nearly constant mass density.
	\item A network architecture is defined by: Connectivity, dimensions and flexibility of network strands.
	\item The weak dependence on chemical compositions helps to manipulate the mechanical response.
	\item In the article they focues in the Shear modulus.
	\item Topological defects: Chain entranglements, dandling chain ends, loops, multiple strands and side chains.
	\item Search of definitions:
		\begin{enumerate}
			\item Shear modulus:
			\item Cross-link functionality:
			\item Degree of polymerization: Number of monomeric units (MO and CL)
			\item Kuhn length:  A theoretical treatment, developed by Werner Kuhn, in which a real polymer chain is considered as a collection of N Kuhn segments each with a Kuhn length b. Each Kuhn segment can be thought of as if they are freely jointed with each other.
		\end{enumerate}
	\item The softness at small deformations is manipulated by changing the cross-link density.
	\item Linear chain Networks
	\begin{itemize}
		\item The cross-linnk density is THE control parameter
		\item The Young's modulus can be tune by introducing loops and dangles.
		\item Modify the junctions can controll the network topology
		\item The cross-link functionality can be increased by adding dendrimes, colloidal particles and microphase domains.
		\item Their equilibrium characteristics are mostly determined by the degree of polymerization of the chemical network strand.
		\item Strand entanglements can create a loweer limit for the shear modulus and upper limit for the elongation-at-break.
		\item Softness can be achieved by solvent swelling.
		\item Swelling leads to strand extensions and therefore firmness enhacement.
	\end{itemize}
	\item Dual Networks
	\begin{itemize}
		\item Networks with permanent cross-links  and dynamic cross-links.
		\item Permamnent cross-link are responsible for elastic properties and sample shape
		\item Dynamic cross-link regulate energy dissipation and strain-rate response.
		\item Dynamic networks depen on strain rate, permanent network does not depend on strain rate.
			\item At higher rates, physical cross-links do not have enough time to break and behave as permanent cross-links, resulting in high modulus and strength.
		\item \textbf{To characterize the dynamic mechanical properties of the dual networks at small deformations, evolution of the stress relaxation Young’s modulus E(t) during uniaxial extension at a constant temperature is calculated by using a linear viscoelasticity approximation.}
	\end{itemize}
	\item Network with Brush-Like strands: I didn't read it with detail, beacuse it is out of my scope/comprehension/interest at the moment. It will be great to read it with more knowledge in the area.
\end{itemize}

Need to check \cite{dealyStructureRheologyMolten2006} for more information about polymer structures and their characterization.

Now, in the sense, that the curve of strain-stress is essential to characterize the material, I search in lammps how to compute the stres and strin in the simulation.
What I found is an page where there explain how to \href{https://docs.lammps.org/Howto_elastic.html}{Calculate elastic constants} and according to that page, the elastic constants (strain and stress) are related as follows:
\begin{gather*}
	s_{ij} = C_{ijkl} e_{kl},
\end{gather*}
where the repeates indices impley summation.$s_{ij}$ are the elemnts of the symmetric stress tensor, $e_{kl}$ are the elements of the symmetric strain tensor and $C_{ijkl}$ are the elemnts of the foruth rank tensor of elastic constants.

Then, they state that the elastic constant can be computed at \textit{zero temperature} or \textit{finite temperature} and metiond the following methods for doing it in LAMMPS,
\begin{itemize}
	\item At zero temperature
		\begin{enumerate}
			\item The $C_{ijkl}$ are estimated by deforming the simulaiton box in one of the six directions using \verb|change_box| command. Then compute the chanfes in the stress tensor.
		\end{enumerate}
	\item At finite temperature
		\begin{enumerate}
				\item Exploit the relation between elastic constants, stress fluctuations, and the Born matrix. \cite{rayStatisticalEnsemblesMolecular1984}
				\item Measure the change in average stress tensor in an NVT simulations when the cell volume undergoes a finite deformation.
				\item Sample the triclinic cell fluctuations that occur in an NPT simulation. 
			\end{enumerate}
\end{itemize}

Finally, in the article \cite{clavierComputationElasticConstants2017} a nice review of the advantages and disadvantages of all of these methods is provided.

So the next week I will check those articles and the examples cripts in lammps.

Possibly, during the weekend's I transfer old entries of multiple daily log in Obsdian, notesbooks, Git and Discord.

%Dealy, J. M.; Larson, R. G. Structure and Rheology of Molten Polymers: From Structure to Flow Behavior and Back Again; Hanser Gardner: Cincinnati, OH, 2006. [It's a book]

% Ferry, J. D. Viscoelastic Properties of Polymers, 3rd ed.; John Wiley: NewYork, 1980.



%\printbibliography

\end{document} % this will automatically generate a references chapter if you cited any references throughout the text
