% This a LaTeX template for a research journal, aimed at 
% being 
% 1. easy to use, so one can simply type the daily entries;
% 2. elegant.
% It was written by Níckolas Alves (alves-nickolas.github.io)

% THIS WAS ORIGINALLY COMPILED WITH LUALATEX, so I suggest going to the Menu (top-left of your screen, if you're on Overleaf) and selecting LuaLaTeX on Settings -> Compiler. I didn't test it on XeLaTeX, but I think it should work fine as well. While the document will still compile on pdfLaTeX, for example, it will not have access to the FiraMath font, and hence math text will look weird (it will be typeset in LaTeX's standard Computer Modern Math). If you don't plan on using mathematics at all, then there is a reasonable chance pdfLaTeX will do just fine

\documentclass[a4paper, 11pt, oneside]{researchjournal} % I wrote the design using a4paper, 11pt, oneside, but feel free to change

%\logo{} can be used to add a small decoration to the top of the cover page. My original idea was to put an \insergraphics command in it and load, e.g., the university logo or something

\author{Fco. Vázquez}

% colors are customizable using xcolor's (https://ctan.org/pkg/xcolor) \definecolor
\definecolor{ChapterBackground}{HTML}{2e3440} %colors to use on chapters
\definecolor{ChapterForeground}{HTML}{eceff4} %colors to use on chapters
\definecolor{DayColor}{HTML}{d08770} %colors to use on newdays and daybibs
\definecolor{CoverBackground}{HTML}{3b4252} %cover background
\definecolor{CoverForeground}{HTML}{e5e9f0} %cover letters
\definecolor{LinkColor}{HTML}{a3be8c} %color for links

\begin{document} % this will automatically generate a simple cover


\newday{2024-07-16} About Voronoi tessellation and characterization techniques in softmatter.
In the article \textcite{yangVoronoiTessellationPacking2002} 

They analyse a system with 5000 particles and vary the radius from $100$ to $0.1\mathrm{\mu m}$ and analyse the number of edges for each polyhedron face, number of faces for each polyhedron face, perimeter, area of a polyhedron face and perimeter, area and volume of a polyhedron.
In a nutshell, they said that the topological and metric properties of Voronoi polyhedra are quantified as a function of particles size and packing density.
Edges, vertices, volume, area in terms of the particle size and packing fraction.
According to the results of the article, the average face number decreases when the particle size and packing density decreases.
The distributions of face number and edges become broader and more asymmetric.
On the other hand, the average perimeter and area of polyhedra increses and the distributions of polyhedron surface area and volume become more flat and can be described by the log-normal distribution.
Finally, the average sphericity coefficient of Voronoi Polyhedra varies with packing density.
[The sphericity, $ \Psi $, of an object is the ratio of the surface area of a sphere with the same volume to the object's surface area $\Psi=\pi^{1/3}\left(6V_{p}\right)^{2/3}/A_{p}$].

They mention the following laws: Aboav-Weaire's law and Lewis's law.

% https://link.springer.com/content/pdf/10.1007/3-540-47789-6_10.pdf

\rule{\textwidth}{0.4pt}

In the article \cite{lazarVoronoiCellAnalysis2022} 
in the section of Stability they state that:
When geometric quantities are used for classification purposes, thresholds for the range of possible values must be chosen, on the other hand, topological ones can change abruptly under small perturbations of particle coordinates.


\rule{\textwidth}{0.4pt}

How we can relate the topological and metric properties with rehological properties?

Relation between the interatomic interaction with the geometric properties.
I thinkg that there is a relation between the superficial area of the voronoi faces with pressure between two particles.

Thinking about the Lennard-Jones potential in three dimensions, it creates an sphere of potential.
It can be computed the pressure in a plane that intersect that sphere.

I think that, that relation, between the potential and the intersecte plane, will help to create a conecction between rehological properties and structural stuff.

\begin{align*}
	\vec{P}_{VC} = -\dv{r}\mathrm{LJ}(r)|_{r\in R^{2}}\frac{1}{A_{VC}}\hat{e}_{r},
\end{align*}
where $A_{VC}$ is the area of the shared plane between two voronoi cells and the force is evaluated at that plane.
The $VC$ indicates, "Voronoi Cell".

Now, this has the same units of the stress tensor.

To introduce the direction of the plane into the pressure, we take dot product with the spherical coodrinate system, an then multiply be the normal vector of the plane,
\begin{align*}
	\vec{P}_{plane} = \left[\vec{P}_{VC}\cdot\left(\hat{e}_r,\hat{e}_\phi,\hat{e}_\theta\right)\right]\hat{n}_{plane}
\end{align*}


Create an histogram of the normal vector.
Histogram of the dot product between the normal vector with the cartessian reference.

The better way to analyse the macro caracteristics is an histogram.


About packing density: Using the volumen of voronoi polyhedra.
It can be said that a system is jammed, when the mean of the volumens of the voronoi polyhedra tends to a minimum.

% https://www.nature.com/articles/nmat4674
\rule{\textwidth}{0.4pt}

About the simulations.
50 simulations with 500 patchy particles took 24 hrs.
This is 500 Monomers and Crosslinker. 
All of them has 50 CL and 450 MO, hence, 200 patches for the 50 CL and 900 patches for MO, hence, 1600 particles in total.

I have no idea how to improve the time of the simulation.
Each simulation takes between 27-28 mins. 

\newday*{2024-07-17}
Key point of the reunion:
\begin{itemize}
	\item Start the script for shear deformation
	\item Keep in mind to vary the following parameters:
	\begin{enumerate}
		\item Cross-Linker concentration
		\item Size of the box (Packing density)
		\item The energy of the Patch-Patch interaction.
	\end{enumerate}
	\item Analyze the density distibution of the voronoi parameters
	\item Make and animation, where the color of the particle represent the volume of the voronoi cell and share the script.
	\item Hint: Directores
\end{itemize}

About the idea of the normla vector: They do not sees a great future.

\rule{\textwidth}{0.4pt}

The deformation script is done, basically I copy the code from Felipe.

\newday{2024-07-19}
Yesterday I finich the scripts for the shear deformation and investigate some papers that help me to understad what I'm trying to do for the thesis project.

I found the artilce by \cite{sheikoArchitecturalCodeRubber2019} and here are some ``main'' points:
\begin{itemize}
	\item Materials response to deformation is expressed by a stress-strain curve measured art controlled strain rate and temperature.
	\item A set of stres-strain curves constitue the ``mechanical phenotype''.
	\item The dependcy of those curves help us to understand the molecular relacation processes, such as conformational tranformation, bond scission and molecule displacement.
	\item In the article, they classify a material, based on their stress-strain curves, in three categories: Thermoplastics, Biological networks and  Synthetic network.
	\item Feature of interset for the article: Firmness is the stiffeness enhancement in response to deformationa and is one of nature's defense mechanisms preventig accidental organ rupture.
	\item Solif thermoplastics, the mechanical properties are predominantly controlled by molecular interactions.
	\item Elastomers, the mechanical properties are controlled by the changes in the system configuration,by unravelling network strands at nearly constant mass density.
	\item A network architecture is defined by: Connectivity, dimensions and flexibility of network strands.
	\item The weak dependence on chemical compositions helps to manipulate the mechanical response.
	\item In the article they focues in the Shear modulus.
	\item Topological defects: Chain entranglements, dandling chain ends, loops, multiple strands and side chains.
	\item Search of definitions:
		\begin{enumerate}
			\item Shear modulus:
			\item Cross-link functionality:
			\item Degree of polymerization: Number of monomeric units (MO and CL)
			\item Kuhn length:  A theoretical treatment, developed by Werner Kuhn, in which a real polymer chain is considered as a collection of N Kuhn segments each with a Kuhn length b. Each Kuhn segment can be thought of as if they are freely jointed with each other.
		\end{enumerate}
	\item The softness at small deformations is manipulated by changing the cross-link density.
	\item Linear chain Networks
	\begin{itemize}
		\item The cross-linnk density is THE control parameter
		\item The Young's modulus can be tune by introducing loops and dangles.
		\item Modify the junctions can controll the network topology
		\item The cross-link functionality can be increased by adding dendrimes, colloidal particles and microphase domains.
		\item Their equilibrium characteristics are mostly determined by the degree of polymerization of the chemical network strand.
		\item Strand entanglements can create a loweer limit for the shear modulus and upper limit for the elongation-at-break.
		\item Softness can be achieved by solvent swelling.
		\item Swelling leads to strand extensions and therefore firmness enhacement.
	\end{itemize}
	\item Dual Networks
	\begin{itemize}
		\item Networks with permanent cross-links  and dynamic cross-links.
		\item Permamnent cross-link are responsible for elastic properties and sample shape
		\item Dynamic cross-link regulate energy dissipation and strain-rate response.
		\item Dynamic networks depen on strain rate, permanent network does not depend on strain rate.
			\item At higher rates, physical cross-links do not have enough time to break and behave as permanent cross-links, resulting in high modulus and strength.
		\item \textbf{To characterize the dynamic mechanical properties of the dual networks at small deformations, evolution of the stress relaxation Young’s modulus E(t) during uniaxial extension at a constant temperature is calculated by using a linear viscoelasticity approximation.}
	\end{itemize}
	\item Network with Brush-Like strands: I didn't read it with detail, beacuse it is out of my scope/comprehension/interest at the moment. It will be great to read it with more knowledge in the area.
\end{itemize}

Need to check \cite{dealyStructureRheologyMolten2006} for more information about polymer structures and their characterization.

Now, in the sense, that the curve of strain-stress is essential to characterize the material, I search in lammps how to compute the stres and strin in the simulation.
What I found is an page where there explain how to \href{https://docs.lammps.org/Howto_elastic.html}{Calculate elastic constants} and according to that page, the elastic constants (strain and stress) are related as follows:
\begin{gather*}
	s_{ij} = C_{ijkl} e_{kl},
\end{gather*}
where the repeates indices impley summation.$s_{ij}$ are the elemnts of the symmetric stress tensor, $e_{kl}$ are the elements of the symmetric strain tensor and $C_{ijkl}$ are the elemnts of the foruth rank tensor of elastic constants.

Then, they state that the elastic constant can be computed at \textit{zero temperature} or \textit{finite temperature} and metiond the following methods for doing it in LAMMPS,
\begin{itemize}
	\item At zero temperature
		\begin{enumerate}
			\item The $C_{ijkl}$ are estimated by deforming the simulaiton box in one of the six directions using \verb|change_box| command. Then compute the chanfes in the stress tensor.
		\end{enumerate}
	\item At finite temperature
		\begin{enumerate}
				\item Exploit the relation between elastic constants, stress fluctuations, and the Born matrix. \cite{rayStatisticalEnsemblesMolecular1984}
				\item Measure the change in average stress tensor in an NVT simulations when the cell volume undergoes a finite deformation.
				\item Sample the triclinic cell fluctuations that occur in an NPT simulation. 
			\end{enumerate}
\end{itemize}

Finally, in the article \cite{clavierComputationElasticConstants2017} a nice review of the advantages and disadvantages of all of these methods is provided.

So the next week I will check those articles and the examples cripts in lammps.

Possibly, during the weekend's I transfer old entries of multiple daily log in Obsdian, notesbooks, Git and Discord.

%Dealy, J. M.; Larson, R. G. Structure and Rheology of Molten Polymers: From Structure to Flow Behavior and Back Again; Hanser Gardner: Cincinnati, OH, 2006. [It's a book]

% Ferry, J. D. Viscoelastic Properties of Polymers, 3rd ed.; John Wiley: NewYork, 1980.

\newday{2024-07-22}

Today I check the examples scripts form \href{https://docs.lammps.org/Howto_elastic.html}{Calculate elastic constants}, specifficly the \textit{Zero temperature} and \textit{Born\_Matrix} examples.
I understud the methodology.
The scripts create a set of particles, then in minimize the system and save that state.
Then, a set of 12 deformations are applied, a pair of deformation per basis: $xx$, $yy$, $zz$, $xz$, $yz$, $xy$, with positive and negative sign, to change the direction. 
Then, the stress and pressure tensors are used to compute the strain tensor.

In the \textit{zero} temperature scheme, this is the end of the simulation, meanwhile, this procedure is repeted thru an interval of time for the \textit{Born\_Matrix} example.

I want to applied that methodology to characterize the hydrogel, but first, the scripts needs to be converted from reduce units to real units.
Also, I want/need to undertad the relations between stres-strain parameters, preassure, stress, the derivatives of the potential energy with stress and strain.
Furthermore, I need to seek information about an order parameter to propose.

\rule{\textwidth}{0.4pt}

Taking into account, that temperature is the mean of the velocity of system of particles, we can create a vague relation betweent the ``structure'' and the temperature.
Assuming that a solid or fluid has more bonds between particles than a gas, a gas has a greater temperature than the others.
However, how can we stablish a range of temperature o a temperutre in which occurs a transition between solid, to liquid to gas.

If we think that the temperature is a mean, hence we can use the standrd deviation as a tool to identify a transition.
When the standard deviation of the temperature is localize, then is in a state, but when the standard deviation is broader, we can stipulate that is happening a phase transition.

Now, here are few point that need to be clarify:
\begin{itemize}
	\item The temperature is a mean of the particle velocities in a system. To analyze the standard deviation of the temperature, that means to take into account $N$ systems at constant number of particles, volume and energy (NVE).
	\item The other points I dont remebebr becuase I take another line of thought.
\end{itemize}

\rule{\textwidth}{0.4pt}

About viscosity, temperature and rheological properties.

The viscosity, temperature and rheological properties are consquence of the composition of the material.
The temperature is more related with the internal velocity of the particles, meanwhile, the viscosoty and rehological properties are more related with the relative position between particles.

I undertand the structure of the material, as the relative position of the particles between the, hence, the viscosity and the rhological properties are observables of the strucutre of the material.
On the other hand, the temperature is an observable of the velocity of the particles.

It is important to take into account, that the velocity of the particles is affected by the position of the particles and the position of the particles is affected by the velocity of them.
Hence, twe can use the hree of them to state phase changes.

In a broad sens, the temperature will help to diffiriantiet between solids and liquid with gasses.
The viscosity between solids and gases with liquids and rhologicla properties between liquids and gasses with solids.

Now, with all that in mind, How we can related those properties with specific structures?
That a set of parameters will help to kwon that a material has chains in ``U'' or will form vessicles or chains.

To be honest, I infer that to crespond that, we also need to take into account the compounds and the electronic proporties of their compounds.
Becuases, that will help to connect the macro with the micro properties.

About Pressure.
Preassure is an external variable.
Pressure is the change of momentum over time per area.
Now, the temperature is the expected value of the kinetic energy, which is related with the mometum of the particles.
So pressure, will help to understad the change of preperties thru time.

The pressure will help to observed the change of temperature over time.
This is, because the temperature is directly related with the velocity and mass, such as the momeutm.
However, the mass is assume to be constant, even in deformationn, there is no change in mass, only in the surface area at constant volume.
So, when the temprature changes, then the preassure register that change in time, because the velocity changes.

However, under deformation, the area also changes, hence, the preassure changes.

Now I need more basic information about SoftMatter stuff, hence I will watching the following videos:  \href{https://www.youtube.com/watch?v=fwratvdgoT4&list=PL7B_29ynGKv3v81zK9C9AcTNtGDCvbojV}{Kent State 2020 course on Statistical Mechanics of SoftMatter}

\rule{\textwidth}{0.4pt}

some stuff of definitions:
\begin{itemize}
	\item Shear Strain: The engineering shear strain ($\gamma xy$) is defined as the change in angle between lines $\bar{AC}$ and $\bar{AB}$.
	\item Shear stress (often denoted by $\tau$) is the component of stress coplanar with a material cross section.
	\item In materials science, shear modulus or modulus of rigidity, denoted by $G$, or sometimes $S$ or $\mu$, is a measure of the elastic shear stiffness of a material and is defined as the ratio of shear stress to the shear strain.
	\item  Young's modulus E describes the material's strain response to uniaxial stress in the direction of this stress (like pulling on the ends of a wire or putting a weight on top of a column, with the wire getting longer and the column losing height),
    \item The Poisson's ratio $\upsilon$ describes the response in the directions orthogonal to this uniaxial stress (the wire getting thinner and the column thicker),
    \item The bulk modulus $K$ describes the material's response to (uniform) hydrostatic pressure (like the pressure at the bottom of the ocean or a deep swimming pool),
    \item The shear modulus $G$ describes the material's response to shear stress (like cutting it with dull scissors).
\end{itemize}

\rule{\textwidth}{0.4pt}

Recomendtions of Dr. Tensor, jsjs

\begin{itemize}
	\item Statistical Physics, A Guenault (Chapman and Hall)
	\item Introductory Statistical Mechanics, DS Betts and RE Turner (Addison Wesley)
	\item Introductory statistical Mechanics, Bowley and Sanchez (Oxford)
	\item Introduction to Statistical Physics, K Huang (Taylor and Francis)
	\item Statistical Mechanics: A Survival Guide, A. M. Glazer and J. S. Wark (Oxford University Press)
\end{itemize}

Other references: 
\href{https://link.springer.com/book/10.1007/978-3-540-28606-6}{Statistical Physics, Daijiro Yoshioka}, 
\href{https://link.springer.com/book/10.1007/978-3-319-21054-4}{Introduction to the Theory of Soft Matter, Jonathan V. Selinger}

\newday*{2024-07-24}

Important points of the reunion:
\begin{itemize}
	\item We are more interested in the curve behaviour.
	\item Changes in the code: Flip yes, remap x
	\item Graphs
	\begin{itemize}
		\item Shear rate vs pressure
		\item Sheare rate vs stress
	\end{itemize}
	\item Oscilatory rheology.
	\item To charetirize the elastic regime.
	\item Change shear rate and then viscosity.
	\item Viscosity in dought.
	\item Then shearch for yield stress.
	\item To see the elastic limit.
	\item For futher simulation: Change the shear rate: Do not let the system to relax.
	\item Experimentally, not neccecesary you are in a stationary state.
	\item First we are going to center in the macroscopic propoerties.
	\item Daniel Bonn.
\end{itemize}

\newday{2024-08-05}

Last week I was on vacation.

Important changes: I added the nve fix to create Brownian Dynamics
Other stuff: The pressure command gives the components of the stress tensor of a collection of atoms.

Sutff about tensors: 
\href{https://www.quora.com/What-is-a-fourth-rank-tensor}{What is a fourth rank tensro} 
\href{https://www.quora.com/?signup_answer_page=17479694}{Answer N}
\href{https://www.quora.com/In-laymans-terms-what-is-a-Tensor-Field}{In layman's terms, what is a Tensor Field?}

About fluctuations in Stress in Young's Modulus: \href{https://matsci.org/t/fluctuations-in-stress-youngs-modulus/33483}{LAMMPS forum}.
Basically: They use moving average over the \verb|ave/time| command in lammps to smotth the curves.


Some stuff to take into account:
\begin{itemize}
	\item \verb|fix nvt/sllod|
\end{itemize}

Observations/questions:
The strain-rate vs stress curves means that the stress at each strain rate is the final stress of the deformation or is like an average of a steady state?

The strain vs stress curves means that the stress at each strain is the final stress or the average of the stress?

When the deformation is applied, there is a strain, which is the ratio of the initial position with the final position of the deformation.
You can create that deformation at different rhythms, hence the strain-rate/strain vs stress curves.

For me, those kinds of graphs makes sense if the stress at each strain is the average with the standard deviation of the deformation process at a constant strain-rate.

Now... in LAMMPS the pressure command gives a 6 elements for the stress tensor, so... I need to take de dot product with that?
I guess, because in all articles they report a scalar.
Also, makes sense to report the stress tensor component that ir parallel to the deformation.


Idea for Tesis keywords?: Characterization of hydrogels \textit{In silico}

Need to finish the reading og the Entropy based fingerprint for local crystaline order article.

Next articles to read:
\begin{itemize}
	\item Importance of many-body correlations in glass trnasition: An example from polydisperse hard spheres
	\item Enhancing Entropy and Enthalpy Fluctuations to Drive Crystallization in Atomistic Simulations
	\item Well-Tempered Metadynamics: A Smoothly Converging and Tunable Free-Energy Method
	\item Escaping free-energy minima
\end{itemize}


So... this is intersting. How a deformation can start a phase transition-

\newday{2024-08-06}

Entropy based fingerprint for local crystalline order 
Metadynamics to enhance the probability of inducing the crystal formation un an accessible computer time.
Metadynamics relies on the identification of appropiate collective variables.
They found that enthalpy and an approximate expression for entropy based on the two body correlation function, were useful collective variables in this constext.
The main thing is that Enthalpy and entropy are global properties and in order to be able to use them as local parmeters we have to project them onto each atom.

The pair correlation function accounts for about 90\% of the configurational entropy, and is given by,
$$S_2 = -2\pi\rho k_B\int_{0}^{\infty}\left[g(r)\ln g(r)-g(r)+1\right]r^2 dr,$$
$\rho$ is the system's density, and $g(r)$ is the radial distribution function.

The projection on atom $i$ can be achieved using the xpression:
$$s_s^i = -2\pi\rho k_B\int_{0}^{r_m}\left[g_m^i(r)\ln g_m^i(r)-g_m^i(r)+1\right]r^2 dr.$$
The radial distribution function is defined as 
$$ g_m^i(r) = \frac{1}{4\pi\rho r^2}\sum_{j}\frac{1}{\sqrt{2\pi\sigma^2}}\exp\left[-\frac{(r-r_j)^2}{2\sigma^2}\right],$$
$j$ are the neighbors of atom $i$,
$r_{ij}$ is the distance between atoms $i$, $j$, 
and $\sigma$ is a broadening parameter.
$\sigma$ shall be so small such that $g_m(r)\approx g(r)$, yet large enough for the derivatives relative to the atomic positions to be manageable.

To distinct the distributions more clearly, they define an average local entropy:
$$\bar{s}_S^i = \frac{\sum_j s_S^j f(r_{ij})+s_S^i}{\sum_j f(r_{ij}) +1}$$,
$j$ runs over the neighbors of atom $i$ and $f(r_{ij})$ is a switching function with cutoff $r_a$:
$$ f(r_{ij}) = \frac{1-(r_{ij}/r_a)^N}{1-(r_{ij}/r_a)^M}, $$
$N=6,~M = 12$.

$\bar{s}_S$ is the entropy fingerprint.
The ability to distinguish sharply between solid.like and liquid-like molecules depend on a wise choice of the parameters $r_m$ and $r_a$.
As $r_m$ is increased the difference beteen liquid and solid more and more evident.
By increasing $r_a$ eventually the locality of the entropy finger print is lost.

The entropy fingerprint is able to distinguish liquid-like from solid-like atomic enviroments and is best achieve if we acompany the definition of local entropy with a measure of local enthalphy.

To distiguish between the phases, it is neede to calculate the joint probability distribution of the enthalpy and entropy finger prints $(P(\bar{s}_H,\bar{s}_S))$.

\rule{\textwidth}{0.4pt}

About the simulations

So... the assembly still wokrs and has been upgraded. 
However, the deformation simulation still broking :).
I decrement the time step and improve a little bit: instead of broking at 600 steps, it brokes at 20 000 steps.

Need to search another technique in LAMMPS for the shear deformation.

\href{https://mattermodeling.stackexchange.com/questions/9374/langevin-vs-nos%C3%A9-hoover-thermostat}{Hoover thermostat vs Langevin thermostat}
\href{https://mattermodeling.stackexchange.com/questions/4245/selection-of-appropriate-langevin-damping-parameter-for-md-of-solid-metal?noredirect=1&lq=1}{Damping in Langevin thermostat}

So ... I change the langevin thermostat for a Hoose-Nosee thermostat, and the soimultion does not longer breaks.

So... new changes in the shear rate, gamma delta and other stuff improves the simulation.

\newday{2024-08-07}

So ... the langevin impelementation is still breaking, but the nvt integration works.

In the Langevin thermostat, when I increment the damp parameter, it helps to create more iterations, but eventually it borkes due to missing particles.
With the nvt assemble, the velocities of the particles gos crazy.

So... I change the shear rate, beccuase I think that I'm trying to force some results.
I want to see a slow deformation.
Also, due to the change of units, my intuition can not be trusted.

So, I will stop to force results, follow the recomendations of the documentation and change the rate of deformation and the maximum strain.


The next step, is to create the graph of the stress components over time.
Then I want to add the computes of the entropy and entalphy.

\rule{\textwidth}{0.4pt}

Well, it's like magic.
As expected, I reduce the shear rate, and the velocities of the particles slow down and the clusters are preserved.
At higher shear rates, the velocities of the particles goes high, and the clusters are no longer preserved.

Now, taking that into account, the number of clusters will help to now if there is a reconfiguration of the system or no.
Also, when the number of clusters decrese at a faster rate, we can say that is in a liqud phase o something like that.

Maybe it will be wise to see how to calculate the elastic constants.

\newday*{2024-08-08}

In the reunion Toño and Caludia said that the langevin and nve commands are correct and star checking at my code.
Then, Toño change the integration command nve to nve\/rigid and the simulation dind't broke.


\newday{2024-08-09}

Now that the simulations are properly coded, it is time to change the parameters.

The command \verb|deform| with the \textit{erate} keyword it adds a relative fraction of the total box length each time step.

The parameters that we are going to chage are:
\begin{itemize}
	\item Shear rate
	\item Number of CrossLinkers
\end{itemize}

I have some doubts about the damp parameter of the Lagevin thermostat.
Because, it is always said that it represent a viscosity and the medium, hence, the rheological properties of the simultion will change, so... it will be interesting how the properties changes with the damp parameter.

Well, also, the main objective is that we want to connect the macroscopic response with the structural properties of the chain.

For that, It will be nice to change the mass of the system, to see if the behaviour changes or not.

\rule{\textwidth}{0.4pt}

Understanding the response of Polyethylene glycol diacrylate Hydrogel Networks: Astatistical Mechanics-Based Framework article.

The derive a microstructurally motivated and energy-based model that captures the three essential features that enable deformation in PEGDA hydrogels:
\begin{itemize}
	\item Entropic contribution of the PEG chains
	\item Deformation of rods
	\item Mechanical PA-PA interactions.
\end{itemize}

The resume of the relations are:
We demonstrate that in the limit of short PEG chains and long PA rods, the behavior is governed by PA−PA interactions, which we attribute to the mechanical restrictions due the high density grafting of PEG chains to the PA cores.
In the case of long PEG chains and short PA rods, the hydrogel behaves entropically due to the deformation of the PEG chains.

Those are the main ideas that I was interested.
The advantage or the characteristic of that polymer is that is a brush like, such that, we can define a system of coordinate arround the principal chain, and then cansider the tranformation of the that system of coordinates after the compression.

Read: A microscopically motivated model for the swelling-induced drastic softening of hydrogen-bond dominated biopolymer networks.

Read: Swelling thermodnamics and phse transitions of polymer gels.

%\printbibliography

\end{document} % this will automatically generate a references chapter if you cited any references throughout the text
