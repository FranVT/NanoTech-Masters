\documentclass[../../main-notes.tex]{subfiles}

\begin{document}

\section{Introduction}


Partimos del teorema del virial en forma tensorial para un sistema de $N$ partículas
\begin{gather*}
    \frac{1}{2}\frac{d^2}{dt^2}\left(\sum_{i=1}^N m_i \mathbf{r}_i \otimes \mathbf{r}_i\right) = \sum_{i=1}^N m_i \mathbf{v}_i \otimes \mathbf{v}_i + \frac{1}{2}\sum_{i=1}^N \sum_{j\neq i} (\mathbf{r}_i - \mathbf{r}_j) \otimes \mathbf{F}_{ij}
\end{gather*}

Momento de inercia tensorial
\begin{gather*}
\mathbf{I} = \sum_{i=1}^N m_i \mathbf{r}_i \otimes \mathbf{r}_i
\end{gather*}

Primera derivada temporal
\begin{gather*}
\frac{d\mathbf{I}}{dt} = \sum_{i=1}^N m_i \left( \mathbf{v}_i \otimes \mathbf{r}_i + \mathbf{r}_i \otimes \mathbf{v}_i \right)
\end{gather*}


Seguinda derivada temporal
\begin{gather*}
\frac{d^2\mathbf{I}}{dt^2} = 2\sum_{i=1}^N m_i (\mathbf{v}_i \otimes \mathbf{v}_i) + \sum_{i=1}^N (\mathbf{F}_i \otimes \mathbf{r}_i + \mathbf{r}_i \otimes \mathbf{F}_i)
\end{gather*}

Descomposición de fuerzas
\begin{gather*}
\mathbf{F}_i = \mathbf{F}_i^{\text{ext}} + \sum_{j\neq i} \mathbf{F}_{ij}
\end{gather*}


Solución para sistemas en equilibrio: $\langle d^2\mathbf{I}/dt^2\rangle=0$
\begin{gather*}
\left\langle \sum_{i=1}^N m_i (\mathbf{v}_i \otimes \mathbf{v}_i) \right\rangle + \frac{1}{2}\left\langle \sum_{i=1}^N \sum_{j\neq i} \mathbf{r}_{ij} \otimes \mathbf{F}_{ij} \right\rangle = -\frac{1}{2}\left\langle \sum_{i=1}^N (\mathbf{F}_i^{\text{ext}} \otimes \mathbf{r}_i + \mathbf{r}_i \otimes \mathbf{F}_i^{\text{ext}}) \right\rangle
\end{gather*}


Tensor de esfuerzos virial
\begin{gather*}
\boxed{\sigma_{\text{virial}} = -\frac{1}{V} \left[ \underbrace{\sum_{i=1}^N m_i (\mathbf{v}_i \otimes \mathbf{v}_i)}_{\sigma_{\text{kin}}} + \underbrace{\frac{1}{2}\sum_{i=1}^N \sum_{j\neq i} \mathbf{r}_{ij} \otimes \mathbf{F}_{ij}}_{\sigma_{\text{conf}}}\right]} 
\end{gather*}

\section{Another prompt}

Partimos del teorema del virial generalizado para un sistema de N partículas sujetas a dinámica de Langevin. Consideramos la ecuación de movimiento en el régimen inercial:
\begin{gather*}
m_i \frac{d^2\mathbf{r}_i}{dt^2} = -\nabla_i U(\{\mathbf{r}_j\}) - \gamma_i m_i \mathbf{v}_i + \sqrt{2\gamma_i m_i k_B T} \boldsymbol{\xi}_i(t)
\end{gather*}

donde $\boldsymbol{\xi}i(t)$ es un proceso estocástico gaussiano con $\langle \xi_i^\alpha(t) \xi_j^\beta(t') \rangle = \delta{ij}\delta_{\alpha\beta}\delta(t-t')$.

Definimos el tensor de momento de inercia del sistema:
\begin{gather*}
\mathbf{I} = \sum_{i=1}^N m_i \mathbf{r}_i \otimes \mathbf{r}_i
\end{gather*}

Primera derivada temporal
\begin{gather*}
\frac{d\mathbf{I}}{dt} = \sum_{i=1}^N m_i \left( \mathbf{v}_i \otimes \mathbf{r}_i + \mathbf{r}_i \otimes \mathbf{v}_i \right)
\end{gather*}

Segunda derivada temporal
\begin{gather*}
\frac{d^2\mathbf{I}}{dt^2} = 2\sum_{i=1}^N m_i (\mathbf{v}_i \otimes \mathbf{v}_i) + \sum_{i=1}^N \left( \mathbf{F}_i \otimes \mathbf{r}_i + \mathbf{r}_i \otimes \mathbf{F}_i \right)
\end{gather*}

Descomposición de fuerzas:
Para fuerzas de Langevin:
\begin{gather*}
\mathbf{F}_i = -\nabla_i U - \gamma_i m_i \mathbf{v}_i + \sqrt{2\gamma_i m_i k_B T} \boldsymbol{\xi}_i(t)
\end{gather*}

Promedio en equilibrio
Aplicando $\langle d^2\mathbf{I}/dt^2 \rangle = 0$ y considerando que los términos estocásticos promedian a cero:
\begin{gather*}
\left\langle \sum_{i=1}^N m_i (\mathbf{v}_i \otimes \mathbf{v}_i) \right\rangle + \frac{1}{2}\left\langle \sum_{i=1}^N \mathbf{r}_i \otimes \mathbf{F}_i^{\text{cons}} \right\rangle = \frac{1}{2}\left\langle \sum_{i=1}^N \gamma_i m_i (\mathbf{v}_i \otimes \mathbf{r}_i + \mathbf{r}_i \otimes \mathbf{v}_i) \right\rangle
\end{gather*}

Tensor de esfuerzos virial:
Definimos el tensor de esfuerzos virial como el promedio volumétrico:
\begin{gather*}
\boxed{\sigma_{\text{virial}} = -\frac{1}{V} \left[ \underbrace{\sum_{i=1}^N m_i \langle \mathbf{v}_i \otimes \mathbf{v}_i \rangle}_{\sigma_{\text{kin}}} + \underbrace{\frac{1}{2}\sum_{i=1}^N \sum_{j\neq i} \langle \mathbf{r}_{ij} \otimes \mathbf{F}_{ij} \rangle}_{\sigma_{\text{conf}}} \right]}
\end{gather*}

Término disipativo (Langevin)
En régimen inercial, la contribución disipativa se expresa como:
\begin{gather*}
\sigma_{\text{diss}} = \frac{1}{V} \sum_{i=1}^N \gamma_i m_i \langle \mathbf{v}_i \otimes \mathbf{r}_i \rangle
\end{gather*}


\subsection{Another prompt}

Fundamento físico-matemático:
La derivación del tensor de esfuerzos virial inicia con la conservación del momento lineal porque este principio encapsula la conexión entre las leyes de Newton a escala atómica y la respuesta mecánica colectiva de los materiales. Al considerar un sistema de partículas, la segunda ley de Newton para la partícula $i$ es:
\begin{gather*}
m_i \frac{d\mathbf{v}_i}{dt} = \mathbf{F}_i^{\text{ext}} + \sum_{j \neq i} \mathbf{F}_{ij}
\end{gather*}

Proceso de derivación:

Momento de inercia tensorial: Se define $\mathbf{I} = \sum_i m_i \mathbf{r}_i \otimes \mathbf{r}_i$ como análogo rotacional del momento lineal.

Relación con momento: La segunda derivada temporal $\frac{d^2\mathbf{I}}{dt^2}$ vincula aceleraciones con fuerzas:
\begin{gather*}
\frac{1}{2}\frac{d^2\mathbf{I}}{dt^2} = \sum_i m_i (\mathbf{v}_i \otimes \mathbf{v}_i) + \frac{1}{2}\sum_i (\mathbf{F}_i \otimes \mathbf{r}_i + \mathbf{r}_i \otimes \mathbf{F}_i)
\end{gather*}

Transición a medio continuo: Al promediar en volumen y tiempo, se obtiene el tensor virial como densidad de flujo de momento:
\begin{gather*}
\sigma_{\text{virial}} = -\frac{1}{V} \left\langle \frac{1}{2}\frac{d^2\mathbf{I}}{dt^2} \right\rangle
\end{gather*}

¿Qué se busca representar?
El tensor virial cuantifica cómo se transfiere momento a través de un volumen microscópico:

Movimiento térmico ($\sigma_{\text{kin}}$) → Transporte convectivo

Fuerzas intermoleculares ($\sigma_{\text{conf}}$) → Transferencia por contacto

Disipación ($\sigma_{\text{diss}}$) → Pérdidas irreversibles

Esta descripción unifica la mecánica de partículas con la hidrodinámica continua, siendo el puente entre la escala atómica y macroscópica.

Transición conceptual: La elección del momento como punto de partida no es arbitraria. Refleja la visión de Boltzmann y Gibbs de que las propiedades termodinámicas emergen del transporte de cantidades conservadas (momento, energía, masa).




\end{document}
