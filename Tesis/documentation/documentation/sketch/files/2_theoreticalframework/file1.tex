\documentclass[../../main-notes.tex]{subfiles}

\begin{document}


\begin{comment}
Theory of simple liquids 
Computational Physics\citep{Thijssen2007}

The opposite situation is given by a system that remains in strong thermal contact with another system at thermal equilib-equilibrium. 
The prototype of such behavior is provided by small particles that interact among themselves with a given force (which may be electrical,magnetic, gravitational, etc., in origin) suspended in a highly viscousliquid (i.e., oil) at a given temperature. 
In the high-viscosity limit the equation of motion of the ith particle is

 The missing term in Eq.A9.1 is the force acting on the particles due to the collisions of themolecules of oil: they produce an extra random force $b(t)$, which shouldbe added to A9.2. 
 We have
This new force prevents the particles from remaining at the minimumposition (xim), and it is the origin of the Brownian motion.

In the simplest case, each collision gives a contribution to theforce, and contributions coming from different collisions are uncorre-lated: the forces acting on the same particle at different times (or ondifferent particles at the same time) are practically uncorrelated. We canthuswrite

Where the bar denotes the average over many repeated experiments: i.e.,we have M identical copies of the same system (or a single system onwhich M measurements are taken at widely separated time intervals) andwe average the $B'(t)$ over these M replicas. The bar denotes theaverage when M goes to infinity. If the number of collisions in a time e isquite large, the $B'(t)$ will be Gaussian-distributed variables with avariance\footnote{Explanaition of why gaussian in page 25}
~\citep{Parisi1988}

\end{comment}


\subsection{Description of the microgel}
Description of the microgel as a colloid to introduce the langevin methodology.

“The dynamics of a macromolecular system is entirely determined by the potential $U(rN)$ associated with the process. For computational and practical reasons, this potential is virtually always an approximation of the real physical potential\footnote{Those potentials are explained in the following section?\\ Would be better to describe the parameters in the implementation and the model here?}.”\citep{paquetMolecularDynamicsMonte2015}


\subsection{Brownian dynamics}\footnote{ 
The problem is characterised by two very different timescales, one associated with the slow relaxation of the initial velocity of the brownian particle and another linked to the frequent collisions that the brownian particle suffers with particles of the bath. (Problema de implementación o representación matemática, eso del tiempo de relajación por condiciones iniciales es por que matemáticamente se reuieren condiciones iniciales para resolver el sistema ecuaciones diferenciales.) 
Langevin assumed that the force acting on the brownian particle consists of two parts: a systematic, frictional force proportional to the velocity $u(t)$, but acting in the opposite sense, and a randomly fluctuating force, $R(t)$, which arises from collisions with surrounding particles\citep{tsl2006}.
}


From a general point of view there are two types of methods to make a quatitative description of systems: one focused on simulating dynamics at the microscale, and the other dedicated to deriving or establishing evolutionary equations at the macroscale\citep{wangMultiscaleModelingSimulation2025}.
Since we assume that the a microgel's mechanical response derives from its internal structure\footnote{Poner citas que desmuestrén que no es hipótesis, si no que se sabe} we choose to simulate the dynamics at the microscale.
Additionally, by treating the microgel as a colloid, permits applying Brownian motion theory to model its response under shear deformation. 
Finally, there are two commonly used mathematical frameworks to model the Brownian motion, the continuous time random walk (CTRW) model and the Langevin equation\citep{wangMultiscaleModelingSimulation2025}, in this work we decided\footnote{Supongo que eventualmente justificaré la desición.} to use the langevin dynamics mathematical framework.

This is because, the solid phase of the colloid has a large mass and will change their momenta after many collisions with the solvent molecules and the picture which emerges is that of the heavy particles forming a system with a much longer time scale than the solvent molecules\citep{Thijssen2007} and Langevin theory takes advantage of this difference in time scale to eliminate the details of the degrees of freedom of the solvent particles and represent their effect by stochastic and dissipative forces allowing longer simulations that would be impossible if the solvent were explicitly included\citep{pastorTechniquesApplicationsLangevin1994}.
However, the representation of the solvent by a stochastic and dissipative force, introduce the problem of characterize two very different timescales, one associated with the slow relaxation of the initial velocity of the brownian particle and another linked to the frequent collisions that the brownian particle suffers with particles of the bath\citep{tsl2006}\footnote{Para traer a colación la sensibilidad de la respuesta mecánica al parámetro de damp.}. 
Therefore, two terms are used to create a mathematical representation of the solvent: a frictional force proportional to the velocity of the brownian particle and a fluctuating force. 
Hence,
\begin{gather}
    m\dv{\vec{v}(t)}{t}=\vec{F}(t)-m\gamma\vec{v}(t)+\vec{R}(t).\label{eqn:BrownianDyn1}
\end{gather}
The friction constant $\gamma$\footnote{Cuidado con las unidades. Hacer análisis dimensional, porque por la condición de correlación en $R$, $\gamma$ ocupa tener unidades de masa entre tiempo, pero en la ecuación, solo ocupa unidades de $1/s$.} parametrises the effect of solvent damping and activation and is commonly referred to as the collision frequency in the simulation literature, even though formally a Langevin description implies that the solute suffers an infinite number of collisions with infInitesimally small momentum transfer.
Also, the fact that the second term is not a function of the position of any of the particles involves the neglect of involves the neglect of hydrodynamic interaction or spatial correlation in the friction kernel spatial correlation in the friction kernel\citep{pastorTechniquesApplicationsLangevin1994}.
On the other hand, $\vec{R}(t)$\footnote{No me acuerdo en donde está que se puede asumir que tiene distribución gaussiana.} is a ``random force''subject to the following conditions
\begin{align*}
    \expval{\vec{R}(t)} &= 0 \\
    \expval{\vec{R}(t)\vec{R}(t')} &= 2k_{B}T\gamma\delta\qty(t-t') 
\end{align*}
The no time correlation is equivalent to assuming that the viscoelastic relaxation of the solvent is very rapid with respect to solute motions\footnote{Grote land Hynes [26] have investigated this assumption for motions involving barrier crossing and have found that while it is seriously in error for passage over sharp barriers (such as 12 recombination); it is quite adequate for conformational transitions such as might be found in polymer motions.\citep{pastorTechniquesApplicationsLangevin1994}}.

In comparing the results of Langevin dynamics with those of other stochastic methods [28-31], the relevant variable is the velocity relaxation time, $\tau_{v}$ which equals $\gamma^{-1}$\citep{pastorTechniquesApplicationsLangevin1994}
The Langevin equation improves conformational sampling over standard molecular dynamics\citep{paquetMolecularDynamicsMonte2015}.

\begin{itemize}
    \item Hablar acerca de que la fuerza aleatoria puede tener distribución gaussiana, pero no necesariamente.
    \item hablar de la ecuación de Green-Kubo: \[\eta=\frac{V}{k_B T}\int_{0}^{\infty}\expval{\sigma_{xy}(t)\sigma_{xy}(0)}\mathrm{d}t\]
    \item No se que tanto hablar de la idea de correlación y su aplicación en estos temas.
\end{itemize}


\begin{comment}
Coso que puede ayudar en futuras re-escrituras.

--------------------------------------------
Agregar para exploraciones futuras

One should notice that the frictional term depends on the previous history of the trajectory (Markovian process). 
The friction term is important in obtaining realistic simulations, as it takes into account the viscosity of the solvent (a feature, which is absent from both MD and MC). 
If one assumes that the frictional term is constant (no history), one obtains the celebrated Langevin equation (LE):\citep{paquetMolecularDynamicsMonte2015}
--------------------------------------------

In other words, the kinetic energy is on average divided equally over the degrees of freedom, the polymers will move much more slowly than the solvent molecules\citep{Thijssen2007}



How can we model the effect of the solvent particles without taking into account their degrees of freedom explicitly? 
When a heavy particle is moving through the solvent, it will encounter more solvent particles at the front than at the back. 
Therefore, the collisions with the solvent particles will on average have the effect of a friction force proportional and opposite to the velocity of the heavy particle.\citep{Thijssen2007}

Reference [34] presents the Langevin equation for polymers engaged in polymerization/depolymerization reactions, by establishing a random diffusion coefficient that correlates with particle size. Reference [35] presents the Langevin equations for continuous time Lévy walks. X. D. Wang et al.present the Langevin description of the Lévy walk [36]. Y. Chen et al.then provide the Langevin description of the Lévy walk with memory [37] and examine the impact of an external force [38,39]\citep{wangMultiscaleModelingSimulation2025}.

\end{comment}


\end{document}
