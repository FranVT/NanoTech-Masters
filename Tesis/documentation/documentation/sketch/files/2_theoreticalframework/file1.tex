\documentclass[../../main-notes.tex]{subfiles}

\begin{document}

Theory of simple liquids 
Computational Physics\citep{Thijssen2007}

The opposite situation is given by a system that remains in strong thermal contact with another system at thermal equilib-equilibrium. 
The prototype of such behavior is provided by small particles that interact among themselves with a given force (which may be electrical,magnetic, gravitational, etc., in origin) suspended in a highly viscousliquid (i.e., oil) at a given temperature. 
In the high-viscosity limit the equation of motion of the ith particle is

 The missing term in Eq.A9.1 is the force acting on the particles due to the collisions of themolecules of oil: they produce an extra random force $b(t)$, which shouldbe added to A9.2. 
 We have
This new force prevents the particles from remaining at the minimumposition (xim), and it is the origin of the Brownian motion.

In the simplest case, each collision gives a contribution to theforce, and contributions coming from different collisions are uncorre-lated: the forces acting on the same particle at different times (or ondifferent particles at the same time) are practically uncorrelated. We canthuswrite

where the bar denotes the average over many repeated experiments: i.e.,we have M identical copies of the same system (or a single system onwhich M measurements are taken at widely separated time intervals) andwe average the B'{t) over these M replicas. The bar denotes theaverage when M goes to infinity. If the number of collisions in a time e isquite large, the $B'(t)$ will be Gaussian-distributed variables with avariance\footnote{Explanaition of why gaussian in page 25}
~\citep{Parisi1988}


-----------------------------------------------------------------------


\subsection{Description of the microgel}
Description of the microgel as a colloid to introduce the langevin methodology.

“The dynamics of a macromolecular system is entirely determined by the potential U(rN) associated with the process. For computational and practical reasons, this potential is virtually always an approximation of the real physical potential\footnote{Those potentials are explained in the following section?\\ Would be better to describe the parameters in the implementation and the model here?}.”\citep{paquetMolecularDynamicsMonte2015}




\subsection{Brownian dynamics}\footnote{Its basis is the stochastic theory used by Langevin to describe the brownian motion of a large and massive particle in a bath of particles that are much smaller and lighter than itself. 
The problem is characterised by two very different timescales, one associated with the slow relaxation of the initial velocity of the brownian particle and another linked to the frequent collisions that the brownian particle suffers with particles of the bath. 
Langevin assumed that the force acting on the brownian particle consists of two parts: a systematic, frictional force proportional to the velocity $u(t)$, but acting in the opposite sense, and a randomly fluctuating force, $R(t)$, which arises from collisions with surrounding particles\citep{tsl2006}.
}

Brief explanation of what is Brwonian motion, what is trying to represent/model, how this can modeld the microgel.

Even though there are different mathematical representations that describes the brownian motion, we use the langevin equation\footnote{No se que más decir a parte de ``porque está implementado en lammps'', jsjs} described with increments of Wiener process,


In the processes of polymerization and depolymerization, polymers exhibit Brownian nonGaussian kinetic characteristics [9]\citep{wangMultiscaleModelingSimulation2025}.


To elucidate the essence of these natural phenomena and uncover the underlying physical mechanisms, scientists employ statistical and mathematical tools to quantify the dynamical behaviors.
Methods for quantitative modeling across multiple scales are primarily categorized into two types: one focused on simulating dynamics at the microscale, and the other dedicated to deriving or establishing evolutionary equations at the macroscale.
There are two commonly used modeling frameworks, the continuous time random walk (CTRW) model and the Langevin equation\footnote{Reference [34] presents the Langevin equation for polymers engaged in polymerization/depolymerization reactions, by establishing a random diffusion coefficient that correlates with particle size. Reference [35] presents the Langevin equations for continuous time Lévy walks. X. D. Wang et al. present the Langevin description of the Lévy walk [36]. Y. Chen et al. then provide the Langevin description of the Lévy walk with memory [37] and examine the impact of an external force [38,39]\citep{wangMultiscaleModelingSimulation2025}.

}\citep{wangMultiscaleModelingSimulation2025}.

It differs from molecular dynamics simulations (based on Newton's equation) in that the solvent is modelled by stochastic and dissipative forces. 
This approximation allows substantially longer simulations than would be possible if the solvent were explicitly included, and Langevin dynamics is an excellent alternative to molecular dynamics for certain systems\citep{pastorTechniquesApplicationsLangevin1994}.



The interesting features of the system are highlighted and the less relevant are treated in an approximate way.
Examples: Eliminate the molecular degreees of freedom by considering as rigid bodies.
Also the Langevin dynamics technique.

the stochastic theory used by Langevin to describe the brownian motion of a large and massive particle in a bath of particles that are much smaller and lighter than itself.\citep{tsl2006}
Consider a solution containing polymers or ions which are much heavier than the solvent molecules\citep{Thijssen2007}. 
The problem is characterised by two very different timescales, one associated with the slow relaxation of the initial velocity of the brownian particle and another linked to the frequent collisions that the brownian particle suffers with particles of the bath\citep{tsl2006}. 
As the kinetic energy is on average divided equally over the degrees of freedom, the ions or polymers will move much more slowly than the solvent molecules\citep{Thijssen2007}.
Moreover, because of their large mass, they will change their momenta only after many collisions with the solvent molecules and the picture which emerges is that of the heavy particles forming a system with a much longer time scale than the solvent molecules\citep{Thijssen2007}.
Langevin assumed that the force acting on the brownian particle consists of two parts: a systematic, frictional force proportional to the velocity $u(t)$, but acting in the opposite sense, and a randomly fluctuating force, $R(t)$, which arises from collisions with surrounding particles\citep{tsl2006}.

One should notice that the frictional term depends on the previous history of the trajectory (Markovian process). 
The friction term is important in obtaining realistic simulations, as it takes into account the viscosity of the solvent (a feature, which is absent from both MD and MC). 
If one assumes that the frictional term is constant (no history), one obtains the celebrated Langevin equation (LE):\citep{paquetMolecularDynamicsMonte2015}



How can we model the effect of the solvent particles without taking into account their degrees of freedom explicitly? 
When a heavy particle is moving through the solvent, it will encounter more solvent particles at the front than at the back. 
Therefore, the collisions with the solvent particles will on average have the effect of a friction force proportional and opposite to the velocity of the heavy particle.\citep{Thijssen2007}
\begin{gather}
    m\dv{\vec{v}(t)}{t}=-m\gamma \vec{v}(t)+\vec{F}(t)+\vec{R}(t).\label{eqn:BrownianDyn1}
\end{gather}

where $R(t)$ is a ``random force''.
Again, the time correlations present in the fluid should show up in this force, but they are neglected once more and the force is subject to the following conditions 
- Average $0$
- No time correlation
- Gaussian distribution
include the rapid variations in the force due to the frequent collisions with solvent particles on top of the coarse-grained friction force.

the correction is formed of two terms: a friction term, which introduces an artificial viscosity, and a stochastic term, which takes into account the unknown nature of the correction\citep{paquetMolecularDynamicsMonte2015}
The Langevin equation improves conformational sampling over standard molecular dynamics\citep{paquetMolecularDynamicsMonte2015}.

The intra-solute [3,4] terms of the force field used for molecular dynamics generally can be carried over for a Langevin dynamics simulation; the critical new parameter is the friction constant, which parametrises the effect of solvent damping and activation\footnote{The generalised Langevin equation can be simplified by assuming that (i) the friction kernel is delta correlated in time, and (ii) it is not a function of the position of any of the particles:

Assumption (i) is equivalent to assuming that the viscoelastic relaxation of the solvent is very rapid with respect to solute motions, and, in fact, $\xi$ is just the zero frequency value of the friction kernel.

Grote land Hynes [26] have investigated this assumption for motions involving barrier crossing and have found that while it is seriously in error for passage over sharp barriers (such as 12 recombination); it is quite adequate for conformational transitions such as might be found in polymer motions.

Assumption (ii) involves the neglect of hydrodynamic interaction or spatial correlation in the friction kernel\citep{pastorTechniquesApplicationsLangevin1994}.

}\citep{pastorTechniquesApplicationsLangevin1994}.

($\gamma$ is commonly referred to as the collision frequency in the simulation literature, even though formally a Langevin description implies that the solute suffers an infInite number of collisions with infInitesimally small momentum transfer. In comparing the results of Langevin dynamics with those of other stochastic methods [28-31], the relevant variable is the velocity relaxation time, tv> which equals y.1.\citep{pastorTechniquesApplicationsLangevin1994})



This difference in time scale can be employed to eliminate the details of the degrees of freedom of the solvent particles and represent their effect by forces that can be treated in a simple way.


\end{document}
