\documentclass[../../main-notes.tex]{subfiles}

\begin{document}

\section{Introduction}

\subsection{Molecular Dynamics}\footnote{\begin{itemize}\item Lagevin thermostat \item Brownian Dynamics\end{itemize}}

\subsection{Virial Stress and Cauchy stress}


The virial stress developed on the virial theorem of Clausis 1870 and Maxwell 1870 is 
\begin{gather}
    \sigma_{ij}^V = \frac{1}{V}\sum_\alpha\qty[\frac{1}{2}\sum_{\beta=1}^N\qty(R_i^\beta-R^\alpha_i)F^{\alpha\beta}_j-m^\alpha v_i^\alpha v_j^\alpha]\label{eqn-mdStress},
\end{gather}
where $(i,j)$ represents the directions $x$, $y$ and $z$.
$\beta$ goes from $1$ to $N$ representing the neighbors of the particle with index $\alpha$.
Therefore, $R^\alpha_i$ is the position of the particle $alpha$ along the direction $i$, meanwhile $F^{\alpha\beta}_j$ is the force on particle $\alpha$ due to the interaction with particle $\beta$ in the $j$ direction.
Finally, $V$ is the total vovlume of the system, $m^\alpha$ is the mass of the particle $\alpha$ and $v^\alpha_i$ is the velocity of the particle $\alpha$ in direction $i$.
It is important to acknowledge that the force $F^{\alpha\beta}_j$ is uniquely defined only for pair potentials and EAM type potentials.\footnote{So\ldots I need to check\citep{Swenson_1983} and\citep{Tsai_1979} to understand how we get that expression from the virial theorem. Also, I don't know what is the virial theorem  }


The virial stress calculated from molecular dynamics (MD) simulations has to be averaged voer time in order for it to be equivalent to the continuum Cauchy stress\citep{Subramaniyan_Sun_2008}.

Viriral stress is indeed an atomistic definition for stress that is equivalent to the continuum Cauchy stress.


Molecular dynamics simulations are typically performed in the Eulerian reference frame\footnote{I don't know what is the defference between the Langrangian framework and the Eulerian reference frame.} and hence will need to have velocity included in the stress definition.

\subsection{Pressure and stress relation}

Pressure and stress are familiar physicla notions. 
Both refer to the force per unit area which two bodies in contact, or two parts of a single body separated by an imaginary plane, exert on one another.
Both tensorial quantities\citep{Tsai_1979}.
Under hydrostatic conditions, the relationship between external pressure and internal stress is particurlarly simple:
\begin{gather}
    P = \frac{1}{3}\qty(\simga_{xx}+\sigma_{yy}+\sigma_{zz}),
\end{gather}
where $\simga_{xx}=\sigma_{yy}=\sigma_{zz}$ and $\sigma_{xy}=\sigma_{yz}=\sigma_{zx}=0$, that is, at equilibirum, the external pressure $P$ is equal to the internal normal stress components and throught the system, the shear components being zero.
Under these conditions, the external pressure may be calculated from the virial theorem:
\begin{gather}
    PV = NkT-\frac{1}{3}\right\langle\left\sum_{i,j<1}^N\vec{r}_{ij}\cdot\pdv{\Phi_{ij}}{r_{ij}}\right\rangle,
\end{gather}
where $V$ is the volume, $N$ is the number of particles, $T$ is the tmeperature of the system, $k$ is the Boltzmann's constante, $r_{ij}$  is the vector joining particles $i$ and $j$ and $\Phi_{ij}$ is the interatomic potential between $i$ and $j$.
The angular brackets denote average over time\footnote{Is the same expression for the scalar pressure used by the compute pressure in lammps: \href{https://docs.lammps.org/compute_pressure.html}{documentation page}.}.

The instantaneous internal stress ar a point is made up of two parts:
\begin{itemize}
    \item The sum of the interatomic forces intercepted by a small area containing the point, averaged over the area.
    \item The momentum flux through this area during a time interval $\Delta t$
\end{itemize}
If an atom moves across the area, carrying momentum $\Delta mv$, then the area also ``feels'' a force equal to the momentum flux $\Delta mv/\Delta t$, and the force also contributes to the stress over the area in the interval $\Delta t$.
The normal component of the sum of the forces gives the normal stress, and the in-plane component gives the tangential stress.
The area may be either stationary or moving at a uniform velocity.
It may also be at the boundary of the system.

The time averages of the instantaneous stress components then ive what may be called the ``measured'' stresses at the point.\citep{Tsai_1979}.
\ldots
This formulation is not new: Cauchy discussed the stress-strain relationship in a crystalline material from the viewpoint of ``region of molecular activity'' as early as 1828.
\ldots
The stress method applies equally to a system not in thermal equilibrium, because the temperature term does not appear explicitly in this formulation.
\ldots
the method of stress calculation may be applied locally, without requiring the system to be in equilibrium or even spatially homogeneous.
\ldots
it should be possible to use this method to obtain the stress distribution in a solid with a crack in it, whereas the virial method would be inapplicable in this case.

They show that the pressure calculated by the vrial method is actually the normal stress in the boundary planes.
The stress method, on the other hand, can be used to calculated the stress not only in the boundary planes, but also in the interior planes.

The virial is defined as 
\begin{gather}
    \Upsilon
\end{gather}

\subsection{Lammps implementation}

\paragraph{Langevin Thermostat}

\paragraph{compute stress/atom and pressure}\footnote{Explain the scalar pressure, pressure tensor and stress tensor. Explain the relation between pressure and stress of the system.}

Virial contribution to the stress and pressure tensors\citep{Thompson_Plimpton_Mattson_2009}.
They find three ways of computing the virial contribution,
\begin{align}
    W(\vec{r}^N) &= \sum_{k\in\vec{0}}\sum_{w=1}^{N_k} \vec{r}^k_{w}\cdot\vec{F}^k_{w} \\
    W(\vec{r}^N) &= \sum_{n\in Z^3}\sum_{i=1}^N \vec{r}_{i\vec{n}}\cdot\left(
    -\sum_{k\in\vec{0}}\dv{\vec{r}_{i\vec{n}}}u_k\qty(\vec{r}^{N_k})\right) \\
        W(\vec{r}^N) &= \sum_{n\in Z^3}\sum_{i=1}^N \vec{r}_{i}\cdot\vec{F}_i 
    +\sum_{\vec{n}\in Z^3}\vec{\mathrm{Hn}}\cdot\sum_{i=1}^N\left(
    -\sum_{k\in\vec{0}}\dv{\vec{r}_{i\vec{n}}}u_k\qty(\vec{r}^{N_k})\right)
\end{align}
\footnote{Skimming the equations \eqref{eqn-mdStress} and that one, the virial term are similars. Need to check if they are equivalents. }

\end{document}
