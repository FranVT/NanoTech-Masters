\documentclass[main.tex]{subfiles}

\begin{document}




\begin{comment}
\subsection{Processing the data}

\begin{itemize}
    \item Temperature
    \item Energy
    \item Pressure
    \item Stress
\end{itemize}

\subsection{Pressure and Stress}

Differences between the compute pressure and compute stress/atom.

\subsection{Pressure}

The \textbf{scalar} pressure is computed as follows,
\begin{gather*}
    P=\frac{Nk_{B}T}{V}+\frac{1}{Vd}\sum_{i=1}^N\vec{r}_i\cdot\vec{f}_i.
\end{gather*}

The pressure \textbf{tensor} is computed as follows



\subsection{Atom stress}

\subsection{Relation between pressure and stress}

\begin{gather*}
    p=-\frac{1}{dV}\sum_{i=1}^{N}\mathrm{Tr}\qty(\sigma_i)
\end{gather*}
where $d$ is the dimension of the simulation, $d=3$, $V$ is the volume of the simulation box.
$\sigma_i$ is the stress per atom calculation.

From the results we do not get the exact behaviour between the stress and pressure, but the difference is constant and small, indicating that are equivalent.
\end{comment}




\end{document}
