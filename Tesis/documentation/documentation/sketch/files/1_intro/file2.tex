\documentclass[../../main-notes.tex]{subfiles}

\begin{document}

\subsection{Description of the micro-gel}

\begin{itemize}
    \item What is a microgel
    \item Connection with colloids 
    \item Link to brownian dynamics
\end{itemize}


\citep{priyaComprehensiveReviewHydrogel2024}

Hydrogels are three-dimensional networks of hydrophilic polymers that can absorb and retain large amounts of water while maintaining their structure\footnote{Their ability to retain a large amount of water is due to their 3D structure, which gives them a gel-like appearance and behaviour.}. 
The water absorption capacity and network stability of hydrogels are controlled by crosslinking, which involves forming covalent or non-covalent bonds between polymer chains\footnote{Agregar lo siguiente: The hydrogels are prepared using different methods like chemical cross-linking of monomers, physical cross-linking using temperature or pH changes, and blending of natural or synthetic polymers.}. 
Crosslinkers play a crucial role in providing secondary interactions with biological tissues, and the presence of hydrophilic groups in the polymer chains enhances water uptake [10]. 
These methods allow researchers to create hydrogels with specific properties suitable for various applications such as tissue engineering, biomedicine, and sensing\footnote{}. 
The properties of hydrogels can be tailored based on the nature and arrangement of their constituent monomers, as well as the preparation method employed.


Hydrogels possess several distinctive characteristics, making them highly valuable for various biomedical applications. 
These characteristics include biodegradability, biocompatibility, hydrophilicity, super absorbency, viscoelasticity, softness, and fluffiness. 
Furthermore, hydrogels are responsive to various stimuli, which adds versatility to their applications. 
These stimuli can include temperature, electric field, magnetic field, biological molecules, and ionic strength [11,12]. 
This resemblance to the extracellular matrix, along with their biocompatibility and biodegradability, makes them ideal for creating scaffolds to support tissue regeneration and growth. 
However, traditional hydrogels struggle to combine electrical conductivity, strong adhesion, and mechanical performance.
Designing hydrogels that adhere well to various surfaces while maintaining these properties is a significant challenge.


The ability of hydrogels to create functional replacement tissues and organs holds immense promise for treating a wide range of injuries and diseases [1–5]. 
This includes advances in biomaterials science, 3D printing, stem cell technology, and bio fabrication methods.
hydrogel materials are used to support the development of new cells and promote repair. 
Normally, metal-based materials, chitosan, cellulose, and hydrogel materials are used for biomedical applications.
They are used in dye and heavy metal adsorption, biosensors, and medical coatings. 
They are highly biocompatible and can mimic the properties of natural extracellular matrices. 
In tissue engineering, hydrogels serve as excellent biomaterials since they can mimic the natural water and polymer environment found in biological systems [9]. 

Hydrogels have gained popularity for their solubility, water retention, and wet environment compatibility. 
In general, hydrogel has large amounts of water, and it is one of the three-dimensional hydrophilic polymers [8]. 


Smart hydrogels are a subcategory that can respond to external stimuli like pH, temperature, specific molecules, solvents, or mechanical force. 
These stimuli-responsive properties allow hydrogels to act as sensors for various applications. 
Additionally, the transportation properties and injectability of hydrogels make them promising candidates for drug delivery systems [13–15]. 
They can encapsulate and release drugs in a controlled manner, which is crucial for targeted and sustained drug applications. 
This review discusses the preparation methods for hydrogels, their characterization properties, and their use in various applications.


Types of Hydrogels
\begin{enumerate}
    \item Natural Polymer: 
            Natural polymer-derived hydrogels, sourced from plants or animals like polysaccharides and proteins.
            These hydrogels are adept at absorbing and retaining water, effectively managing pesticide release in soil to boost efficacy and minimize environmental harm caused by excessive application.
            Despite challenges like mechanical strength variations inherent in natural sources, natural polymer-derived hydrogels hold great promise for sustainable agriculture and environmental conservation.
            Examples: Cellulose, derivatives such as crboxymethyl celluose, Chitosan, Sodium alginate
    \item Synthetic Polymer: 
            Synthetic polymer hydrogels, such as those made from polyacrylamide (PAM) and PVA.
            controllable structures, mechanical strength, and chemical stability.
            PAM is especially favoured for its water retention and non-toxic nature, making it prevalent in biomedicines and agriculture. 
            However, the use of PAM comes with concerns. 
            Acrylamide, used in PAM synthesis, is potentially neurotoxic and may release unreacted particles, posing environmental and health risks. 
            Additionally, these hydrogels have low biodegradability, causing environmental residues and potential contamination. 
            Production of these synthetic polymers often involves harmful chemicals, increasing costs and raising further health and environmental concerns
    \item Natural-Synthetic Polymer: 
            blending natural polymers like alginate and xanthan gum with synthetic counterparts such as PAM and PVA.
            These hydrogels enhance biodegradability and biocompatibility, mitigate long-term soil and water contamination risks, and provide robust mechanical strength and chemical stability.
\end{enumerate}


Preparation methods of Hydrogels\footnote{
    \begin{itemize}
        \item Physical Cross-Linking
        \item Chemical Cross-Linking
        \item Irradiation based cross linking
    \end{itemize}
}

Hydrogels are composed of polymer chains, and the properties of the hydrogel are influenced by the properties of the polymer used.

Polymer chains connect through cross-links to form a 3D network in hydrogels. 
Crosslinking affects various physical properties of the hydrogel, such as elasticity, viscosity, solubility, glass transition temperature, strength, toughness, and melting point


It has higher glass transition temperature due to limited rotational motion between the polymer chains. 
Cross-linking increases the molecular weight of the polymer chains, which, in turn, limits their translational movement and decreases the solubility of the polymer.
These kinds of polymers are insoluble but can absorb a large amount of solvent, resulting in the formation of gels.
The amount of solvent absorbed by the hydrogel depends on the cross-linking density, which increases as the interactions between the polymer chains and solvent molecules decrease.



Ionizing radiations, such as X-rays, gamma rays, accelerated electrons, ion beams, and high-energy ultraviolet rays, can be used for polymerization reactions and crosslinking of polymers. 
These radiations have sufficient energy to break chemical bonds and initiate polymerization or crosslinking processes. 
In the context of polymerization reactions, ionizing radiation is used to initiate the polymerization of monomers and create long polymer chains. 
This process is known as radiation polymerization or radiation curing. It is commonly used in industries like coatings, adhesives, and 3D printing, where rapid and efficient polymerization is desired. 

Crosslinking is another essential process that can be controlled and intentionally modified using ionizing radiations. 
Crosslinking involves the formation of chemical bonds between polymer chains, creating a three-dimensional network structure. 

This network enhances the mechanical properties, stability, and other desirable characteristics of the polymer material. 
Controlling the crosslinking process with ionizing radiations can be achieved by maniipulating various parameters during exposure, such as exposure time of radiaiton, frquency, temperature and pressure.

Hydrogel Fromation Materials

Materials used for hydrogel formation include polyethylene oxide, PVA, poly(acrylic acid) (PAA), poly(propylene furmarate-co-ethylene glycol) (P(PF-co-EG)), and polypeptides. 
Some kind of materials are naturally derived polymers, including agarose, alginate, chitosan, collagen, fibrin, gelatin, and hyaluronic acid. 
Table 1 shows hydrogel preparation methods and doping elements.

Physical Cross-Linking
In physically cross-linked hydrogels, the interactions between polymer chains are not covalent but rather based on physical interactions. 
These interactions can include hydrogen bonding, van der Waals forces, hydrophobic interactions, or coordination bonds. 
Unlike chemical cross-linking, physical cross-linking is reversible under certain conditions, which means that the hydrogel can undergo structural changes without breaking any covalent bonds. 
This characteristic makes physically cross-linked hydrogels more responsive to external stimuli like temperature, pH, or ionic strength. 
They may exhibit unique properties, such as “self-healing” behaviour, where the gel can reform after being broken. 

The hydrogels prepared by these interactions are uniquely physical gels and have high water sensitivity and thermal reversibility. 
These kinds of hydrogel have a short lifespan, in the range of a few days to a maximum of a month, in the physiological media. 
Therefore, in this form, hydrogels are used when a short-term drug release is required. 

These kinds of hydrogels are safe to use for clinical purposes because the gelation does not require any toxic covalent crosslinking molecules. 
This method is used to prepare hydrogels through non-covalent approaches, such as electrostatic, hydrogen bonding, and hydrophobic forces among polymer chains. 
The physical methods used to prepare hydrogel have certain advantages, such as high-water sensitivity and thermal reversibility [21]. 
When prepared using this method, hydrogels are very safe for clinical purposes as the gelation does not require any toxic covalent crosslinking molecules. 
Chitosan with small anionic molecules such as sulphates, phosphates, and citrates of Pt, Pd, and Mo can be used for hydrogel preparation using physical methods. 
The synthesized hydrogels depend on the charge and size of anions and the concentration of deacetylation of chitosan.


Chemical Cross-Linking

The formation of physical gels through clustering of molecules causes formation of free chain loops and thus inhomogeneity that signifies short lived network imperfections. 
This preparation of hydrogel networks is easy to control when compared to physical hydrogels as their preparation and the applications they are used for are not dependent on their pH. 

The chemical crosslinking method can be used to transform the physical properties of the hydrogels. 
In chemically cross-linked hydrogels, covalent bonds form between the polymer chains.
These covalent bonds are strong and stable, resulting in a 3D network of interconnected polymer chains. 

The cross-links are typically formed through chemical reactions, such as polymerization or cross-linking agent-induced reactions. 
The presence of covalent bonds makes the hydrogel structure more robust and resistant to changes in environmental conditions, such as temperature and pH. 
As a result, chemically cross-linked hydrogels generally exhibit greater mechanical strength and long-term stability. 

Chemically crosslinked hydrogels are easier to control as compared to physical hydrogels as their preparation method and applications are not dependent on their pH. 
Tan et al. [33] synthesized N-succinyl chitosan function alizedhyaluronic acid injectable composite hydrogels through a Schiff base mechanism. 
It was determined that the compressive modulus, which is an important factor for cartilage tissue engineering, improved with an increasing amount of N-succinyl chitosan in the hybrid hydrogel [33]. 
Ito et al. [34] also reported the same preparation method for hydrogel using cellulose and alginate. The amino group’s derivatives form hydrogels using Michael’s addition reactions. 
The amino groups react with the vinyl group of other polymers. 
In these kinds of preparations, hydrogels enhanced mucoadhesive properties. This method has some disadvantages, such as multistep preparation and purification method. Polymers might become cytotoxic after functionalization with the reactive groups.


Irradiation Based Cross Linking

Irradiation-based crosslinking is an attractive approach for hydrogel synthesis, especially in applications where rapid gelation and cost efficiency are critical factors. 
By harnessing light-sensitive functional groups and UV irradiation, researchers can achieve effective hydrogel formation in a short period, thus enabling various potential applications in biomedicine, tissue engineering, drug delivery, and other fields [13,14]. 
However, it is essential to consider specific requirements and potential limitations, such as the compatibility of the chosen light-sensitive moieties with the target application and the sensitivity of the hydrogel to environmental factors like light and temperature. 

The use of irradiation-based crosslinking with light-sensitive functional groups is a proposed method for the synthesis of hydrogels, and it offers several advantages over traditional chemical crosslinking methods [31]. 
It has some advantages such as 
    (i) Speedy preparation: The hydrogel formation can occur rapidly when using light-sensitive functional groups and UV irradiation. This allows for a faster production process compared to chemical crosslinking methods, which may involve longer reaction times; 
    (ii) Low cost of production: The use of light-sensitive functional groups and UV irradiation may reduce the need for expensive crosslinking agents or catalysts, potentially making the production process more cost-effective [32]. 

The proposed method by Ono et al. [35] for the synthesis of UV-light irradiated chitosan hydrogels is based on the use of light-sensitive moieties, specifically azide and lactose. 
Azide and lactose are introduced into the chitosan polymer. 
These moieties are selected because they are sensitive to UV light and can undergo specific reactions upon exposure. 
The chitosan polymer with the introduced azide and lactose moieties is exposed to UV light. 
UV light serves as a trigger to initiate a reaction involving the azide group. 
Upon UV irradiation, the azide group on the chitosan polymer is converted into a nitrene group. 
This conversion is likely a photochemical reaction, where the high-energy UV light breaks the N-N triple bond in the azide group, generating a reactive nitrene intermediate. 
The generated nitrene intermediate then reacts with the amino groups present in chitosan. 
The nitrene group binds covalently to the amino groups, resulting in the crosslinking of chitosan chains. 
This crosslinking process forms a three-dimensional network, leading to the formation of a hydrogel. 
The irradiation-based cross linking method reported by Yoo et al. [36] for the synthesis of UV-irradiated chitosan hydrogels involves pre-functionalization with photo-sensitive acrylates of chitosan and pluronic acid. 
However, this method has some drawbacks such as need for a light sensitizer, delayed irradiation, and increase in local temperature.


Characterization Methods

Swelling

The swelling of a hydrogel refers to the process of absorbing a significant amount of water, leading to an increase in its volume and weight. 
Has a significant impact on various characteristics, including the degree of cross-linking, mechanical properties, and rate of degradation.
measure the swelling of hydrogels, such as gel fraction study, swelling ratio measurements, and the weight loss method. 
These methods are relatively simple and widely used to assess the swelling behaviour of hydrogels.
By understanding their swelling properties, researchers can gain insights into the structure and performance of hydrogels
Additionally, the swelling behaviour can provide valuable information about the hydrogel’s response to changes in the surrounding environment, making it an important consideration in material design and engineering. The swelling of a hydrogel is determined using gel fraction



Microstructural Performance 

The microstructure of hydrogels plays a crucial role in determining their physical properties and behaviour. 
Several key parameters are used to characterize the microstructure of hydrogels such as volume fraction of the polymer in the swollen state, number average molecular weight between cross-links, mesh size of the network, and other environmental factors. 
The mesh size depends on the degree of cross-linking, chemical structure of the monomers, and the external environment, especially pH, temperature, and ionic strength. 
The mesh size calculates the physical properties of the hydrogel, including mechanical strength and the degradability and diffusivity of a releasing molecule [39]. 
The microstructure of hydrogels is vital for tailoring their properties to specific applications, such as drug delivery, tissue engineering, and biomaterial design. 
By controlling the degree of cross-linking and mesh size, researchers can customize the mechanical properties, degradation rate, and release kinetics of the hydrogel, making it a versatile and valuable material in various biomedical and engineering fields. 




Mechanical Properties

The mechanical properties were determined from normal techniques such as tensile testing, compression testing, indentation testing, bulge testing, cyclical testing, and the strip extension method. 
This test is demonstrated in the following ways. 
Tensile testing is a common technique used to determine the mechanical response of a material to an applied tensile force. 
In the case of hydrogels, this involves subjecting a strip of the hydrogel to a tensile force using specialized grips in an instrument. 
The resulting stress-strain data can be used to calculate important mechanical properties, such as Young’s modulus, yield strength, and ultimate tensile strength [43]. 
Compression testing involves placing a hydrogel sample between two plates and applying pressure to compress it. 
This helps determine the compression distance or how the hydrogel responds to compressive forces. 
By analysing the deformation and load data during compression, researchers can understand how the hydrogel behaves under compressive stresses. 
Indentation testing involves pressing a small, hard object into the surface of a hydrogel. 
The indentation depth and the force applied are measured to understand the hydrogel’s hardness and elastic modulus in the indentation region. 

The bulge test is a specialized test used to measure the mechanical properties of thin films or membranes, including hydrogels. 
In this test, a hydrogel sheet is constrained around its edges while being subjected to internal pressure. 
By monitoring the deformation, researchers can extract properties like the elastic modulus and Poisson’s ratio of the hydrogel. 
Cyclical testing, also known as fatigue testing, involves subjecting the hydrogel to repetitive loading and unloading cycles. 
This helps assess the material’s fatigue behaviour, which is essential in understanding its long-term mechanical durability under repeated stresses.






Morphology Study

Morphology studies in the context of hydrogel research are essential for understanding the structural characteristics and physical properties of the prepared hydrogels. 
FE-SEM is used to examine the surface morphology and topography of hydrogels. 
It provides highresolution images of the hydrogel’s surface, revealing its surface features, such as porosity, roughness, and the presence of any structural defects [46]. 

Environmental Scanning Electron Microscopy (ESEM) allows for imaging under controlled environmental conditions, including in the presence of water, making it suitable for studying hydrogels. 
ESEM can reveal the morphological changes that hydrogels undergo in a wet state, which is crucial for understanding their behaviour in physiological or aqueous environments. 
Transmission Electron Microscopy (TEM) is used to study the internal structure of hydrogels at the nanoscale.
It provides information about the particle size and shape of components within the hydrogel [47]. 
It can also reveal the distribution of nanoparticles or other inclusions, if present. For FE-SEM and ESEM, the hydrogel samples are typically dehydrated, coated with a conductive layer like gold or carbon, and imaged under high vacuum conditions or in a controlled environmental chamber. 
For TEM, ultrathin sections of the hydrogel or nanoparticles within it are prepared, stained or contrasted if necessary, and examined using high-vacuum TEM.




FTIR


The chemical structure and bonds are identified by FTIR. 
Those chemical bonds can be excited and absorb infrared light at frequencies that are typically based on structure and bonds.
Molecules have characteristic vibrational frequencies associated with their chemical bonds, and when these bonds are exposed to infrared radiation, they absorb energy at specific frequencies [49]. 
Each type of bond absorbs infrared light at distinct frequencies based on factors such as bond strength and the masses of the atoms involved. 
Analysing the absorption pattern of infrared light by a hydrogel helps identify the functional groups and chemical structures of the compounds present.







Antimicrobial Characteristics 

Self Healing Characteristics 

Self-healing materials, which can autonomously repair damage, are increasingly common in hydrogels. 
These mechanisms depend on dynamic covalent bonds (e.g., Diels–Alder, imine, disulfide) and non-covalent interactions (e.g., hydrogen bonds, hydrophobic interactions). 
Natural polysaccharides like alginate and chitosan, as well as synthetic polymers like PEG and PVA, are used to create biocompatible, strong, and flexible self-healing hydrogels. 
Combining Li alginate with poly(acrylamide-co-stearyl methacrylate) results in hydrogels with high fracture energy and fire resistance [55]. 
Rapid self-healing hydrogels utilize flexible polymer networks for quick repairs, and innovations include fluorochromic and Schiff base linkage-based hydrogels that self-repair at room temperature [56]. 
Two hydrogels of different colours (one red with rhodamine B) were cut and placed together. 
After 10 min, they autonomously adhered and healed without external force, demonstrating fast self-healing. 
Some dye diffusion blurred the interface, but the hydrogel could still stretch without cracking, indicating recovery of its 3D structure and mechanical strength. 
Rheological analysis showed rapid recovery of the hydrogels’ internal structure, with storage modulus G’ rising from 200 Pa to 2016 Pa, matching the original value. Under varying oscillatory forces, the hydrogels exhibited a thixotropic, elastic response, confirming their self-healing ability [57].




Delivery Characteristics 

Viscoelastic Properties

Rheology or compression testing is commonly used to evaluate the viscoelastic properties of hydrogels, and these properties are summarized in Table 1. 
The viscoelastic behaviour of hydrogels can be modulated by varying polymer and crosslinker concentrations, which influences their stiffness and elasticity [60]. 
For instance, higher crosslinker concentrations in glutaraldehyde-crosslinked gelatin hydrogels increase stiffness and shift the material toward a more elastic state. 
Additionally, modifying the viscosity of the aqueous phase with dextran in agarose and polyacrylamide hydrogels allows for control over viscoelastic properties while maintaining a stable elastic modulus, enhancing the design of hydrogels for biomedical applications. 
Viscoelasticity in hydrogels arises from several molecular mechanisms, particularly in physically or non-covalently crosslinked systems.
When stress is applied, crosslinkers can detach and allow the polymer matrix to flow, then reattach, as seen in weakly crosslinked collagen gels, alginate gels, and PEG hydrogels. 
Other factors, such as polymer entanglement and protein unfolding, also contribute to viscoelasticity by dissipating energy and enabling reversible elastic responses [61]. 
Even in well-crosslinked hydrogels, the significant water content leads to energy dissipation, resulting in a measurable loss modulus without inducing plasticity.


Applications 

\begin{itemize}
    \item Wound Healing
    \item Contact Lenses
    \item Tissue Engeneering
    \item 3D Bioprinting
    \item Biosensors
    \item Supercapacitor
    \item Catalysis
\end{itemize}

--------


\citep{ahmedHydrogelPreparationCharacterization2015a}

hydrogels are polymer networks extensively swollen with water.
Hydrophilic gels that are usually referred to as hydrogels are networks of polymer chains that are sometimes found as colloidal gels in which water is the dispersion medium [1].

Researchers, over the years, have defined hydrogels in many different ways. 
The most common of these is that hydrogel is a water-swollen, and cross-linked polymeric network produced by the simple reaction of one or more monomers. 
Another definition is that it is a polymeric material that exhibits the ability to swell and retain a significant fraction of water within its structure, but will not dissolve in water. 
Hydrogels have received considerable attention in the past 50 years, due to their exceptional promise in wide range of applications [2–4]. 
They possess also a degree of flexibility very similar to natural tissue due to their large water content.

The ability of hydrogels to absorb water arises from hydrophilic functional groups attached to the polymeric backbone, while their resistance to dissolution arises from cross-links between network chains. 
Many materials, both naturally occurring and synthetic, fit the definition of hydrogels

natural Hydrogels were gradually replaced by synthetic hydrogels which has long service life, high capacity of water absorption, and high gel strength. Fortunately, synthetic polymers usually have well-defined structures that can be modified to yield tailor able degradability and functionality. Hydrogels can be synthesized from purely synthetic components.

Recently, hydrogels have been defined as two- or multicomponent systems consisting of a three-dimensional network of polymer chains and water that fills the space between macromolecules. 
Depending on the properties of the polymer (polymers) used, as well as on the nature and density of the network joints, such structures in an equilibrium can contain various amounts of water; 
typically in the swollen state, the mass fraction of water in a hydrogel is much higher than the mass fraction of polymer. 
In practice, to achieve high degrees of swelling, it is common to use synthetic polymers that are water-soluble when in non-cross-linked form.

Hydrogels may be synthesized in a number of ‘‘classical’’ chemical ways. 
These include one-step procedures like polymerization and parallel cross-linking of multifunctional monomers, as well as multiple step procedures involving synthesis of polymer molecules having reactive groups and their subsequent cross-linking, possibly also by reacting polymers with suitable cross-linking agents. 
The polymer engineer can design and synthesize polymer networks with molecular-scale control over structure such as cross-linking density and with tailored properties, such as biodegradation, mechanical strength, and chemical and biological response to stimuli.


Classification of hydrogel products 

\begin{enumerate}
    \item By source 
    \begin{enumerate}
        \item Natural
        \item Synthetic
    \end{enumerate}
    \item Polymeric composition
    \begin{enumerate}
        \item Homopolymeric hydrogels 
        \item Copolymeric hydrogels 
        \item Multipolymer Interpenetrating polymeric hydrogel 
    \end{enumerate}
    \item Configuration
    \begin{enumerate}
        \item Amorphous
        \item Semicrystaline 
        \item  Crystalline
    \end{enumerate}
    \item type of cross-linking 
    \begin{enumerate}
        \item Chemical
        \item Physical
    \end{enumerate}
    \item Physical appearance
    \begin{enumerate}
        \item matrix
        \item film
        \item microspheres
    \end{enumerate}
    \item network electrica charge
    \begin{enumerate}
        \item Nonionic
        \item Ionic
        \item Amphoteric electrolyte
        \item Zwitterionic
    \end{enumerate}
\end{enumerate}


They may perform dramatic volume transition in response to a variety of physical and chemical stimuli, where the physical stimuli include temperature, electric or magnetic field, light, pressure, and sound, while the chemical stimuli include pH, solvent composition, ionic strength, and molecular species
The extent of swelling or de-swelling in response to the changes in the external environment of the hydrogel could be so drastic that the phenomenon is referred to as volume collapse or phase transition [12]. Synthetic hydrogels have been a field of extensive research for the past four decades, and it still remains a very active area of research today.


Technologies adopted in hydrogel preparation

By definition, hydrogels are polymer networks having hydrophilic properties. 
While hydrogels are generally prepared based on hydrophilic monomers, hydrophobic monomers are sometimes used in hydrogel preparation to regulate the properties for specific applications

In general, hydrogels can be prepared from either synthetic polymers or natural polymers. 
The synthetic polymers are hydrophobic in nature and chemically stronger compared to natural polymers. 
Their mechanical strength results in slow degradation rate, but on the other hand, mechanical strength provides the durability as well. 
These two opposite properties should be balanced through optimal design [35]. 
Also, it can be applied to preparation of hydrogels based on natural polymers provided that these polymers have suitable functional groups or have been functionalized with radically polymerizable groups [36]. 
In the most succinct sense, a hydrogel is simply a hydrophilic polymeric network cross-linked in some fashion to produce an elastic structure. 
Thus, any technique which can be used to create a cross-linked polymer can be used to produce a hydrogel. 
Copolymerization/cross-linking free-radical polymerizations are commonly used to produce hydrogels by reacting hydrophilic monomers with multifunctional cross-linkers. 
Water-soluble linear polymers of both natural and synthetic origin are cross-linked to form hydrogels in a number of ways:
1. Linking polymer chains via chemical reaction. 
2. Using ionizing radiation to generate main-chain free radicals which can recombine as cross-link junctions. 
3. Physical interactions such as entanglements, electrostatics, and crystallite formation.

Any of the various polymerization techniques can be used to form gels, including bulk, solution, and suspension polymerization.

In general, the three integral parts of the hydrogels preparation are monomer, initiator, and cross-linker. 
To control the heat of polymerization and the final hydrogels properties, diluents can be used, such as water or other aqueous solutions. 
Then, the hydrogel mass needs to be washed to remove impurities left from the preparation process. 
These include nonreacted monomer, initiators, cross-linkers, and unwanted products produced via side reactions

Preparation of hydrogel based on acrylamide, acrylic acid, and its salts by inverse-suspension polymerization [37] and diluted solution polymerization have been investigated elsewhere. 
Fewer studies have been done on highly concentrated solution polymerization of acrylic monomers, which are mostly patented [38]. 
Chen [39] produced acrylic acid-sodium acrylate superabsorbent through concentrated (43.6 wt%) solution polymerization using potassium persulphate as a thermal initiator. 

Hydrogels are usually prepared from polar monomers. 
According to their starting materials, they can be divided into natural polymer hydrogels, synthetic polymer hydrogels, and combinations of the two classes. 
From a preparative point of view, they can be obtained by graft polymerization, cross-linking polymerization, networks formation of water-soluble polymer, and radiation cross-linking, etc. 
There are many types of hydrogels; mostly, they are lightly cross-linked copolymers of acrylate and acrylic acid, and grafted starch-acrylic acid polymers prepared by inversesuspension, emulsion polymerization, and solution polymerization. 


Polymerization techniques

\begin{itemize}
    \item Bulk polymerization
    \item Solution polymerization/cross-linking
    \item Suspension polymerization or inverse-suspension polymerization 
    \item Polimerization by irradiation 
\end{itemize}


Hydrogel technical features  

The functional features of an ideal hydrogel material can be listed as follows [48]:  
The highest absorption capacity (maximum equilibrium swelling) in saline. 
Desired rate of absorption (preferred particle size and porosity) depending on the application requirement. 
The highest absorbency under load (AUL). 
The lowest soluble content and residual monomer. 
The lowest price. 
The highest durability and stability in the swelling environment and during the storage. 
The highest biodegradability without formation of toxic species following the degradation. pH-neutrality after swelling in water. 
Colorlessness, odorlessness, and absolute non-toxic. 
Photo stability. 
Re-wetting capability (if required) the hydrogel has to be able to give back the imbibed solution or to maintain it; depending on the application requirement (e.g., in agricultural or hygienic applications).  

Obviously, it is impossible that a hydrogel sample would simultaneously fulfill all the above mentioned required features. 
In fact, the synthetic components for achieving the maximum level of some of these features will lead to inefficiency of the rest. 
Therefore, in practice, the production reaction variables must be optimized such that an appropriate balance between the properties is achieved. 
For example, a hygienic products of hydrogels must possess the highest absorption rate, the lowest re-wetting, and the lowest residual monomer, and the hydrogels used in drug delivery must be porous and response to either pH or temperature.


The rest of the article is about how to sythentize them.

--------

\citep{hartMaterialPropertiesApplications2021}

A key feature that controls the properties of a polymeric material is its architecture. 
Beyond the conventional linear polymer, architectures such as branched, cyclic, bottlebrush, star and block copolymers have expanded the property profile of polymeric materials and offered opportunities for polymer research and applications. 
Recently, the polymer field has seen the emergence of a new class of polymer architecture: mechanically interlocked polymers (MIPs), which are polymers that  include a mechanical bond.
The mechanical bond itself is not a new idea, occurring when two (or more) molecular components are constrained in space without being covalently bonded together (Fig. 1). 
Mechanically interlocked molecules (MIMs) possess large conformational freedom while maintaining a permanent spatial association between the components1–4. 
MIMs have played an important role in the field of molecular switches and molecular machines, and were recognized in 2016 with the Nobel Prize in Chemistry being awarded to Jean-Pierre Sauvage, Sir J. Fraser Stoddart and Bernard L. Feringa5–8. 
However, MIMs are not confined to this field, and have been explored in applications that range from drug delivery to catalysis9–15.

MIPs present an attractive frontier in polymer science, as the presence of the mechanical bond allows for unprecedented degrees of motion within the polymer architecture. 
Conceptually, there are a myriad of ways the mechanical bond can be incorporated into polymer architectures (Fig. 1), and these unique and varied structures can enable property profiles that have never been seen before.


The most common MIP is based on the rotaxane architecture: a ring (macrocycle) threaded onto a dumbbell-like component (Fig. 1a), where the ring is able to slide back and forth along the dumbbell component but is prevented from dethreading by the presence of bulky stoppers. 
A polymeric analogy to this would be the main-chain polyrotaxane (Fig. 1b), in which the ring(s) are trapped on the polymer backbone and can undergo a similar sliding motion over a much longer distance. 
Thus, by expanding or limiting the range of the rings’ sliding motion (for example, by controlling the length of the polymer backbone or the number of rings on it), the properties of the polyrotaxane can be tuned. 

Another important class of MIPs is slide-ring materials (SRMs), which are polymer networks where the crosslink itself is a rotaxane moiety. 
The term slide-ring gels, SRGs, is also commonly used to describe these materials, particularly if the network is swollen with a solvent. 
The most common of these systems contain figure-of-eight crosslinks (Fig. 1c) and the resulting MIP networks have mobile junctions that can slide freely along the polymer backbone. 

The final class of polyrotaxanes to be discussed are ‘daisy-chain’ architectures, which feature interlocked monomeric units composed of two of the key polyrotaxane elements (the ring and the thread) covalently bound together; the simplest form of this structure is the dimeric cyclic [c2]daisy chain (Fig. 1d).

The other main class of MIPs is mechanically interlocked rings, or catenanes (such as a [2]catenane, Fig. 1e). 
This chain-link structure can be incorporated into a polymer in many ways: as monomeric subunits (poly[2]catenane), pendant moieties or even composing the entirety of the polymer (polymeric [2]catenane, Olympic gel, poly[n]catenane (Fig. 1e)). 
Mechanically bonded rings do not allow for the long-range translational motion present in the polyrotaxanes; however, the components of a catenane possess several unusual degrees of freedom, such as elongation, twisting and rotating motions (Fig. 1e). 
Although MIP synthesis remains a challenge, developments have been made over the years that have allowed access to a range of MIPs16,17. 

In this Review, we will focus on polyrotaxanes, SRMs and polycatenanes to examine the current understanding of how the mechanical bond impacts the properties of a polymeric material and the potential applications of such materials. 
Although syntheses will be touched upon, it will not be the focus of this Review and readers interested in synthetic approaches to MIPs are directed elsewhere18–21. 
Other classes of interlocked polymers, such as pseudopolyrotaxanes or knot-based materials10,20, will not be discussed herein.

--------
Connecting Elasticity and Effective Interactions of Neutral Microgels: The Validity of the Hertzian Model\citep{rovigattiConnectingElasticityEffective2019}

%Colloidal suspensions have been used for decades as model systems for investigating fundamental condensed matter  phenomena.1−6 Compared to atomic and molecular systems, colloids have much larger characteristic time- and lengthscales, which makes them more accessible from an experimental point of view. In this context, the most iconic (and probably studied) soft matter system is certainly hard  spheres.1,7


--------

Stress localization, stiffening, and yielding in a model colloidal gel\citep{colomboStressLocalizationStiffening2014}

Colloidal gels, which can form in suspensions of colloidal particles in the presence of attractive effective interactions, are particularly appealing as materials whose functions can be in principle designed at the level of the nanoscale (particle) components

In colloidal suspensions, gels can form even in extremely dilute systems via aggregation of the particles into a rich variety of network structures that can be suitably tuned by changing the solid volume fraction, the physico-chemical environment, or the processing conditions. Hence, these handles could be used to design a specific complex mechanical response in addition to the selected nano-particle properties

Colloidal gels are typically very soft, but the variety of microstructures may lead to an equal variety in the mechanics

The microstructural complexity may also enable adjustments of the mechanical response to the external deformation. Soft gels can be in principle made to yield relatively easily, but in certain cases, a significant strain hardening has been observed before yielding finally occurs

The yielding of colloidal gels, due to breaking and reorganization of the network structure, can be accompanied by strong inhomogeneity of stresses and strains throughout the material\footnote{Pos hay otros fenómenos que llevan a yielding}

----------

Polymer Networks: From Plastics and Gels to Porous Frameworks\citep{guPolymerNetworksPlastics2020}


When bifunctional molecules are linked together, linear macromolecules, or “linear polymers,” with high molecular weights can form. 
Analogously, when molecules with functionality greater than two are linked together, three-dimensional (3D) macromolecules, or “polymer networks,” with very high (classically referred to as “infinite”[1,2]) molecular weights can form. 
Early organic chemists referred to polymerization, the process used to form linear and 3D polymers, as a “chemical combination involving the operation of primary valence forces,” further stating that “the term polymer should not be used (as it frequently is by physical and inorganic chemists) to name loose or vaguely defined molecular aggregates.”[1] 
Similarly, Wallace Carothers defined polymerization as, “any chemical combination of a number of similar molecules to form a single molecule.”[1] 
These notions either implicitly or explicitly defined polymers as being composed of strongly (covalently) bonded constituents. 
Today, however, it is widely accepted that linear polymers and polymer networks can be constructed from covalent and/ or non-covalent bonds; indeed, the full spectrum of bonding interactions, reaction mechanisms, and chemical compositions (e.g., organic, inorganic, biological) can be leveraged to design fascinating new polymer networks with exceptional properties.

From a structural perspective, polymer networks are composed of network “junctions” (in some cases, these can also be referred to as “crosslinks”, defined as a bond that links one strand to another), which have three or more groups (the exact “branch functionality” we refer to as f) emanating from a core, connected together by f “strands.” 
Strands can be linear polymer chains, flexible short molecules, rigid struts/ linkers, etc. 
As noted above, junctions and strands in polymer networks can be linked together via physical interactions (e.g., van der Waals interactions, hydrophobic interactions, Coulombic interactions, metalligand coordination) or covalent bonds. 
Hence, polymer networks are conventionally classified as “physical” (supramolecular) or “chemical” (covalent) networks. 
It should be noted that this classification does not always accurately reflect material properties; bond strengths and exchange rates are much more informative. 
For example, given sufficiently strong and static physical interactions, physical networks can behave identically to chemical networks; alternatively, the incorporation of mechanisms for covalent bond exchange can result in chemical networks that exhibit adaptable mechanical properties regulated by external stimuli. 
Thus, the properties of polymer networks can vary widely depending on the composition of the junctions and strands as well as the formation and use conditions. 
With this broad view in mind, nearly all polymer networks, regardless of their colloquial name, structure, properties, etc. can generally be divided into one of four major classes: thermosets, thermoplastics, elastomers, and gels.

Gels are polymer networks constructed from either covalent or supramolecular bonds that are swollen in liquid media such as water or organic solvents. 
The network structure ensures that the liquid is held within the material. 
Gels are usually very soft (Young s moduli of 103–104 Pa) but are often capable of undergoing relatively large deformation. 
Examples of gels include gelatin, fibrin, and polyacrylamide hydrogel.

As perhaps the most important, useful, and broadly studied class of materials from theoretical, academic, and industrial perspectives, polymer networks can have many unique properties, including elasticity, tunable mechanical strength, porosity, and swellability. 
These properties and others have led to numerous applications of polymer networks in everyday life, such as adhesives, cosmetics, sorbents, membranes, rubber products, coating materials, and food packaging. 
Moreover, as recent developments have imparted unconventional properties (e.g., malleability, self-healing, conductivity, ultra-high permanent porosity, enhanced crystallinity, and stimuli-responsiveness) into polymer networks, they continue to hold great promise in advanced applications including drug delivery systems,[3] tissue engineering scaffolds,[4] soft actuators,[5] gas storage,[6] catalysis,[6–8] and electronic materials.[9] 
Thus, though polymer networks have been widely studied for more than a century, there are features of their structure that have only recently been leveraged to impart new properties; an even deeper understanding is needed to realize the next-generation of functional, and ideally sustainable, polymer networks. 

In this review, we introduce key concepts related to the formation, characterization, and properties of polymer networks. 
Our goal is to provide newcomers to the field with broad and up-to-date knowledge that can serve as a starting point for more detailed investigations of topics of interest. 
Major focus is devoted to polymer network structure, which includes both chemical and topological aspects. 
Additionally, several types of recently developed polymer networks with exceptional properties are highlighted.



\end{document}
