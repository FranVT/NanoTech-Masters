\documentclass[../../main-notes.tex]{subfiles}

\begin{document}

\subsection{Description of the micro-gel}

\begin{itemize}
    \item What is a microgel
    \item Connection with colloids 
    \item Link to brownian dynamics
\end{itemize}


\citep{priyaComprehensiveReviewHydrogel2024}

\begin{comment}
Hydrogels possess several distinctive characteristics, making them highly valuable for various biomedical applications. 
These characteristics include biodegradability, biocompatibility, hydrophilicity, super absorbency, viscoelasticity, softness, and fluffiness. 
Furthermore, hydrogels are responsive to various stimuli, which adds versatility to their applications. 
These stimuli can include temperature, electric field, magnetic field, biological molecules, and ionic strength [11,12]. 
This resemblance to the extracellular matrix, along with their biocompatibility and biodegradability, makes them ideal for creating scaffolds to support tissue regeneration and growth. 
However, traditional hydrogels struggle to combine electrical conductivity, strong adhesion, and mechanical performance.
Designing hydrogels that adhere well to various surfaces while maintaining these properties is a significant challenge.

The ability of hydrogels to create functional replacement tissues and organs holds immense promise for treating a wide range of injuries and diseases [1–5]. 
This includes advances in biomaterials science, 3D printing, stem cell technology, and bio fabrication methods.
hydrogel materials are used to support the development of new cells and promote repair. 
Normally, metal-based materials, chitosan, cellulose, and hydrogel materials are used for biomedical applications.
They are used in dye and heavy metal adsorption, biosensors, and medical coatings. 
They are highly biocompatible and can mimic the properties of natural extracellular matrices. 
In tissue engineering, hydrogels serve as excellent biomaterials since they can mimic the natural water and polymer environment found in biological systems [9]. 

Hydrogels have gained popularity for their solubility, water retention, and wet environment compatibility. 
In general, hydrogel has large amounts of water, and it is one of the three-dimensional hydrophilic polymers [8]. 


Smart hydrogels are a subcategory that can respond to external stimuli like pH, temperature, specific molecules, solvents, or mechanical force. 
These stimuli-responsive properties allow hydrogels to act as sensors for various applications. 
Additionally, the transportation properties and injectability of hydrogels make them promising candidates for drug delivery systems [13–15]. 
They can encapsulate and release drugs in a controlled manner, which is crucial for targeted and sustained drug applications. 
This review discusses the preparation methods for hydrogels, their characterization properties, and their use in various applications.

Preparation methods of Hydrogels\footnote{
    \begin{itemize}
        \item Physical Cross-Linking
        \item Chemical Cross-Linking
        \item Irradiation based cross linking
    \end{itemize}
}

Hydrogels are composed of polymer chains, and the properties of the hydrogel are influenced by the properties of the polymer used.

Polymer chains connect through cross-links to form a 3D network in hydrogels. 

It has higher glass transition temperature due to limited rotational motion between the polymer chains. 

These kinds of polymers are insoluble but can absorb a large amount of solvent, resulting in the formation of gels.

\end{comment}


Ionizing radiations, such as X-rays, gamma rays, accelerated electrons, ion beams, and high-energy ultraviolet rays, can be used for polymerization reactions and crosslinking of polymers. 
These radiations have sufficient energy to break chemical bonds and initiate polymerization or crosslinking processes. 
In the context of polymerization reactions, ionizing radiation is used to initiate the polymerization of monomers and create long polymer chains. 
This process is known as radiation polymerization or radiation curing. 
It is commonly used in industries like coatings, adhesives, and 3D printing, where rapid and efficient polymerization is desired. 

This network enhances the mechanical properties, stability, and other desirable characteristics of the polymer material. 

Hydrogel Fromation Materials

Materials used for hydrogel formation include polyethylene oxide, PVA, poly(acrylic acid) (PAA), poly(propylene furmarate-co-ethylene glycol) (P(PF-co-EG)), and polypeptides. 
Some kind of materials are naturally derived polymers, including agarose, alginate, chitosan, collagen, fibrin, gelatin, and hyaluronic acid. 
Table 1 shows hydrogel preparation methods and doping elements.

Physical Cross-Linking

These kinds of hydrogels are safe to use for clinical purposes because the gelation does not require any toxic covalent crosslinking molecules. 
This method is used to prepare hydrogels through non-covalent approaches, such as electrostatic, hydrogen bonding, and hydrophobic forces among polymer chains. 
The physical methods used to prepare hydrogel have certain advantages, such as high-water sensitivity and thermal reversibility [21]. 
When prepared using this method, hydrogels are very safe for clinical purposes as the gelation does not require any toxic covalent crosslinking molecules. 
Chitosan with small anionic molecules such as sulphates, phosphates, and citrates of Pt, Pd, and Mo can be used for hydrogel preparation using physical methods. 
The synthesized hydrogels depend on the charge and size of anions and the concentration of deacetylation of chitosan.


Chemical Cross-Linking

The formation of physical gels through clustering of molecules causes formation of free chain loops and thus inhomogeneity that signifies short lived network imperfections. 

The chemical crosslinking method can be used to transform the physical properties of the hydrogels. 

 
Tan et al. [33] synthesized N-succinyl chitosan function alizedhyaluronic acid injectable composite hydrogels through a Schiff base mechanism. 
It was determined that the compressive modulus, which is an important factor for cartilage tissue engineering, improved with an increasing amount of N-succinyl chitosan in the hybrid hydrogel [33]. 
Ito et al. [34] also reported the same preparation method for hydrogel using cellulose and alginate. The amino group’s derivatives form hydrogels using Michael’s addition reactions. 
The amino groups react with the vinyl group of other polymers. 
In these kinds of preparations, hydrogels enhanced mucoadhesive properties. This method has some disadvantages, such as multistep preparation and purification method. Polymers might become cytotoxic after functionalization with the reactive groups.


Irradiation Based Cross Linking

Irradiation-based crosslinking is an attractive approach for hydrogel synthesis, especially in applications where rapid gelation and cost efficiency are critical factors. 
By harnessing light-sensitive functional groups and UV irradiation, researchers can achieve effective hydrogel formation in a short period, thus enabling various potential applications in biomedicine, tissue engineering, drug delivery, and other fields [13,14]. 
However, it is essential to consider specific requirements and potential limitations, such as the compatibility of the chosen light-sensitive moieties with the target application and the sensitivity of the hydrogel to environmental factors like light and temperature. 

The use of irradiation-based crosslinking with light-sensitive functional groups is a proposed method for the synthesis of hydrogels, and it offers several advantages over traditional chemical crosslinking methods [31]. 
It has some advantages such as 
    (i) Speedy preparation: The hydrogel formation can occur rapidly when using light-sensitive functional groups and UV irradiation. This allows for a faster production process compared to chemical crosslinking methods, which may involve longer reaction times; 
    (ii) Low cost of production: The use of light-sensitive functional groups and UV irradiation may reduce the need for expensive crosslinking agents or catalysts, potentially making the production process more cost-effective [32]. 

The proposed method by Ono et al. [35] for the synthesis of UV-light irradiated chitosan hydrogels is based on the use of light-sensitive moieties, specifically azide and lactose. 
Azide and lactose are introduced into the chitosan polymer. 
These moieties are selected because they are sensitive to UV light and can undergo specific reactions upon exposure. 
The chitosan polymer with the introduced azide and lactose moieties is exposed to UV light. 
UV light serves as a trigger to initiate a reaction involving the azide group. 
Upon UV irradiation, the azide group on the chitosan polymer is converted into a nitrene group. 
This conversion is likely a photochemical reaction, where the high-energy UV light breaks the N-N triple bond in the azide group, generating a reactive nitrene intermediate. 
The generated nitrene intermediate then reacts with the amino groups present in chitosan. 
The nitrene group binds covalently to the amino groups, resulting in the crosslinking of chitosan chains. 
This crosslinking process forms a three-dimensional network, leading to the formation of a hydrogel. 
The irradiation-based cross linking method reported by Yoo et al. [36] for the synthesis of UV-irradiated chitosan hydrogels involves pre-functionalization with photo-sensitive acrylates of chitosan and pluronic acid. 
However, this method has some drawbacks such as need for a light sensitizer, delayed irradiation, and increase in local temperature.


Characterization Methods

Swelling

The swelling of a hydrogel refers to the process of absorbing a significant amount of water, leading to an increase in its volume and weight. 
Has a significant impact on various characteristics, including the degree of cross-linking, mechanical properties, and rate of degradation.
measure the swelling of hydrogels, such as gel fraction study, swelling ratio measurements, and the weight loss method. 
These methods are relatively simple and widely used to assess the swelling behaviour of hydrogels.
By understanding their swelling properties, researchers can gain insights into the structure and performance of hydrogels
Additionally, the swelling behaviour can provide valuable information about the hydrogel’s response to changes in the surrounding environment, making it an important consideration in material design and engineering. The swelling of a hydrogel is determined using gel fraction



Microstructural Performance 

The microstructure of hydrogels plays a crucial role in determining their physical properties and behaviour. 
Several key parameters are used to characterize the microstructure of hydrogels such as volume fraction of the polymer in the swollen state, number average molecular weight between cross-links, mesh size of the network, and other environmental factors. 
The mesh size depends on the degree of cross-linking, chemical structure of the monomers, and the external environment, especially pH, temperature, and ionic strength. 
The mesh size calculates the physical properties of the hydrogel, including mechanical strength and the degradability and diffusivity of a releasing molecule [39]. 
The microstructure of hydrogels is vital for tailoring their properties to specific applications, such as drug delivery, tissue engineering, and biomaterial design. 
By controlling the degree of cross-linking and mesh size, researchers can customize the mechanical properties, degradation rate, and release kinetics of the hydrogel, making it a versatile and valuable material in various biomedical and engineering fields. 




Mechanical Properties

The mechanical properties were determined from normal techniques such as tensile testing, compression testing, indentation testing, bulge testing, cyclical testing, and the strip extension method. 
This test is demonstrated in the following ways. 
Tensile testing is a common technique used to determine the mechanical response of a material to an applied tensile force. 
In the case of hydrogels, this involves subjecting a strip of the hydrogel to a tensile force using specialized grips in an instrument. 
The resulting stress-strain data can be used to calculate important mechanical properties, such as Young’s modulus, yield strength, and ultimate tensile strength [43]. 
Compression testing involves placing a hydrogel sample between two plates and applying pressure to compress it. 
This helps determine the compression distance or how the hydrogel responds to compressive forces. 
By analysing the deformation and load data during compression, researchers can understand how the hydrogel behaves under compressive stresses. 
Indentation testing involves pressing a small, hard object into the surface of a hydrogel. 
The indentation depth and the force applied are measured to understand the hydrogel’s hardness and elastic modulus in the indentation region. 

The bulge test is a specialized test used to measure the mechanical properties of thin films or membranes, including hydrogels. 
In this test, a hydrogel sheet is constrained around its edges while being subjected to internal pressure. 
By monitoring the deformation, researchers can extract properties like the elastic modulus and Poisson’s ratio of the hydrogel. 
Cyclical testing, also known as fatigue testing, involves subjecting the hydrogel to repetitive loading and unloading cycles. 
This helps assess the material’s fatigue behaviour, which is essential in understanding its long-term mechanical durability under repeated stresses.






Morphology Study

Morphology studies in the context of hydrogel research are essential for understanding the structural characteristics and physical properties of the prepared hydrogels. 
FE-SEM is used to examine the surface morphology and topography of hydrogels. 
It provides highresolution images of the hydrogel’s surface, revealing its surface features, such as porosity, roughness, and the presence of any structural defects [46]. 

Environmental Scanning Electron Microscopy (ESEM) allows for imaging under controlled environmental conditions, including in the presence of water, making it suitable for studying hydrogels. 
ESEM can reveal the morphological changes that hydrogels undergo in a wet state, which is crucial for understanding their behaviour in physiological or aqueous environments. 
Transmission Electron Microscopy (TEM) is used to study the internal structure of hydrogels at the nanoscale.
It provides information about the particle size and shape of components within the hydrogel [47]. 
It can also reveal the distribution of nanoparticles or other inclusions, if present. For FE-SEM and ESEM, the hydrogel samples are typically dehydrated, coated with a conductive layer like gold or carbon, and imaged under high vacuum conditions or in a controlled environmental chamber. 
For TEM, ultrathin sections of the hydrogel or nanoparticles within it are prepared, stained or contrasted if necessary, and examined using high-vacuum TEM.




FTIR


The chemical structure and bonds are identified by FTIR. 
Those chemical bonds can be excited and absorb infrared light at frequencies that are typically based on structure and bonds.
Molecules have characteristic vibrational frequencies associated with their chemical bonds, and when these bonds are exposed to infrared radiation, they absorb energy at specific frequencies [49]. 
Each type of bond absorbs infrared light at distinct frequencies based on factors such as bond strength and the masses of the atoms involved. 
Analysing the absorption pattern of infrared light by a hydrogel helps identify the functional groups and chemical structures of the compounds present.







Antimicrobial Characteristics 

Self Healing Characteristics 

Self-healing materials, which can autonomously repair damage, are increasingly common in hydrogels. 
These mechanisms depend on dynamic covalent bonds (e.g., Diels–Alder, imine, disulfide) and non-covalent interactions (e.g., hydrogen bonds, hydrophobic interactions). 
Natural polysaccharides like alginate and chitosan, as well as synthetic polymers like PEG and PVA, are used to create biocompatible, strong, and flexible self-healing hydrogels. 
Combining Li alginate with poly(acrylamide-co-stearyl methacrylate) results in hydrogels with high fracture energy and fire resistance [55]. 
Rapid self-healing hydrogels utilize flexible polymer networks for quick repairs, and innovations include fluorochromic and Schiff base linkage-based hydrogels that self-repair at room temperature [56]. 
Two hydrogels of different colours (one red with rhodamine B) were cut and placed together. 
After 10 min, they autonomously adhered and healed without external force, demonstrating fast self-healing. 
Some dye diffusion blurred the interface, but the hydrogel could still stretch without cracking, indicating recovery of its 3D structure and mechanical strength. 
Rheological analysis showed rapid recovery of the hydrogels’ internal structure, with storage modulus G’ rising from 200 Pa to 2016 Pa, matching the original value. Under varying oscillatory forces, the hydrogels exhibited a thixotropic, elastic response, confirming their self-healing ability [57].




Delivery Characteristics 

Viscoelastic Properties

Rheology or compression testing is commonly used to evaluate the viscoelastic properties of hydrogels, and these properties are summarized in Table 1. 
The viscoelastic behaviour of hydrogels can be modulated by varying polymer and crosslinker concentrations, which influences their stiffness and elasticity [60]. 
For instance, higher crosslinker concentrations in glutaraldehyde-crosslinked gelatin hydrogels increase stiffness and shift the material toward a more elastic state. 
Additionally, modifying the viscosity of the aqueous phase with dextran in agarose and polyacrylamide hydrogels allows for control over viscoelastic properties while maintaining a stable elastic modulus, enhancing the design of hydrogels for biomedical applications. 
Viscoelasticity in hydrogels arises from several molecular mechanisms, particularly in physically or non-covalently crosslinked systems.
When stress is applied, crosslinkers can detach and allow the polymer matrix to flow, then reattach, as seen in weakly crosslinked collagen gels, alginate gels, and PEG hydrogels. 
Other factors, such as polymer entanglement and protein unfolding, also contribute to viscoelasticity by dissipating energy and enabling reversible elastic responses [61]. 
Even in well-crosslinked hydrogels, the significant water content leads to energy dissipation, resulting in a measurable loss modulus without inducing plasticity.


Applications 

\begin{itemize}
    \item Wound Healing
    \item Contact Lenses
    \item Tissue Engeneering
    \item 3D Bioprinting
    \item Biosensors
    \item Supercapacitor
    \item Catalysis
\end{itemize}

--------


\citep{ahmedHydrogelPreparationCharacterization2015a}

The ability of hydrogels to absorb water arises from hydrophilic functional groups attached to the polymeric backbone, while their resistance to dissolution arises from cross-links between network chains. 
Many materials, both naturally occurring and synthetic, fit the definition of hydrogels

natural Hydrogels were gradually replaced by synthetic hydrogels which has long service life, high capacity of water absorption, and high gel strength. Fortunately, synthetic polymers usually have well-defined structures that can be modified to yield tailor able degradability and functionality. 
Hydrogels can be synthesized from purely synthetic components.

 
Depending on the properties of the polymer (polymers) used, as well as on the nature and density of the network joints, such structures in an equilibrium can contain various amounts of water; 
typically in the swollen state, the mass fraction of water in a hydrogel is much higher than the mass fraction of polymer. 
In practice, to achieve high degrees of swelling, it is common to use synthetic polymers that are water-soluble when in non-cross-linked form.

Hydrogels may be synthesized in a number of ‘‘classical’’ chemical ways. 


Classification of hydrogel products 

\begin{enumerate}
    \item By source 
    \begin{enumerate}
        \item Natural
        \item Synthetic
    \end{enumerate}
    \item Polymeric composition
    \begin{enumerate}
        \item Homopolymeric hydrogels 
        \item Copolymeric hydrogels 
        \item Multipolymer Interpenetrating polymeric hydrogel 
    \end{enumerate}
    \item Configuration
    \begin{enumerate}
        \item Amorphous
        \item Semicrystaline 
        \item  Crystalline
    \end{enumerate}
    \item type of cross-linking 
    \begin{enumerate}
        \item Chemical
        \item Physical
    \end{enumerate}
    \item Physical appearance
    \begin{enumerate}
        \item matrix
        \item film
        \item microspheres
    \end{enumerate}
    \item network electrica charge
    \begin{enumerate}
        \item Nonionic
        \item Ionic
        \item Amphoteric electrolyte
        \item Zwitterionic
    \end{enumerate}
\end{enumerate}


They may perform dramatic volume transition in response to a variety of physical and chemical stimuli, where the physical stimuli include temperature, electric or magnetic field, light, pressure, and sound, while the chemical stimuli include pH, solvent composition, ionic strength, and molecular species
The extent of swelling or de-swelling in response to the changes in the external environment of the hydrogel could be so drastic that the phenomenon is referred to as volume collapse or phase transition [12]. Synthetic hydrogels have been a field of extensive research for the past four decades, and it still remains a very active area of research today.


Technologies adopted in hydrogel preparation

By definition, hydrogels are polymer networks having hydrophilic properties. 
While hydrogels are generally prepared based on hydrophilic monomers, hydrophobic monomers are sometimes used in hydrogel preparation to regulate the properties for specific applications

In general, hydrogels can be prepared from either synthetic polymers or natural polymers. 
The synthetic polymers are hydrophobic in nature and chemically stronger compared to natural polymers. 
Their mechanical strength results in slow degradation rate, but on the other hand, mechanical strength provides the durability as well. 
These two opposite properties should be balanced through optimal design [35]. 

Also, it can be applied to preparation of hydrogels based on natural polymers provided that these polymers have suitable functional groups or have been functionalized with radically polymerizable groups [36]. 
In the most succinct sense, a hydrogel is simply a hydrophilic polymeric network cross-linked in some fashion to produce an elastic structure. 

Thus, any technique which can be used to create a cross-linked polymer can be used to produce a hydrogel. 

Copolymerization/cross-linking free-radical polymerizations are commonly used to produce hydrogels by reacting hydrophilic monomers with multifunctional cross-linkers. 

Water-soluble linear polymers of both natural and synthetic origin are cross-linked to form hydrogels in a number of ways:
1. Linking polymer chains via chemical reaction. 
2. Using ionizing radiation to generate main-chain free radicals which can recombine as cross-link junctions. 
3. Physical interactions such as entanglements, electrostatics, and crystallite formation.

Any of the various polymerization techniques can be used to form gels, including bulk, solution, and suspension polymerization.

Preparation of hydrogel based on acrylamide, acrylic acid, and its salts by inverse-suspension polymerization [37] and diluted solution polymerization have been investigated elsewhere. 
Fewer studies have been done on highly concentrated solution polymerization of acrylic monomers, which are mostly patented [38]. 
Chen [39] produced acrylic acid-sodium acrylate superabsorbent through concentrated (43.6 wt%) solution polymerization using potassium persulphate as a thermal initiator. 

Hydrogels are usually prepared from polar monomers. 
According to their starting materials, they can be divided into natural polymer hydrogels, synthetic polymer hydrogels, and combinations of the two classes. 
From a preparative point of view, they can be obtained by graft polymerization, cross-linking polymerization, networks formation of water-soluble polymer, and radiation cross-linking, etc. 
There are many types of hydrogels; mostly, they are lightly cross-linked copolymers of acrylate and acrylic acid, and grafted starch-acrylic acid polymers prepared by inversesuspension, emulsion polymerization, and solution polymerization. 


Polymerization techniques

\begin{itemize}
    \item Bulk polymerization
    \item Solution polymerization/cross-linking
    \item Suspension polymerization or inverse-suspension polymerization 
    \item Polimerization by irradiation 
\end{itemize}


Hydrogel technical features  

The functional features of an ideal hydrogel material can be listed as follows [48]:  
The highest absorption capacity (maximum equilibrium swelling) in saline. 
Desired rate of absorption (preferred particle size and porosity) depending on the application requirement. 
The highest absorbency under load (AUL). 
The lowest soluble content and residual monomer. 
The lowest price. 
The highest durability and stability in the swelling environment and during the storage. 
The highest biodegradability without formation of toxic species following the degradation. pH-neutrality after swelling in water. 
Colorlessness, odorlessness, and absolute non-toxic. 
Photo stability. 
Re-wetting capability (if required) the hydrogel has to be able to give back the imbibed solution or to maintain it; depending on the application requirement (e.g., in agricultural or hygienic applications).  

Obviously, it is impossible that a hydrogel sample would simultaneously fulfill all the above mentioned required features. 
In fact, the synthetic components for achieving the maximum level of some of these features will lead to inefficiency of the rest. 
Therefore, in practice, the production reaction variables must be optimized such that an appropriate balance between the properties is achieved. 
For example, a hygienic products of hydrogels must possess the highest absorption rate, the lowest re-wetting, and the lowest residual monomer, and the hydrogels used in drug delivery must be porous and response to either pH or temperature.


The rest of the article is about how to sythentize them.

--------

\citep{hartMaterialPropertiesApplications2021}


The most common MIP is based on the rotaxane architecture: a ring (macrocycle) threaded onto a dumbbell-like component (Fig. 1a), where the ring is able to slide back and forth along the dumbbell component but is prevented from dethreading by the presence of bulky stoppers. 
A polymeric analogy to this would be the main-chain polyrotaxane (Fig. 1b), in which the ring(s) are trapped on the polymer backbone and can undergo a similar sliding motion over a much longer distance. 
Thus, by expanding or limiting the range of the rings’ sliding motion (for example, by controlling the length of the polymer backbone or the number of rings on it), the properties of the polyrotaxane can be tuned. 

Another important class of MIPs is slide-ring materials (SRMs), which are polymer networks where the crosslink itself is a rotaxane moiety. 
The term slide-ring gels, SRGs, is also commonly used to describe these materials, particularly if the network is swollen with a solvent. 
The most common of these systems contain figure-of-eight crosslinks (Fig. 1c) and the resulting MIP networks have mobile junctions that can slide freely along the polymer backbone. 

The final class of polyrotaxanes to be discussed are ‘daisy-chain’ architectures, which feature interlocked monomeric units composed of two of the key polyrotaxane elements (the ring and the thread) covalently bound together; the simplest form of this structure is the dimeric cyclic [c2]daisy chain (Fig. 1d).

The other main class of MIPs is mechanically interlocked rings, or catenanes (such as a [2]catenane, Fig. 1e). 
This chain-link structure can be incorporated into a polymer in many ways: as monomeric subunits (poly[2]catenane), pendant moieties or even composing the entirety of the polymer (polymeric [2]catenane, Olympic gel, poly[n]catenane (Fig. 1e)). 
Mechanically bonded rings do not allow for the long-range translational motion present in the polyrotaxanes; however, the components of a catenane possess several unusual degrees of freedom, such as elongation, twisting and rotating motions (Fig. 1e). 
Although MIP synthesis remains a challenge, developments have been made over the years that have allowed access to a range of MIPs16,17. 

In this Review, we will focus on polyrotaxanes, SRMs and polycatenanes to examine the current understanding of how the mechanical bond impacts the properties of a polymeric material and the potential applications of such materials. 
Although syntheses will be touched upon, it will not be the focus of this Review and readers interested in synthetic approaches to MIPs are directed elsewhere18–21. 
Other classes of interlocked polymers, such as pseudopolyrotaxanes or knot-based materials10,20, will not be discussed herein.

--------

\citep{correaTranslationalApplicationsHydrogels2021}

Since their discovery in the 1960s,1 synthetic hydrogels have become increasingly useful for engineering biological systems.

Hydrogels come in many flavors, with diverse capabilities and limitations, but in general these systems can all be described as cross-linked macromolecular networks that retain a significant amount of water. 
As much as 99\% of the weight of a hydrogel can be water, which makes these materials quite friendly to water-enriched biological environments such as the human body. 

In earlier technologies, harsh mechanisms for macromolecular cross-linking (e.g., toxic agents, radiation,  etc.)24−28 meant that gelation needed to occur prior to introducing gels to biological systems. 
Unsurprisingly, this limited the bioengineering applications of hydrogels to superficial environments such as the surface of the eye, an open wound, or an exposed surgical bed. 

Subsequent work developed safer cross-linking mechanisms, which began a trend toward triggering gelation in situ after injection, providing a minimally invasive way of administering  hydrogels to practically any organ or tissue.29,30 
The most biocompatible iterations of these injectable in situ gelling platforms use specific cues from the body to trigger gelation:  physiological temperature,31 pH,32 or ionic strength.33 
Unlike earlier hydrogels that relied on covalent cross-links, some of these hydrogels have self-healing properties and possess mechanical properties akin to native tissue, capable of countering natural forces and stresses of a body in motion.


More recently, shear-thinning hydrogels were developed that are formed through dynamic and reversible cross-linking.34 
For example, physical hydrogels use noncovalent interactions (e.g., supramolecular chemistries) between soluble building blocks in order to self-assemble into a dynamic, reversibly cross-linked  network.35,36 

Likewise, reversible covalent cross-linking strategies can yield dynamic networks with similar properties.37,38 
These “dynamic hydrogels” assembled through reversible cross-links afford the unique property of being injectable even after having formed a gel, due to their shear-thinning and selfhealing behaviors. 
Current research on dynamic hydrogels has revealed novel and useful capabilities that have opened new frontiers for this technology. 
For example, they can stabilize delicate protein and cellular cargoes to combat pharmaceutical  cold-chain limitations,39 they can adhere strongly to tissues to  form protective barriers and bandages,40 and they can be delivered through spray applications to coat complex biological  geometries.41


While dynamic hydrogels are opening up new translational possibilities, significant progress is also being made to introduce unprecedented levels of functionality into biomaterials. 
This includes features such as nanoscale patterning of  bioactive molecules,42,43 programmable drug release,44,45 and  stimuli-responsive behaviors.46,47 
As a consequence, much of the research in this space is trending toward increasingly interdisciplinary projects that recruit the expertise of nanotechnologists, chemists, protein engineers, and synthetic biologists to develop sophisticated multifunctional hydrogels. 
These novel systems include the rise of programmable behavior in hydrogels reminiscent to the behaviors we now  associate with digital technology.48 
For example, significant advancements have been made to transform simple PEG-based hydrogels into responsive systems based on Boolean-logic gating decisions (e.g., YES, AND, OR operations) by incorporating functional peptides and proteins into the  hydrogel network.45,49,50 
Programmable biotechnologies are already leading to smart injectable materials with the potential to degrade or release drugs based on either endogenous or  exogenous triggers.51,52 
As these capabilities continue to mature, multifunctional and programmable hydrogels may provide the technological foundation for platforms that can engage more effectively with the complex, multistage biological events that govern processes such as tissue regeneration and immunity.


Mechanical considerations for designing injectable hydrogels

Hydrogels are a broad class of materials that exhibit mechanical and chemical properties that are especially useful for a variety of medical interventions. 
Noninjectable hydrogels represent the bulk of the literature as they were the first to be discovered and developed, and their usefulness for both drug and cell delivery led to broad enthusiasm for developing hydrogels for biomedical applications.53−56  
However, static covalent cross-links ultimately introduced translational challenges for clinical implementation, since traditional covalent gels require invasive surgical implantation to access nonsuperficial tissues.  
Additionally, new manufacturing processes, such as 3D printing, require dynamic rheological properties during processing, disqualifying the use of traditional covalent  hydrogels.57  
Interest in further developing the translational potential of hydrogels led to innovative methods to implant them through minimally invasive means, of which the most clinically relevant is injection through a needle or catheter (Figure 3).  
Initial success for injectable systems came about with systems that could gel in situ, which allowed liquid polymer solutions to be injected into tissues where they subsequently solidify.  
For example, dual-syringe devices can coinject two  solutions that react to form a hydrogel when mixed.58−60  
Similarly, microencapsulation of gel-inducing molecules could slow down gelation to provide an injection “window” after  combining the components of the gel.61  
Alternatively, stimuliresponsive polymers have been developed that undergo sol− gel transitions based on environmental factors such as temperature, pH, and ionic strength.  
These systems are engineered to remain liquid under nonphysiological conditions (e.g., room-temperature, acidic pH, salt-free) but solidify when introduced into the body (e.g., 37 °C, neutral pH, millimolar  salt concentration).62−64  
While these systems are injectable, many experience problems with gelation kinetics.  
For example, they may gel too quickly and solidify within the syringe or gel too slowly and prematurely release cargo in vivo, and poor  mixing may further cause heterogeneous gelation.62,65−67


To overcome these limitations, significant attention has been devoted to dynamic hydrogels, which can seamlessly transitionback and forth from solid-like to liquid-like during injection thanks to their shear-thinning and self-healing capabilities.  
These materials, which are gelled within the syringe before injection, additionally have the ability to stabilize drugs over broad temperature ranges and maintain homogeneously mixed  cell solutions.68−70  
Here, we define dynamic hydrogels as any hydrated polymer network cross-linked via reversible chemistries, which can include both covalent and noncovalent chemistries.  
Early reports of the unique rheology of dynamic networks emerged in the late 1980s with polysaccharide-based networks covalently cross-linked through boric esters, which  identified intriguing self-healing capabilities.71−73  
However, it was only in the early 2000s that noncovalent chemistries began to be leveraged to make shear-thinning supramolecular  hydrogels based on cyclodextrins,74 engineered peptides,75 and the physical interactions resulting from biopolymer  blends.76  
Although they can be prepared through diverse chemistries, dynamic hydrogels share unique rheological properties that are directly related to their translational potential as injectable systems.  
In this section, we will review the principle rheological considerations that ought to be taken into account when designing an injectable dynamic hydrogel, as well as a range of techniques to properly characterize these complex systems.


Rehological Considerations for injectable dynamic hydrogels 

Injectable hydrogels have enabled minimally invasive strategies to deliver therapeutic drug and cellular cargo without surgical implantation.  
The applicability of hydrogels in clinical settings is seemingly limitless, from applications that require localization in different regions of the body to the delivery of a wide range of cargo.  
Importantly, the rheological properties of these hydrogels are constrained by the need for administration by direct injection or catheter delivery.   

Here, we focus on and discuss the rheological properties of existing injectable hydrogels and emphasize the need for determining property−function relationships to facilitate their design for clinical translation.   

Injectable therapeutic hydrogels must be compatible with a three-stage administration process (Figure 4).  
First, their formulation must be compatible with the incorporation of drug, cellular, or other therapeutic cargo (e.g., the hydrogel must not react with or otherwise compromise the bioactivity of cargo).  
Second, they must be injectable.  
Third, they should provide the desired terminal function within the body, which ranges broadly from cell expansion to controlled release of molecular cargo of diverse types.  
Typically, the terminal function within the body is the key target in the design process, yet the performance of the hydrogel during formulation and administration must not be neglected.  
From a translational perspective, the injectability of a particular formulation may change as the relevant dimensions and geometries of the injection process changes when moving from the lab to the clinic.  
Going forward, it is helpful to provide an explicit definition of “injectability”.   
Here, we define injectability as the capability of a formulation to flow at a clinically relevant flow rate through an administration needle, catheter, or autoinjector using clinically relevant applied pressures.  
According to this definition, injectability is necessarily dependent on the intended application and will vary depending on the needle gauge and length (i.e., subcutaneous vs catheter injections) and other processing constraints (e.g., administration volumes, syringe geometries, and desired flow rates).   

The application-specific requirements imposed by each stage of the administration process can impose paradoxical constraints on the rheological properties of injectable materials, simultaneously requiring flowability for injection and solid-like retention at the injection site (e.g., sustained localized delivery). 
There have been several hydrogel compositions with varying chemistries and cross-linking modalities developed that address this paradoxical constraint and are capable of both injectability and solid-like retention after injection. 

For a material to flow, it must demonstrate liquid-like behavior, whereby the constituent molecules are able to move past each other, under relevant processing conditions. 
Most covalent materials cannot flow because their covalent bonds prevent relative movement of their constituent molecules. 
Consequently, “static” covalent hydrogels require the injection of prepolymer systems that gel upon injection or stimuli responsive polymers that cross-link in response to temperature, UV, pH, or other external stimuli. 

More recently, there has been an increased interest in the use of dynamically crosslinked hydrogels as injectable materials.77−80 
The specific cross-linking strategies vary and include both dynamic covalent and noncovalent supramolecular cross-linking, but generally these approaches imbue hydrogels with dynamic, yielding, and self-healing rheological responses. 
The various cross-linking strategies and description of the hydrogels for the delivery of  therapeutics have been outlined in several reviews.36,81−83 
We highlight that although dynamic hydrogel compositions vary, they demonstrate similar rheological functions. 
In general, the physical cross-links create a hydrogel network with solid-like material properties under static conditions. 
Yet, when deformed, the dynamic cross-links can be disrupted, dissipating stress and resulting in liquid-like behavior. 
Since the cross-links are reversible, they can reassociate after deformation to restore the network structure and its solid-like behavior. 
The rapid development of dynamic hydrogels for injectable material platforms has enabled new therapeutic strategies without the need for in situ chemical reaction strategies. 
However, our understanding of structure−property-function relationships (which relate a hydrogel’s rheological properties to their functional performance) for dynamically cross-linked  hydrogels is still rather poorly developed.84−86 

Dynamically cross-linked hydrogels are complex fluids, where their reversible cross-links result in bulk material behaviors that include yielding, shear-thinning, thixotropy, and viscoelasticity. 
To date, designing injectable hydrogels from dynamically cross-linked networks with the desired combination of properties for new applications remains challenging. 
Indeed, researchers in the rheological community have focused on creating engineering design strategies for dynamically crosslinked hydrogels.87−91 

For injectable therapeutic applications, there is a desire to design hydrogel materials with tunable viscoelasticity to deliver stem cells and control their differentiation,92−98 a need for strategies to control the release of  small molecular cargo,36,77,81,83,99−101 and a push toward  materials that provide stabilization of pharmaceuticals.102 
With structure−property−function relationships in place, it becomes easier to answer important design questions before heading to the bench. 

Questions such as 
how does one design a hydrogel’s terminal function (i.e., local depot formation for sustained release of molecular cargo) without compromising performance in formulation or during administration by injection? 
How can one identify if an existing hydrogel formulation would meet the demands of a new application, eliminating the need for starting anew with laborious and costly trial-and-error efforts? Unveiling property−function relationships facilitates the design process of injectable hydrogels. 
Knowledge of these relationships allows for rapidly identifying and satisfying the functional constraints across a broad variety of administration conditions while optimizing the performance of the injectable hydrogel in vivo. 

The following sections provide a concise review of key property-function relationships of dynamically cross-linked hydrogels for injectable therapeutic applications. 
We briefly discuss structure−property relationships in the context of the rheological properties that are introduced but leave a detailed discussion to  other excellent reviews.90,98,103,104 
Since cross-linking strategies and network structure result in similar rheological behaviors (i.e., shear thinning, yield stress), the property−function relationships shown here are useful across many hydrogel compositions. 
Next, we discuss rheological characterization strategies for complex fluids, such as physically cross-linked hydrogels, and provide information about best practices during the characterization process. 
We intend these sections to help scientists and engineers design future biomaterials and also highlight key areas where more investigations are needed.


Pre-and Postinjection constraints of injectable Hydrogels 

The applications of injectable hydrogels dictate the requisite properties for the hydrogel during formulation and after injection.  
The details of the requirements for these applications are left to the other sections of this review.  
From a rheological perspective, the rheological modifications required by each application must be considered alongside the constraints of injectability.  
A common requirement is the localization of a hydrogel after injection, which depends strongly on the rate at which the hydrogel self-heals after injection.  
During injection, the high shear destroys the structure of the dynamic hydrogel.  
After injection, most dynamic hydrogels do not return to their initial viscosity immediately but rather demonstrate a recovery  of viscosity over time.92,96,104−110  
The transient recovery of viscosity after the cessation of flow (i.e., once in the implantation site after injection) is called thixotropy.  
Thixotropic behavior in dynamic hydrogels depends heavily on the cross-linking motif, whereby some motifs result in hydrogels that require a significant amount of time to recover (strongly thixotropic), while some show only mild thixotropy and recover their properties rapidly (weakly thixotropic).  
For injectable drug delivery applications, the thixotropy of a hydrogel provides valuable insight for the time scales over which a hydrogel will be susceptible to burst release or flowing away from the site of injection before establishing a depot.


Relevant Rheological Properties for injectability 

The viscosity of a hydrogel is related to its injectability, elucidating the constraints that injectability places on the viscosity of injectable biomaterials.  
For clinical applications, injectable hydrogels must be delivered through a needle or catheter to the site of injection.  
The injectability of a fluid depends on how much pressure is required to drive this process of injection over relevant time frames.  
This pressure is a function of the fluid viscosity, injection geometry, and desired flow rate.   

Here, we review how injectability constrains the allowable rheological properties of injectable hydrogels.  
To elucidate these constraints on rheological properties, it is important to understand the physical process of injection.   

Injection, in its simplest form, is the flow of a fluid through a circular tube of constant diameter and length.  
Often, there is a maximum pressure that can be applied and a minimum flow rate that is desired.  
A syringe injection, for example, would be limited to the amount of force the average healthcare  personnel could comfortably apply to a syringe plunger.111  
An autoinjector on the other hand would be limited by the maximum pressure the mechanism could generate.  
Intuitively, there is a limit to the viscosity (i.e., resistance to flow) of the materials which can be injected under a prescribed set of injection conditions and geometries.  
Therefore, it is critical to understand how viscosityand its dependence on shear rate affects injectability, enabling researchers to use simple rheological measurements to design their materials for injectability.  

Steady state flow models are used to model the relationship between a hydrogel’s viscosity and the pressure required to  inject it (Figure 5a).112−114  
In the case of polymer solutions and physically associated hydrogel materials, the viscosity often obeys a power law (eq 1) that is described by the consistency  index, K, and shear-thinning parameter, n.115,116 
A shearthinning parameter of n = 1 describes a Newtonian fluid with constant viscosity as the shear rate is increased. 
A value of n < 1 represents a shear-thinning fluid with a viscosity that decreases as the shear rate is increased. 
Assuming power law shear-thinning behavior, the constitutive relationship shown in eq 2 can be used to describe the relationship between the shear stress and shear rate on the fluid. 
The governing equation for steady state flow through a pipe (eq 3) is derived using this constitutive relationship to model the injection pressure (P) as a function of flow rate (Q), radius (R), length (l), and  viscosity.117−119 
This model has been used by Paxton et al to predict the bioprinting window for a variety of 3D printing  materials.113 Almendinger et al validated the model for shearthinning antibody solutions and used it to predict the extrusion  pressure in a variety of injection scenarios.112,120−122 
Our group validated the model for physically cross-linked hydrogels, demonstrating its applicability for two physical hydrogels with distinct cross-linking mechanisms (polymer−nanoparticle  interactions and ionic cross-linking).123

In addition to validating the model, our work also demonstrated the utility of using the model inversely for materials design to elucidate the consistency indexes and shearthinning parameters that correspond to hydrogel injectability across applications with varying geometries or flow rate constraints. 
Ashby style plots of the consistency index vs shear-thinning parameter were used in combination with the model developed in eq 3 to reveal a parameter space for readily injectable hydrogels under typical process constraints for  syringe injection (Figure 5b).85,87−89 
It is therefore critical that in the development of injectable biomaterials, researchers employ flow models accompanied by rheological characterization to avoid developing material platforms with properties that could never scale to the clinic. 
This is perhaps most notable for deep-tissue delivery of dynamic hydrogels, which can impose significantly different constraints in a preclinical model versus clinical practice. 
For example, the primary model for oncological research is mice, where delivery to any organ is possible through a short 0.5−1.0 in syringe. 
In contrast, the equivalent in a human patient could require injection through a long catheter and necessitate injection forces that are impractical. 
To avoid these types of pitfalls on the road to translation, it is imperative that the flow properties of dynamic hydrogels be measured within application relevant shear rate/shear stress regimes to determine the appropriate constitutive relationship for each hydrogel. 
It is important to note that injections through small diameter needles can result in shear rates which are dramatically higher than the typical shear rates range used for characterization on a rheometer (Figure 6). 
For example, oncological treatments have improved patient comfort and are  delivered at flow rates between 1 and 2.3 mL/min.124,125.

Using eq 4 which describes the maximum shear rate for a Newtonian fluid in a pipea flow rate of 2.3 mL/min in a 27gauge needle (standard for subcutaneous injections) results in  shear rates up to 42 × 103 s−1. 
Researchers should be cautious of extrapolating constitutive relationships beyond the range of characterization, as this can lead to significant errors and often poor approximations for fluids as complex as dynamically  cross-linked hydrogels.123 
There are limitations to the model presented in eq 3, which simplifies the flow of these hydrogels by assuming a simple power law shear-thinning response (eq 2), steady state conditions, no slip, a negligible yield stress under the flow  conditions, and negligible effects of fluid extensibility.123 
While these assumptions help make the problem simpler to analyze, there may indeed be cases where these simplifications fail to capture the relevant phenomena necessary to describe the flow of more complex fluids. 
Good practice is to validate the flow model within the target flow regimes and with the appropriate rheological data measured within the correct shear-rates. 
In the rare cases where the simple model in eq 3 fails to adequately describe the flow behavior, there is extensive literature on the flow of non-Newtonian fluids that should be explored.114,117−119,126−128 
Alternative models have been developed to account for slip, significant yield stresses, and nonconstant geometries, though these models should be validated with the target materials and desired flow regimes prior to broad utilization by the community.


Rheological Characterization of Injectble hydrogels

As we have shown, the rheological properties of injectable hydrogels dictate the function and ultimately a significant fraction of the performance as injectable therapeutic strategies. 
Here, we provide a brief review of characterization methods for measuring the rheological behavior of injectable hydrogels. 
For an in-depth discussion of these methods, we point the reader  to reports by Ewoldt,129 Larson,115 and Macosko.116 
Dynamic hydrogels demonstrate rheological behavior that may comprise a combination of yielding, shear-thinning, thixotropic,  viscoelastic, and extensible behaviors.130−135 
Consequently, their characterization is nontrivial and requires a combination of rheological tests to characterize comprehensively. 
Here, we present the state-of-the-art rheological methods for measuring the viscoelastic and flow behaviors of injectable hydrogels.

\textbf{Viscoelasticity}

The viscoelasticity of a hydrogel is most often measured using dynamic mechanical analysis to measure the bulk elastic and viscous responses of a hydrogel to  an imposed oscillatory shear strain or stress.116 
Methods also exist to measure the viscoelasticity of hydrogels at various length scales, which may be important in cell-based applications where they interact with the hydrogel at different  length scales than the bulk.136,137 
In bulk oscillatory measurements, the amplitude and frequency of the imposed oscillations are varied, and the oscillatory response of the hydrogel is measured. 
Small amplitude oscillatory shear (SAOS) is the most common experimental method for measuring a hydrogel’s viscoelastic response. 
The oscillatory response measured in response to the oscillatory input is analyzed and typically represented through dynamic storage and loss moduli (Figure 7a), which describe the elastic and viscous responses of the hydrogel, respectively. 
When the storage modulus is greater than the loss modulus, the material is said to be solid-like. 
When the loss modulus is greater than the storage modulus, the material is said to be liquid-like. 
The storage and loss moduli are only well-defined when experiments are performed within the linear viscoelastic regime (Figure 7), where the hydrogel network responds linearly to the imposed strain or stress amplitude. 
Frequency sweeps are performed at a constant strain or stress amplitude, and the frequency of the oscillation is varied to probe the material’s time-dependent viscoelasticity. 
For irreversible covalent hydrogels, solid-like behavior is observed for all frequencies without any significant frequency dependence due to the permanent cross-links in the network. 
For dynamic hydrogels, the response to oscillatory shear can be more complex, showing both solid and liquid like responses that depend on the frequency of oscillation. 
The point at which the storage modulus is equal to the loss modulus is the crossover frequency and denotes the transition between solid and liquid like states. 
In general, the viscoelastic response is a function of the thermodynamics and kinetics of the physical  cross-link and network topology.133,135,138−143 
Craig et al. have demonstrated for nonentangled physically cross-linked networks that the relaxation time (τ) of the hydrogel is equal to  the dissociation rate of the physical cross-link.142 

In a simple system, where the only cross-links originate from physical cross-links, the equilibrium constant (Keq) of the interaction describes the equilibrium concentration of cross-links and therefore the magnitude of the rubbery plateau. 
Though not discussed in detail here, stress relaxation experimentswhere a constant deformation is applied and the temporal decrease in stress is monitoredare also a valuable experimental tool for measuring the relaxation time (τ) of dynamically cross-linked  hydrogels.115 
Stress relaxation experiments are especially useful when the relaxation time is longer than the measurable relaxation times in SAOS experiments. 
Time sweep SAOS measurementswhere the amplitude and frequency of oscillation are kept constantare useful when measuring the transition of a hydrogel or its components from liquid to solid or vice versa, such as in the gelation of covalent and physically cross-linked materials. 

The oscillatory response is measured at a constant frequency and amplitude over an extended period of time. 
For irreversible covalent materials, mixing of two components can be performed immediately before measuring the materials viscoelastic response. 
The temporal evolution of the dynamic moduli reveals the kinetics of gelation, where the gelation point is assigned to the time at which the storage modulus surpasses the loss modulus at a crossover time. 
For dynamically crosslinked hydrogels, time sweep SAOS measurements are used to probe their self-healing behavior. 
The amplitude of the applied shear strain or stress is transitioned from low-to-high or highto-low to probe the response of the dynamic cross-links. 

The temporal viscoelastic response of a dynamic hydrogel is measured to quantify the kinetics of recovery and degree of self-healing. 
This process is often alternated and repeated several times to demonstrate reversible self-healing of dynamically cross-linked hydrogels. 
For injectable applications, dynamic hydrogels typically undergo transitions from a static equilibrium state to a nonlinear flow state and then return to a static equilibrium state. 
The properties of the hydrogels during and after these transitions influence their performance as injectable therapeutics. 
Measuring these properties, however, is challenging due to the transition from the linear to nonlinear regime. 
Nonlinear oscillatory shear measurements that go beyond the linear viscoelastic regime, such as large amplitude oscillatory shear  (LAOS), have been a recent area of research focus.86,144−150 
The storage and loss moduli become ill-defined in the nonlinear regime, and methods for quantifying a hydrogels’ response are more challenging. 
There have been recent advances in the analysis of nonlinear rheological data using Fourier transform analysis methods and a sequence of physical process methods, which provide more insight into the nonlinear properties of injectable biomaterials.


\textbf{Flow Rheology}

The flow properties of a hydrogel are measured using a rheometer or capillary viscometer.  
In these instruments, a simple shear flow is applied to measure the relationship between the shear rate and shear stress of a fluid. 
This relationship is shown in a flow curve (and is extracted through an analysis of the imposed viscometric  flows).116 
A typical flow curve for a yielding, physically crosslinked hydrogel is shown in (Figure 8a). 
The viscosity is the ratio of the stress and shear rate and can be constant across shear rates (Newtonian) or be shear rate dependent (nonNewtonian). 
This section will discuss the acquisition of a steady state flow curve, introduce the important features of a flow curve for injectable hydrogels, and discuss the measurement transient thixotropic behaviors (time-dependent change in properties). 
Typical flow curves for physically cross-linked hydrogels show three distinct regimes: (1) preyield, (2) yielding, and (3) flow. 
We highlight that although discussion about a true yield stress has been a long contentious area of discourse in the scientific literature, the engineering reality of  its effects is readily evident for injectable hydrogels.130,151−153 
In a rheometer, an angular velocity is applied to a rotating geometry, and the resulting torque on the geometry is measuredor vice versa. 

With a known geometry, such as parallel plates or a cone-and-plate, the angular velocity is converted to shear rate and the torque is converted to stress. 
In a capillary viscometer, a constant flow rate is applied, and the pressure required to drive the flow is measured. 
The shear stress is determined from the geometry and pressure, and the shear rate is calculated from the geometry, flow rate, and pressure using the Weissenberg−Rabinowitsch−Mooney analysis.116,154 

For Newtonian fluids, the shear rate is a function of the flow rate and channel geometry. 
For non-Newtonian fluids, the shear rate is also a function of the fluid’s viscosity in addition to the flow rate and channel geometry. 
Deciding between a rheometer and viscometer depends on the viscosity of the fluids being measured and on the shear rates that are of  interestoutlined by Pipe et al.154 
Generally, it is simpler to measure high-shear-rate flow curves using a capillary viscometer. 

In rheometers, there are significant challenges at high shear rates. 
The shear rate in rheometers is proportional  to (gap size)−1 and is increased by decreasing the gap size between the two shearing surfaces. 
As the gap size is decreased to increase the shear rate, significant errors arise due to geometrical imperfections. 
Rheometers also suffer from radial migration of the sample and subsequent ejection of a sample at high shear rates. 
Capillary viscometers provide an alternative strategy for measuring the viscosity of fluids at high shear rates and use a closed capillary that is not prone to technical issues such as sample ejection. 
Regardless of the measurement technique used, the outcome is a measurement of the stress− shear rate relationship of a fluid. 

The yield stress demarcates the minimum required stress necessary to induce flow for the fluid, and several strategies for  measuring it have been developed.153,155,156 
Here, we review the use of flow data to measure the yield stress. 
Using a stress vs shear rate curve (Figure 8a), the yield stress manifests as a nonzero intercept with the stress axis. 
The yield stress is then calculated using a Herschel−Bulkley model (eq 5) that is fit to the stress-shear rate data. 
Here, σy is the yield stress with units in Pascals, n is the shear-thinning parameter (unitless), and K is the consistency index with units in Pascal-secondsn. 
The modified Herschel−Bulkley model is often preferred because it yields fitting parameters with constant units and more intuitive meaning. 
The consistency index is replaced with γ̇critical, which  is the critical shear rate with units in s−1 at which the flow stress is double that of the yield stress. 

Alternatively, some authors suggest plotting viscosity vs stress (Figure 8b), where a dramatic decrease of several orders of magnitude in the viscosity is observed for a small increase in the  stress.130,131,152,156−158 
Here, the stress at which the viscosity decreases is assigned as the yield stress. 
In plots of viscosity vs shear rate (Figure 8c), the preyield regime appears as shearthinning. 
Before the yielding event, the stress is constant at increasing shear rates, resulting in a viscosity which appears to  decrease.115 
On a log−log plot of viscosity vs shear rate, this phenomenon is observed as a slope of −1. 
In practice, the visualization of rheological data showing flow data with plots of viscosity vs shear rate alone makes it challenging to understand important details about the rheological response of a dynamic hydrogel. 
For this reason, it is recommended thatat a minimumboth stress vs shear rate and viscosity vs shear rate data be shown when characterizing yield stress fluids. 

The flow curve (stress vs shear rate) shows yielding, while the viscosity versus shear rate plot of the flow regime more clearly shows the degree of shear-thinning for the hydrogel. 
Postyield, physical hydrogels flow with shear-thinning behavior, where viscosity decreases as the shear rate is increased. 
Seen as a series of progressively decreasing slopes on a stress vs shear rate plot and as a negative, linear decline (slope = n − 1, where n is between 0 and 1) in a log-base plot  of viscosity vs shear rate (Figure 8c).115 

In a yielding hydrogel, the Hershel−Bulkley model provides information about the non-Newtonian viscosity of a hydrogel, where the consistency index and shear-thinning parameter describe the power law shear thinning of the hydrogel in flow. 
Alternatively, the flow portion of the viscosity versus shear rate curve can be fit to a power law (eq 1) to find the consistency index and shearthinning parameter fits. 
It is critical that only the flow portion of the viscosity versus shear rate plot is used when fitting a power law to the rheological data of a dynamic hydrogel. 

In general, because it is difficult to distinguish between the preyield and flow regime, it is important to be cautious when demarcating the flow regime in viscosity vs shear rate plots before measuring the degree of shear thinning with a model fit (Figure 8c). 
When measuring a flow curve, it is critical to consider the effects of thixotropy and take appropriate precautions with test protocols. 
Intuitively, materials that are strongly thixotropic have a significant delay in restructuring, resulting in a transient response until equilibrium is reached during a deformation (Figure 9a). 
For dynamically cross-linked systemswhich possess both solid-like and liquid-like behaviorsthe dynamics of the cross-links and network often result in transient material response when changing the shear rate, especially before or  near the yield point. 

It is common to observe an overshoot in the viscosity (Figure 9b) on the startup of shear as the network structure yields and breaks down to the new equilibrium state.109,157 
At faster shear rates, the viscosity overshoot is more pronounced and depends on the yielding and relaxation behavior of the material. 
Figure 9c, shows the viscosity of a thixotropic material when the applied stress is instantly decreased (flow cessation). 
Instead of the viscosity increasing instantly, the viscosity slowly increases as the structure within the material rebuilds. 
The recovery of the viscosity can be fit to an exponential to determine the characteristic thixotropic time scales.109 
The phenomena observed upon the sudden application or removal of shear shown in Figure 9b and c probe similar phenomena as described in self-healing SAOS experiments discussed above. 
Authors will often choose between either self-healing SAOS experiments or flow cessation experiments to demonstrate reversible self-healing. 
Experimentally, thixotropy can significantly affect the acquisition of a flow curve, making it challenging to determine the equilibrium viscosities, shear-thinning, and yielding behavior of materials.156,157 
Flow experiments that do not account for thixotropy are often irreproducible and can demonstrate significant hysteresis (Figure 9d). 
A simple strategy for measuring the equilibrium flow curve is to perform flow experiments using stepwise changes in the stress or shear rate and not ramped protocols.109 
Stepwise experiments can be designed to apply a deformation until equilibrium is reached before taking a measurement.115 


\textbf{Outlook for Rheological Characterization of Injectable Hydrogels}

In this section, we’ve highlighted the importance of functional constraints on the rheological behavior of dynamically crosslinked hydrogels. 
The constraints are often paradoxical, requiring higher yield stresses or viscosities for localization upon injection, yet also demand low viscosities to allow for facile injection. 

Here, we’ve reviewed the property−function relationship between the rheological properties of power-law shear-thinning fluids and the pressure required for injection. 
These relationships elucidate materials design targets for future injectable material platforms, specifically target viscosities which allow for facile injection at the shear rates relevant to the clinic. 
Designing these materials requires careful and rigorous characterization of viscoelastic and flow properties, which include viscoelasticity, shear-thinning, yielding, and thixotropy. 

We’ve briefly provided a survey of the more standard characterization techniques and point the reader to some reliable resources that provide a more rigorous description of the techniques. 
Most notably, stress relaxation and creep experiments are critical for understanding the longtime relaxation behaviors of materials and are not suitably characterized using SAOS. 

Together, this section provides the reader with a foundation to understand how the rheological behavior of existing hydrogels may translate to a desired function within their application. 
As more hydrogels are developed in the field of therapeutic delivery and new challenges arise, the property−function relationships shown here will enable more effective materials selection strategies to down-select materials and create rheological targets for new applications.

--------
Connecting Elasticity and Effective Interactions of Neutral Microgels: The Validity of the Hertzian Model\citep{rovigattiConnectingElasticityEffective2019}

%Colloidal suspensions have been used for decades as model systems for investigating fundamental condensed matter  phenomena.1−6 Compared to atomic and molecular systems, colloids have much larger characteristic time- and lengthscales, which makes them more accessible from an experimental point of view. In this context, the most iconic (and probably studied) soft matter system is certainly hard  spheres.1,7


--------

Stress localization, stiffening, and yielding in a model colloidal gel\citep{colomboStressLocalizationStiffening2014}

Colloidal gels, which can form in suspensions of colloidal particles in the presence of attractive effective interactions, are particularly appealing as materials whose functions can be in principle designed at the level of the nanoscale (particle) components

In colloidal suspensions, gels can form even in extremely dilute systems via aggregation of the particles into a rich variety of network structures that can be suitably tuned by changing the solid volume fraction, the physico-chemical environment, or the processing conditions. Hence, these handles could be used to design a specific complex mechanical response in addition to the selected nano-particle properties

Colloidal gels are typically very soft, but the variety of microstructures may lead to an equal variety in the mechanics

The microstructural complexity may also enable adjustments of the mechanical response to the external deformation. Soft gels can be in principle made to yield relatively easily, but in certain cases, a significant strain hardening has been observed before yielding finally occurs

The yielding of colloidal gels, due to breaking and reorganization of the network structure, can be accompanied by strong inhomogeneity of stresses and strains throughout the material\footnote{Pos hay otros fenómenos que llevan a yielding}

----------

Polymer Networks: From Plastics and Gels to Porous Frameworks\citep{guPolymerNetworksPlastics2020}


When bifunctional molecules are linked together, linear macromolecules, or “linear polymers,” with high molecular weights can form. 
Analogously, when molecules with functionality greater than two are linked together, three-dimensional (3D) macromolecules, or “polymer networks,” with very high (classically referred to as “infinite”[1,2]) molecular weights can form. 
Early organic chemists referred to polymerization, the process used to form linear and 3D polymers, as a “chemical combination involving the operation of primary valence forces,” further stating that “the term polymer should not be used (as it frequently is by physical and inorganic chemists) to name loose or vaguely defined molecular aggregates.”[1] 
Similarly, Wallace Carothers defined polymerization as, “any chemical combination of a number of similar molecules to form a single molecule.”[1] 
These notions either implicitly or explicitly defined polymers as being composed of strongly (covalently) bonded constituents. 
Today, however, it is widely accepted that linear polymers and polymer networks can be constructed from covalent and/ or non-covalent bonds; indeed, the full spectrum of bonding interactions, reaction mechanisms, and chemical compositions (e.g., organic, inorganic, biological) can be leveraged to design fascinating new polymer networks with exceptional properties.

From a structural perspective, polymer networks are composed of network “junctions” (in some cases, these can also be referred to as “crosslinks”, defined as a bond that links one strand to another), which have three or more groups (the exact “branch functionality” we refer to as f) emanating from a core, connected together by f “strands.” 
Strands can be linear polymer chains, flexible short molecules, rigid struts/ linkers, etc. 
As noted above, junctions and strands in polymer networks can be linked together via physical interactions (e.g., van der Waals interactions, hydrophobic interactions, Coulombic interactions, metalligand coordination) or covalent bonds. 
Hence, polymer networks are conventionally classified as “physical” (supramolecular) or “chemical” (covalent) networks. 
It should be noted that this classification does not always accurately reflect material properties; bond strengths and exchange rates are much more informative. 
For example, given sufficiently strong and static physical interactions, physical networks can behave identically to chemical networks; alternatively, the incorporation of mechanisms for covalent bond exchange can result in chemical networks that exhibit adaptable mechanical properties regulated by external stimuli. 
Thus, the properties of polymer networks can vary widely depending on the composition of the junctions and strands as well as the formation and use conditions. 
With this broad view in mind, nearly all polymer networks, regardless of their colloquial name, structure, properties, etc. can generally be divided into one of four major classes: thermosets, thermoplastics, elastomers, and gels.

Gels are polymer networks constructed from either covalent or supramolecular bonds that are swollen in liquid media such as water or organic solvents. 
The network structure ensures that the liquid is held within the material. 
Gels are usually very soft (Young s moduli of 103–104 Pa) but are often capable of undergoing relatively large deformation. 
Examples of gels include gelatin, fibrin, and polyacrylamide hydrogel.

As perhaps the most important, useful, and broadly studied class of materials from theoretical, academic, and industrial perspectives, polymer networks can have many unique properties, including elasticity, tunable mechanical strength, porosity, and swellability. 
These properties and others have led to numerous applications of polymer networks in everyday life, such as adhesives, cosmetics, sorbents, membranes, rubber products, coating materials, and food packaging. 
Moreover, as recent developments have imparted unconventional properties (e.g., malleability, self-healing, conductivity, ultra-high permanent porosity, enhanced crystallinity, and stimuli-responsiveness) into polymer networks, they continue to hold great promise in advanced applications including drug delivery systems,[3] tissue engineering scaffolds,[4] soft actuators,[5] gas storage,[6] catalysis,[6–8] and electronic materials.[9] 
Thus, though polymer networks have been widely studied for more than a century, there are features of their structure that have only recently been leveraged to impart new properties; an even deeper understanding is needed to realize the next-generation of functional, and ideally sustainable, polymer networks. 

In this review, we introduce key concepts related to the formation, characterization, and properties of polymer networks. 
Our goal is to provide newcomers to the field with broad and up-to-date knowledge that can serve as a starting point for more detailed investigations of topics of interest. 
Major focus is devoted to polymer network structure, which includes both chemical and topological aspects. 
Additionally, several types of recently developed polymer networks with exceptional properties are highlighted.



\end{document}
