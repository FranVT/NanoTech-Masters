\documentclass[../../main-notes.tex]{subfiles}

\begin{document}

\subsection{Description of the micro-gel}

\begin{itemize}
    \item What is a microgel
    \item Connection with colloids 
    \item Link to brownian dynamics
\end{itemize}

--------
Connecting Elasticity and Effective Interactions of Neutral Microgels: The Validity of the Hertzian Model\citep{rovigattiConnectingElasticityEffective2019}

%Colloidal suspensions have been used for decades as model systems for investigating fundamental condensed matter  phenomena.1−6 Compared to atomic and molecular systems, colloids have much larger characteristic time- and lengthscales, which makes them more accessible from an experimental point of view. In this context, the most iconic (and probably studied) soft matter system is certainly hard  spheres.1,7


--------

Stress localization, stiffening, and yielding in a model colloidal gel\citep{colomboStressLocalizationStiffening2014}

Colloidal gels, which can form in suspensions of colloidal particles in the presence of attractive effective interactions, are particularly appealing as materials whose functions can be in principle designed at the level of the nanoscale (particle) components

In colloidal suspensions, gels can form even in extremely dilute systems via aggregation of the particles into a rich variety of network structures that can be suitably tuned by changing the solid volume fraction, the physico-chemical environment, or the processing conditions. Hence, these handles could be used to design a specific complex mechanical response in addition to the selected nano-particle properties

Colloidal gels are typically very soft, but the variety of microstructures may lead to an equal variety in the mechanics

The microstructural complexity may also enable adjustments of the mechanical response to the external deformation. Soft gels can be in principle made to yield relatively easily, but in certain cases, a significant strain hardening has been observed before yielding finally occurs

The yielding of colloidal gels, due to breaking and reorganization of the network structure, can be accompanied by strong inhomogeneity of stresses and strains throughout the material\footnote{Pos hay otros fenómenos que llevan a yielding}

----------

Polymer Networks: From Plastics and Gels to Porous Frameworks\citep{guPolymerNetworksPlastics2020}


When bifunctional molecules are linked together, linear macromolecules, or “linear polymers,” with high molecular weights can form. 
Analogously, when molecules with functionality greater than two are linked together, three-dimensional (3D) macromolecules, or “polymer networks,” with very high (classically referred to as “infinite”[1,2]) molecular weights can form. 
Early organic chemists referred to polymerization, the process used to form linear and 3D polymers, as a “chemical combination involving the operation of primary valence forces,” further stating that “the term polymer should not be used (as it frequently is by physical and inorganic chemists) to name loose or vaguely defined molecular aggregates.”[1] 
Similarly, Wallace Carothers defined polymerization as, “any chemical combination of a number of similar molecules to form a single molecule.”[1] 
These notions either implicitly or explicitly defined polymers as being composed of strongly (covalently) bonded constituents. 
Today, however, it is widely accepted that linear polymers and polymer networks can be constructed from covalent and/ or non-covalent bonds; indeed, the full spectrum of bonding interactions, reaction mechanisms, and chemical compositions (e.g., organic, inorganic, biological) can be leveraged to design fascinating new polymer networks with exceptional properties.

From a structural perspective, polymer networks are composed of network “junctions” (in some cases, these can also be referred to as “crosslinks”, defined as a bond that links one strand to another), which have three or more groups (the exact “branch functionality” we refer to as f) emanating from a core, connected together by f “strands.” 
Strands can be linear polymer chains, flexible short molecules, rigid struts/ linkers, etc. 
As noted above, junctions and strands in polymer networks can be linked together via physical interactions (e.g., van der Waals interactions, hydrophobic interactions, Coulombic interactions, metalligand coordination) or covalent bonds. 
Hence, polymer networks are conventionally classified as “physical” (supramolecular) or “chemical” (covalent) networks. 
It should be noted that this classification does not always accurately reflect material properties; bond strengths and exchange rates are much more informative. 
For example, given sufficiently strong and static physical interactions, physical networks can behave identically to chemical networks; alternatively, the incorporation of mechanisms for covalent bond exchange can result in chemical networks that exhibit adaptable mechanical properties regulated by external stimuli. 
Thus, the properties of polymer networks can vary widely depending on the composition of the junctions and strands as well as the formation and use conditions. 
With this broad view in mind, nearly all polymer networks, regardless of their colloquial name, structure, properties, etc. can generally be divided into one of four major classes: thermosets, thermoplastics, elastomers, and gels.

Gels are polymer networks constructed from either covalent or supramolecular bonds that are swollen in liquid media such as water or organic solvents. 
The network structure ensures that the liquid is held within the material. 
Gels are usually very soft (Young s moduli of 103–104 Pa) but are often capable of undergoing relatively large deformation. 
Examples of gels include gelatin, fibrin, and polyacrylamide hydrogel.

As perhaps the most important, useful, and broadly studied class of materials from theoretical, academic, and industrial perspectives, polymer networks can have many unique properties, including elasticity, tunable mechanical strength, porosity, and swellability. 
These properties and others have led to numerous applications of polymer networks in everyday life, such as adhesives, cosmetics, sorbents, membranes, rubber products, coating materials, and food packaging. 
Moreover, as recent developments have imparted unconventional properties (e.g., malleability, self-healing, conductivity, ultra-high permanent porosity, enhanced crystallinity, and stimuli-responsiveness) into polymer networks, they continue to hold great promise in advanced applications including drug delivery systems,[3] tissue engineering scaffolds,[4] soft actuators,[5] gas storage,[6] catalysis,[6–8] and electronic materials.[9] 
Thus, though polymer networks have been widely studied for more than a century, there are features of their structure that have only recently been leveraged to impart new properties; an even deeper understanding is needed to realize the next-generation of functional, and ideally sustainable, polymer networks. 

In this review, we introduce key concepts related to the formation, characterization, and properties of polymer networks. 
Our goal is to provide newcomers to the field with broad and up-to-date knowledge that can serve as a starting point for more detailed investigations of topics of interest. 
Major focus is devoted to polymer network structure, which includes both chemical and topological aspects. 
Additionally, several types of recently developed polymer networks with exceptional properties are highlighted.



\end{document}
