\documentclass[../../main-notes.tex]{subfiles}

\begin{document}


\textbf{General description of a hydrogel}

We can describe a hydrogel as networks formed by cross-linked polymer chains that exhibits the abilitiy to swell and retain a significant fraction of water within its structure, but will not dissolve in water\citep{ahmedHydrogelPreparationCharacterization2015a,ahmedHydrogelsMicrogelsDriving2025,priyaComprehensiveReviewHydrogel2024}.\footnote{the main difference with the microgels, is the size. Hydrogel is bulk, and microgelgel is particle.}

cross-linked polymer
networks:

... composed of polymer chains, the properties of the hydrogel are influence by the polymer chains, furthermore, the crosslinking affects various physical properties of the hydrogel\citep{priyaComprehensiveReviewHydrogel2024}.

depending on the properties of the polymer used as well as n the nature and density of the network joints, such structures can contain various amounts of water\citep{ahmedHydrogelPreparationCharacterization2015a}.

hydrgels are generally prepared based on hydrophilic monomers that can reulate the properties for specific applications\citep{ahmedHydrogelPreparationCharacterization2015a}



-------------------------------------------------

One of the most common definition of hydrogel is a water-swollen, and cross-linked polymeric network produced by the simple reaction of one or more monomers\citep{ahmedHydrogelPreparationCharacterization2015a}.

hydrogels are polymer networks extensively swollen with water\citep{ahmedHydrogelPreparationCharacterization2015a}.

Hydrophilic gels that are usually referred to as hydrogels are networks of polymer chains that are sometimes found as colloidal gels in which water is the dispersion medium [1]\citep{ahmedHydrogelPreparationCharacterization2015a}.



Hydrogels have received considerable attention in the past 50 years, due to their exceptional promise in wide range of applications [2–4]\citep{ahmedHydrogelPreparationCharacterization2015a}. 

They possess also a degree of flexibility very similar to natural tissue due to their large water content\citep{ahmedHydrogelPreparationCharacterization2015a}.

Recently, hydrogels have been defined as two- or multicomponent systems consisting of a three-dimensional network of polymer chains and water that fills the space between macromolecules\citep{ahmedHydrogelPreparationCharacterization2015a}.


Hydrogels are three-dimensional networks of hydrophilic polymers that can absorb and retain large amounts of water while maintaining their structure\footnote{Their ability to retain a large amount of water is due to their 3D structure, which gives them a gel-like appearance and behaviour.}\citep{priyaComprehensiveReviewHydrogel2024}. 

The water absorption capacity and network stability of hydrogels are controlled by crosslinking, which involves forming covalent or non-covalent bonds between polymer chains\footnote{The hydrogels are prepared using different methods like chemical cross-linking of monomers, physical cross-linking using temperature or pH changes, and blending of natural or synthetic polymers.}\citep{priyaComprehensiveReviewHydrogel2024}. 

Crosslinkers play a crucial role in providing secondary interactions with biological tissues, and the presence of hydrophilic groups in the polymer chains enhances water uptake [10]\citep{priyaComprehensiveReviewHydrogel2024}. 
These methods allow researchers to create hydrogels with specific properties suitable for various applications such as tissue engineering, biomedicine, and sensing\citep{priyaComprehensiveReviewHydrogel2024}. 
The properties of hydrogels can be tailored based on the nature and arrangement of their constituent monomers, as well as the preparation method employed\citep{priyaComprehensiveReviewHydrogel2024}.


\textbf{Types of Hydrogels} or classification

From\citep{priyaComprehensiveReviewHydrogel2024} 
\begin{enumerate}
    \item Natural Polymer: 
            Natural polymer-derived hydrogels, sourced from plants or animals like polysaccharides and proteins.
            These hydrogels are adept at absorbing and retaining water, effectively managing pesticide release in soil to boost efficacy and minimize environmental harm caused by excessive application.
            Despite challenges like mechanical strength variations inherent in natural sources, natural polymer-derived hydrogels hold great promise for sustainable agriculture and environmental conservation.
            Examples: Cellulose, derivatives such as crboxymethyl celluose, Chitosan, Sodium alginate
    \item Synthetic Polymer: 
            Synthetic polymer hydrogels, such as those made from polyacrylamide (PAM) and PVA.
            controllable structures, mechanical strength, and chemical stability.
            PAM is especially favoured for its water retention and non-toxic nature, making it prevalent in biomedicines and agriculture. 
            However, the use of PAM comes with concerns. 
            Acrylamide, used in PAM synthesis, is potentially neurotoxic and may release unreacted particles, posing environmental and health risks. 
            Additionally, these hydrogels have low biodegradability, causing environmental residues and potential contamination. 
            Production of these synthetic polymers often involves harmful chemicals, increasing costs and raising further health and environmental concerns
    \item Natural-Synthetic Polymer: 
            blending natural polymers like alginate and xanthan gum with synthetic counterparts such as PAM and PVA.
            These hydrogels enhance biodegradability and biocompatibility, mitigate long-term soil and water contamination risks, and provide robust mechanical strength and chemical stability.
\end{enumerate}




\end{document}
