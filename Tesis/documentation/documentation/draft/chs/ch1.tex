%======================================================================
\chapter{Introduction}\label{ch1:Intro}

\markright{Introduction}
%======================================================================

\paragraph{Curiosity/phenomenology} Paragraph that will tell the reader that hydrogels are cool.

\paragraph{Applications/Market size of the applications sectors} If the previous paragraph does not convince the reader, well my last hope is that money does.

Besides, because of such a wide variety of response triggers, hydrogels can serve as sensors or actuators or can be utilized in controlled drug delivery systems, biosensors, tissue engineering scaffolds, and others [20], because of their biomimetic properties and multi functionalities [21]\citep{bustamante-torresHydrogelsClassificationAccording2021}.

In particular, biomedical applications are very popular and include cell culture [5], wound dressing and healing [2,6], drug delivery [2,7,8], tissue engineering scaffolds [9], bone repair [10], and cartilage regeneration [11]\citep{picchioniHydrogelsBasedDynamic2018}. 
 

\paragraph{Description of the Thesis} What the reader will find in each chapter and section.

\paragraph{Why computers and not rheometers?} Explain how in silico experiments can help to understand the relation between the network and the mechanical response.

\section{State of the art: Hydrogels}\label{ch1:StateArt}

\begin{itemize}
    \item Characteristics
    \item Descriptions 
    \item Synthesis techniques
    \item Cross-linking (Bond breaking)
\end{itemize}

\paragraph{General description of a hydrogel}
A hydrogel is commonly describe as a material composed by a network of polymers chains that exhibits the abilitiy to swell and retain a significant fraction of water within its structure, but will not dissolve in water\citep{ahmedHydrogelPreparationCharacterization2015a,ahmedHydrogelsMicrogelsDriving2025,priyaComprehensiveReviewHydrogel2024,bustamante-torresHydrogelsClassificationAccording2021}.\footnote{the main difference with the microgels, is the size. Hydrogel is bulk, and microgelgel is particle.}
The water absorption capacity, network stability of hydrogels, and the conformation of the network with polymer chains are attributable to crosslinking mechanisms\citep{priyaComprehensiveReviewHydrogel2024,ahmedHydrogelPreparationCharacterization2015a}.
Meanwhile, the polymer chains are predominantly composed with hydrophilic functional groups and can be modified to suit a variety of applications\citep{ahmedHydrogelPreparationCharacterization2015a,priyaComprehensiveReviewHydrogel2024,bustamante-torresHydrogelsClassificationAccording2021}.

While the analysis of the impact of functional groups is intriguing, the present project prioritizes the examination of mechanisms that are more pertinent to the mechanical response. 
The crosslinking mechanisms\footnote{The hydrogels are prepared using different methods like chemical cross-linking of monomers, physical cross-linking using temperature or pH changes, and blending of natural or synthetic polymers.}, in particular, are of particular interest, as they are responsible for resisting dissolution. 
This suggests that crosslinking mechanisms enable the network to undergo modification by external stimuli.

The subsequent sections will present the essential information to facilitate a comprehensive understanding of the crosslinking mechanisms, their relationship to the mechanical response, the reported mechanical response of hydrogels, and the correlation between rheology experiments and stress curves.

\subsection{Crosslinking mechanisms}\label{ch1:Cross-linking}

\paragraph{Intro to cross linking}
A crosslinker is a molecule that functions as a bridge between polymer chains, thereby facilitating the formation of an interconnected network.
As previously suggested, it is pertinent to understand the mechanisms of crosslinking in order to gain insight into the correlation between these mechanisms and mechanical properties, including elasticity, viscosity, solubility, glass transition temperature, strength, toughness, and melting point stiffness, swelling capacity, viscosity, and so forth\citep{priyaComprehensiveReviewHydrogel2024}.
The elements under consideration form stable bonds, which are comonly categorized into two main types: covalent (permanent) and physical (reversible)\citep{bustamante-torresHydrogelsClassificationAccording2021}.
However, recent mechanisms, such as mechanical crosslinker mechanics, have been demostrated to form bridges due to the topology of the constituents of the hydrogel.

\begin{figure}[!h]
    \centering
    \includegraphics[width=0.8\textwidth]{pics/LogoTec.jpg}
    \caption{Image with the three different crosslinker mechanisms}
\end{figure}

\paragraph{Difference between physical and chemical bonds}
The distinguishing characteristic between reversible and irreversible crosslinking mechanisms is determined by the energy required to break the bonds and the conditions in which the bonds can be re-formed.
\dots more sentences with references.

\paragraph{General consequences of the differences of energy}
In general, physical crosslinking mechanisms are weaker than chemical ones.
However, numerous interactions contribute to complex behaviors\citep{bustamante-torresHydrogelsClassificationAccording2021}.
On the other hand, chemical crosslinking mechanisms are easier to control than physical crosslinking mechanisms because their preparation is independent of pH\citep{bustamante-torresHydrogelsClassificationAccording2021}. 
However, they are very brittle due to structural inhomogeneity and lack of energy dissipation\citep{xuRoleChemicalPhysical2018}.


\begin{comment}
The incorporation of crosslinkers results in an increase in the molecular weight of the polymer chains, thereby limiting their translational movement and decreasing their solubility, and augment the interactions between the polymer chains\citep{priyaComprehensiveReviewHydrogel2024}.

The cohesion forces that allow the cross-linking of the polymer have a covalent character, and other forces such as electrostatic, hydrophobic, dipole–dipole interactions, or hydrogen bonds intervene [11–13]\citep{bustamante-torresHydrogelsClassificationAccording2021}. 

Similarly, hydrophilic polymers can be cross-linked through chemical bonds, leading to the formation of hydrogels, which are materials that have attracted particular attention in the biomedical field [17]\citep{bustamante-torresHydrogelsClassificationAccording2021}. 

Moreover, since the hydrogel has covalent crosslinking, it can return to its original state after unloading\citep{xuRoleChemicalPhysical2018}.

It was also found that the energy dissipation of HGel with a low chemical crosslinking density was dependent on the hydrophobic interaction at the initial deformation stage\citep{xuRoleChemicalPhysical2018}\footnote{Pascals has the same units as the ``dissipation'' energy, ``dissipation'' energy is the area under the loading and unloading curves.}.
Molecular chains between two chemical crosslinking points were crimped and curled, and therefore the physical crosslinking would play a major role for energy dissipation.

As a result, we can conclude that at initial deformation state the physical crosslinking would play the major role of energy dissipation, whereas the chemical and physical crosslinking would work synergistically in large deformation dissipated more energy to enhance mechanical strength of HGel\citep{xuRoleChemicalPhysical2018}.
The mechanical property of H-Gel was also affected by the ratio of MBA, AAm, HMA and core-shell LPs\citep{xuRoleChemicalPhysical2018}. 
\end{comment}

%\citep{priyaComprehensiveReviewHydrogel2024}
\paragraph{Reversible Cross-linking}
In physically cross-linked hydrogels, polymer chains are held together by molecular entanglements or physicochemical interactions, including van der Waals forces, hydrogen bonds, hydrophobic interactions, charge condensation, and supramolecular chemistry\citep{bustamante-torresHydrogelsClassificationAccording2021}.
The aforementioned interactions enable hydrogels to undergo structural changes without the rupture of any covalent bonds. Consequently, these materials exhibit enhance responsiveness to external stimuli, such as temperature, pH, or ionic strength. Additionally, hydrogels demonstrate high water sensitivity and thermal reversibility\citep{bustamante-torresHydrogelsClassificationAccording2021,priyaComprehensiveReviewHydrogel2024}.
These materials are known to exhibit distinctive properties, including "self-healing" behavior, where the gel can reform after being broken.
The lifespan of these hydrogels is brief, ranging from a few days to a maximum of a month, when maintained within physiological media.

\begin{comment}
Crystalization 
    (This crystallization process includes crystal nucleation and crystal growth)
Amphiphilic Copolymers 
    (amphiphilic copolymers can aggregate in water to form micelles and hydrogels in which the hydrophobic segments of the polymer self-assembly [40]\citep{bustamante-torresHydrogelsClassificationAccording2021}, Cross-linking is thought to occur by hydrophobic interactions [41])
Hydrogel Cross-linking by charge interactions
    Cross-linking (or de-crosslinking) can be achieved in situ by pH changes that cause ionization or protonation of ionic functional groups and cause gelation\citep{bustamante-torresHydrogelsClassificationAccording2021}.
Interaction by Hydrogen Bonds
    Hydrogen bonding interactions can be used to produce hydrogels in vitro by freezethaw cycles. 
Stereo-Complexing
    Stereo-complexing refers to the interactions between polymeric chains, or small molecules, of the same chemical composition but different stereochemistry. 
Protein interaction
    Cross-linking by protein interactions can be accomplished through the use of genetically engineered proteins or antigen–antibody interactions\citep{bustamante-torresHydrogelsClassificationAccording2021}. 
\end{comment}

%\citep{priyaComprehensiveReviewHydrogel2024}
\paragraph{Irreversible Cross-linking}
In the context of chemically cross-linked hydrogels, the formation of covalent bonds between polymer chains is a significant phenomenon.
This process is achieved by polymerization and reactions induced by cross-linking agents.
These covalent bonds exhibit a high degree of strength and stability, leading to a structural arrangement of interconnected polymer chains that is more robust and resistant to alterations in environmental conditions, such as temperature and pH.
Consequently, chemically cross-linked hydrogels generally exhibit greater mechanical strength and long-term stability.  
Furthermore, it generally contains regions of the high cross-linking density and low degree of swelling (clusters), dispersed in the regions of the low cross-linking density and high swelling index due to the hydrophobic aggregation of the cross-linking agent\citep{bustamante-torresHydrogelsClassificationAccording2021}.. 

\begin{comment}
Graft Copolymerization and Irradiation Crosslinking
    This method allows radiation to act on the polymer matrix, inducing the formation of reactive sites that may interact with a molecule to be grafted, initiating a free radical polymerization process [80].
Reactive function groups
    These are covalent reactions between the functional groups of the polymers (mainly -OH, -COOH, -NH2) that provide solubility to water-soluble polymers [99,100]. 
Enzymatic Method
    Cross-linking of the hydrogel occurs under mild conditions without the need for the use of low molecular weight compounds (monomers, initiators, cross-linking agents), irradiation, or prior polymer functionalization to favor its cross-linking [31,119]. Enzymes often exhibit a high degree of substrate specificity, potentially avoiding side reactions during cross-linking. 
\end{comment}

\paragraph{Mecanical bonds}
As previously mentioned at the beggining of the section, a novel class of polymer architecture has recently emerged within the field of polymer science kwnon as mechanically interlocked polymers (MIPs). 
These polymers are distinguished by the presence of a mechanical bond.
A mechanical bond is defined as the constraint of two (or more) molecular components in space without the formation of covalent bonds\citep{hartMaterialPropertiesApplications2021}.
While these types of hydrogels exhibit substantial conformational flexibility while preserving a persistent spatial correlation between their components, their synthesis remains challenging.



\begin{comment}
A key feature that controls the properties of a polymeric material is its architecture. 
Beyond the conventional linear polymer, architectures such as branched, cyclic, bottlebrush, star and block copolymers have expanded the property profile of polymeric materials and offered opportunities for polymer research and applications. 

1–4. 
Conceptually, there are a myriad of ways the mechanical bond can be incorporated into polymer architectures (Fig. 1), and these unique and varied structures can enable property profiles that have never been seen before.
\end{comment}



\begin{comment}
Variations in concentrations, structure, functionality of the monomer, and the cross-linker used in such gels can modify the structure [4]\citep{bustamante-torresHydrogelsClassificationAccording2021}.

Since they are systems in an aqueous medium, it is necessary to consider the traditional variables such as temperature, concentration, pH, and ionic strength\citep{bustamante-torresHydrogelsClassificationAccording2021}. 
Moreover, some hydrogels are capable of responding to external stimuli such as pH, temperature, electricity, light and biological molecules as enzymes during the swelling and shrinking process\citep{bustamante-torresHydrogelsClassificationAccording2021}. 
Therefore, several studies have made it possible to improve mechanical, optical, or swelling behavior, adding another compound with hydrophobic properties to the hydrophilic monomer\citep{bustamante-torresHydrogelsClassificationAccording2021}. 
\end{comment}


\subsection{Mechanical response of hydrogels}\label{ch1:Cross-linking}

\paragraph{Network-mechanical response relation} Introduce the idea of how by understanding the network we can manipulate/control the mechanical response.

The research of hydrophilic polymers has been complex because the physical properties of solubility or swellability depend on different factors, such as the type of polymer, molecular weight, the ratio of polar groups, and degree of cross-linking\citep{bustamante-torresHydrogelsClassificationAccording2021}.
High molecular weight and a high degree of cross-linking will reduce the hydrophilicity of the molecule [18,19]\citep{bustamante-torresHydrogelsClassificationAccording2021}. 


\paragraph{Tunnable mechanical response with applications} Review of articles of applications 



Just describe the phenomena and say that it depends on the structure and so on.

\paragraph{Viscoelasticity}

\paragraph{Yield stress}

\paragraph{Shear thinning}


\paragraph{Transition to talk about printing hydrogels}\citep{correaTranslationalApplicationsHydrogels2021}
Subsequent work developed safer cross-linking mechanisms, which began a trend toward triggering gelation in situ after injection, providing a minimally invasive way of administering  hydrogels to practically any organ or tissue.29,30\footnote{Why the gelation in situ after injection is important?}

Unlike earlier hydrogels that relied on covalent cross-links, some of these hydrogels have self-healing properties and possess mechanical properties akin to native tissue, capable of countering natural forces and stresses of a body in motion.
More recently, shear-thinning hydrogels were developed that are formed through dynamic and reversible cross-linking.34 
For example, physical hydrogels use noncovalent interactions (e.g., supramolecular chemistries) between soluble building blocks in order to self-assemble into a dynamic, reversibly cross-linked  network.35,36 
These “dynamic hydrogels” assembled through reversible cross-links afford the unique property of being injectable even after having formed a gel, due to their shear-thinning and selfhealing behaviors. 
Current research on dynamic hydrogels has revealed novel and useful capabilities that have opened new frontiers for this technology. 

For example, they can stabilize delicate protein and cellular cargoes to combat pharmaceutical  cold-chain limitations,39 they can adhere strongly to tissues to  form protective barriers and bandages,40 and they can be delivered through spray applications to coat complex biological  geometries.41

static covalent cross-links ultimately introduced translational challenges for clinical implementation, since traditional covalent gels require invasive surgical implantation to access nonsuperficial tissues.  
Additionally, new manufacturing processes, such as 3D printing, require dynamic rheological properties during processing, disqualifying the use of traditional covalent  hydrogels.57  
Interest in further developing the translational potential of hydrogels led to innovative methods to implant them through minimally invasive means, of which the most clinically relevant is injection through a needle or catheter (Figure 3).

static covalent cross-links ultimately introduced translational challenges for clinical implementation, since traditional covalent gels require invasive surgical implantation to access nonsuperficial tissues.  
Additionally, new manufacturing processes, such as 3D printing, require dynamic rheological properties during processing, disqualifying the use of traditional covalent  hydrogels.57



Hybrid crosslinking hydrogels consist of covalent and noncovalent crosslinking\citep{xuRoleChemicalPhysical2018}. 
The dynamic physical crosslinking could effectively dissipate energy via destruction and reorganization, and the chemical crosslinking could sustain skeleton construction [19,20]\citep{xuRoleChemicalPhysical2018}. 

The molecular reversibility can be actually achieved in two different ways: either by making use of equilibrium reactions (e.g., the Diels-Alder one) or through dynamic exchange reactions (e.g., reaction of an excess amino groups with epoxide ones)\citep{picchioniHydrogelsBasedDynamic2018}.

The general idea is that the use of dynamic covalent bonds allows the polymeric network to adjust itself as a result of an external stimulus\citep{picchioniHydrogelsBasedDynamic2018}. 
This can be achieved in principle through other weaker interactions, e.g., hydrogen bonding\citep{picchioniHydrogelsBasedDynamic2018}. 

the use of covalent bonds displays two distinct and clear advantages [25]. 
In first instance, the network is still covalently linked, which renders it quite robust against small random fluctuations in environmental conditions such as temperature. 
Furthermore, exchange reactions such as the one of an amine with an imine are often kinetically controlled by the use of catalysts. 
In turn, this allows the possibility to freeze the network conformation (by slowing the kinetics) when desired\citep{picchioniHydrogelsBasedDynamic2018}. 
The general concept behind the use of reversible interactions for the hydrogel polymeric chains is the (reversible) network disruption with immediate release of any loading (Figure 3).

Reversible bonds can be incorporated along the backbone (red circles) or at the crosslinking point (green triangles). 
The network, when subjected to an appropriate external stimulus, can then break at the crosslinking point (route A) or along the backbone (route B). 
This generates network fragments that can be quite different in terms of chemical structure even if in both cases the loading (blue circles) will be released. 
As a result of the network disruption, the load is released as the polymeric chains become soluble and not able anymore to entrap the load\citep{picchioniHydrogelsBasedDynamic2018}. 

\subsection{Rheology/stress}\label{ch1:NetworkStructure}

Main review:\citep{guPolymerNetworksPlastics2020,sheikoArchitecturalCodeRubber2019}

\paragraph{Bridge of the experiments and interpretation} Hysteresis curves to get the sotred energy and the dissipated energy.

\paragraph{Name some network structures} The correlation between the structure with the hysteresis loops

\paragraph{Link to mechanical response} Same as before.


\paragraph{What if we can change the structure on command and in real time?} Bridge to crosslinkers.

\paragraph{How crosslinking affects the mechanical response}

\begin{comment}
These include one-step procedures like polymerization and parallel cross-linking of multifunctional monomers, as well as multiple step procedures involving synthesis of polymer molecules having reactive groups and their subsequent cross-linking, possibly also by reacting polymers with suitable cross-linking agents\citep{priyaComprehensiveReviewHydrogel2024}. 

The polymer engineer can design and synthesize polymer networks with molecular-scale control over structure such as cross-linking density and with tailored properties, such as biodegradation, mechanical strength, and chemical and biological response to stimuli\citep{priyaComprehensiveReviewHydrogel2024}.

In general, the three integral parts of the hydrogels preparation are monomer, initiator, and cross-linker. 
To control the heat of polymerization and the final hydrogels properties, diluents can be used, such as water or other aqueous solutions\citep{ahmedHydrogelPreparationCharacterization2015a}. 
Then, the hydrogel mass needs to be washed to remove impurities left from the preparation process\citep{ahmedHydrogelPreparationCharacterization2015a}. 
These include nonreacted monomer, initiators, cross-linkers, and unwanted products produced via side reactions\citep{ahmedHydrogelPreparationCharacterization2015a}.

\end{comment}


Crosslinking is another essential process that can be controlled and intentionally modified using ionizing radiations\citep{priyaComprehensiveReviewHydrogel2024}. 




\begin{comment}

-------------------------------------------------

cross-linked polymer

In general, the cross-linkers increases the molecular weight of the polymer chains, which, in turn, limits their translational movement and decreases the solubility of the polymer\citep{priyaComprehensiveReviewHydrogel2024}.

Controlling the crosslinking process with ionizing radiations can be achieved by maniipulating various parameters during exposure, such as exposure time of radiaiton, frquency, temperature and pressure\citep{priyaComprehensiveReviewHydrogel2024}.




-------------------------------------------------

Hydrogels have received considerable attention in the past 50 years, due to their exceptional promise in wide range of applications [2–4]\citep{ahmedHydrogelPreparationCharacterization2015a}. 

They possess also a degree of flexibility very similar to natural tissue due to their large water content\citep{ahmedHydrogelPreparationCharacterization2015a}.

Recently, hydrogels have been defined as two- or multicomponent systems consisting of a three-dimensional network of polymer chains and water that fills the space between macromolecules\citep{ahmedHydrogelPreparationCharacterization2015a}.

Hydrogels are three-dimensional networks of hydrophilic polymers that can absorb and retain large amounts of water while maintaining their structure\footnote{Their ability to retain a large amount of water is due to their 3D structure, which gives them a gel-like appearance and behaviour.}\citep{priyaComprehensiveReviewHydrogel2024}. 

Crosslinkers play a crucial role in providing secondary interactions with biological tissues, and the presence of hydrophilic groups in the polymer chains enhances water uptake [10]\citep{priyaComprehensiveReviewHydrogel2024}. 

These methods allow researchers to create hydrogels with specific properties suitable for various applications such as tissue engineering, biomedicine, and sensing\citep{priyaComprehensiveReviewHydrogel2024}. 
The properties of hydrogels can be tailored based on the nature and arrangement of their constituent monomers, as well as the preparation method employed\citep{priyaComprehensiveReviewHydrogel2024}.


\textbf{Types of Hydrogels} or classification

From\citep{priyaComprehensiveReviewHydrogel2024} 
\begin{enumerate}
    \item Natural Polymer: 
            Natural polymer-derived hydrogels, sourced from plants or animals like polysaccharides and proteins.
            These hydrogels are adept at absorbing and retaining water, effectively managing pesticide release in soil to boost efficacy and minimize environmental harm caused by excessive application.
            Despite challenges like mechanical strength variations inherent in natural sources, natural polymer-derived hydrogels hold great promise for sustainable agriculture and environmental conservation.
            Examples: Cellulose, derivatives such as crboxymethyl celluose, Chitosan, Sodium alginate
    \item Synthetic Polymer: 
            Synthetic polymer hydrogels, such as those made from polyacrylamide (PAM) and PVA.
            controllable structures, mechanical strength, and chemical stability.
            PAM is especially favoured for its water retention and non-toxic nature, making it prevalent in biomedicines and agriculture. 
            However, the use of PAM comes with concerns. 
            Acrylamide, used in PAM synthesis, is potentially neurotoxic and may release unreacted particles, posing environmental and health risks. 
            Additionally, these hydrogels have low biodegradability, causing environmental residues and potential contamination. 
            Production of these synthetic polymers often involves harmful chemicals, increasing costs and raising further health and environmental concerns
    \item Natural-Synthetic Polymer: 
            blending natural polymers like alginate and xanthan gum with synthetic counterparts such as PAM and PVA.
            These hydrogels enhance biodegradability and biocompatibility, mitigate long-term soil and water contamination risks, and provide robust mechanical strength and chemical stability.
\end{enumerate}



\end{comment}
 



\newpage
