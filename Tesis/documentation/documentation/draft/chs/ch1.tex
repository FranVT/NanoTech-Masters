%======================================================================
\chapter{Introduction}\label{ch1:Intro}

\markright{Introduction}
%======================================================================

\paragraph{Curiosity/phenomenology} Paragraph that will tell the reader that hydrogels are cool.

\paragraph{Applications/Market size of the applications sectors} If the previous paragraph does not convince the reader, well my last hope is that money does.

\paragraph{Description of the Thesis} What the reader will find in each chapter and section.

\section{State of the art}\label{ch1:StateArt}

\paragraph{Network-mechanical response relation} Introduce the idea of how by understanding the network we can manipulate/control the mechanical response.

\paragraph{Tunnable mechanical response with applications} Review of articles of applications 

\paragraph{Why computers and not rheometers?} Explain\footnote{that Tec didn't pay the bills for a lab.} how in silico experiments can help to understand the relation between the network and the mechanical response.


\subsection{Hydrogels}\label{ch1:Hydrogels}

\begin{itemize}
    \item Characteristics
    \item Descriptions 
    \item Synthesis techniques
    \item Cross-linking (Bond breaking)
\end{itemize}


\paragraph{General description of a hydrogel}
We can describe a hydrogel as networks formed by cross-linked polymer chains that exhibits the abilitiy to swell and retain a significant fraction of water within its structure, but will not dissolve in water\citep{ahmedHydrogelPreparationCharacterization2015a,ahmedHydrogelsMicrogelsDriving2025,priyaComprehensiveReviewHydrogel2024}.\footnote{the main difference with the microgels, is the size. Hydrogel is bulk, and microgelgel is particle.}
The water absorption capacity and network stability of hydrogels can be controlled by crosslinking mechanisms, which involves forming covalent or non-covalent bonds between polymer chains\footnote{The hydrogels are prepared using different methods like chemical cross-linking of monomers, physical cross-linking using temperature or pH changes, and blending of natural or synthetic polymers.}\citep{priyaComprehensiveReviewHydrogel2024,ahmedHydrogelPreparationCharacterization2015a}. 
On the other hand, hydrogels are generally prepared based on hydrophilic monomers that can reulate the properties for specific applications\citep{ahmedHydrogelPreparationCharacterization2015a,priyaComprehensiveReviewHydrogel2024}.

\paragraph{Transtition to talk about crosslink}
A general correlation exists between the mechanical properties\footnote{Such as elasticity, viscosity, solubility, glass transition temperature, strength, toughness, and melting point\citep{priyaComprehensiveReviewHydrogel2024}} of the hydrogel and its crosslinking mechanisms.
The incorporation of crosslinkers results in an increase in the molecular weight of the polymer chains, thereby limiting their translational movement and decreasing their solubility, and augment the interactions between the polymer chains\citep{priyaComprehensiveReviewHydrogel2024}.
Since we are interested in the mechanical response of the material, our focus in this section is to explore the crosslinking mechanisms in hydrogels.

\paragraph{Difference between physical and chemical bonds}
Crosslinking mechanisms involves the formation of \textit{bonds}\footnote{I'm not very sure to use this word.} between polymer chains that can be reversible or irreversible.
Normally referred as \textit{physical cross-linking} and \textit{chemical cross-linking}.
Sentences about theebergy difference to break those bonds.
I have the intuition that the basic difference is the energy required for braking in given conditions.


\paragraph{Physical Cross-linking}\citep{priyaComprehensiveReviewHydrogel2024}
In physically cross-linked hydrogels, the interactions between polymer chains are not covalent but rather based on physical interactions. 
These interactions can include hydrogen bonding, van der Waals forces, hydrophobic interactions, or coordination bonds and are reversible under certain conditions, which means that the hydrogel can undergo structural changes without breaking any covalent bonds.
This characteristic makes physically cross-linked hydrogels more responsive to external stimuli like temperature, pH, or ionic strength and have high water sensitivity and thermal reversibility. 
They may exhibit unique properties, such as “self-healing” behaviour, where the gel can reform after being broken.
These kinds of hydrogel have a short lifespan, in the range of a few days to a maximum of a month, in the physiological media. 


\paragraph{Chemical Cross-linking}\citep{priyaComprehensiveReviewHydrogel2024}
In chemically cross-linked hydrogels, covalent bonds form between the polymer chains.
These covalent bonds are strong and stable, resulting in a 3D structure of interconnected polymer chains more robust and resistant to changes in environmental conditions, such as temperature and pH. 
The cross-links are typically formed through chemical reactions, such as polymerization or cross-linking agent-induced reactions. 
As a result, chemically cross-linked hydrogels generally exhibit greater mechanical strength and long-term stability. 

Chemically crosslinked hydrogels are easier to control as compared to physical hydrogels as their preparation method and applications are not dependent on their pH.

This preparation of hydrogel networks is easy to control when compared to physical hydrogels as their preparation and the applications they are used for are not dependent on their pH. 

\paragraph{Mecanical bonds}\citep{hartMaterialPropertiesApplications2021}
Mechanical bond is when two (or more) molecular components are constrained in space without being covalently bonded together.

The  itself is not a new idea, occurring  (Fig. 1)

Mechanically interlocked molecules (MIMs) possess large conformational freedom while maintaining a permanent spatial association between the components1–4. 

MIMs have played an important role in the field of molecular switches and molecular machines, and were recognized in 2016 with the Nobel Prize in Chemistry being awarded to Jean-Pierre Sauvage, Sir J. Fraser Stoddart and Bernard L. Feringa5–8. 

However, MIMs are not confined to this field, and have been explored in applications that range from drug delivery to catalysis9–15.

MIPs present an attractive frontier in polymer science, as the presence of the mechanical bond allows for unprecedented degrees of motion within the polymer architecture. 
Conceptually, there are a myriad of ways the mechanical bond can be incorporated into polymer architectures (Fig. 1), and these unique and varied structures can enable property profiles that have never been seen before.



A key feature that controls the properties of a polymeric material is its architecture. 
Beyond the conventional linear polymer, architectures such as branched, cyclic, bottlebrush, star and block copolymers have expanded the property profile of polymeric materials and offered opportunities for polymer research and applications. 
Recently, the polymer field has seen the emergence of a new class of polymer architecture: mechanically interlocked polymers (MIPs), which are polymers that include a mechanical bond.


\begin{comment}
These include one-step procedures like polymerization and parallel cross-linking of multifunctional monomers, as well as multiple step procedures involving synthesis of polymer molecules having reactive groups and their subsequent cross-linking, possibly also by reacting polymers with suitable cross-linking agents\citep{priyaComprehensiveReviewHydrogel2024}. 

The polymer engineer can design and synthesize polymer networks with molecular-scale control over structure such as cross-linking density and with tailored properties, such as biodegradation, mechanical strength, and chemical and biological response to stimuli\citep{priyaComprehensiveReviewHydrogel2024}.

In general, the three integral parts of the hydrogels preparation are monomer, initiator, and cross-linker. 
To control the heat of polymerization and the final hydrogels properties, diluents can be used, such as water or other aqueous solutions\citep{ahmedHydrogelPreparationCharacterization2015a}. 
Then, the hydrogel mass needs to be washed to remove impurities left from the preparation process\citep{ahmedHydrogelPreparationCharacterization2015a}. 
These include nonreacted monomer, initiators, cross-linkers, and unwanted products produced via side reactions\citep{ahmedHydrogelPreparationCharacterization2015a}.

\end{comment}


Crosslinking is another essential process that can be controlled and intentionally modified using ionizing radiations\citep{priyaComprehensiveReviewHydrogel2024}. 




\begin{comment}

-------------------------------------------------

cross-linked polymer

In general, the cross-linkers increases the molecular weight of the polymer chains, which, in turn, limits their translational movement and decreases the solubility of the polymer\citep{priyaComprehensiveReviewHydrogel2024}.

Controlling the crosslinking process with ionizing radiations can be achieved by maniipulating various parameters during exposure, such as exposure time of radiaiton, frquency, temperature and pressure\citep{priyaComprehensiveReviewHydrogel2024}.




-------------------------------------------------

Hydrogels have received considerable attention in the past 50 years, due to their exceptional promise in wide range of applications [2–4]\citep{ahmedHydrogelPreparationCharacterization2015a}. 

They possess also a degree of flexibility very similar to natural tissue due to their large water content\citep{ahmedHydrogelPreparationCharacterization2015a}.

Recently, hydrogels have been defined as two- or multicomponent systems consisting of a three-dimensional network of polymer chains and water that fills the space between macromolecules\citep{ahmedHydrogelPreparationCharacterization2015a}.

Hydrogels are three-dimensional networks of hydrophilic polymers that can absorb and retain large amounts of water while maintaining their structure\footnote{Their ability to retain a large amount of water is due to their 3D structure, which gives them a gel-like appearance and behaviour.}\citep{priyaComprehensiveReviewHydrogel2024}. 

Crosslinkers play a crucial role in providing secondary interactions with biological tissues, and the presence of hydrophilic groups in the polymer chains enhances water uptake [10]\citep{priyaComprehensiveReviewHydrogel2024}. 

These methods allow researchers to create hydrogels with specific properties suitable for various applications such as tissue engineering, biomedicine, and sensing\citep{priyaComprehensiveReviewHydrogel2024}. 
The properties of hydrogels can be tailored based on the nature and arrangement of their constituent monomers, as well as the preparation method employed\citep{priyaComprehensiveReviewHydrogel2024}.


\textbf{Types of Hydrogels} or classification

From\citep{priyaComprehensiveReviewHydrogel2024} 
\begin{enumerate}
    \item Natural Polymer: 
            Natural polymer-derived hydrogels, sourced from plants or animals like polysaccharides and proteins.
            These hydrogels are adept at absorbing and retaining water, effectively managing pesticide release in soil to boost efficacy and minimize environmental harm caused by excessive application.
            Despite challenges like mechanical strength variations inherent in natural sources, natural polymer-derived hydrogels hold great promise for sustainable agriculture and environmental conservation.
            Examples: Cellulose, derivatives such as crboxymethyl celluose, Chitosan, Sodium alginate
    \item Synthetic Polymer: 
            Synthetic polymer hydrogels, such as those made from polyacrylamide (PAM) and PVA.
            controllable structures, mechanical strength, and chemical stability.
            PAM is especially favoured for its water retention and non-toxic nature, making it prevalent in biomedicines and agriculture. 
            However, the use of PAM comes with concerns. 
            Acrylamide, used in PAM synthesis, is potentially neurotoxic and may release unreacted particles, posing environmental and health risks. 
            Additionally, these hydrogels have low biodegradability, causing environmental residues and potential contamination. 
            Production of these synthetic polymers often involves harmful chemicals, increasing costs and raising further health and environmental concerns
    \item Natural-Synthetic Polymer: 
            blending natural polymers like alginate and xanthan gum with synthetic counterparts such as PAM and PVA.
            These hydrogels enhance biodegradability and biocompatibility, mitigate long-term soil and water contamination risks, and provide robust mechanical strength and chemical stability.
\end{enumerate}



\end{comment}
 



\newpage
