\chapter{Patchy particle scheme for hydrophilic polymeric networks}

Now that we have covered the theoretical framework, we can delve into the numerical tools that will help us find relations between the polymeric network and the mechanical response.
First, we will describe the patchy particle scheme for simulating PNIPAM cross-linked networks.
Then, we will describe the numerical simulation protocol.
Next, we will introduce the LAMMPS package and explain how it can be used to simulate these systems.
Finally, we will present and analyze the simulation results.

\section{Simulation protocol}

One of the microgels that has been the focus of significant research is the type that is based on PNIPAM cross-linked networks.
In the article \textit{In silico Synthesis of Microgel Particles}, the authors present a flexible numerical protocol capable of designing individual microgel particles based on PNIPAM corr-linked networks. 
This protocol can generate particles with properties comparable to the experimental ones.
In this project, we employ a similar protocol to explore its versatility and identify a numerical tool that can facilitate connections between network configuration and mechanical response.

Our primary focus is on creating networks without spherical confinement and without mimicking the swelling behavior of PINIPAM microgels with temperature.
Therefore, the central strategy involves the implementation of a binary mixture of patchy particles to generate a disroded polymeric network structure, followed by the application of shear deformation.
The primary benefit of this protocol is that previous numerical efforts in microgel modeling have predominantly concentrated on unrealistic networks consisting of chains of equivalent length, frequently establishing cross-linked connections on crystalline lattice regions or where closed polymer networks are assembled by directly integrating randomly dispersed cross-linkers with polymer chains.

\subsection{Patchy particles representation}

A patchy particle\citep{bianchiPhaseDiagramPatchy2006,bianchiTheoreticalNumericalStudy2008} can be defined as a sphere with radius $r$ containing $n$ spheres of radius $l<r$ on its surface.
The smaller spheres are typically referred to as ``patches'' and the number of patches is often refer to as ``functionality''.
The center of the patches can be placed on the surface of the central particle. 
However, it can also be modified to be at a point inside the enclosed volume of the main particle.

The implementation of patchy particles as monomers and crosslinkers is a highly effective strategy.
This is due to the fact that it facilitates the integration of the infinitesimal representation by the Langevin dynamics with a particle that possesses volume and functionality.
The functionality representation is important because it allows for the representation of the monomer and cross-linker molecules that can form a polymeric network.
However, it is important to recognize that the geometry of the monomers and functional groups is assume to be spherical.

Finally, to define the volume of the particle, a repulsive pairwise interaction is defined between the central particles.
Meanwhile, to the formation of a polymeric network is facilitated by an attractive pairwise interaction defined between patches.
Because this model is designed to simulate the final network, not the synthesis process, the pairswise interaction between central particles and patches is not defined.

In contrast, the softness explain by particle interactions is characterized by the form of the repulsive pair potential between two particles.
Finally, the particle volume fraction contributes to the ability of the particles to deform or compress, in contrast to hard spheres\footnote{The patchy particles are hard spheres, but the hydrogel network is a soft ``particle''}\citep{vlassopoulosTunableRheologyDense2014}.

\subsection{Description of the system}

\paragraph{Interaction potentials} We start by describing the interaction potentials between patchy particles.
The interaction between the central particles is modeled with a Weeks-Chandler-Andersen repulsive potential,
\begin{gather}
    U_{WCA}(r_{i,j}) =\left\{ 
        \begin{array}{ll}
            4\epsilon_{i,j}\left[\qty(\frac{\sigma}{r_{i,j}})^{12}-\qty(\frac{\sigma}{r_{i,j}})^6\right]+\epsilon_{i,j}, & r_{i,j}\in[0,2^{1/6}\sigma], \\
            0, & r_{i,j}>2^{1/6}\sigma
        \end{array}
\right.
    ,\label{eqn:CL-MO_interaction}
\end{gather}
where $r_{i,j}$ is the distance between the center of the central particles, $\sigma$ is the diameter of the particles and $\epsilon_{i,j}$ is the energy of the interacton.
On the other hand, the patch-patch interaction is modeled with an attractive potential,
\begin{gather}
    U_{\mathrm{patchy}}\qty(r_{\mu\upsilon}) = \left\{
        \begin{array}{ll}
            2\epsilon_{\mu\upsilon}\left(\frac{\sigma_p^4}{2 r_{\mu\upsilon}^4}-1\right)\exp\left[\frac{\sigma_p}{\qty(r_{\mu\upsilon}-r_{c})}+2\right], & r_{\mu\upsilon}\in\qty[0,r_c], \\
            0, & r_{\mu,\upsilon}>r_c,
        \end{array}
            \right.\label{eqn:patch-patch_interaction}
\end{gather}
where $r_{\mu\upsilon}$ is the distance between two patches, $\sigma_p$ is the diameter of the patches, $r_c$ is the cut distance of interaction set to $1.5\sigma_p$ and $\epsilon_{\mu,\upsilon}$ is the interaction energy between the patches.
This potential can be interpreted as a reversible interaction.

It is important to say that if we let the polymeric network form with only those potentials, the patches are going to form cluster of more then 2 patches, which is not desirble since this will mean that every single monomer can be a crossliker\footnote{Mejorar este enunciado/idea}.
Hence, the interaction between patches is complemented by a three-body repulsive potential, defined in terms of~\eqref{eqn:patch-patch_interaction}, that provides an efficient bond-swapping mechanism making possible to easily equilibrate the system at extremely low temperatures, while at the same time, reataining the single-bond-per-patch condition\citep{sciortinoThreebodyPotentialSimulating2017},
\begin{gather}
    U_{\mathrm{swap}}(r_{l,m},r_{l,n}) = w\sum_{l,m,n}\epsilon_{m,n}U_3\qty(r_{l,m})U_3\qty(r_{l,n}),\quad r_{l,n}\in\qty[0,r_c],\label{eqn:swap_interaction}
\end{gather}
where
\begin{gather}
    U_{3}\qty(r) = \left\{
        \begin{array}{ll}
            1 & r\in\qty[0,r_{\min}], \\
            -U_{\mathrm{patchy}}\qty(r)/\epsilon_{m,n}, & r\in\qty[r_{\min},r_c]
        \end{array}
        \right.\label{eqn:swapmod_interaction}.
\end{gather}
The sum in~\eqref{eqn:swap_interaction} runs over all triples of bonded patches (patch $l$ bonded both with $m$ and $n$).
$r_{l,m}$ and $r_{l,n}$ are the distances between the reference patch and the other two patches.
The parameter $\epsilon_{m,n}$ is the energy of repulsion and $w$ is used to tuned the swapping ($w=1$) and non-swapping bonds ($w\gg1$). 
The cut off distance $r_c$ is the same as in the potential of interaction between patches, meanwhile the minimum distance $r_{\min}$ is the distance at the minimum of~\eqref{eqn:patch-patch_interaction}, \textit{i.e.} $\epsilon_{m,n}\equiv\abs{U_{\mathrm{patchy}}(r_{\min})}$.
Finally, the energy of interaction between crosslinker patches ($\epsilon_{\mu^i,\mu^i}$) are set to $0$ to allow only crosslinker-monomer and monomer-monomer bonding (figure~\ref{fig:intento2}).

\paragraph{Polymeric network parameters}
Now that the interaction between pathcy particles have been described, we can describe the control parameters for the simulations.
We set a constant number of patchy particles $N_p$, a packing fraction $\phi$ and a cross-link concentration $c$. 
From this parameters we compute the volume of the box and the number of patchy particles of functionality 2 and the patchy particles of functionality 4.

From time and computational constrains we set the total number of partiles to be $N_p=\num{8000}$.
Which is a lower number of particles with respect other simulations[cites].
Therefore, to compesate that we take assemble averages of 5 experiments, which is the same as simulating a system of \num{40000} particles, but with a more manageable computation requirements.

The volume of the box was computed by computing the volume of the patchya particle A and B and scaling those values by the number of particles and the desire packing fraction.
\begin{align*}
    V_{\mathrm{box}} &= \frac{N_{\mathrm{patchyA}}V_{\mathrm{patchyA}}+N_{\mathrm{patchyB}}V_{\mathrm{patchyB}}}{\phi}
\end{align*}
The number of patchy particle of type A is computed as $N_{\mathrm{patchyA}} = c N_p$ and the number of patchy particles of type B as $N_{\mathrm{patchyB}}= N_p - N_{\mathrm{patchyA}}= N_p(1 - c )$.

The temperature was set to be constat thru all the assembly process $T=0.05$ in Lennard-Jones units.
The damp parameter was set to $\mathrm{damp}=0.01$, and it was a mess to get that value.

The damp controlls the viscous response from the interaction between the thermal bath and the particles, which represents the interaction between water molecules and the polymer network.

\subsection{Ddescription of the deformation}

We use couette flow deformation or shear deformation because that is the deformation that we can implement computationationally.

We can describe any deformaiton as a combinaiton of tensile and shear deformation.
Since the most common application, such as printing\ldots, are describe as a couette or poiselle flow are shear deformations, we decided to explore a shear deformation.
Tensile or compression deformation are not cool because buuu.


We decided to apply shear deformation because we c

Shear forces dominate biological environments where hydrogels are typically deployed. 
In vivo, tissues and materials experience complex multiaxial loading conditions rather than simple uniaxial tension.

Shear testing provides a more uniform stress field throughout the hydrogel sample compared to tensile testing. 
In rheological measurements using parallel plate or cone-and-plate geometries, the applied shear stress is distributed evenly across the sample, eliminating edge effects and stress concentrations that plague tensile testing.
Shear rheometry excels at characterizing the complex viscoelastic properties that define hydrogel functionality.
Many hydrogels exhibit shear-thinning behavior that is critical for applications like injection and 3D bioprinting. 


\paragraph{Deformation protocol} Constant shear rate, vary shear rate between simulations.
First goes the assembly protocol then the shear deformation.
Go beyond the elastic limit to see what's up.


\section{LAMMPS}

LAMMPS (Large-scale Atomic/Molecular Massively Parallel Simulator) is a highly flexible, open-source molecular dynamics software solution used for simulating atomic, molecular, and mesoscale systems. 
And is widely recognized for its extensibility, complete documentation, and active support in scientific communities focused on materials modeling and molecular simulation.
It is designed to efficiently model materials science, chemistry, and physics problems by enabling large-scale simulations on parallel computing architectures.
Furthermore, its parallelized structure allows efficient computation of large or complex systems, and supports integration with other computational tools and machine learning methods.

To define a simulation with this software we need to create an input script.
This input script is defined as a series of lines, with each line beginning with a command name and followed by one or more arguments separated by whitespace.
The program incorporates programming commands that define variables, perform conditional tests, execute loops, or invoke shell commands to launch an external program.
The input script is parsed and executed one line at a time. 
This feature allows a single script to run a simulation in stages, alter one or more parameters between stages, or run a series of independent simulations where the entire system is reinitialized multiple times.

\begin{itemize}
    \item \verb|table| command
    \item \verb|langevin| command
    \item \verb|threebody| command
    \item \verb|boundary| command
\end{itemize}

although on lammps we use the fix nve, the simulation does not goes with constant energy.
This fix invokes the velocity form of the Stoermer-Verlet time integration algorithm (velocity-Verlet).
Which in the case of this molecular dynamics is makes a constant temperature.
Which allows us to create a NVT ensemblle.
Also, the constant volume it ensure with the fix deform, because Note that changing the tilt factors of a triclinic box does not change its volume.


\subsection{Three-body potential}

%An importante technica issue to address is the tabulation of the threebody potential to introduce the swap potential into LAMMPS.
This is because the forces on all three particles $I$, $J$, and $K$ of a triplet of this type of three-body interaction potential lie within the plane defined by the three inter-particle distance vectors $\vec{r}_{IJ}$, $\vec{r}_{IK}$ and $\vec{r}_{JK}$.
This property is used to project the forces onto the inter-particle distance vectors.
Hence, we need to create a table taking that into consideration, for that we \textbf{\ldots}

Deformation.

\section{Results}

\subsection{Mechanical response}

Strain stress graph

\begin{figure}[ht!]
    \centering
    \includegraphics[width=\textwidth]{figs/ComputaitonalResults/strain-vs-stressxy.png}
    \caption{results from computational results test}
\end{figure}


\subsection{Network analysis}

I guess that figures of the final network and parameter of order or size of porous or whatever.

