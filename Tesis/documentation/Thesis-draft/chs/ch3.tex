\chapter{Patchy particle scheme for hydrophilic polymeric networks}

\section{Simulation protocol}

Description of the system

\subsection{Description of the system}

\paragraph{Physical Parameters} Particle size, box length, packing fraction, temperature.

\paragraph{Interaction potentials} The interaction between the central particles is modeled with a Weeks-Chandler-Andersen repulsive potential,
\begin{gather}
    U_{WCA}(r_{i,j}) =\left\{ 
        \begin{array}{ll}
            4\epsilon_{i,j}\left[\qty(\frac{\sigma}{r_{i,j}})^{12}-\qty(\frac{\sigma}{r_{i,j}})^6\right]+\epsilon_{i,j}, & r_{i,j}\in[0,2^{1/6}\sigma], \\
            0, & r_{i,j}>2^{1/6}\sigma
        \end{array}
\right.
    ,\label{eqn:CL-MO_interaction}
\end{gather}
where $r_{i,j}$ is the distance between the center of the central particles, $\sigma$ is the diameter of the particles and $\epsilon_{i,j}$ is the energy of the interacton.
The patch-patch interaction is modeled with an attractive potential,
\begin{gather}
    U_{\mathrm{patchy}}\qty(r_{\mu\upsilon}) = \left\{
        \begin{array}{ll}
            2\epsilon_{\mu\upsilon}\left(\frac{\sigma_p^4}{2 r_{\mu\upsilon}^4}-1\right)\exp\left[\frac{\sigma_p}{\qty(r_{\mu\upsilon}-r_{c})}+2\right], & r_{\mu\upsilon}\in\qty[0,r_c], \\
            0, & r_{\mu,\upsilon}>r_c,
        \end{array}
            \right.\label{eqn:patch-patch_interaction}
\end{gather}
where $r_{\mu\upsilon}$ is the distance between two patches, $\sigma_p$ is the diameter of the patches, $r_c$ is the cut distance of interaction set to $1.5\sigma_p$ and $\epsilon_{\mu,\upsilon}$ is the interaction energy between the patches.
Moreover, the interaction between patches is complemented by a three-body repulsive potential, defined in terms of~\eqref{eqn:patch-patch_interaction}, that provides an efficient bond-swapping mechanism making possible to easily equilibrate the system at extremely low temperatures, while at the same time, reataining the single-bond-per-patch condition\citep{sciortinoThreebodyPotentialSimulating2017},
\begin{gather}
    U_{\mathrm{swap}}(r_{l,m},r_{l,n}) = w\sum_{l,m,n}\epsilon_{m,n}U_3\qty(r_{l,m})U_3\qty(r_{l,n}),\quad r_{l,n}\in\qty[0,r_c],\label{eqn:swap_interaction}
\end{gather}
where
\begin{gather}
    U_{3}\qty(r) = \left\{
        \begin{array}{ll}
            1 & r\in\qty[0,r_{\min}], \\
            -U_{\mathrm{patchy}}\qty(r)/\epsilon_{m,n}, & r\in\qty[r_{\min},r_c]
        \end{array}
        \right.\label{eqn:swapmod_interaction}.
\end{gather}
The sum in~\eqref{eqn:swap_interaction} runs over all triples of bonded patches (patch $l$ bonded both with $m$ and $n$).
$r_{l,m}$ and $r_{l,n}$ are the distances between the reference patch and the other two patches.
The parameter $\epsilon_{m,n}$ is the energy of repulsion and $w$ is used to tuned the swapping ($w=1$) and non-swapping bonds ($w\gg1$). 
The cut off distance $r_c$ is the same as in the potential of interaction between patches, meanwhile the minimum distance $r_{\min}$ is the distance at the minimum of~\eqref{eqn:patch-patch_interaction}, \textit{i.e.} $\epsilon_{m,n}\equiv\abs{U_{\mathrm{patchy}}(r_{\min})}$.
Finally, the energy of interaction between crosslinker patches ($\epsilon_{\mu^i,\mu^i}$) are set to $0$ to allow only crosslinker-monomer and monomer-monomer bonding (figure~\ref{fig:intento2}).

\subsection{LAMMPS}

A LAMMPS input script (text file) is simply a series of lines each beginning with a command name followed by one or more whitespace separated arguments. 
Programming like commands are included which define variables, perform conditional tests, execute loops, or invoke shell commands, to launch a program external to LAMMPS.
The input script is parsed and executed one line at a time which means a single script can be used to run a simulation in stages, altering one or more parameters between the stages, or to run a series of independent simulations where the entire system is reinitialized multiple times.



\paragraph{Implementation} About tables and the general script.


\section{Results}

So.. The results and stuff

\subsection{Mechanical response}

Strain stress graph

\subsection{Network analysis}

I guess that figures of the final network and parameter of order or size of porous or whatever.

