%======================================================================
\chapter{Theoretical framework}
%======================================================================


Introducción

Objetivo y pendejada y media de la Tesis

Aplicaciones de los hidrogeles.
Cuáles son las preguntas abiertas.



Marco Teórico

Introducción
Iniciar con redes poliméricas.
Despés pasar a geles.
Después a hidrogeles.

Respuesta mecánica
Modelos ideales
Tipos de respuesta mecánicas
Respuesta mecánica observada en hidrogeles

Conección Part 1
Relación entre la estructura con la respuesta mecánica en los hidrogeles

Conección parte 2
Buscamos hacer una ecuación constitutiva usando dinámica molecular.
En vez de analizar una ecuación constitutiva, buscamos un hamiltoniano que permita modelar la dinámica entre partículas para obtener la respuesta mecánica.


Métodos numéricos

----------

De acuerdo a la literatura, un hidrogel es una red polimérica que se puede hinchar en un solvente sin que la red polimérica se disuelva.
Gran parte de la descripción del hidrogel se centra en describir la red polimérica del material.
El marco téorico es una recopilación decente de las características del hidrogel que se puede encontrar en la literatura.
Posteriormente, se hablará de la modelación del sistema y la metodología para resolver las ecuaciones diferenciales que modelan el sistema.

El objetivo de la Tesis es buscar una herramienta que permita explorar la evolución de la red polimerica de un hidrogel durante procesos de deformación para poder conectar la respuesta macroscópica del material con la estructura del material.


En un intento medio futil de explorar más a lo que se refieren con hidrogeles, el resto de la sección se dedicará a describir,
- características la red polimérica
- las características de los monomeros que forman la red



