%======================================================================
\chapter{Theoretical framework}
%======================================================================


Introducción

Objetivo y pendejada y media de la Tesis

Aplicaciones de los hidrogeles.
Cuáles son las preguntas abiertas.



Marco Teórico

Introducción
Iniciar con redes poliméricas.
Despés pasar a geles.
Después a hidrogeles.

Respuesta mecánica
Modelos ideales
Tipos de respuesta mecánicas
Respuesta mecánica observada en hidrogeles

Conección Part 1
Relación entre la estructura con la respuesta mecánica en los hidrogeles

Conección parte 2
Buscamos hacer una ecuación constitutiva usando dinámica molecular.
En vez de analizar una ecuación constitutiva, buscamos un hamiltoniano que permita modelar la dinámica entre partículas para obtener la respuesta mecánica.


Métodos numéricos

----------

De acuerdo a la literatura, un hidrogel es una red polimérica que se puede hinchar en un solvente sin que la red polimérica se disuelva.
Gran parte de la descripción del hidrogel se centra en describir la red polimérica del material.
El marco téorico es una recopilación decente de las características del hidrogel que se puede encontrar en la literatura.
Posteriormente, se hablará de la modelación del sistema y la metodología para resolver las ecuaciones diferenciales que modelan el sistema.

El objetivo de la Tesis es buscar una herramienta que permita explorar la evolución de la red polimerica de un hidrogel durante procesos de deformación para poder conectar la respuesta macroscópica del material con la estructura del material.


En un intento medio futil de explorar más a lo que se refieren con hidrogeles, el resto de la sección se dedicará a describir,
- características la red polimérica
- las características de los monomeros que forman la red

# About hydrogels

From bibliographic review, we can say that a hydrogel is a polymeric network that has the capacity of swelling[cites].
Some examples are,
    polyacrylamide,
    sodium polyacrylate,
    Poly(vinyl alcohol),
    Poly(ethylene glycol),
    Poly(hydroxyyethyl methacrylate),
    agarose,
    alginate,
    gelatin,
    pectin,
    starch,
    cellulose-based networks,
    protein networks,
    among others.
In order to try to understand (some of) the properties of the hydrogels, we are going to explore what is a polymeric network and what is the capacity of swelling.

In general terms, a polymeric network is a three-dimensional structure formed by long polymer chains that are interconnected.
Meanwhile, the swellability is defined as the capacity to absorb significant amounts of a solvent without dissolving, resulting in an increase volume.
Since the swellability is a ``capacity'' of the network, we are going to start by exploring what is a polymeric structure.

## About polymeric networks

From a structural perspective, polymer networks consist of network ``junctions'', which consist of three or more strands connected by a mechanism. 
This mechanism is commonly refer as ``crosslink'' and can be describe through physical interactions or covalent bonds.
On the other hand, we recall that a polymer is a macromolecule composed by monomers that are covalently bondend together forming a strand.

### Dtour to bonds and stuff
Before proceeding it is beneficial to dive into the difference of physical interactions and covalents bonds in order to clarify the ``crosslink'' mechanisms.
This is because, bond strengths and exchange rates are more informative to accuratly reflect the material properties.
For example, given sufficiently strong and static physical interactions, physical networks can behave identically to covalent bonded networks; 
alternatively, the incorporation of mechanisms for covalent bond exchange can result in chemical networks that exhibit adaptable mechanical properties regulated by external stimuli [cite]. 

A covalent bond is a type of chemical bond formed when two atoms share one or more pairs of electrons. 
On the other hand, a physical interaction refers to a non-covalent force that describe how atoms, ions, or molecules attract or repel each other without forming new chemical entities. 
The covalent bond arises from the quantum mechanical interactions between the atomic orbitals, where the shared electrons occupy a molecular orbital that extends over both atoms, lowering the overall energy and stabilizing the molecule.
In contrast, physical interactions arise from electrostatic, van der Waals, or dipole forces, and they are fundamentally due to the redistribution of electron density and the resulting energy changes between particles.


### Stuff that I already wrote


### Cross-linking

## About swelability capacity




All of the examples are composed of, or conatin a significat proportion of, hydrophilic monomers or polymer segments.
Some of the key hydrophilic groups in those examples are,
    hydroxyl group,
    amide group,
    carboxylate anion,
    ether group.


These groups interact with water primarly through hydrogen bonding and dipole-dipole interactions.



This is because, the hydrophilic property of the polymeric network allows the swellability ability.
(What is the boundary of "disolve" in water)

All of these examples of hydrogels have in common that 



# About the intentions of the thesis
What is a material?
Why classify?
In what I'm interseted?
Focus into the relations, not into the classifications.

Mechanical response
Phase diagram
Network structure

RElate the network structure with the mechanical response.
Understand the mechanical response through the network structure.

- Try to describe a network structure
- Try to explain the ability of swelling

Constitutive relations: Stress and temperature.
Pressure and Stress, phase diagram


