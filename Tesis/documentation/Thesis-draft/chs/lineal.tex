Introducción

Objetivo y pendejada y media de la Tesis

Aplicaciones de los hidrogeles.
Cuáles son las preguntas abiertas.



Marco Teórico

Introducción
Iniciar con redes poliméricas.
Despés pasar a geles.
Después a hidrogeles.

Respuesta mecánica
Modelos ideales
Tipos de respuesta mecánicas
Respuesta mecánica observada en hidrogeles

Conección Part 1
Relación entre la estructura con la respuesta mecánica en los hidrogeles

Conección parte 2
Buscamos hacer una ecuación constitutiva usando dinámica molecular.
En vez de analizar una ecuación constitutiva, buscamos un hamiltoniano que permita modelar la dinámica entre partículas para obtener la respuesta mecánica.


Métodos numéricos

----------

# About the intentions of the thesis
What is a material?
Why classify?
In what I'm interseted?
Focus into the relations, not into the classifications.

Mechanical response
Phase diagram
Network structure

RElate the network structure with the mechanical response.
Understand the mechanical response through the network structure.

- Try to describe a network structure
- Try to explain the ability of swelling

Constitutive relations: Stress and temperature.
Pressure and Stress, phase diagram


