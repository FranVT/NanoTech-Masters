\chapter{Conclusion}




\paragraph{Hydrogels}
A hydrogel is a polymeric network comprised of interconnected polymer chains that swell by absorbing solvents without dissolving. 
Key examples include polyacrylamide and alginate. 
Understanding hydrogels necessitates knowledge of polymeric networks, characterized by junctions of three or more strands linked via crosslinks, either through physical interactions or covalent bonds. 
Hydrogels form a continuous network necessary for specific properties, achieved through adequate crosslinking density.

Crosslinking is crucial for maintaining hydrogel structure, affecting viscoelastic responses and achieving the gel point. 
These mechanisms are either physical, involving non-covalent forces, or covalent, involving electron sharing.
Physical interactions can imitate covalent networks, while dynamic covalent chemistry allows for adaptable properties. 
Mechanical crosslinking enhances material toughness and elasticity.

Polymeric networks can be classified as reversible or irreversible based on crosslinking mechanisms. 
Physical crosslinks can easily reform, enabling dynamic behavior and self-healing. 
In contrast, dynamic covalent networks require external stimuli to modify bonds. 
Understanding these crosslink mechanisms is essential for advancing polymer development and mechanical strength in hydrogels.


A hydrogel is a polymeric network that can swell, composed of interconnected polymer chains. 
Examples include polyacrylamide and alginate. 
The swellability is a key property, allowing the hydrogel to absorb solvents and increase in volume without dissolving. 
Understanding hydrogels begins with defining polymeric networks and their swelling capabilities.

Polymeric networks are structures formed by junctions of three or more strands linked by crosslinks, which can be physical interactions or covalent bonds. 
They are categorized into thermosets, thermoplastics, elastomers, and gels. 
Gels are liquid-swollen networks, either chemically or physically crosslinked, that trap solvent molecules, imparting both solid and fluid characteristics. 
Hydrogels, a type of gel with water as the solvent, must constitute a continuous polymer network spanning the entire volume to exhibit certain properties. 
Achieving the gel point requires adequate crosslinking density and reactive sites; otherwise, the polymers remain soluble and do not form a gel.

Crosslinking processes are essential in maintaining the structure of hydrogels during swelling, influencing their viscoelastic-viscoplastic responses and the development of the gel point. 
Crosslinking mechanisms are classified into physical (involving non-covalent interactions like hydrogen bonding or van der Waals forces) and covalent types (involving the sharing of electrons). 
Strong physical interactions can mimic covalent networks, while dynamic covalent chemistry allows for adaptable properties through bond exchange. 
Additionally, mechanical crosslinking occurs through physical entanglements of polymer chains. 
These approaches yield different network qualities: covalent networks are stable with high mechanical strength, physical networks are softer and more dynamic, while mechanical crosslinking enhances toughness and elasticity in materials like slide-ring gels.

Research categorizes polymeric networks as either reversible or irreversible based on their crosslink mechanisms. 
Physical crosslinks, like polymer junctions, can be undone and redone easily due to thermal fluctuations and environmental changes, facilitating dynamic behavior and self-healing. 
Conversely, dynamic covalent networks require external stimuli to activate crosslinkers, which influence bond formation and dissolution. 
This adaptability is vital for maintaining mechanical strength while altering the network topology, highlighting the significance of understanding crosslink mechanisms in polymer development. 






We conclude that we have a conclusion in two years.

The patchy particle protocol is a vlid methodology to simulate the mechanical response of polymeric networks.

In order to simulate hydrogels we need no add a FENE potential.

The reverible interactions are more sutiable to model microgel particle response.

For future works we can \dots analyze the following important topological measures associated with the entanglement network: the primitive path countour length, the number of entanglements per chain, the end-to-end length of an entanglement strand and the number of central particles per entanglement strand.



