\chapter{Theoretical Framework}\label{ch1:Intro}

\section{Hydrogels}

\paragraph{Introduction} From bibliographic review, we can say that a hydrogel is a polymeric network that has the capacity of swelling[cites].
Some examples are,
    polyacrylamide,
    sodium polyacrylate,
    Poly(vinyl alcohol),
    Poly(ethylene glycol),
    Poly(hydroxyyethyl methacrylate),
    agarose,
    alginate,
    gelatin,
    pectin,
    starch,
    cellulose-based networks,
    protein networks,
    among others.
In order to try to understand (some of) the properties of the hydrogels, we are going to explore what is a polymeric network and what is the capacity of swelling.

In general terms, a polymeric network is a three dimensional structure formed by long polymer chains that are interconnected.
Meanwhile, the swellability is defined as the capacity to absorb significant amounts of a solvent without dissolving, resulting in an increase volume.
Since the swellability is a ``capacity'' of the network, we are going to start by exploring what is a polymeric structure.

\paragraph{Polymeric networks} From a structural perspective, polymer networks consist of network ``junctions'', which consist of three or more strands connected by a mechanism. 
This mechanism is commonly refer as ``crosslink'' and can be describe through physical interactions or covalent bonds.
On the other hand, we recall that a polymer is a macromolecule composed by monomers that are covalently bondend together forming a strand.

\paragraph{Bonds} Before proceeding it is beneficial to dive into the difference of physical interactions and covalents bonds in order to clarify the ``crosslink'' mechanisms.
This is because, bond strengths and exchange rates are more informative to accuratly reflect the material properties.
For example, given sufficiently strong and static physical interactions, physical networks can behave identically to covalent bonded networks; 
alternatively, the incorporation of mechanisms for covalent bond exchange can result in chemical networks that exhibit adaptable mechanical properties regulated by external stimuli [cite]. 

A covalent bond is a type of chemical bond formed when two atoms share one or more pairs of electrons. 
On the other hand, a physical interaction refers to a non-covalent force that describe how atoms, ions, or molecules attract or repel each other without forming new chemical entities. 
The covalent bond arises from the quantum mechanical interactions between the atomic orbitals, where the shared electrons occupy a molecular orbital that extends over both atoms, lowering the overall energy and stabilizing the molecule.
In contrast, physical interactions arise from electrostatic, van der Waals, or dipole forces, and they are fundamentally due to the redistribution of electron density and the resulting energy changes between particles.


