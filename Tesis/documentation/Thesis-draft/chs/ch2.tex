\chapter{Theoretical Framework}\label{ch1:Intro}

\section{Hydrogels}

\paragraph{Introduction} From bibliographic review, we can say that a hydrogel is a polymeric network that has the capacity of swelling[cites].
Some examples are,
    polyacrylamide,
    sodium polyacrylate,
    Poly(vinyl alcohol),
    Poly(ethylene glycol),
    Poly(hydroxyyethyl methacrylate),
    agarose,
    alginate,
    gelatin,
    pectin,
    starch,
    cellulose-based networks,
    protein networks,
    among others.
In order to try to understand (some of) the properties of the hydrogels, we are going to explore what is a polymeric network and what is the capacity of swelling.

In general terms, a polymeric network is a three dimensional structure formed by long polymer chains that are interconnected.
Meanwhile, the swellability is defined as the capacity to absorb significant amounts of a solvent without dissolving, resulting in an increase volume.
Since the swellability is a ``capacity'' of the network, we are going to start by exploring what is a polymeric structure.

\subsection{Polymeric networks}

\paragraph{Introduction} From a structural perspective, polymer networks consist of network ``junctions'', which consist of three or more strands connected by a mechanism. 
This mechanism is commonly refer as ``crosslink'' and can be describe through physical interactions or covalent bonds.
On the other hand, we recall that a polymer is a macromolecule composed by monomers that are covalently bondend together forming a strand.
Monomers can possess specific functional groups or reactive sites, which determine the manner in which monomers bind together. 
This, in turn, influences the structural and property characteristics of the resulting polymer.
Part of the swelling capacity can be explained by the type of monomers in the network, meanwhile the structural frameworks allows to explain the mechanical response and the swelling.

\paragraph{Swellability} The capacity for substantial solvent absorption and expansion of hydrogels is attributable to the expansion of the network due to osmotic pressure and the hydrophilic funcional groups of the monomers that constitute hydrogels.
Some of the key hydrophilic groups are,
    hydroxyl group,
    amide group,
    carboxylate anion,
    ether group.
The hydrophilic phenomenon, from a chemical perspective, occurs when molecules possess polar or charged functional groups that spontaneously form hydrogen bonds or electrostatic interactions with water, enabling water to diffuse over the surface.
Nevertheless, the network's integrity remains intact due to the crosslink mechanism.

\paragraph{Crosslink} The underlying principles of crosslink mechanism are the physical interactions and covalent bonds.
However it is important to acknolwdge that, for example, given sufficiently strong and static physical interactions, physical networks can behave identically to covalent bonded networks; 
alternatively, the incorporation of mechanisms for covalent bond exchange can result in chemical networks that exhibit adaptable mechanical properties regulated by external stimuli [cite]. 
Consequently, an emphasis on bond strengths and exchange rates provides more informative insights for accurately inferring the properties of hydrogels.

With this understanding, a covalent bond is defined as a specific type of chemical bond that occurs when two atoms share one or more pairs of electrons. 
On the other hand, a physical interaction is defined as a non-covalent force that describes how atoms, ions, or molecules attract or repel each other without forming new chemical entities. 
The covalent bond is the result of quantum mechanical interactions between atomic orbitals.
In these interactions, shared electrons occupy a molecular orbital that extends over both atoms. 
In contrast, physical interactions are attributed to electrostatic, van der Waals, or dipole forces, arising from the redistribution of electron density and the consequent energy alterations between particles.

Now we can dive into the different types of polymer networks and the different types of crosslink mechanisms.
After that, then we are going to spend some paragraphs into explor the ideas of mechanical response through constitutive relations.
In order to end with the mechanical response of hydrogels and some conections with the polymeric network.

\subsection{Types of polymeric networks: Gels}

\paragraph{Introduction} In general terms a polymeric network can be divided into one of four major classes: 
    thermosets,
    thermoplastics,
    elastomers,
    and gels.
Thermosets are rigid, covalently bonded polymer networks with high modulus, which are insoluble and degrade rather than melt upon heating. 
In contrast, thermoplastics are held together by strong physical interactions, allowing them to be remolded and recycled when heated. 
Elastomers are soft, deformable with covalent networks used above their glass transition temperature, capable of large reversible extensions. 
Finally, gels are liquid-swollen networks, either covalent or physical interactions, that are soft and highly deformable.

\paragraph{Gels} In detail, a gel is a three-dimensional, crosslinked polymer network formed by physical or chemical crosslinks, which serve to trap the solvent molecules via intermolecular interactions, including hydrogen bonding and osmotic forces. This process prevents the fluid from flowing freely.
This results in a material with both solid and liquid characteristics—elasticity from the polymer network and fluidity from the entrapped solvent.
This description resembles that of a hydrogel.
The hydrogel, however, is a specific type of gel in which the solvent is water and the polymers are hydrophilic.

\paragraph{Gel point} The classification of a polymer network as a gel is characterized by the formation of a continuous, system-spanning (infinite) network through the process of polymerization or crosslinking mechanisms.
This phenomenon is indicative of the percolation\footnote{In physics, percolation describes the emergence of large-scale connectivity in disordered systems. On the other hand, mathematically, is the study of cluster formation in a random graph or lattice when sites or bonds are occupied with a given probability.} of the polymeric network.
And it is known as the gel point.
At this point, the polymer chains become sufficiently interconnected through crosslinks to create a macroscopic network that spans the entire volume, causing the material to gain elasticity and lose fluidity, transitionning\footnote{This transition is a percolation threshold where the molecular weight and network correlation length diverge.} it from a viscous liquid (sol) to a solid-like gel.

Finally, it must be mentioned that not all polymer networks can achieve a gel point.
To reach this point requires sufficient network connectivity due to enough reactive sites and a high enough crosslinking density.
For instance, linear, unbranched, or insufficiently crosslinked polymers remain soluble and do not gel.

\subsection{Crosslink mechanisms}


(Hablar de la porpiedad de swelling debido a los grupos funcionales)


(Hablar de las distintos tipos de networks según su estructura)

(Hablar de crosslinkers)



