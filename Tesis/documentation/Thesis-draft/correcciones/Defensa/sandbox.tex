
\section{Gels}


\paragraph{Polymeric gels}

Definition of a Gel (Soft Matter Perspective)
A gel is a soft, solid-like material formed by a continuous, three-dimensional network that spans the volume of a liquid, immobilizing it through surface tension and capillary forces.

The chemical gelation transition, as it is typically observed in polymer systems, transforms a solution of polymeric molecules, the sol, from a viscous fluid to an elastic disordered solid, the gel\citep{gadoUnifyingModelChemical2003}.
Gelation phenomena are also observed in colloidal systems, that are suspensions of mesoscopic particles interacting via short range attraction: due to aggregation phenomena at low density (colloidal gelation) these systems display gel states with a power law behaviour of the viscosity coefficient and of the elastic modulus [7], as in chemical gelation\citep{gadoUnifyingModelChemical2003}.



\paragraph{Colloidal gels}
A colloidal gel is a space-spanning, percolated network formed by the aggregation of Brownian (sub-micron) particles. 
The gel forms when attractive forces between particles (e.g., van der Waals, depletion) cause them to stick upon contact, preventing equilibrium condensation and instead trapping the structure in a non-equilibrium state.




1. Colloidal Gels
Original Framework: Colloid Science & Statistical Mechanics.

The definition arises from the study of suspensions of solid particles (e.g., silica, clay, proteins) in a liquid medium. 
The gelation is seen as an arrested phase separation driven by particle-particle interactions.



Micron-sized constituent colloids are large enough to be imaged with light, while still small enough to have their dynamics driven by kBT, the thermal energy, where T is the  temperature and kB, the Boltzmann constant.

With sufficiently high concentration of adde polymers in a colloid-polymer mixture, these mixtures form gels, that is, networks  of particles that span across the volume in a disordered arrested structure, which can  sustain shear stresses, despite the small packing fraction φ occupied by the colloidal  particles\citep{zaccarelliGelationArrestedPhase2008}.


From this general definition it  follows that a low density disordered arrested state which does not flow but possess  solid-like properties as a yield stress, is commonly named a gel\citep{zaccarelliColloidalGelsEquilibrium2007}.

The terminology of sol-gel transition refers to a liquid mixture where solute (sol) particles (ranging from monomers to biological macromolecules) are suspended in a solvent\citep{zaccarelliColloidalGelsEquilibrium2007}.
Initially the sol particles are  separated, but, under appropriate conditions, they aggregate until a percolating network is formed. 
In the following the conditions under which such percolating network can be defined as a gel will be discussed. 
Colloidal gels are often formed by particles dispersed in a liquid solvent. 
However, in polymers and silica-gels the solvent is not a liquid or it is missing\citep{zaccarelliColloidalGelsEquilibrium2007}.

Chemical gelation studies were initiated in the framework of cross-linking polymers, whose gelation transition was associated to the formation of an infinite network with  finite shear modulus and infinite zero-shear viscosity.
One possible example of polymer gel-forming systems is provided by  epoxy resins. 
In these systems, polymer chains grow step-wise by reactions mediated  by end-groups or cross-linkers (step polymerization).


If the (average)  functionality of the monomers is greater than two, to allow the establishment of a  branched structure with junction points, a fully connected network, spanning the whole  space, is built[30] and a gel is obtained.

Percolation it is based on the connectivite properties of the system.


Physical gels are gels in which bonds originate from physical interactions of the order of $k_B T$.

A colloidal suspension is a mixture in which microscopically dispersed insoluble particles are suspended thorughout another substance, forming a heterogeneous system with distinct phases.



\paragraph{Non-covalent} Noncovalent bonds are characterized by their relative low energy (typically a few $k_b T$), which is modulated smoothly by external variables such as temperature, pH, and solvent\citep{formanekGelFormationReversibly2021a}.


\section{References}

@article{zaccarelliGelationArrestedPhase2008,
  title = {Gelation as Arrested Phase Separation in Short-Ranged Attractive Colloid-Polymer Mixtures},
  author = {Zaccarelli, Emanuela and Lu, Peter J. and Ciulla, Fabio and Weitz, David A. and Sciortino, Francesco},
  year = 2008,
  month = dec,
  journal = {Journal of Physics: Condensed Matter},
  volume = {20},
  number = {49},
  eprint = {0810.4239},
  primaryclass = {cond-mat},
  pages = {494242},
  issn = {0953-8984, 1361-648X},
  doi = {10.1088/0953-8984/20/49/494242},
  urldate = {2025-11-22}
}

@article{zaccarelliColloidalGelsEquilibrium2007,
  title = {Colloidal Gels: Equilibrium and Non-Equilibrium Routes},
  author = {Zaccarelli, Emanuela},
  year = 2007,
  month = aug,
  journal = {Journal of Physics: Condensed Matter},
  volume = {19},
  number = {32},
  eprint = {0705.3418},
  primaryclass = {cond-mat},
  pages = {323101},
  issn = {0953-8984, 1361-648X},
  doi = {10.1088/0953-8984/19/32/323101},
  urldate = {2025-11-22}
}

@article{formanekGelFormationReversibly2021a,
  title = {Gel Formation in Reversibly Cross-Linking Polymers},
  author = {Formanek, Maud and Rovigatti, Lorenzo and Zaccarelli, Emanuela and Sciortino, Francesco and Moreno, Angel J.},
  year = 2021,
  month = jul,
  journal = {Macromolecules},
  volume = {54},
  number = {14},
  pages = {6613--6627},
  issn = {0024-9297, 1520-5835},
  doi = {10.1021/acs.macromol.0c02670},
  urldate = {2025-11-22}
}

@article{gadoUnifyingModelChemical2003,
  title = {A Unifying Model for Chemical and Colloidal Gels},
  author = {Gado, E. Del and Fierro, A. and de Arcangelis, L. and Coniglio, A.},
  year = 2003,
  month = jul,
  journal = {Europhysics Letters},
  volume = {63},
  number = {1},
  pages = {1},
  publisher = {IOP Publishing},
  issn = {0295-5075},
  doi = {10.1209/epl/i2003-00468-4},
  urldate = {2025-11-22}
}

