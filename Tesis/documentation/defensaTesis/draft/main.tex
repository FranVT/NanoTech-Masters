%%%%%%%%%%%%%%%%%%%%%%%%%%%%%%%%%%%%%%%%%
% Tufte Essay
% LaTeX Template
% Version 2.0 (19/1/19)
%
% This template originates from:
% http://www.LaTeXTemplates.com
%
% Authors:
% The Tufte-LaTeX Developers (https://www.ctan.org/pkg/tufte-latex)
% Vel (vel@LaTeXTemplates.com)
%
% License:
% Apache License, version 2.0
%
%%%%%%%%%%%%%%%%%%%%%%%%%%%%%%%%%%%%%%%%%

%----------------------------------------------------------------------------------------
%	PACKAGES AND OTHER DOCUMENT CONFIGURATIONS
%----------------------------------------------------------------------------------------

\documentclass[a4paper]{tufte-handout} % Use A4 paper by default, remove 'a4paper' for US letter

\usepackage{graphicx} % Required for including images
\setkeys{Gin}{width=\linewidth, totalheight=\textheight, keepaspectratio} % Default images settings
\graphicspath{{Figures/}{./}} % Specifies where to look for included images (trailing slash required)

\usepackage{booktabs} % Required for better horizontal rules in tables

\usepackage{subfiles}

%----------------------------------------------------------------------------------------
%	TITLE SECTION
%----------------------------------------------------------------------------------------

\title{Defensa de Tesis}

\author{Francisco Vazquez-Tavares}

\date{Last compilation: \today} % Date, use \date{} for no date


%----------------------------------------------------------------------------------------
%	COMMANDS SECTION
%----------------------------------------------------------------------------------------


%----------------------------------------------------------------------------------------

\begin{document}

\maketitle % Print the title section

%----------------------------------------------------------------------------------------
%	ABSTRACT/SUMMARY
%----------------------------------------------------------------------------------------

\begin{abstract}
    Sketch de la presentación para la defensa de Tesis
\end{abstract}

%----------------------------------------------------------------------------------------
%	ESSAY BODY
%----------------------------------------------------------------------------------------

\justifying

\section{Introducción}

Hola, muchas gracias por estar aquí.
La inteción de esta exposición es compartirles los resultados que surgieron durante dos años de maestria.
Cómo podemos ver en el título, hablaré de hidrogeles, respuesta mecánica y simulación numérica.

\section{Geles}

Comenzaremos esclareciendo lo que es un hidrogel.
con la palabra podemos intuir que hay dos elementos importante, hidro y gel.
Hidro, hace referencia a agua, mientrás que gel, pues nos hace referencia a un gel.
Así que vamos a hablar de geles.

Pensemos en una esponja.
La esponja tiene una estructura porosa, que cuando la ponemos en agua, la esponja la absorbe.
Y cuando apretamos la esponga, el agua que quedó atrapada sale.
Y eso es algo simiar a un gel.
Hay un líquido disperso en algo sólido.
Solo que una epsonja no es un gel.
Para que la esponja se un gel, la esponja debe de ser tamaño microscopico.

Entonces, ejemplos de geles que tenemos en nuestro día a día, es el gel para cabello, para las quemadura o la gelatina.
Y cuando estamos interactuando con los geles, podemos notar que estos son más pareceidos a un fluido, pero cuando dejas de manipularlo se quedan quietos, como si fuera un solido.
Además, también podemos notar que son sensibles a la temperatura.
Cuando se deja la gelatina en la mesa, se empieza a guadar.

A lo que me refiero con el ejemplo de la esponja microscopica, es lo siguiente:
Cuando estamos preparando gelatina, podemos un polvo, algo sólido y después le ponemos agua acaliente, le revolvemos y de ahí al refrigerador.
Algo pasa entre revolver el agua y el refrigerador, que el agua con polvo deja de ser agua con polvo y se convierte en una microesponja con agua.

A grandes rasgos, el polvito son cadenas poliméricas y cuando las ponemos en agua caliente, las mezaclamos, permitimos que esas cademas se empiezan a conectar y generar la esponja.
Se empieza a crear una red polimérica.
Y cuando la temperatura sube, esta red se des hace y se aguada la gelatina.

Entonces, el gel, en términos más científicos, es un sistema coloidal donde la fase dispersa es líquida, mientrás que la fase continúa es sólida.
Es decir, un líquido disperso en un sólido.
Agua, dispersa en grenetina.

La parte de hidro en el hidrogel, quiere decir que son geles que la fase dispersa es únicamente agua.
Entonces, la gelatina es un hidrogel.
Si de alguna forma lográn hacer bolitas de gelatina del tañamo de 100 nanometros a 100 micrometros, entonces habrán hecho micro-hidrogel y si hace gelatina aún más chica, pues serán nano-hidrogel.

Además de que algunos hidrogeles sean comida, cuando se cambia la red polimérica por otros grupos funcionales y demás, pues se pueden aplicar en la liberación de fármacos, la creación de adámios para tejidos, sensores, lentes de contacto, tratamiento de heridas, capatición/purificación de agua, entre muchas más aplicaciones.

Lo que buscamos en el proyecto de Tesis es poder simular hidrogeles para analizar su respuesta mecánica.
Es decir, queremos explicar que mecanismos moleculares determinan las condiciones de ruptura del hidrogel o cuantas repeticiones se puede hacer cuando se usa como actuador o saber que tanta fuerza hay que aplicar para liberar el fármaco.
Entonces, ahora daremos paso a hablar de lo que es una respuesta mecánica y porque es importante para esto de la aplicaciones.

\marginpar{Hablar de lo importante de la red polímerica con la respuesta mecánica.}

\section{Respuesta mecánica}

Una de las respuestas mecánicas más conocidas en la respuesta elática.
Fuertemente asociada con la respuesta de ligas o resortes.
Cuando se aplica una fuerza al material, este se deforma instanámente y se mantiene la deformación.
Cuando se detiene la aplicación de fuerza, el material regresa instantánemente a su posición original.

Por el otro lado, la deformación plástica es cuando al aplicar una fuerza en el material, este se deforma instanánemaente, pero cuando sueltas el objeto, este ya no regresa a su forma original.
Como los clips o los metales en general.

Finamente, hay una respues llamada viscosa.
Esta respuesta, en vez de tene runa deformación inmediata con la fuerza aplica, el material se va deformando lentamente y una vez que se termina de aplicar la deformación, ya no regresa a su posición original.

Entonces, tenemos estas respuestas mecánicas.
Si combinamos estas tres respuestas, obtenemos a muy grandes rasgos la respuesta mecánica de los hidrogeles.
Teóricamente, se describe estas deformaciones como viscoelástica o viscoplástica.
Se describe que cuando se introduce una fuerza, la deformación inicial es inmediata.
Pero después de un tiempo, la respuesta empeiza a ser viscosa.
Y si al quitar la fuerza, el hydrogel regresa a su posición original, entonces es una deformación viscoelástica, mientrás que si no regressa al mismo lugar, es una deformación viscoplástica.

Sin entrar mucho en detalle, los mecanismos moleculares que permiten explicar en que momento ocurre la trancisión de regimen elástico a plástico aún no están muy claros.
Aunque normalment se le asigna a las propiedades de la red polímera la respuesta elástica y la respuesta viscosa es explicada por la disipación de enrgía que hay entre las colisiónes del agua con la cadena polimérica.

Entonces, al buscar explicar la respuesta mecánica desde primerio principios, es importante relacionar los tipos de deformaciones con procesos moleculares.
Se asocia a una deformación elástica cuando ocurren estiramientos entre los enlaces, pero sin llegar a romperse.
Después, la deformación plástica es cuando se rompen lo enlaces.
Mientrás que la respues viscosa, sigue siendo explicada por el intercambio de momenta entre las moléculas del agua con los monómeros.

También es importante notar, que normalmente se describe la respuesta mecánica en términos de estrés con respecto a la deformación a plicada.
En este tipo de gráfica se introduce el concpeto de \emph{yield}.

\section{Dinamica Molecular}

Ahora es momento de explicar que onda con dinámica molecular.

\section{Resultados}


\end{document}
