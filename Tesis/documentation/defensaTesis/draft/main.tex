%%%%%%%%%%%%%%%%%%%%%%%%%%%%%%%%%%%%%%%%%
% Tufte Essay
% LaTeX Template
% Version 2.0 (19/1/19)
%
% This template originates from:
% http://www.LaTeXTemplates.com
%
% Authors:
% The Tufte-LaTeX Developers (https://www.ctan.org/pkg/tufte-latex)
% Vel (vel@LaTeXTemplates.com)
%
% License:
% Apache License, version 2.0
%
%%%%%%%%%%%%%%%%%%%%%%%%%%%%%%%%%%%%%%%%%

%----------------------------------------------------------------------------------------
%	PACKAGES AND OTHER DOCUMENT CONFIGURATIONS
%----------------------------------------------------------------------------------------

\documentclass[a4paper]{tufte-handout} % Use A4 paper by default, remove 'a4paper' for US letter

\usepackage{graphicx} % Required for including images
\setkeys{Gin}{width=\linewidth, totalheight=\textheight, keepaspectratio} % Default images settings
\graphicspath{{Figures/}{./}} % Specifies where to look for included images (trailing slash required)

\usepackage{booktabs} % Required for better horizontal rules in tables

\usepackage{subfiles}

%----------------------------------------------------------------------------------------
%	TITLE SECTION
%----------------------------------------------------------------------------------------

\title{Defensa de Tesis}

\author{Francisco Vazquez-Tavares}

\date{Last compilation: \today} % Date, use \date{} for no date


%----------------------------------------------------------------------------------------
%	COMMANDS SECTION
%----------------------------------------------------------------------------------------


%----------------------------------------------------------------------------------------

\begin{document}

\maketitle % Print the title section

%----------------------------------------------------------------------------------------
%	ABSTRACT/SUMMARY
%----------------------------------------------------------------------------------------

\begin{abstract}
    Sketch de la presentación para la defensa de Tesis
\end{abstract}

%----------------------------------------------------------------------------------------
%	ESSAY BODY
%----------------------------------------------------------------------------------------

\section{Introducción}
Hola, mcuhas gracias por estar aquí.

Mi principal objetivo para estos 40 minutos es compartirles mi investigación a lo largo de 2 años de maestría.
En términos prácticos, la investigación se basó en replicar una metodología computacional que simula microgeles y adaptarla para simular hidrogeles.
Y lo que veremos a continuación es explicarles, qué metodología usé, qué es microgel y qué es una hidrogel, la diferencia entre esos dos, por qué esos materiales y también qué es lo que analisamos de esa simulación.
Así que empezemos con el tema de hidrogeles y microgeles.

\section{Hidrogeles}

Por las palabras vemos que tenemos geles, uno que es hidro y el otro que es micro.
Así que comenzaremos con definer un gel y después nos aveturamos a difeernciar entre hidro y microgel.

En términos muy general, se puede decir que un gel es una combinación de algo líquido disperso en algo sólido.
Podemos pensar en una esponja para lavar trastes.
El agua está dispersa en la esponja.
Cuando estamos lavando trastes, el agua se queda en la esponja.
También puede ser que dejemos la esponja en un traste con agua y la esponja se empieza expandir por absorber agua.
Hay otras que no, pero en ambos casos, cuando las sacamos del traste lleno con agua, gran parte del agua se queda en la esponja, hasta que las apretamos.
Solo que hay un ligero detalle.
Una esponja no es considerado un gel.

Algunos ejemplos de geles que conocemos son, la gelatina, gel para el cabello o el gel para skin care.
Y la gran diferencia con la esponja, es que podemos ver la esponja, y en los geles no podemos ver la parte sólida.
De los ejemplos de geles, la gelatina es lo más intuitivo, porque al hacerla ponemos un polvo en agua caliente.
Estamos poniendo algo sólido en algo líquido.
Nada más que esa mezcla de agua caliente con polvo no es un gel.
Es agua caliente con polvo.
Entonces, la gelatina es considerada gel hasta que la revolvemos, la metemos en el refrigerador y esperamos quién sabe cuantas horas y como que se solidifica.

Además, este caso de la gelatina está curioso, porque cuando la sacamos del refri, se quedá ahí.
Cómo si fuera el gel para cabello, así medio rígido.
Pero conforme va pasando el timepo, se empieza a medio derretir y se parece más al gel para skin care.

Y eso abre varias dudas.
Qué es lo que hace que la gelatina se mantenga rígida y después ya no.
Porqué el gel de cabello nos lo podemos poner, pero el cabello no está todo mojado.
Igual con el skin care.
La piel no se quedá mojada todo el rato. 
Se siente baboso, pero después como que se seca.
Así que vamos a intentar responder estás preguntas usando física y demás.

\subsection{Geles}

Pues la definición así más formal de un gel es una sistema coloidal con una fase dispersa líquida en una fase cóntínua sólida.
Es decir, que en la gelatina, la fase dispersa es el agua y la fase continua es el polvito que le ponemos.
Pero es contraintuitivo y al mismo tiempo es cohesivo.
Porque cuando ponemos el polivto, pues vemos claramente que hay polvo en el agua.
Pero después de mezclar y poner en el refri, ya no distinguimos el agua del polvo.
Y es hasta ese punto en que la gelatina es un gel.
Entonces algo pasa con el polvito que hace esa gran diferencia entre que la gelatina sea considerada como gel o no.

https://www.anec.org/es/knowledge/biology/what-is-gelatin-35-271.htm

https://theory.labster.com/es/gelatin/

https://theory.labster.com/es/gel/

Esa gran diferencia, es que el polvito genera una una red en todo el volumen.
En otras palabras, el polvo se convierte en una red que percola el sistema.
Y cuando existe está red, se dice que se alcanzo el punto de gelación.
Y ahora tenemos una red sólida que atrapa agua.



\subsection{Crosslinkers: Reticulantes}


\section{Dinámica Molecular}

\section{Resultados}


\end{document}
