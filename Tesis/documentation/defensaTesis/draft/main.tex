%%%%%%%%%%%%%%%%%%%%%%%%%%%%%%%%%%%%%%%%%
% Tufte Essay
% LaTeX Template
% Version 2.0 (19/1/19)
%
% This template originates from:
% http://www.LaTeXTemplates.com
%
% Authors:
% The Tufte-LaTeX Developers (https://www.ctan.org/pkg/tufte-latex)
% Vel (vel@LaTeXTemplates.com)
%
% License:
% Apache License, version 2.0
%
%%%%%%%%%%%%%%%%%%%%%%%%%%%%%%%%%%%%%%%%%

%----------------------------------------------------------------------------------------
%	PACKAGES AND OTHER DOCUMENT CONFIGURATIONS
%----------------------------------------------------------------------------------------

\documentclass[a4paper]{tufte-handout} % Use A4 paper by default, remove 'a4paper' for US letter

\usepackage{graphicx} % Required for including images
\setkeys{Gin}{width=\linewidth, totalheight=\textheight, keepaspectratio} % Default images settings
\graphicspath{{Figures/}{./}} % Specifies where to look for included images (trailing slash required)

\usepackage{booktabs} % Required for better horizontal rules in tables

\usepackage{subfiles}

%----------------------------------------------------------------------------------------
%	TITLE SECTION
%----------------------------------------------------------------------------------------

\title{Defensa de Tesis}

\author{Francisco Vazquez-Tavares}

\date{Last compilation: \today} % Date, use \date{} for no date


%----------------------------------------------------------------------------------------
%	COMMANDS SECTION
%----------------------------------------------------------------------------------------


%----------------------------------------------------------------------------------------

\begin{document}

\maketitle % Print the title section

%----------------------------------------------------------------------------------------
%	ABSTRACT/SUMMARY
%----------------------------------------------------------------------------------------

\begin{abstract}
    Sketch de la presentación para la defensa de Tesis
\end{abstract}

%----------------------------------------------------------------------------------------
%	ESSAY BODY
%----------------------------------------------------------------------------------------

\justifying


Bueno, muchas gracias por aceptar ser parte del comité de evaluación de este trabajo de investigación\footnote{Diap 1}.

Hoy les compartiré el trabajo de investigación que estuve realizando a lo largo del programa de maestría.
Iniciaremos con definir que e sun hidrogel y a que nso referimos con respuesta mećanica\footnote{Diap 2}.
Después les compartiré la metodología que usamos para simular numéricamente este material y los resultados computacionales\footnote{Diap 3}.

\section{Introducción: Geles}

Uno de los ejemplos más cómunes de geles son el gel para cabello, la gelatina o el gel que nos ponemos para tratar quemaduras.
Así que exploremos uno de estos ejemplos para obtener una definción para este material\footnote{Diap 4}.

Cuando hacemos gelatina ponemos un polvo en agua caliente, mezclamos y después lo dejamos en el refrigerador\footnote{Diap 5}.
Si analizamos con más detalle el proceso, algo sucede con el polvo que hace que la mezcla de agua con polvo se convierta en gelatina.
También podemos intuir que la temperatura del agua tiene un papel importante en el proceso.
Porque si dejamos el polvo mezclado en el agua afuera del refrigerador y lo vovlemos a revisar más tarde, pues veremos agua con polvo.
Pero cuando mezclamos el polvo y después lo ponemos en el refrigerador, algo suscede con el povo que hace que se solidifique.

Cuando consideramos que el polvo es colágeno degradado, podemos representar este polvo como un conjunto de polímeros sólidos\footnote{Diap 6}.
Y cuando estos polímeros empiezan a formar una red interconectada y que está red se expande a lo largo de todo el volúmen del recipiente, el agua con polvo deja de ser agua con polvo y ahora es considerada como una gelatina.

El proceso en el que las cademas poliméricas se unen para formar una red que se expande a lo largo de todo el volúmen se le conoce como punto de gelación o también se dice que el sistema percoló.
Por otro lado, el mecanísmo que se usa para describir cómo se forma la red se le conoce como crosslinking, pero se detallará más adelante.

Por ahora, esto nos permite dar una definición de gel.
Se puede definir un gel como un sistema coloidal compuesto por una fase sólida y una fase líquida\footnote{Diap 7}.
Normalmente se menciona que la sólida es la fase contínua y atrapa al fase líquida.
Estó indica que la red polímerica es porosa y que en esos poros el agua es atrapada\footnote{Diap 8}.

Con esto en mente, podemos diferenciar entre gel, microgel y nanogel.
Cuando la red polimérica está en estos rangos se dice que es microgel y cuando es aún más pequeño se dice que es nanogel.

De igual forma, se puede terminar de definir lo que es un hydrogel.
Ya que sabemos la definición de gel, la parte hydro, implica que la fase líquida es agua.
Por lo que la gelatina es un hydrogel.
Si lograramos hacer gelatina de tamaño micro, entonces habremos hecho un microgel y si lo hacemos más pequeño, entonces tendremos un nanogel.

Esto nos permite explicar las aplicaciones de los hidrogeles\footnote{Diap 9}.
Al restringuir que la fase líquida sea agua, esto incrementa la biocompatibilidad del material.
Con la selección adecuada de monomeros para formar la red polimérica, los hidrogeles se pueden usar para crear lentes de contacto, suministro de medicamento controlado, andamios para tejidos, como organos o piel.
Fuera de las apliaciones biologicas, estos materiales también se pueden aplicar en sistemas de captación de agua y de retención de metales pesados, actuadores o electrónica flexible.

Esto justifica el análisis de la respuesta mecánica de los hidrogeles usando dinámica molecular\footnote{Diap 10}.
A pesar de tener tantas aplicaciones aún no se tiene muy en claro cuáles son los mecanismos moleculares que les permite a los hidrogeles tener respuestas mecánicas adecuadas para su aplicación.
Es decir, saber si se puede romper durante el proceso de manufactura o cuál el tiempo de vida útil.
Entonces, ahora vamos a entrar a entender lo que es una respuesta mecánica y cuál es la respuesta mecánica reportada para los hidrogeles.

La respuesta mecánica más conocida es la respuesta elástica\footnote{Diap 11}


La baja tmeperatura ayuda a anailzar mejor la respuesta mecánica debido a la red polimérica al disminuir el efecto viscoso.

\end{document}
