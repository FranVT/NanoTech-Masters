\chapter{Tips \& Tricks} \label{ch:Tips}

In this chapter we will see how to utilize and even extend capabilities of \TeXtured{}.
Additionally, there will be sprinkled miscellaneous tips on how to improve the quality of your document.

\section{Structure}%
\label{sec:Structure}

\subsection{Headings}%
\label{sub:Headings}

\begin{itemize}
    \item numbered and \enquote{lettered} chapters
          \begin{Todo}
              Describe \custommacro{\chapternotnumbered}, and \enquote{lettered} chapters in front matter.
          \end{Todo}
    \item Use nicely named \textcolor{gray}{sub}sections --- much easier to navigate, since it leads to better ToC and Index
          \begin{Todo}
              Describe \macro{\texorpdfstring}.
          \end{Todo}
\end{itemize}

\subsection{Structure Environments}%
\label{sub:Structure Environments tips}

\begin{itemize}
    \item Utilize structure (remark, definition, ...) environments to make the document more structured and easier to read.
          Including a brief description as an optional argument can help to foreshadow the content of the environment.
          Important concepts will then stick out more and will be remembered better.
          \begin{remark}[Spacing at the End of Structure Environments]
              Structure environments ending with displayed math or a list may need a bit of tweaking to ensure proper spacing at their end.

              This is most easily achieved using the \custommacro{\qedhere} macro on the line, which should be the last one in the environment.
              This uses the mechanism of the \macro{\qedhere} macro from \package{amsthm} package, but now has also a starred variant for extra vertical space (for equations containing big operators), or even an optional argument for a completely custom vertical shift.
          \end{remark}
          \begin{Todo}
              Describe creation of new \enquote{structure} environments.
          \end{Todo}
    \item Try to motivate every definition/theorem with \enquote{normal} text, do not let the document degenerate just into a listing of definitions/theorems/proofs/...
    \item Use references to other remarks/definitions/sections to make the document more interconnected, which can help the reader to look at a bigger picture, recollect necessary information to proceed further, or to understand the context better.
          \begin{Todo}
              Describe \custommacro{\Cref}, \custommacro{\Nref}.
          \end{Todo}
          \begin{Todo}
              Show using \macro{\autocite{TODO}} in the text \autocite{TODO}.
              Helps to not forget to add the citation later.
          \end{Todo}
\end{itemize}


\section{Typography}%
\label{sec:Typography}

\begin{itemize}
    \item use \macro{~} to enter non-breakable space, or also after dot in initials/after titles
          (otherwise one gets bigger space than is proper), for example \macro{M.Sc.~Name Surname}
    \item proper usage of hyphens/dashes --- learn when to use hyphen - (\macro{-}), when en-dash -- (\macro{--}), and when em-dash --- (\macro{---})
    \item use \emph{emphasis} with \macro{\emph} for the names of new and important concepts
    \item for quotation marks use \macro{\enquote} from \package{csquotes} package
    \item sometimes using gray text instead of parentheses may result in a cleaner look, for example instead of \enquote{(pseudo-)Riemannian} just gray out \enquote{pseudo-} like \enquote{\textcolor{gray}{pseudo-}Riemannian}
    \item choose capitalization style of titles, and stick with it --- I chose \enquote{titlecase}
\end{itemize}


\section{Mathematics \& Physics}%
\label{sec:MathematicsandPhysics}

\subsection{Math Typesetting}%
\label{sub:Math Typesetting}

Learn stuff in \package{amsmath} and \package{mathtools} packages.
Then it is possible to write stuff like the following inclusion map
\begin{alignat*}{3}
    % NOTE: maybe there is a better way to center parameters (t,r,\overline{x})?
    \iota\colon & (\Sphere^{1}, \R_{\ge 0}, \Sphere^{\spacedim - 1})                                                                           && \longrightarrow \coset{\AdS}{\Z} \\
                & (\centerhphantom{t}{\Sphere^{1}},\centerhphantom{r}{\R_{\ge 0}},\centerhphantom{\overline{\bm{x}}}{\Sphere^{\spacedim - 1}}) && \longmapsto \bm{x} =
    \iota(t,r,\overline{\bm{x}}) \equiv \smash[t]{
        \begin{dcases}
            \begin{aligned}
                \bm{x}^{\shortminusone} & = \sq{\ell^{2} + r^{2}} \cos(t/\ell), \\
                \bm{x}^{0}              & = \sq{\ell^{2} + r^{2}} \sin(t/\ell), \\
                \bm{x}^{i}              & = r \overline{\bm{x}}^{i} \qq{for} i=1,\ldots,\spacedim
            \end{aligned}
        \end{dcases}
    }
    \eqend
\end{alignat*}
\begin{remark}[Math Ending Punctuation]
    Make sure to use \custommacro{\eqend} or \custommacro{\eqcomma} macro (when appropriate) to properly end a math environment with a period or a comma, respectively.
    They add a small space before the punctuation to make the formula look better.
\end{remark}

\begin{Todo}
    Maybe show diagrams with \package{TikZ} package.
\end{Todo}
\begin{Todo}
    Describe \macro{\DeclareDocumentCommand}, ...
\end{Todo}

\subsection{Numbers and Units}%
\label{sub:Numbers and Units}

Use \package{siunitx} package for convenient typesetting numbers and units.
Examples are shown in \Cref{tab:siunitx}.

\begin{Note}
    The \package{siunitx} package is very powerful and flexible.
    It can be even used to nicely align numbers in tables.
    As of now, this feature is not customized in any way in \TeXtured{}.
    Suggestions for improvements are welcome.
\end{Note}
\begin{table}[ht!]
    \begin{booktabs}{stretch=1.1}
        \toprule
        \textbf{Command}                           & \textbf{Output}                & \textbf{Usage}     \\
        \midrule
        \fakemacro{\num{123.45 e-8}}               & \num{123.45 e-8}               & numbers            \\
        \fakemacro{\si{\meter\per\second\squared}} & \si{\meter\per\second\squared} & units              \\
        \fakemacro{\SI{123.45}{m/s^2}}             & \SI{123.45}{m/s^2}             & numbers with units \\
        \fakemacro{\SIrange{1}{10}{\kilo\meter}}   & \SIrange{1}{10}{\kilo\meter}   & ranges             \\
        \fakemacro{\SIlist{1;3;5}{A}}              & \SIlist{1;3;5}{A}              & lists              \\
        \fakemacro{\SI{1.23 +- 0.45}{\celsius}}    & \SI{1.23 +- 0.45}{\celsius}    & uncertainties      \\
        \bottomrule
    \end{booktabs}
    \caption{Examples of \package{siunitx} package usage.}
    \label{tab:siunitx}
\end{table}


\section{\texorpdfstring{\LaTeX{}}{LaTeX} Coding}%
\label{sec:LaTeX Coding}

\begin{Todo}
    Describe how to create custom macros with \macro{\NewDocumentCommand}, \macro{\RenewDocumentCommand}, \macro{\NewCommandCopy}, ...
\end{Todo}
\begin{Question}
    Difference between \enquote{macro} and \enquote{function} in \LaTeX{}?
    Which nomenclature is appropriate?
\end{Question}
\begin{remark}[Macro Space Handling]
    Using macro inside text in the form \macro{\foo} can swallow the following whitespace.
    When this is not the desired behavior, call the macro like \macro{\foo{}}.
    In this way an empty argument is passed to the macro, leaving the following whitespace intact.
\end{remark}
\begin{Todo}
    Describe \macro{\makeatletter} and \macro{\makeatother}.
\end{Todo}
\begin{Todo}
    Describe \macro{\ensuremath}.
    When math macro is used often outside math mode (alone as \macro{...\(\foo\)...}), defining it wrapped in \macro{\ensuremath} can lead to perhaps easier use (as just \macro{...\foo{}...}).
\end{Todo}
\begin{Todo}
    Describe \custommacro{\includeonlysmart}.
\end{Todo}
\begin{Note}
    Be careful about implicit end of line spaces in function definitions, sometimes necessary to use \macrobox{\texttt{\%}} after last command on the line.
    \todo{Describe this in more detail.}
\end{Note}
\begin{Todo}
    Describe WIP mode (particularly with \hologo{LuaLaTeX}).
\end{Todo}
\begin{Note}
    Some comments in source code refer to files from \TeX{}Live installation on Arch Linux.
    On other distributions or operating systems the paths might be different.
\end{Note}
