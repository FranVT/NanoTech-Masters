\documentclass[12pt,colorful,boxey]{tufte-style-thesis}

\usepackage{amsmath,amsthm,amsfonts,amssymb,amscd,latexsym}         % ..if you need lots of math symbols
\usepackage{bm}
\usepackage{physics}
\usepackage{mathptmx}
\usepackage{mathtools}
\usepackage{cancel}
\usepackage{centernot} 		% Notacion matematica
\usepackage{empheq}			% Recuadros para ecuaciones
\usepackage[most]{tcolorbox}
\usepackage{comment}
\usepackage{ragged2e}
\usepackage{siunitx}
\usepackage{svg}

\usepackage{algorithm2e}
\usepackage{notoccite}

\usepackage{layouts}

\usepackage[section,toc,automake,symbols,nonumberlist]{glossaries}
%\usepackage[automake,symbols, acronym]{glossaries-extra}   
\newglossary{symbolsmech}{symm}{symn}{Mechanical Response Symbols}
\newglossary{symbolsmd}{symd}{synd}{Molecular Dynamics Symbols}
\newglossary{symbolsmr}{symr}{synm}{Methods and Results Symbols}
\makeglossaries

% Define custom glossary style with groups
\newglossarystyle{grouped}{
  \setglossarystyle{listgroup}%
  \renewcommand{\glsgroupskip}{}%
}

\setglossarystyle{grouped}

% Mechanical Response symbols
\newglossaryentry{epsilon-mech}{
  type=symbolsmech,
  name=\ensuremath{\bm{\epsilon},~\epsilon_{ij}},
  description={Strain tensor (Dimensionless)}
}
\newglossaryentry{sigma-mech}{
  type=symbolsmech,
  name=\ensuremath{\bm{\sigma},~\sigma_{ij}},
  description={Stress tensor (\SI{}{\pascal})}
}
\newglossaryentry{C-mech}{
  type=symbolsmech,
  name=$C_{ijkl}$,
  description={Youn'g modulus tensor (\SI{}{\pascal})}
}
\newglossaryentry{dotEps-mech}{
  type=symbolsmech,
  name=\ensuremath{\dot{\bm{\epsilon}},~\dot{\epsilon_{ij}}},
  description={Time derivative of the Strain tensor (\SI{}{\per\second})}
}
\newglossaryentry{sigsuby-mech}{
  type=symbolsmech,
  name=$\sigma_y$,
  description={Yield point (\SI{}{\pascal})}
}
\newglossaryentry{eta-mech}{
  type=symbolsmech,
  name=$\eta$,
  description={Viscosity (\SI[per-mode=fraction]{}{\kilo\gram\per\meter\per\second})}
}
\newglossaryentry{gamma-mech}{
  type=symbolsmech,
  name=$\dot{\gamma}$,
  description={Shear rate (\SI[per-mode=fraction]{}{\per\second})}
}
\newglossaryentry{Gt-mech}{
  type=symbolsmech,
  name=$G(t)$,
  description={Time dependent modulus (\SI[per-mode=fraction]{}{\pascal})}
}
\newglossaryentry{T-mech}{
  type=symbolsmech,
  name=$\vec{T}$,
  description={Traction vector (\SI[per-mode=fraction]{}{\newton})}
}
\newglossaryentry{n-mech}{
  type=symbolsmech,
  name=$\vec{n}$,
  description={Unit normal vector (Dimensionless)}
}
\newglossaryentry{vr-mech}{
  type=symbolsmech,
  name=$\vec{r}$,
  description={Position vector (\SI{}{\meter})}
}
\newglossaryentry{vp-mech}{
  type=symbolsmech,
  name=$\vec{p}$,
  description={Momentum vector (\SI{}{\kilo\gram\meter\per\second})}
}
\newglossaryentry{dvr-mech}{
  type=symbolsmech,
  name=$\dot{\vec{r}}$,
  description={Time derivative of position vector (\SI{}{\meter\per\second})}
}
\newglossaryentry{m-mech}{
  type=symbolsmech,
  name=$m$,
  description={Mass (\SI{}{\kilo\gram})}
}
\newglossaryentry{bW-mech}{
  type=symbolsmech,
  name=$\bm{W}$,
  description={Virial stress tensor (\SI{}{\pascal})}
}
\newglossaryentry{bT-mech}{
  type=symbolsmech,
  name=$\bm{T}$,
  description={Kinetic stress tensor (\SI{}{\pascal})}
}
\newglossaryentry{V-mech}{
  type=symbolsmech,
  name=$V$,
  description={Volume (\SI{}{\meter\tothe{3}})}
}
\newglossaryentry{F-mech}{
  type=symbolsmech,
  name=$\vec{F}$,
  description={Force vector (\SI{}{\newton})}
}

% Molecular dynamics Symbols
\newglossaryentry{r-md}{
  type=symbolsmd,
  name=$r$,
  description={Distance (Dimensionless)}
}
\newglossaryentry{gamma-md}{
  type=symbolsmd,
  name=$\gamma$,
  description={Friction constant (\SI{}{\kilo\gram\per\second})}
}
\newglossaryentry{kb-md}{
  type=symbolsmd,
  name=$\mathrm{k}_\mathrm{B}$,
  description={Boltzmann constant (\SI[per-mode=fraction]{}{\joule\per\kelvin})}
}
\newglossaryentry{T-md}{
  type=symbolsmd,
  name=$T$,
  description={T (\SI[per-mode=fraction]{}{\kelvin})}
}
\newglossaryentry{delta-md}{
  type=symbolsmd,
  name=$\delta$,
  description={Dirac's delta (Dimensionless)}
}
\newglossaryentry{dt-md}{
  type=symbolsmd,
  name=$dt$,
  description={Time diferential (\SI{}{\second})}
}

% Methods and results
\newglossaryentry{sigma-mr}{
  type=symbolsmr,
  name=$\sigma$,
  description={Particle diameter (Dimensionless)}
}

%\bibliographystyle{abbrvnat}
\bibliographystyle{unsrtnat}

% Command to make clear the changes  
\newcommand{\cor}[1]{\textcolor{blue}{#1}}

% Insight of mechanical response of hydrogels via patchy particle methodology
% INFO : used in the titlepage, copyright and stuff.
\author{Francisco Javier Vazquez Tavares}
\title{Exploration of mechanical response of \cor{polymeric gel} via \cor{molecular dynamics}}
%\subtitle{This is the subtitle}
\university{Instituto Tecnológico de Estudios Superiores de Monterrey \\ Campus Monterrey}
\lab{School of Engineering and Sciences}

\person{Supervisor}{Antonio Ortiz Ambriz}{~}
\person{Cosupervisor}{Claudia Elena Ferreiro Cordova}{~}
\person{Jury member}{Susana Marin Aguilar}{~}
\person{Jury member}{Alejandro Martinez Borquez}{~}
\logo{templateStuff/Logo_del_ITESM.svg.png}
%\logo{example-image-b}
%\logo{example-image-c}
\shoutouts{As recognition of stubbornness}


\begin{document}

\maketitle

\justifying

\chapter*{Abstract}

Specific, problem statement, previous solutions/attemps and our solution.

This thesis addresses the theoretical framework required for analyzing computer simulations of a patchy particle network's mechanical response under shear deformation.
This network is modeled after a hydrophilic polymeric network composed by PNIPAM microgel numerical protocol.
The protocol has been implemented in LAMMPS to take advantage of the parallelization of the velocity Verlet algorithm to solve the Langevin equations for a many particle system.
It was found that the methodology simulates a polymeric network with physical crosslinkers and qualitatively captures the mechanical response as well as the network structure caused by shear deformation.
\cor{clarify that PNIPAM uses chemical CL, while the polymeric gel in the simulaiton has physical crosslinkers.}
\cor{What are the difference between physical and chemical}

\cor{Contrast with other work. Add numeric values.}

\chapter*{Acknowledgements}

I would want to thank my advisers for the time they have provided me over the last two years.
I would also like to acknowledge the research group that emerged during the master's program.
Of course, I appreciate my family's and partner's emotional support and patience.
And I would like to thank Tecnológico de Monterrey for the tuition grant that allowed me to continue with my master's studies.
Finally, this work recieved support from Luis Aguilar, Alejandro de León, Alejandro Ávalos and Jair García of the Laboratorio Nacional de Visualización Científica Avanzada.

\tableofcontents
\listoffigures
%\listoftables
%\listoflistings



% Print glossary with custom grouping
%\printglossary[title=List of Symbols, toctitle=List of Symbols]
%\printglossary[type=symbolsmech]
%\printglossary[type=symbolsmd]


% 4. Use the entries to populate the glossaries
%\gls{sigma-mech}, \gls{sigma-md}.


\glsaddall

% 5. Print glossaries on the same page
%\chapter*{List of Symbols}

\clearpage
%\printglossary[title=List of Symbols, toctitle=List of Symbols]
\printglossary[type=symbolsmech]
\printglossary[type=symbolsmd]
\printglossary[type=symbolsmr]

%\begingroup
%\let\newpage\relax
%\printglossary[type=symbolsmech]
%\medskip
%\printglossary[type=symbolsmd]
%\endgroup


\mainmatter


\begin{align*}
    \epsilon_{ij} &\text{: Strain tensor} \\
    \sigma_{ij}  &\text{: Stress tensor} \\
    C_{ijkl} &\text{: Elastic Young's modulus} \\
    \dot{\epsilon}_p &\text{: Time derivative of the strain after plastic limit} \\
    \sigma_{y} &\text{: Yield point} \\
    \eta &\text{: Constant viscosity} \\
    \dot{\gamma} &\text{: Shear rate} \\
    G(t) &\text{: time-dependent relaxation modulus} \\
    \vec{T} &\text{: Traction vector} \\
    \vec{n} &\text{: Unit normal vector} \\
    \bm{\sigma} &\text{: Stress tensor} \\
    \expval{s} &\text{: Expected value of variable $s$} \\
    \vec{r} &\text{: Position vector} \\
    \vec{p} &\text{: Momentum} \\
    \dot{\vec{r}} &\text{: Time derivative of position vector} \\
    m &\text{: mass} \\
    \bar{s} &\text{: Time average of quantity $s$} \\
    \bm{W} &\text{: Virial Tensor} \\
    \bm{T} &\text{: Kinetic tensor} \\
    V &\text{: Volume} \\
    \vec{F} &\text{: Force vector} \\
    \gamma &\text{: Viscous/friction} \\
    \vec{R} &\text{: Random force} \\
    k_{\mathrm{B}} &\text{: Boltzmann's constant} \\
    T &\text{: Temperature} \\
    \gamma &\text{: damp} \\
    \vec{\eta} &\text{: Random force} \\
    dt &\text{: Time diferenctial/time step} \\
    \epsilon_{i,j} &\text{: Interaction energy} \\
    r_{i,j} &\text{: Distance} \\
    \sigma &\text{Diameter of the particles} \\
    r_c &\text{: Cutt off distance} \\
    K &\text{: energy constant} \\
    R_o^2 & \text{: Distance} \\
    w &\text{: Tunning parameter} \\
    \phi &\text{: Packing fraction} \\
    N_p &\text{: Number of particles} \\
    c &\text{: Cl concentration} \\
    T &\text{: Temperature} \\
    \nabla &\text{: Gradient} \\
    \hat{e} &\text{Basis vector} \\
    L &\text{Box length} \\
    S_{ab} &\text{: Stress lammps} \\
    W_{ab} &\text{: Virial stress lammps} \\
    \mathrm{damp} &\text{: Damp parameter} 
\end{align*}


\begin{align*}
    \eta :&~\text{Viscosity} \\
    \mu :&~\text{Friction} \\
    \gamma :&~\text{Shear strain} \\
    \dot{\gamma}:&~\text{Shear rate} \\
    \bm{\sigma}:&~\text{Stress tensor} \\
    \varepsilon :&~\text{Energy} \\
    \epsilon :&~\text{Strain}
\end{align*}

\newpage

\chapter{Introduction}\label{ch1:Intro}

%Try to describe the mechanical response of materials from first principles is an intelectual exercise that I found to be interesting enough to dedicate 2 years of my time and maybe more.

% About this thesis
\cor{The aim of this work is to explore the patchy particle methodology for molecular simulations to approximate the mechanical response of polymeric gels under a shear deformation.}
The research, conducted over two years, establishes a basis for further research, including the development of synthesis methods for polymeric gels, such as hydrogels, with customized mechanical properties, or the design of hydrogels to achieve specific mechanical performance requirements.
The proposed methodology relies on prior research\citep{gnanSilicoSynthesisMicrogel2017} in which the polymeric network of PNIPAM microgels was replicated using patchy particles and molecular dynamics.
To align with the institution's research, the computational approach was modified to simulate polymeric gels.
This was done to facilitate the gradual modification of the methodology for hydrogels.

A review of the literature reveals that hydrogels have a wide range of applications.
However, the precise origin of the mechanical response in \cor{polymeric gels} remains a topic of debate.
\cor{The major objective of this thesis is to} investigate a computational methodology \cor{to determine network characteristics that describe the mechanical response of polymeric gels.}
\cor{The network characteristics can be expressed as parameters that can be included into a viscoelastic relationship} in order to predict the mechanical response of polymeric gels.
It is important to clarify that the explicit formulation of such an equation is not within the scope of this thesis. 
%The final goal is to identify a relation between the network characteristics and its mechanical response.
To that end, the following specific objectives have been established:
\begin{itemize}
    \item Replicate the existing numerical protocol using patchy particles proposed in reference~\citep{gnanSilicoSynthesisMicrogel2017}.
    \item Adapt the protocol to create a more general network.
    \item Apply controlled shear deformations.
    \item Visualize the network before and after shear.
    \item Analyze the strain-stress curves of the material under different shear conditions and different network parameters.
\end{itemize}
\cor{The qualitative analysis of the strain-stress curves will serve as a tool for validating the methods used to approximate the mechanical response of polymeric gels.}

The following part explores the uses of hydrogels found in the literature, followed by an analysis of their distinctive mechanical responses. 
The subsequent chapter on the theoretical framework starts by describing the quantification of material attributes and the application of numerical simulations for system analysis. 
The final section of this document discusses the analysis of the numerical results and the conclusions.


\section{About Hydrogels}

A hydrogel is defined as a material characterized by a three-dimensional crosslinked polymer network.
The numerical protocol, originally designed to simulate PNIPAM microgels, was subsequently adapted to model polymeric gels.
Microgels are colloidal-scale particles composed of cross-linked polymer networks that can swell and deswell in response to external stimuli\citep{gnanSilicoSynthesisMicrogel2017}.
Gels are polymer networks formed through covalent or supramolecular bonds and are swollen in liquid media, including water or organic solvents\citep{guPolymerNetworksPlastics2020}.

Hydrogels generally demonstrate viscoelastic and viscoplastic mechanical behaviors.
The material's viscoelastic properties enable it appropriate for shock absorption, vibration damping, and mimicking biological tissues\cor{\citep{xuRapidlyDampingHydrogels2024a}}.
In addition, viscoplasticity allows the material appropriate for energy dissipation.
The following part of the section briefly presents specific application examples and closes with a basic overview of the hydrogel's mechanical response, serving as an introduction to the theoretical framework.

\subsection{Applications}

Three principal considerations influence the selection of application examples.
Initially, due to the imminent threat that climate change presents to the global ecosystem, research into environmentally relevant applications is vital.
Secondly, ought to align to the institution's strategic research priorities in biological applications.
Finally, examples from the smart materials field were included based on previous academic experience.

%\paragraph{Environmental applications} 
Hydrogels can effectively remove a wide range of toxic compounds, such as heavy metals, organic pollutants, and pathogens from aqueous environments.
The review~\citep{randoFunctionalBioBasedPolymeric2024} discusses an increasing trend of stimuli-responsive smart hydrogels, emphasizing their potential for both adsorption and detection of water contaminants.
One example can be found in reference~\citep{cinfrigniniGoldRushDesigning2024}, where it explores easy-to-make poly(acrylamide-co-acrylic acid) hydrogels as adsorbents for gold recovery from industrial wastewater containing other precious metals.
In addition, reviews~\citep{darbanHydrogelBasedAdsorbentMaterial2022a,songSynthesisHydrogelsTheir2022} explain the synthesis and adsorption mechanisms, crosslinking methods, their corresponding limitations, and outstanding contributions of applications in the fields of removing environmental pollutants of hydrogel-based adsorbent materials.

%\paragraph{Medical applications} 
Moreover, hydrogels have received a lot of attention in the biomedical field because of their high water content, biocompatibility, and adaptability into biological entities.
%They imitate natural tissue conditions, which improves cell survival and function.
For example, reference~\citep{wuAdvancementsHydrogelsCorneal2024} discusses hydrogels and their significance in corneal tissue engineering.  
It also investigates several formats, including stimuli-responsive versions.  
Reference~\citep{kaurHydrogelsPotentialBiomaterial2024} highlights some of the latest fascinating applications of bioactive hydrogels in the realm of antibacterial wound healing, oral drug delivery, cancer immunotherapy, tissue regeneration, and related potential biomedical aspects.  
And, reference~\citep{thummaIntroductionClassificationApplications2025} investigates the manufacture and therapeutic uses of several naturally occurring and synthesized hydrogels for cancer therapy, mostly through 3D modeling and printing.

%\paragraph{Smart materials} 
Finally, stimuli-responsive hydrogels are considered smart materials due to their ability to change chemical and physical characteristics in response to a variety of stimuli such as pH, temperature, chemicals, pressure, electricity, and light.
In~\citep{bishnoiCellulosebasedSmartMaterials2024}, they investigated the role and overview of cellulose-based hydrogels in energy storage systems.
According to~\citep{zhaoIntelligentHydrogelActuators2021}, near infrared laser-driven intelligent hydrogel actuators with a high response rate were manufactured by three-dimensional printing and hydrothermal synthesis.
In~\citep{shomePhotoresponsiveSmartHydrogels2024}, the basic mechanics of photo-responsiveness in hydrogels are reviewed, as well as their possible applications.
And, reference~\citep{duttaSmartMaterialsFlexible2024}, they discuss the state-of-the-art applications of hydrogels in flexible electronics, such as energy storage, touch panels, memristor devices, and sensors like temperature, gas, humidity, chemical, strain, and textile sensors, as well as the latest hydrogel synthesis methods.
To close, figure~\ref{fig:applications}, retrieved from~\citep{petelinsekToughHydrogelsLoadBearing2024}, presents a graphical representation of some of the prior and other applications.

% Clousure/transition paragraph
\begin{figure}[ht!]
    \centering
    \includegraphics[width=12cm]{figs/applications.png}
    \caption{Examples of tough hydrogels applied in the areas of (a) tissue engineering, (b) soft electronics, (c) shape memory materials, (d) 3D printing, (e) biomedical devices, and (f) adhesives. 
    It is worth noting that the majority of the highlighted examples have biomedical applications.\citep{petelinsekToughHydrogelsLoadBearing2024}.}\label{fig:applications}
\end{figure}


\subsection{Mechanical response}

Although hydrogels have been experimentally implemented in a variety of sectors, an explanation of their properties based on first principles is still lacking due to the materials intricate multi-scale nature\citep{senffTemperatureSensitiveMicrogel1999}.
The significance of developing a fundamental understanding of this material arises from the objective of tailoring the hydrogel for specific applications, while also considering its performance and durability throughout manufacturing and processing.
%\cor{How the reological properties are related?}
\cor{For instance, in \citep{townsendFlowBehaviorPrior2019}, it is highlighted that improved rheological characterisation of hydrogel precursor formulations and standardized testing for surgical application or 3D bioprinting is necessary.
According to \citep{ghicaFlowThixotropicParameters2016}, the authors develop many sodium carboxymethylcellulose hydrogels incorporating a BCS class II model drug to evaluate their flow and thixotropic characteristics.
According to \citep{yanRheologicalPropertiesPeptidebased2010}, the researchers discuss the literature regarding the rheological characterization of the bulk mechanical properties of peptide and polypeptide hydrogels, as these mechanical properties and their molecular mechanisms are essential in determining the suitability of these biomaterials for biotechnological applications.}
The intention of this work is therefore to validate a simulation methodology with the ultimate goal of providing a basis for facilitating the design process. 
%Having established this context, we now explore the mechanical properties that enable such a wide range of applications.

In a more detailed description, hydrogels consist of heterogeneous, often disordered polymer networks swollen with water, where molecular interactions (covalent bonds, physical crosslinks, entanglements, and solvent–polymer interactions) collectively determine macroscopic elasticity and viscoelasticity response~\citep{sheikoArchitecturalCodeRubber2019,naritaViscoelasticPropertiesPolyvinyl2013,kongEffectCrossLinkHomogeneity2024,varela-feijooMultiscaleInvestigationViscoelastic2023}.
Accurately bridging atomic-scale forces and chemical bond dynamics to bulk mechanical behavior involves coupling nonlinear polymer physics, solvent effects, and dynamic crosslink kinetics. 
Which presents a significant challenge to current theoretical and computational models.
This thesis used patchy particles to mimic crosslink kinetics and nonlinear polymer physics, with Langevin dynamics accounting for solvent effects and the time evolution of the system.

\cor{Generally, the deformation of a hydrogel leads to modifications in the bonds, intermolecular distances, molecular structure, and chain orientation.}
%When a hydrogel is deformed, the bonds, intermolecular distances, molecular conformation, and chain orientation are modified accordingly\cor{In which sense?}.
Also, any molecular event resulting in energy dissipation contributes to viscoelasticity, given the direct correlation between viscosity and energy dissipation.
One such factor is the polymer molecular weight, which affects the movement of entangled polymers.
Another factor that promotes molecular mobility is the interconnectivity and cohesion of the network.
These processes enable network flow under an applied force, thereby inducing viscoelasticity.

Viscoelastic materials have a mechanical reaction that varies between elastic and plastic deformation.
\cor{What is the diff. between plastic and viscous.}
Hydrogels initially respond elastically to a force before undergoing continual viscous deformation.
The elastic response is when a constant force over the surface of a material (stress) is applied and the deformation, also known as strain, appears instantly and maintains that deformation until the load is removed and the strain goes back into its initial value. 
In contrast, in a viscous response, the strain is no longer instantaneous; instead, it is time-dependent and undergoes continuous deformation during the applied force, and the strain remains after the force is removed.
Therefore, an ideal viscoelastic response occurs when the strain arises instantly and then becomes a time-dependent relationship; after the force is removed, the strain gradually returns its starting value.
As indicated in figure~\ref{fig:mechresponse0-a}.

\begin{figure}[ht!]
    \centering
    \centering
    \includegraphics[width=10cm]{figs/mechResponse/viscoElasticResponse-1.png}
    \caption{Mechanical response that arises from a constant applied force.
            The elastic response is instant, and when the force is removed, the strain returns to its starting value instantly.
            The viscous response is time dependent, and the strain no longer returns to its starting value after the applied force is removed.
            Finally, the ideal viscoelastic response combines both responses: an instant strain response followed by a time-dependent relation and a time-dependent recovery to the strain starting value once the applied force is removed.
    Figure retrieve from\citep{courbotRoleExtracellularMatrix2025}}\label{fig:mechresponse0-a}
\end{figure}

%An elastic responses occurs when the deformation is instantaneous and reversible in reaction to applied forces, whereas a viscous response occurs when the deformation is time-dependent and irreversible.
%An ideal viscoelastic response happens if the deformation appears immediately and then enters a time-dependent relationship; once the force is removed, the deformation disappears.

The time-dependent strain response is known as ``creep'' and it occurs when a constant stress is applied over a long period or the temperature is sufficiently high below the yield point~\citep{khaliliNanoindentationResponse3D2023,wangHyperelasticSwellingTough2024}.
On the other hand, when subjected to a constant strain, viscoelastic materials experience ``stress relaxation'', whereby stress decreases over time after the yield point~\citep{chaudhuriHydrogelsTunableStress2016}.
\cor{Delving} into more specific viscoelastic properties, figure~\ref{fig:mechresponse0-b} shows the creep and stress relaxation strain-stress relation for viscoelastic response.
It is understood that a more viscoelastic response is related with an increase in the strain value at a constant applied force.
On the other hand, when the strain is applied to the material, the stress relaxation phenomena appears as a viscoelastic response.
That is, that the stress decreases over time as the applied deformation takes place and once the strain it is removed, the stress returns to its starting value.

\cor{Talk about hysteresis to talk abput viscoelastic materials}

\begin{figure}[ht!]
    \centering
    \centering
    \includegraphics[width=10cm]{figs/mechResponse/viscoElasticResponse-2.png}
    \caption{The time dependent strain response to an applied stress it is also kwnon as \emph{creep}.
        The viscous response can be also consider as \emph{creep} phenomena.
        On the other hand, when the strain it is applied, the stress relaxation phenomena emerges, by a decrease of the stress under a constant deformation.
        Figure retrieve from\citep{courbotRoleExtracellularMatrix2025}}\label{fig:mechresponse0-b}
\end{figure}

It's essential to remember that elastic and viscoelastic responses are idealized models, and it's very uncommon to explain a material's mechanical reaction using only those responses.
As a result, it is necessary to discuss the viscoplastic response.
This mechanical response is similar to the viscoelastic response, with the main difference being that after the stress is released, the strain gradually decreases to a different value relative to its original.
This is illustrated in figure~\ref{fig:mechresponse0-c}.

\begin{figure}[ht!]
    \centering
    \centering
    \includegraphics[width=10cm]{figs/mechResponse/viscoElasticResponse-3.png}
    \caption{The viscoplastic response also shows the creep phenomenon; however, when the applied force is removed, the strain returns to a different value than when it began.
        Figure retrieve from\citep{courbotRoleExtracellularMatrix2025}}\label{fig:mechresponse0-c}
\end{figure}

In hydrogels an important factor influencing viscoelastic behaviour is the strength (\cor{Which strength.}) of the crosslinking mechanisms in the network.
In the case of weak crosslink mechanisms, stress relaxation is observed; however, in strongly crosslinked hydrogels, this process is prevented. 
This is due to the fact that weak bonds permit the release of energy and the separation of elements under force, thereby facilitating network relaxation or creep.
Furthermore, weak bonds are capable of responding to force by dissociating and rebinding, which can result in plastic or permanent deformations.
This is in contrast to strong crosslink mechanisms, which prevent the network from flowing and dissipating energy, thereby promoting the material's predominantly elastic properties.
It is widely accepted that permanent cross-links are responsible for the elastic properties of the material, while dynamic cross-links regulate energy dissipation~\citep{ProgressHydrogelToughening,moEnergydissipativeDualcrosslinkedHydrogels2021,gcohnGettingControlHydrogel2021}.

Finally, the water content has a significant impact on the viscoelasticity of the hydrogels.
Applying pressure to the hydrogels causes water to flow in or out of the network, resulting in a time-dependent response due to changes in volume. This leads to strand extension and therefore firmness enhancement.
This phenomenon is known as poroelasticity. 
It is important to note that these changes result in viscous dissipation, which in turn impacts the viscoelastic properties\citep{sheikoArchitecturalCodeRubber2019,courbotRoleExtracellularMatrix2025}.

In this regard, the thesis's main focus is on investigating how molecular dynamics can be used to evaluate how much of the mechanical response of hydrogels can be explained by the topological properties of polymeric networks.
The mathematical tools used for quantifying this relationship are explained in the following chapter, along with a more in-depth examination of the hydrogel's mechanical response.
Then, a detailed description of how to model hydrogels with molecular dynamics and obtain mechanical responses is provided.
Finally, the numerical results are shown in relation to the mechanical response described in the literature.

      % Chapter 1 Intro
%======================================================================
\chapter{Theoretical framework}\label{ch2:Theoframework}

\markright{Theoretical framework}
%======================================================================

\section{Soft colloids}\label{ch2:SoftColloids}

\paragraph{What is a colloid?} and type of colloids.

\paragraph{what is soft colloid} Link to hydrogels

\subsection{Gels}

\paragraph{What is a gel?} And examples and so on.

\paragraph{Argument} Why we can use a simulation protocol for microgels to modeled hydrogels?


\begin{itemize}
    \item Why we can model hydrogels as Soft colloids?
    \item Idea of patchy particles and insterpretaion of interaction rules
    \item teaser of simulation experiments
\end{itemize}


Hydrophilic gels that are usually referred to as hydrogels are networks of polymer chains that are sometimes found as colloidal gels in which water is the dispersion medium [1]\citep{ahmedHydrogelPreparationCharacterization2015a}.


Ahora no se si la intro solo centrarme en aplicaciones de hidrogeles y poner la discusion de cross link aquí y así.


\paragraph{Intro} From a structural standpoint, polymer networks are characterized by network ``juctions'' also known as ``crosslinks'', which consist of three or more groups departing from a core, interconnected by ``strands''.
The precise functionality of these branches is referred to as ``f''.
Strand configurations may include linear polymer chains, flexible short molecules, rigid struts/linkers, and other possible forms\citep{guPolymerNetworksPlastics2020}.

\paragraph{CrossLinking} The junctions and strands in polymer networks can be linked together via physical interactions (e.g., van der Waals interactions, hydrophobic interactions, Coulombic interactions, metalligand coordination) or covalent bonds.
Consequently, polymer networks are conventionally classified as either ``physical'' (supramolecular) or ``chemical'' (covalent) networks.
It is important to acknowledge that this classification does not always accurately reflect material properties. 
Bond strengths and exchange rates are much more informative\footnote{To illustrate, when sufficiently strong and static physical interactions are present, physical networks can exhibit behavior analogous to that of chemical networks. 
On the other hand, the incorporation of mechanisms for covalent bond exchange can result in chemical networks that demonstrate adaptable mechanical properties regulated by external stimuli.},
because the properties of polymer networks exhibit significant variability, contingent on the composition of the junctions and strands, as well as the formation and utilization conditions.

From this point of view, virtually all polymer networks, irrespective of their colloquial designation, structural characteristics, and physical properties, can be categorically classified into one of four predominant classes: thermosets, thermoplastics, elastomers, and gels\citep{guPolymerNetworksPlastics2020}.
Given our research focus on polymer networks associated with gels, we will not delve extensively into the properties of other polymer networks. 
However, it should be noted that these networks share certain characteristics.

\paragraph{Gel point} For any polymer network formation, a critical point exists at which the reaction phase transitions from liquid to solid. 
This point is referred to as the gel point.
Although gelation is not the focus of the research, the ``gel point'' is a crucial concept that demonstrates how certain characteristics can be shared.
At this point, many properties of the polymer networks change abruptly, and the properties that are more useful for applications can be reached beyond the gel point.
Consequently, numerous theoretical models\footnote{The classical approches to predict the relationship between gelation and the extent of reaction in step-growth polymerzation are the Carothers model and Flory-Stockmayer\citep{guPolymerNetworksPlastics2020}.} have been developed to predict the gel point for various network formation processes, including mean-field theory, critical percolation theory, and the kinetic gelation model\citep{guPolymerNetworksPlastics2020}.


\paragraph{Gels} Once the polymer network exceeds its gelation point, gels can be described as polymer networks that form through crosslinks or supramolecular bonds. 
These gels can become swollen in liquid media, such as water or organic solvents.
The network structure guarantees that the liquid is retained within the material.
Gels generally display Young's moduli within the range of 103–104 Pa, yet they often exhibit the capacity for significant deformation.
Examples of gels include gelatin, fibrin, and polyacrylamide hydrogel\citep{guPolymerNetworksPlastics2020}.

\subsection{Crosslinking mechanisms}\label{ch1:Cross-linking}

What a low hysteresis means in a hydrogel?

Hydrogels are made up of cross-linked polymer chains. 
The cross-linking creates a three-dimensional network with spaces (pores) between the polymer chains.  
These pores can absorb and hold a significant amount of water, sometimes up to  thousands of times their dry weight. 
The water molecules are retained within the  polymer network, which contributes to the porous nature of hydrogels. 
The polymer chains in hydrogels are flexible, allowing them to stretch and deform under  stress. 
Though hydrogels can be categorized by the types of polymer, cross-linking,  physical appearance, and network electrical charge [31], their common mechanical characteristics are porosity and elasticity. 
These two attributes make hydrogels  highly versatile and useful in various applications, such as in medical devices, drug  delivery systems, and tissue engineering, as discussed above.


\paragraph{Intro to cross linking}
A crosslinker is a molecule that functions as a bridge between polymer chains, thereby facilitating the formation of an interconnected network.
As previously suggested, it is pertinent to understand the mechanisms of crosslinking in order to gain insight into the correlation between these mechanisms and mechanical properties, such as elasticity, viscosity, solubility, glass transition temperature, strength, toughness, and melting point stiffness, swelling capacity, viscosity, and so forth\citep{priyaComprehensiveReviewHydrogel2024}.
The elements under consideration form stable bonds, which are comonly categorized into two main types: covalent (permanent) and physical (reversible)\citep{bustamantetorresHydrogelsClassificationAccording2021}.
However, recent mechanisms, such as mechanical crosslinker mechanics, have been demostrated to form bridges due to the topology of the constituents of the hydrogel.

\begin{figure}[!ht]
    \centering
    \includegraphics[width=0.8\textwidth]{figs/crosslinker_mechanisms.png}
    \caption{Image with the three different crosslinker mechanisms}
\end{figure}

\paragraph{Difference between physical and chemical bonds}
Although the concept of bonding is central to comprehending chemical structures and reactions.
The criteria employed to characterize a chemical bond, its physical origin, and its nature remain subjects of debate\citep{kumarDevelopingCriterionCharacterize2021}.
Consequently, establishing a precise distinction between ``covalent'' and ``non-covalent'' bonds remains challenging.
Therefore, the description of crosslinker mechanisms is limited to the principal interactions reported in articles and the synthesis process, rather than focusing on the classification of interactions as ``covalent'' or ``non-covalent'', but rather as ``reversible'' or ``irrevarsible''.
Also in the recent work \citep{picchioniHydrogelsBasedDynamic2018} it is shown a ``covalent'' reversible network. 
Nonetheless, a general consensus exists that non-covalent bonds are, as a rule, recognized as being weaker than covalent bonds and it is widely accepted that a distinguishing characteristic between covalent and noncovalent bonds is the energy of interaction and equilibrium bond distance\citep{kumarDevelopingCriterionCharacterize2021,novikovNonCovalentInteractionsPolymers2023}.

\begin{figure}[!ht]
    \centering
    \includegraphics[width=0.8\textwidth]{figs/bonds_energy.png}
    \caption{Bond energies of various types of permanent covalent crosslinks, weak physical cross-links, and dynamic covalent crosslinks.}
\end{figure}

\paragraph{Irreversible Cross-linking}
In irreverisble cross-linked hydrogels, polymer chains are synthesized by chain growth polymerization, graft copolymerization, addition and condensation polymerization, ezymatic method, reactive functions groups and gamma and electron beam polymerization\citep{maitraCrosslinkingHydrogelsReview2014,bustamantetorresHydrogelsClassificationAccording2021}.
This types of crosslinking mechanisms exhibit a high degree of strength and stability, leading to a structural arrangement of interconnected polymer chains that is more robust and resistant to alterations in environmental conditions, such as temperature and pH\citep{maitraCrosslinkingHydrogelsReview2014}.


\paragraph{Reversible Cross-linking}
In reversible cross-linked hydrogels, polymer chains are held together by molecular entanglements or physicochemical interactions, including van der Waals forces, hydrogen bonds, hydrophobic interactions, charge condensation, crystallite formation, and supramolecular chemistry\citep{bustamantetorresHydrogelsClassificationAccording2021,maitraCrosslinkingHydrogelsReview2014}.
Some of the syhtesis methods for reverisble crosslinkin mechanisms are ionic interaction, crystallization, stereocomplex formation, hydrophobized polysaccharides, protein interaction, amphilic copolymers and hydrogen bond\citep{maitraCrosslinkingHydrogelsReview2014,bustamantetorresHydrogelsClassificationAccording2021}.
Furthermore, molecular reversibility can be actually achieved in two different ways: either by making use of equilibrium reactions (e.g., the Diels-Alder one) or through dynamic exchange reactions (e.g., reaction of an excess amino groups with epoxide ones)\citep{picchioniHydrogelsBasedDynamic2018}. 


\paragraph{Mecanical bonds}
As previously mentioned, a novel class of polymer architecture has recently emerged within the field of polymer science kwnon as mechanically interlocked polymers (MIPs). 
These polymers are distinguished by the presence of a mechanical bond, that is, a constraint of two (or more) molecular components in space without the formation of covalent bonds\citep{hartMaterialPropertiesApplications2021}.
While these types of hydrogels exhibit substantial conformational flexibility while preserving a persistent spatial correlation between their components, their synthesis remains challenging.

\subsection{Still I'm stuck}\label{ch1:MechRepsonse}

\paragraph{General mechanical properties of the croslsinking}
Eventhough, physical crosslinking mechanisms are weaker than chemical ones, there numerous interactions contribute to complex behaviors.
Meanwhile chemical crosslinking mechanisms are easier to control than physical crosslinking mechanisms because their preparation is independent of pH\citep{bustamantetorresHydrogelsClassificationAccording2021} and they are very brittle due to structural inhomogeneity and lack of energy dissipation\citep{xuRoleChemicalPhysical2018}.


\paragraph{Reversible crosslinking}
The aforementioned interactions enable hydrogels to undergo structural changes without the rupture of any covalent bonds. 
Consequently, these materials exhibit enhance responsiveness to external stimuli, such as temperature, pH, or ionic strength. 
Additionally, hydrogels demonstrate high water sensitivity and thermal reversibility\citep{bustamantetorresHydrogelsClassificationAccording2021,priyaComprehensiveReviewHydrogel2024}.
These materials are known to exhibit distinctive properties, including ``self-healing'' behavior, where the gel can reform after being broken.
The lifespan of these hydrogels is brief, ranging from a few days to a maximum of a month, when maintained within physiological media.


\paragraph{Ir-reversible croslsinkg}
Consequently, chemically cross-linked hydrogels generally exhibit greater mechanical strength and long-term stability.  
Furthermore, it generally contains regions of the high cross-linking density and low degree of swelling (clusters), dispersed in the regions of the low cross-linking density and high swelling index due to the hydrophobic aggregation of the cross-linking agent\citep{bustamantetorresHydrogelsClassificationAccording2021}..


\paragraph{Network-mechanical response relation} Introduce the idea of how by understanding the network we can manipulate/control the mechanical response.

The research of hydrophilic polymers has been complex because the physical properties of solubility or swellability depend on different factors, such as the type of polymer, molecular weight, the ratio of polar groups, and degree of cross-linking\citep{bustamantetorresHydrogelsClassificationAccording2021}.
High molecular weight and a high degree of cross-linking will reduce the hydrophilicity of the molecule [18,19]\citep{bustamantetorresHydrogelsClassificationAccording2021}.


\paragraph{Tunnable mechanical response with applications} Review of articles of applications 

Just describe the phenomena and say that it depends on the structure and so on.


\section{Deformation and Stress}

\paragraph{What is stress?} Relation stress-strain.

\paragraph{How we model a material using stress-strain relations} Constitutive equations?

\subsection{Mechanical response}\label{ch2:MechResponse}

\paragraph{Viscoelasticity} An overview with equations.

\paragraph{Yield stress} and overview with equations.

\paragraph{Shear thinning} an overview with equations.


\section{Molecular dynamics}\label{ch2:MD}

\begin{itemize}
    \item Langevin equation
    \item Velocity Verlet
    \item Periodic Boundary Conditions
\end{itemize}

\subsection{Langevin dynamics}

From a general point of view there are two types of methods to make a quatitative description of systems: one focused on simulating dynamics at the microscale, and the other dedicated to deriving or establishing evolutionary equations at the macroscale\citep{wangMultiscaleModelingSimulation2025}.
Since the assumption is made that the mechanical response of a hydrogel is predominantly derived from its internal structure\footnote{Poner citas que desmuestrén que no es hipótesis, si no que se sabe} we choose to simulate the dynamics at the microscale.
Additionally, by treating the hyrogel as a colloid, permits applying molecular dynamics to model its response under shear deformation. 
Finally, there are two commonly used mathematical frameworks to model the molecular dynamics, the continuous time random walk (CTRW) model and the Langevin equation\citep{wangMultiscaleModelingSimulation2025}, in this work we decided\footnote{Supongo que eventualmente justificaré la desición.} to use the langevin dynamics mathematical framework.

This is because, the solid phase of the colloid has a large mass and will change their momenta after many collisions with the solvent molecules and the picture which emerges is that of the heavy particles forming a system with a much longer time scale than the solvent molecules\citep{Thijssen2007} and Langevin theory takes advantage of this difference in time scale to eliminate the details of the degrees of freedom of the solvent particles and represent their effect by stochastic and dissipative forces allowing longer simulations that would be impossible if the solvent were explicitly included\citep{pastorTechniquesApplicationsLangevin1994}.
However, the representation of the solvent by a stochastic and dissipative force, introduce the problem of characterize two very different timescales, one associated with the slow relaxation of the initial velocity of the brownian particle and another linked to the frequent collisions that the brownian particle suffers with particles of the bath\citep{tsl2006}\footnote{Para traer a colación la sensibilidad de la respuesta mecánica al parámetro de damp.}. 
Therefore, two terms are used to create a mathematical representation of the solvent: a frictional force proportional to the velocity of the particle and a fluctuating force. 
Hence,
\begin{gather}
    m\dv{\vec{v}(t)}{t}=\vec{F}(t)-m\gamma\vec{v}(t)+\vec{R}(t).\label{eqn:BrownianDyn1}
\end{gather}
The friction constant $\gamma$\footnote{Cuidado con las unidades. Hacer análisis dimensional, porque por la condición de correlación en $R$, $\gamma$ ocupa tener unidades de masa entre tiempo, pero en la ecuación, solo ocupa unidades de $1/s$.} parametrises the effect of solvent damping and activation and is commonly referred to as the collision frequency in the simulation literature, even though formally a Langevin description implies that the solute suffers an infinite number of collisions with infinitesimally small momentum transfer.
Also, the fact that the second term is not a function of the position of any of the particles involves the neglect of hydrodynamic interaction or spatial correlation in the friction kernel spatial correlation in the friction kernel\citep{pastorTechniquesApplicationsLangevin1994}.
On the other hand, $\vec{R}(t)$\footnote{No me acuerdo en donde está que se puede asumir que tiene distribución gaussiana.} is a ``random force'' subject to the following conditions
\begin{align*}
    \expval{\vec{R}(t)} &= 0 \\
    \expval{\vec{R}(t)\vec{R}(t')} &= 2k_{B}T\gamma\delta\qty(t-t') 
\end{align*}
The no time correlation is equivalent to assuming that the viscoelastic relaxation of the solvent is very rapid with respect to solute motions\footnote{Grote land Hynes [26] have investigated this assumption for motions involving barrier crossing and have found that while it is seriously in error for passage over sharp barriers (such as 12 recombination); it is quite adequate for conformational transitions such as might be found in polymer motions.\citep{pastorTechniquesApplicationsLangevin1994}}.

In comparing the results of Langevin dynamics with those of other stochastic methods [28-31], the relevant variable is the velocity relaxation time, $\tau_{v}$ which equals $\gamma^{-1}$\citep{pastorTechniquesApplicationsLangevin1994}
The Langevin equation improves conformational sampling over standard molecular dynamics\citep{paquetMolecularDynamicsMonte2015}.

\begin{itemize}
    \item Hablar acerca de que la fuerza aleatoria puede tener distribución gaussiana, pero no necesariamente.
    \item hablar de la ecuación de Green-Kubo: \[\eta=\frac{V}{k_B T}\int_{0}^{\infty}\expval{\sigma_{xy}(t)\sigma_{xy}(0)}\mathrm{d}t\]
    \item No se que tanto hablar de la idea de correlación y su aplicación en estos temas.
\end{itemize}

\subsection{Velocity Verlet}

L Verlet. Computer” experiments” on classical fluids. I. Thermodynamical properties of Lennard-Jones molecules. Physical Review, 159(1):98103, 1967.

J. M. Haile. Molecular Dynamics Simulation: Elementary Methods.  John Wiley \& Sons, Inc., New York, NY, USA, 1st edition, 1992.

Richard L. Burden and J. Douglas Faires. Numerical Analysis. Brooks  Cole, 8 edition, 12 2008.

Molecular Dynamics, Method For, and Microscale Heat Transfer. Molecular Dynamics Method. Bioinformatics, 2(Md):189–226, 2000.

Shichi Nose. A molecular dynamics method for simulations in the canonical ensemble. Molecular Physics, 52(2):255–268, 1984.

M Tuckerman, B J Berne, and G J Martyna. Reversible multiple time  scale molecular dynamics. The Journal of Chemical Physics, 97(3):1990,  1992.



\paragraph{Overview of the method}

\paragraph{Characteristics of the method}


\subsection{Periodic Boundary Conditions}

\subsection{Molecular stress}

%To characterize the behaviour of materials, constitutive relations serve as an input to the continuum theory\dots\footnote{Capaz e ir introduciendo ideas del Clausius\citep{clausiusXVIMechanicalTheorem1870}}

\paragraph{Motivation: Molecular stress is equivalent to continuum stress} \dots This derivation can be found in the apendix of\citep{admalUnifiedInterpretationStress2010}\footnote{Describe more if what is done in this article}.\footnote{(Eventualmente pondré esto en párrafo) Notation:
    $\bm{\sigma}$ Tensor, $\vec{\sigma}$ vector, $\sigma_{i,j}$ tensor, $\overline{\sigma}$ time average, 
}
Consider a system of $N$ interacting particles with each particle position given by
\begin{equation}
    \vec{r}_{\alpha} = \vec{r} + \vec{s}_{\alpha}\label{eqn:DerVirTen1},
\end{equation}
where $\vec{r}$ is the position of the center of mass of the system and $\vec{s}_\alpha$ is the position of each point relative to the center of mass.
Hence, we can express the momentum of each particle as
\begin{equation}
    \vec{p}_\alpha = m_\alpha\qty(\dot{\vec{r}}+\dot{\vec{s}}_\alpha) = m_\alpha\qty(\dot{\vec{r}}+\vec{\upsilon}_\alpha^{\mathrm{rel}}).\label{eqn:DerVirTen2}
\end{equation}
Before starting the procedure, lets take into account that the center of mass of the system is given by
\begin{equation}
    \vec{r} = \frac{\sum_{\alpha}m_\alpha\vec{s}_\alpha}{\sum_{\alpha}m_\alpha}\label{eqn:DerVirTen3},
\end{equation}
and by replacing~\eqref{eqn:DerVirTen1} in~\eqref{eqn:DerVirTen2} we get the following relations, which will be used later,
\begin{equation}
    \sum_\alpha m_\alpha\vec{r}_\alpha = \vec{0},\quad
    \sum_\alpha m_\alpha\vec{\upsilon}_\alpha^{\mathrm{rel}} = \vec{0}.\label{eqn:DerVirTen4}
\end{equation}

Now we can start by computing the time derivative of tensorial product $\vec{r}_\alpha\otimes\vec{p}_\alpha$\footnote{It is interesting to note that the tensorial product $\vec{r}_\alpha\otimes\vec{p}_\alpha$ has units of action and by tacking the time derivative we are dealing with terms that has units of energy.
},
\begin{equation}
    \dv{t}\qty(\vec{r}_\alpha\otimes\vec{p}_\alpha) = 
    \underbrace{\vec{\upsilon}_\alpha^{\mathrm{rel}}\otimes\vec{p}_\alpha}_{\mathrm{Kinetic~term}} 
        +
        \underbrace{\vec{r}_\alpha\otimes\vec{f}_\alpha}_{\mathrm{Virial~term}},\label{eqn:DerVirTen5}
\end{equation}
which is known as the \textit{dynamical tensor virial theorem} and it is simply an alternative form to express the balance of linear momentum.
This theorem becomes useful after making the assumption that there existis a time scale $\tau$, which is short relative to macroscopic processes but long relative to the characteristic time of the particles in the system, over which the particles remain close to their original positions with bounded positions and velocities.
Taking advantage of this property we can compute the time average of~\eqref{eqn:DerVirTen5},
\begin{equation}
    \frac{1}{\tau}\qty(\vec{r}_\alpha\otimes\vec{p}_\alpha)\bigg|_{0}^{\tau} = 
    \overline{\vec{\upsilon}_\alpha^{\mathrm{rel}}\otimes\vec{p}_\alpha} 
        +
    \overline{\vec{r}_\alpha\otimes\vec{f}_\alpha}.\label{eqn:DerVirTen6}
\end{equation}
Assuming that $\vec{r}_\alpha\otimes\vec{p}_\alpha$ is bounded, and the time scales between microscopic and continuum processes are large enough, the term on the left-hand side can be as small as desired by tacking $\tau$ sufficiently large and by summing over all particles we achieve the \textit{tensor virial theorem}:
\begin{equation}
    \overline{\mathbf{W}} = -2\overline{\mathbf{T}},\label{eqn:DerVirTen7}
\end{equation}
where
\begin{equation}
    \overline{\mathbf{W}} = \sum_\alpha\overline{\vec{r}_\alpha\otimes\vec{f}_\alpha}\label{eqn:DerVirTen8}
\end{equation}
is the time-average virial tensor and
\begin{equation}
    \overline{\mathbf{T}}=\frac{1}{2}\sum_\alpha\overline{\vec{\upsilon}_\alpha^{\mathrm{rel}}\otimes\vec{p}_\alpha}\label{eqn:DerVirTen9}
\end{equation}
is the time-average kinetic tensor.
This expression for the tensor virial theorem applies equally to continuum systems that are not in macroscopic equilibrium as well as those that are at rest.

The assumption of the difference between the time scales allow us to simplify the relation by replacing~\eqref{eqn:DerVirTen2} in~\eqref{eqn:DerVirTen9}, so that,
\begin{equation}
    \overline{\mathbf{T}}=
        \frac{1}{2}\sum_\alpha m_\alpha\overline{\vec{\upsilon}_\alpha^{\mathrm{rel}}\otimes\vec{v}_\alpha^{\mathrm{rel}}}
        +
        \frac{1}{2} \left[\overline{\sum_\alpha m_\alpha\vec{\upsilon}_\alpha^{\mathrm{rel}}}\right]\otimes\dot{\vec{r}}\label{eqn:DerVirTen10},
\end{equation}
which is not the simplification we expected, however, by the relations from~\eqref{eqn:DerVirTen4}, equation~\eqref{eqn:DerVirTen10} simplifies to\footnote{No estoy muy seguro si incluir una discusión acerca del término cinético en la expresión del virial. Posiblemente un párrafo\dots posiblemente lo ponga en la interpretación del teorema.
También, no se si ir metiendo interpretación durante la derivación o no, pero bueno.}
\begin{equation}
    \overline{\mathbf{T}}=
        \frac{1}{2}\sum_\alpha m_\alpha\overline{\vec{\upsilon}_\alpha^{\mathrm{rel}}\otimes\vec{\upsilon}_\alpha^{\mathrm{rel}}}\label{eqn:DerVirTen11}.
\end{equation}
On the other hand, instead of reducing the expression, we start to create the conection with the Cauchy stress tensor by distributing~\eqref{eqn:DerVirTen8} into an internal and external contributions,
\begin{equation}
    \overline{\mathbf{W}} = 
    \underbrace{\sum_\alpha\overline{\vec{r}_\alpha\otimes\vec{f}_\alpha^{\mathrm{int}}}}_{\overline{\mathbf{W}}_{\mathrm{int}}}
        +
        \underbrace{\sum_\alpha\overline{\vec{r}_\alpha\otimes\vec{f}_\alpha^{\mathrm{ext}}}}_{\overline{\mathbf{W}}_{\mathrm{ext}}}.\label{eqn:DerVirTen12}
\end{equation}
The time-average internal virial tensor takes into account the interaction between particle $\alpha$ with the other particles in the system, meanwhile, the time-average external virial tensor considers the interaction with atoms outside the system, via a traction vector $\vec{t}$ and external fields acting on the system represented by $\rho\vec{b}$, where $\rho$ is the mass density of it and $\vec{b}$ is the body force per unit mass applied by the external field.
Therefore we can express the following,
\begin{equation}
    \sum_\alpha\overline{\vec{r}_\alpha\otimes\vec{f}_\alpha^{\mathrm{ext}}}
    :=
    \int_{\delta\Omega}\vec{\xi}\otimes\vec{t}dA 
    +
    \int_{\Omega}\vec{\xi}\otimes\rho\vec{b}dV.\label{eqn:DerVirTen13}
\end{equation}
Where $\vec{\xi}$ is a position vector within the domain $\Omega$ occupied by the system of particles with a continuous closed surface $\delta\Omega$.
Assuming that $\Omega$ is large enough to express the external forces acting on it in the form of the continuum traction vector $\vec{t}$.

With this we can substitute the traction vector with $\vec{t}=\bm{\sigma}\vec{n}$, where $\bm{\sigma}$ represent the Cauchy stress tensor and applying the divergence theorem in~\eqref{eqn:DerVirTen13}, we have 
\begin{equation}
    \overline{\mathbf{W}}_{\mathrm{ext}}
     =\int_{\Omega}
        \left[
            \vec{\xi}\otimes\rho\vec{b}+\mathrm{div}_{\vec{\xi}}\qty(\vec{\xi}\otimes\bm{\sigma})
        \right]dV
        =
    \int_{\Omega}
        \left[
            \bm{\sigma}^{\mathrm{T}}
            +
            \vec{\xi}\otimes\qty(\mathrm{div}_{\vec{\xi}}\bm{\sigma}+\rho\vec{b})
        \right]dV\label{eqn:DerVirTen14}
\end{equation}
Since we assume that we are under equilibrium conditions, the term $\mathrm{div}_{\vec{\xi}}\bm{\sigma}+\rho\vec{b}$ is zero~\eqref{eqn:DerVirTen14} it simplifies to
\begin{equation}
    \overline{\mathbf{W}}_{\mathrm{ext}}
    =V\bm{\sigma}^{\mathrm{T}}\label{eqn:DerVirTen15}.
\end{equation}
By tacking into account that we integrate over the domain $\Omega$ we can say that we compute the spatial average of the Cauchy stress tensor,
\begin{equation}
    \bm{\sigma}_{\mathrm{av}} =\frac{1}{V}\int_\Omega\bm{\sigma}dV\label{eqn:DerVirTen16},
\end{equation}
in which $V$ is the volume of the domain $\Omega$.
Replacing~\eqref{eqn:DerVirTen15} into~\eqref{eqn:DerVirTen12}, the tensor virial theorem~\eqref{eqn:DerVirTen7} can be expressed as,
\begin{equation}
    \sum_\alpha\overline{\vec{r}_\alpha\otimes\vec{f}_\alpha^{\mathrm{int}}}
    +
    V\bm{\sigma}_{\mathrm{av}}^{\mathrm{T}}
    =
    -\sum_\alpha m_\alpha\overline{\vec{\upsilon}_\alpha^{\mathrm{rel}}\otimes\vec{\upsilon}_\alpha^{\mathrm{rel}}}.\label{eqn:DerVirTen17}
\end{equation}
Finally, solving for the Cauchy Stress tensor we get,
\begin{equation}
    \bm{\sigma}_{\mathrm{av}}
    =
    -\frac{1}{V}
    \left[
        \sum_\alpha\overline{\vec{f}_\alpha^{\mathrm{int}}\otimes\vec{r}_\alpha}
        +
        \sum_\alpha m_\alpha\overline{\vec{\upsilon}_\alpha^{\mathrm{rel}}\otimes\vec{\upsilon}_\alpha^{\mathrm{rel}}}
    \right],\label{eqn:DerVirTen18}
\end{equation}
an expression that describe the macroscopic stress tensor in terms of microscopic variables\footnote{It is important to acknowledge that several mathematical subtleties were not taken into consideration, however all the mathematical formality is adressed by Nikhil Chandra Admal and E. B. Tadmor in~\citep{admalUnifiedInterpretationStress2010}}.

To end the section it is important to show that~\eqref{eqn:DerVirTen18} is symmetric.
Therefore, we rewrite the internal force as the sum of forces between the particles,
\begin{equation}
    \vec{f}^{\mathrm{int}}_\alpha = \sum_{{\beta}_{\beta\neq\alpha}}\vec{f}_{\alpha\beta}\label{eqn:DerVirTen19},
\end{equation}
and substituting~\eqref{eqn:DerVirTen19} into~\eqref{eqn:DerVirTen18}, we have
\begin{equation}
    \bm{\sigma}_{\mathrm{av}}
    =
    -\frac{1}{V}
    \left[
        \sum_{{\alpha,\beta}_{\beta\neq\alpha}}\overline{\vec{f}_{\alpha\beta}\otimes\vec{r}_\alpha}
        +
        \sum_\alpha m_\alpha\overline{\vec{\upsilon}_\alpha^{\mathrm{rel}}\otimes\vec{\upsilon}_\alpha^{\mathrm{rel}}}
    \right].\label{eqn:DerVirTen20}
\end{equation}
Due to the property $\vec{f}_{\alpha\beta}=-\vec{f}_{\beta\alpha}$ we obtain the following identity
\begin{equation}
    \sum_{{\alpha,\beta}_{\beta\neq\alpha}}\vec{f}_{\alpha\beta}\otimes\vec{r}_\alpha 
    =
    \frac{1}{2}\sum_{{\alpha,\beta}_{\beta\neq\alpha}}\left(\vec{f}_{\alpha\beta}\otimes\vec{r}_\alpha+\vec{f}_{\beta\alpha}\otimes\vec{r}_\beta\right)
    =
    \frac{1}{2}\sum_{{\alpha,\beta}_{\beta\neq\alpha}}\vec{f}_{\alpha\beta}\otimes\left(\vec{r}_\alpha-\vec{r}_\beta\right).\label{eqn:DerVirTen21}
\end{equation}
Therefore, by replacing the identity of~\eqref{eqn:DerVirTen21} into~\eqref{eqn:DerVirTen20}, we have
\begin{equation}
    \bm{\sigma}_{\mathrm{av}}
    =
    -\frac{1}{V}
    \left[
        \frac{1}{2}
        \sum_{{\alpha,\beta}_{\beta\neq\alpha}}\overline{\vec{f}_{\alpha\beta}\otimes\left(\vec{r}_\alpha-\vec{r}_\beta\right)}
        +
        \sum_\alpha m_\alpha\overline{\vec{\upsilon}_\alpha^{\mathrm{rel}}\otimes\vec{\upsilon}_\alpha^{\mathrm{rel}}}
    \right],\label{eqn:DerVirTen22}
\end{equation}
expressed with indexical notation and using the eistein summation convention,
\begin{equation}
    \sigma^{\mathrm{av}}_{ij}
    =
    -\frac{1}{V}
    \left[
        \frac{1}{2}
        \sum_{{\alpha,\beta}_{\beta\neq\alpha}}\overline{f^{\alpha\beta}_{i}r^\alpha_{j} + f^{\beta\alpha}_{i}r^\beta_{j}}
        +
        \sum_\alpha m_\alpha\overline{\upsilon^{\alpha~\mathrm{rel}}_{i}\upsilon^{\alpha{\mathrm{rel}}}_j}
    \right],\label{eqn:DerVirTen23}
\end{equation}
which is the same expression implemented in~LAMMPS\citep{LAMMPS}.\footnote{No se si poner la referencia a la pagina de documentacion\href{https://docs.lammps.org/compute_stress_atom.html}{https://docs.lammps.org/compute\_stress\_atom.html}}


\newpage
      % Chapter 2 Theoretical framework
\chapter{Methods and Results}
% Patchy particle scheme for hydrophilic polymeric networks

Now that the theoretical framework was covered, lets delve into the numerical tools that will help to find relations between the polymeric network and the mechanical response. 
First, the patchy particle scheme for simulating PNIPAM cross-linked networks is presented, detailing the numerical simulation protocol.
Also an introduction to the LAMMPS software and how it was used to simulate this system is covered.
To finished the chapter, the numerical results are analyzed.

\section{Simulation protocol}

One of the microgels that has been the focus of significant research is the type that is based on PNIPAM cross-linked networks.
A flexible numerical portocol capable of designing these individual microgel particles has been presented in the literure \citep{gnanSilicoSynthesisMicrogel2017}.
The authors present a flexible numerical protocol capable of designing individual microgel particles based on PNIPAM crosslinked networks. 
This protocol can generate particles with properties comparable to the experimental ones.
In this project, we employ a similar protocol to explore its versatility and identify a numerical tool that can facilitate connections between network configuration and mechanical response.

The primary focus is on creating networks without spherical confinement and without mimicking the swelling behavior of PINIPAM microgels with temperature.
As a result, the central technique involves using a binary mixture of patchy particles to create a disordered polymeric network structure, which is then deformed using shear forces.
The primary benefit of this protocol is that previous numerical efforts in microgel modeling have predominantly concentrated on unrealistic networks consisting of chains of equivalent length, frequently establishing cross-linked connections on crystalline lattice regions or where closed polymer networks are assembled by directly integrating randomly dispersed cross-linkers with polymer chains.

\subsection{Patchy particles representation}

A patchy particle\citep{bianchiPhaseDiagramPatchy2006,bianchiTheoreticalNumericalStudy2008} can be defined as a sphere with radius $r$ containing $n$ spheres of radius $l<r$ on its surface.
The smaller spheres are typically referred to as ``patches,'' and the number of patches is often referred to as ``functionality''.
The center of the patches can be placed on the surface of the central particle. 
However, it can also be modified to be at a point inside the enclosed volume of the main particle.

The use of patchy particles as monomers and crosslinkers is a highly effective method since it allows for the integration of Langevin dynamics' infinitesimal representation with a particle with volume and functionality.
The functionality is important because it allows for the representation of the monomer and cross-linker molecules that can form a polymeric network.
However, it is important to note that the monomers and interaction sites are considered to be spherical.

To define the volume of the particle, a repulsive pairwise interaction is defined between the central particles.
Meanwhile, the formation of the polymeric network is encouraged by a pairwise interaction between patches.
Because this model is designed to simulate the final network, not the synthesis process, the pairwise interaction between central particles and patches is not defined.
%In contrast, the softness explained by particle interactions is characterized by the form of the repulsive pair potential between two particles.
Finally, the particle volume fraction contributes to the ability of the particles to deform or compress, in contrast to hard spheres\citep{vlassopoulosTunableRheologyDense2014}.%\footnote{The patchy particles are hard spheres, but the hydrogel network is a soft ``particle''.}

\subsection{Description of the system}

%\paragraph{Interaction potentials}
Let's start by describing the interaction potentials between patchy particles.
The interaction between the central particles is modeled using a Weeks-Chandler-Andersen repulsive potential.
\begin{gather}
    U_{\mathrm{WCA}}(r_{i,j}) =\left\{ 
        \begin{array}{ll}
            4\epsilon_{i,j}\left[\qty(\frac{\cor{O}}{r_{i,j}})^{12}-\qty(\frac{\cor{O}}{r_{i,j}})^6\right]+\epsilon_{i,j}, & r_{i,j}\in[0,2^{1/6}\cor{O}], \\
            0, & r_{i,j}>2^{1/6}\cor{O}
        \end{array}
\right.
    ,\label{eqn:CL-MO_interaction}
\end{gather}
where $r_{i,j}$ is the distance between the center of the central particles, \cor{$O$} is the diameter of the particles and $\epsilon_{i,j}$ is the energy of the interacton.
It is noteworthy that the WCA potential can be understood as the Lennard-Jones potential with a vertical shift of $\epsilon$ and a cutoff, resulting in entirely repulsive interactions.
On the other hand, the patch-patch interaction is modeled with an attractive potential,
\begin{gather}
    U_{\mathrm{patchy}}\qty(r_{\mu\upsilon}) = \left\{
        \begin{array}{ll}
            2\epsilon_{\mu\upsilon}\left(\frac{\cor{O}^4}{2 r_{\mu\upsilon}^4}-1\right)\exp\left[\frac{\cor{O}_p}{\qty(r_{\mu\upsilon}-r_{c})}+2\right], & r_{\mu\upsilon}\in\qty[0,r_c], \\
            0, & r_{\mu,\upsilon}>r_c,
        \end{array}
            \right.\label{eqn:patch-patch_interaction}
\end{gather}
where $r_{\mu\upsilon}$ is the distance between two patches, $\cor{O}_p$ is the diameter of the patches, $r_c$ is the cut distance of interaction set to $1.5\cor{O}_p$ and $\epsilon_{\mu,\upsilon}$ is the interaction energy between the patches.
%This potential can be interpreted as a reversible interaction.

In order to put in context the type of polymeric network that this potentials models, equation~\eqref{eqn:patch-patch_interaction} is compared with the FENE potential,
\begin{equation}
    U_{\mathrm{FENE}}(r) = -\frac{1}{2}KR_o^2\ln\left[1-\left(\frac{r}{R_o}\right)^2\right]+U_{WCA}(r),\label{eqn:FENEpot}
\end{equation}
where $R_o$ is the maximum extent of the bond and $K$ is the energy of the bond and has units of energy over area.
This potential it is used to represent covalent bonds between monomers in a polymers.

Figure~\ref{fig:patchpatchpot} compares the patch-patch interaction potential, the Lennard-Jones potential, and the FENE potential.
The Lennard-Jones and patch-patch potentials are qualitatively similar in that they both tend to zero after a certain cutoff distance.
In contrast, the FENE potential goes to infinity if the distance is too short or too wide.
This key distinction enables us to design the patch-patch interaction as a reversible interaction rather than the FENE, which is better suited to representing non-reversible interactions.
As a result, the computational methodology simulates a polymeric network with physical crosslinkers.

\begin{figure}[ht!]
    \centering
    \includegraphics[width=12cm]{figs/numerical/patchpatch.png}
    \caption{Comparisson between the potentials used in the simulation with standard potentials.
        Orange and green line are the potentials used in the simulation for the interaction between central particles and patches, respectively.
        While, the blue and red are the Lennard-Jones and FENE potentials, commonly used to represent interparticle interactions.
        All distances where modifed to match the patches particles for comparison reasons.
}\label{fig:patchpatchpot}
\end{figure}

The patch-patch interaction differs from the Lennard-Jones potential in that the minimum potential is horizontally translated, and the rate of change from the minimum to the cut distance is more apparent.
Equation~\eqref{eqn:patch-patch_interaction} represents a stronger interaction between the patches compared to a Lennard-Jones potential.

However, if the simulation for the polymeric network is simply performed using the Lennard-Jones and patch-patch potentials, the patches will form interactions of more than two patches, which is undesirable.
Hence, the interaction between patches is complemented by a three-body repulsive potential, defined in terms of~\eqref{eqn:patch-patch_interaction}, that provides an efficient bond-swapping mechanism, making it possible to easily equilibrate the system even at low temperatures, while at the same time retaining the single bond per patch condition\citep{sciortinoThreebodyPotentialSimulating2017}.
This interaction is given by:
\begin{gather}
    U_{\mathrm{swap}}(r_{l,m},r_{l,n}) = w\sum_{l,m,n}\epsilon_{m,n}U_3\qty(r_{l,m})U_3\qty(r_{l,n}),\quad r_{l,n}\in\qty[0,r_c],\label{eqn:swap_interaction}
\end{gather}
where
\begin{gather}
    U_{3}\qty(\cor{\alpha}) = \left\{
        \begin{array}{ll}
            1 & \cor{\alpha}\in\qty[0,\cor{\alpha}_{\min}], \\
            -U_{\mathrm{patchy}}\qty(\cor{\alpha})/\epsilon_{m,n}, & \cor{\alpha}\in\qty[\cor{\alpha}_{\min},\cor{\alpha}_c]
        \end{array}
        \right.\label{eqn:swapmod_interaction}.
\end{gather}
\cor{Where $\alpha$ is the distance between the patches $r_{l,m}$ or $r_{l,n}$.}
The sum in~\eqref{eqn:swap_interaction} runs over all triples of bonded patches (patch $l$ bonded both with $m$ and $n$).
$r_{l,m}$ and $r_{l,n}$ are the distances between the reference patch and the other two patches.
The parameter $\epsilon_{m,n}$ is the energy of repulsion, and $w$ is used to tune the swapping ($w=1$) and non-swapping bonds ($w\gg1$). 
The cutoff distance $\alpha_c$ is the same as in the potential of interaction between patches, meanwhile the minimum distance $\alpha_{\min}$ is the distance at the minimum of~\eqref{eqn:patch-patch_interaction}, \textit{i.e.} $\epsilon_{m,n}\equiv\abs{U_{\mathrm{patchy}}(\alpha_{\min})}$.

Figure~\ref{fig:swappot} shows the patch-patch potential with the swap potential.
The patch-patch potential in the figure is the energy of the interaction between patch $i$ and patch $k$.
The swap potential is the energy between patch $i$ and patch $k$, leaving the distance between patch $i$ and patch $j$ fixed.
Taking into account this, when the patches $i$ and $j$ are at the potential well ($r_{ij}=\cor{O}$), the interaction between $i$ and $k$ is null.
When the distance between patches $i$ and $j$ are bigger than the potential well but smaller than the cutoff distance ($r_{\mathrm{cut}}>r_{ij}>\cor{O}$), the interaction between $i-k$ is mildly attractive.
Finally, when the patches $i-j$ are bigger than the cutoff distance ($r_{\mathrm{cut}}<r_{ij}$), the interaction between $i-k$ is repulsive.

\begin{figure}[ht!]
    \centering
    \includegraphics[width=12cm]{figs/numerical/swapPotential.png}
    \caption{Swap potential for patch-patch interaction to ensure single bond per patch condition.
        Blue line represents the patch-patch interaction potential.
        Orange, green and red lines represent the swap potential with a fix distance between patch $i$ with patch $j$ and a free patch $k$.
        When the patches $i$ and $j$ are at the potential well, the interaction between partches $i-k$ and $j-k$ is repulsive (orange line).
        However, when the distance between patches $i$ and $j$ starts to increase, the repulsive interaction with patch $k$ diminishes.
    }\label{fig:swappot}
\end{figure}

Before proceeding to the description of the parameters for the simulation, it is necessary to describe the interaction between the central particle and the patches.
The patches and the central particles are linked by harmonic potentials.
\begin{align}
    E_r &= K_r\qty(r-r_{o})^2, \\
    E_\theta &= K_\theta\qty(\theta-\theta_{o})^2.
\end{align}
Where $r_o$ and $\theta_o$ represent the equilibrium bond distance and angle.
Meanwhile, $K$ is equal to $K=k/2$, where $k$ is the energy of the bond.
This allows us to create patchy particles with patches fixed at a convenient position.

\begin{figure}[ht!]
    \centering
    \includegraphics[width=8cm]{figs/numerical/interactionPatches.png}
    \caption{Interaction between patchy particles.
    Panels (a) and (b) represent the monomer-monomer and monomer-crosslink interactions, respectively.
    Meanwhile, panels (c) and (d) represent the swap potential interaction when the patches are not at the potential well distance, allowing the swap of bonds.
    }\label{fig:interactionPatches}
\end{figure}

All of the above descriptions are summarized in figure~\ref{fig:interactionPatches}.
Orange central particles indicate patchy particles with valence \num{2}, while red central particles indicate patchy particles with valence \num{4}, as illustrated in panels a and b.
On panels c and d, we can see the swap dynamic enabled by the three-body potential.
On panel C, a patchy particle approaches a pair of bonded patches.
Because of the swap potential, instead of forming a bond between the three patches, the interaction can shift to form a bond with the incoming particle.

\subsection{\cor{Reduce units}}

\cor{Before moving to a more extensive description of the system, it is crucial to recognize that the numerical simulations are run using the reduced Lennard-Jones units for computational and physical convenience.
The Lennard-Jones reduced units are achieved by rescaling the distance, energy, and mass to a characteristic quantity and setting $k_B=1$.
The distance is rescaled by the diameter of the particle, $r^*=r/O$; the energy is rescaled by the potential well of the interaction potentials $\epsilon^{*} = E/\epsilon$; meanwhile, the mass is rescaled to a characteristic quantity of the system $m^* = m/m_{\mathrm{ref}}$.
Finally, by setting $k_B$ to $1$ with the other quantities, the time is rescaled by $t^* = t\sqrt{\epsilon/(m_{\mathrm{ref}}O^2)}$.}

\cor{Although energy is not a fundamental unit, $k_\mathrm{B}$ is selected for rescaling as it is essential for establishing the reduced unit for temperature.
The reduced temperature is rescaled by $T^{*} = k_{\mathrm{B}}T/\epsilon$, or}
\begin{equation}
    \cor{\epsilon T^* = k_{\mathrm{B}}T.}
\end{equation}
\cor{Recognizing that $k_{\mathrm{B}}$ serves as a conversion factor between temperature and energy, this decision facilitates a straightforward comparison of thermal energy with interaction energy.
By establishing $T^*=1$, the thermal energy is equated to the potential well depth $\epsilon$.
Conversely, establishing $T^*=0.5$ represents a system having half the thermal energy relative to the potential well depth.}

\cor{From this moment onward, all quantities are presented in reduced Lennard-Jones units.
For clarity, quantities will be represented with an asterisk in their symbols, for example, $r \to r^{*}$ or $T \to T^{*}$, and so on.}

\subsection{\cor{Parameters}}

%\paragraph{Polymeric network parameters}
Let us now move on to the simulation's parameters.
Each system had a fixed number of patchy particles $N_p$, packing fraction $\phi$, and cross-link concentration $c$.
Based on these characteristics, we calculate the box's volume as well as the number of patchy particles of functionality 2 (PB) and functionality 4 (PA).

Due to limitations related to time and computing resources, the total number of particles is set to be $N_p=\num{8000}$.
This is a lower number of particles when compared with other simulations~\citep{gnanSilicoSynthesisMicrogel2017}.
Therefore, in order to compensate, we take the mean of five experiments.
In addition the time step is set to \num{0.001}$\cor{t^{*}}$ \cor{to ensure numerical stability. 
The stability criterion is denoted as $\mathrm{damp}^{*}dt^{*}<2$\citep{leimkuhlerContractionConvergenceRates2024}.
Since $\mathrm{damp}^{*}dt^{*} =\num{1d-4}$, numerical stability is ensured.}
This allow us to obtain good average observables.
Finally, the values of $\cor{K_r^{*}}$ and $\cor{K_\theta^{*}}$ are set to \num{100} to create a strong bond and prevent stretching of monomers (patchy particles B) and crosslinkers (patchy particles A).
The value of $r_o$ is set to \num{0.45}.
The value of $\theta_o$ is set to \SI{180}{\degree} for PB particles and \SI{109.4712}{\degree} for PA particles.

Once the parameters for the synthesis were set, the volume of the box was calculated by determining the volume of the patchy particles A and B and then scaling those values by the number of particles and the desired packing fraction.
\begin{align*}
    \cor{V_{\mathrm{box}}^{*}} &= \frac{N_{\mathrm{patchyA}}\cor{V_{\mathrm{patchyA}}^{*}}+N_{\mathrm{patchyB}}\cor{V_{\mathrm{patchyB}}^{*}}}{\phi}
\end{align*}
The number of patchy particles of type A is computed as $N_{\mathrm{patchyA}} = c N_p$, and the number of patchy particles of type B as $N_{\mathrm{patchyB}}= N_p - N_{\mathrm{patchyA}}= N_p(1 - c )$.
Finally, the temperature was set to be constant through all the assembly process, $\cor{T^*}=\num{0.05}$ in Lennard-Jones units; meanwhile, the damp parameter was set to \cor{$\mathrm{damp}=\num{0.1}$}.
It is very important to note that the damp controls the viscous response caused by the interaction between the thermal bath and the particles, which symbolizes the interaction of water molecules with the polymer network. The diameter of the central particle is set to $\cor{O^*}=1$ and the diameter of the patches at $\cor{O_p^*}=0.4$.

The energy of interaction between central particles is $\cor{\epsilon_{i,j}^*}=1$.
In contrast, the energy of interaction between patches is defined as follows:
The interaction of patchy particles B is set to $\cor{\epsilon_{\mu,\upsilon}^*}=1$, whereas the interaction between patchy particles A is set to $\cor{\epsilon_{\mu,\upsilon}^*}=0$, and the interaction between patches of patchy particles A with patchy particles B is set to $\cor{\epsilon_{\mu,\upsilon}^*}=1$.
This is to allow only crosslinker-monomer and monomer-monomer bonding.

%\paragraph{Deformation protocol}
Once the assembly simulation of the patchy particle network was done, we performed a shear deformation to the resulting network.
We select the shear deformation because shear forces dominate biological environments where hydrogels are typically deployed. 
Also, shear testing provides a more uniform stress field throughout the hydrogel sample compared to tensile testing. 
In rheological measurements using parallel plate or cone-and-plate geometries, the applied shear stress is distributed evenly across the sample, eliminating edge effects and stress concentrations that plague tensile testing.
Furthermore, shear rheometry excels at characterizing the complex viscoelastic properties that define hydrogel functionality.
Many hydrogels exhibit shear-thinning behavior that is critical for applications like injection and 3D bioprinting. 
The shear deformation was done at a constant shear rate in the $xy$ plane.

The shear rates were in the \num{d-3}\cor{[$1/t^*$]} order of magnitude, and the final strain was \num{15}.
This is to characterize the mechanical response and to deform beyond the plastic deformation limit.
Also, the variation of the shear rate was performed to see the viscoelastic response of the material.
The temperature and dampness parameters were set to be the same as in the assembly process.

\subsection{LAMMPS implementation}

As said before, the LAMMPS software is used to solve the Langevin equation in a many-particle system.
However, it is useful to explain how the simulation is defined in this software.
In this regard, the following paragraphs briefly discuss the damp parameter, the implementation of the swap potential~\eqref{eqn:swapmod_interaction}, the shear deformation, and the calculation of the stress tensor.

%\paragraph{damp}
Comparin equation~\eqref{eqn:MolDylammps1} with equation~\eqref{eqn:MolDylammps2}, the viscosity parameter $\gamma$ of the Langevin equation~\eqref{eqn:BrownianDyn1} does not appear; instead, the $\mathrm{damp}$ parameter appears.
This parameter is specified in time units and determines how rapidly the temperature is relaxed so that it can be more easily used as a thermostat\citep{LAMMPS}.
That is, if damp is set to \num{100}, the temperature will relax in a timespan of roughly \num{100} time units.
By making dimensional analysis, the damp factor can be thought of as inversely related to the viscosity of the solvent.
This tells us that a small damp represents a high-viscosity solvent and vice versa.
Therefore, since the damp is set to \cor{\num{0.1}}, the polymeric network is in a high-viscosity solvent.
In addition it is important to mention that all the simulations are done with Lennard Jones units.

%\paragraph{Three-body potential}
In regard to the swap potential, the \verb|threebody/table| pair style command is used to implement generic tabulated three-body interactions.
However, in LAMMPS, the tabulation is done on a three-dimensional plane of the two distances $\cor{r_{ij}^*}$ and $\cor{r_{ik}^*}$ with the angle $\theta_{ijk}$, where the forces on all three particles $I$, $J$, and $K$ lie within the plane defined by the three inter-particle distance vectors $\vec{\cor{r_{IJ}^*}}$, $\vec{\cor{r_{IK}^*}}$, and $\vec{\cor{r_{JK}^*}}$\citep{LAMMPS}.
Allowing the following property to project the forces onto the inter-particle distance vectors,
\begin{equation}
    \begin{pmatrix}\vec{\cor{f_i^*}} \\ \vec{\cor{f_j^*}} \\ \vec{\cor{f_k^*}}\end{pmatrix}
    =
    \begin{pmatrix}\cor{f_{i1}^*} & \cor{f_{i2}^*} & 0 \\ \cor{f_{j1}^*} & 0 & \cor{f_{j2}^*} \\ 0 & \cor{f_{k1}^*} & \cor{f_{k2}^*} \end{pmatrix}
    \begin{pmatrix}\vec{\cor{r_{ij}^*}} \\ \vec{\cor{r_{ik}^*}} \\ \vec{\cor{r_{jk}^*}}\end{pmatrix}.
\end{equation}
And due to symmetry interactions, $\cor{f_{i1}^*}=-\cor{f_{j1}^*}$, $\cor{f_{i2}^*}=-\cor{f_{k1}^*}$, and $\cor{f_{j2}^*}=-\cor{f_{k2}^*}$.

Therefore, to have a correct tabulation, it is necessary to project the force into the inter-particle plane.
Recalling that the force is equivalent to $-\nabla \cor{U^*}(\cor{r^*})$ and the potential has only radial dependence, the force can be expressed as
\begin{gather}
    \vec{\cor{f_n^*}} = -\pdv{\cor{U_{\mathrm{swap}}^{*}}(\cor{r_m^*},\cor{r_l^*})}{\cor{r^*}}\hat{e}_{\cor{r^*}},
\end{gather}
where $n$ represent the particle $i$, $j$ or $k$, while $m$ and $l$ are place holders for distances $ij$, $ik$ and $jk$.
Hence, 
\begin{align}
    \vec{\cor{f_{i}^*}} &= -\pdv{\cor{U_{\mathrm{swap}}^{*}}(\cor{r_{ij}^*},\cor{r_{ik}^*})}{\cor{r^*}}\hat{e}_{\cor{r^*}}\label{eqn:3body1}, \\
    \vec{\cor{f_{j}^*}} &= -\pdv{\cor{U_{\mathrm{swap}}^{*}}(\cor{r_{ji}^*},\cor{r_{jk}^*})}{\cor{r^*}}\hat{e}_{\cor{r^*}}\label{eqn:3body2}, \\
    \vec{\cor{f_{k}^*}} &= -\pdv{\cor{U_{\mathrm{swap}}^{*}}(\cor{r_{ki}^*},\cor{r_{kj}^*})}{\cor{r^*}}\hat{e}_{\cor{r^*}}\label{eqn:3body3}.
\end{align}
The projection of the force into the plane is computed via the dot product between $\hat{e}_{\cor{r^*}}$ and the basis that represents the plane. 
Which can be defined by the following 2-dimensional basis, $\hat{e}_1=\qty[1,0] $ and $\hat{e}_2=\qty[\cos\theta,\sin\theta] $, and $\theta$ is the angle between distances $\vec{\cor{r_{ij}^*}}$ and $\vec{\cor{r_{ik}^*}}$ since the software defines the plane using the distances between the particles.
With this we can compute the following projections:
\begin{align}
    \hat{e}_{\cor{r^*}} \cdot \hat{e}_1 &= 1\\
    \hat{e}_{\cor{r^*}} \cdot \hat{e}_2 &= \cos\theta. 
\end{align}
Also, the following vector can be defined as $\hat{e}_3 = \hat{e}_1 - \hat{e}_2 = \qty[1-\cos\theta,-\sin\theta]$ to represent the $j-k$ distance, and the projection will be
\begin{align}
    \hat{e}_{\cor{r^*}} \cdot \hat{e}_3 &= 1-\cos\theta.
\end{align}

With these projections the forces can be expressed in~\eqref{eqn:3body1},~\eqref{eqn:3body2}, and~\eqref{eqn:3body3} as follows:
\begin{align}
    \cor{f_{i1}^*} &= -\pdv{\cor{U_{\mathrm{swap}}^{*}}(\cor{r_{ij}^*},\cor{r_{ik}^*})}{\cor{r^*}}\label{eqn:3body4a}, \\
    \cor{f_{i2}^*} &= -\pdv{\cor{U_{\mathrm{swap}}^{*}}(\cor{r_{ij}^*},\cor{r_{ik}^*})}{\cor{r^*}}\cos\theta\label{eqn:3body4b}, \\
    \cor{f_{j2}^*} &= -\pdv{\cor{U_{\mathrm{swap}}^{*}}(\cor{r_{ji}^*},\cor{r_{jk}^*})}{\cor{r^*}}\qty(1-\cos\theta)\label{eqn:3body5}.
\end{align}
And due to the symmetry relations, the potential can be tabulated into the LAMMPS software to introduce the one-bond-per-patch condition and mitigate numerical instability during shear deformation.

The \verb|fix deform| command simulates a shear deformation on the $xy$ plane.
It uses the engineering deformation rate (\verb|erate| style) to adjust the box's dimension at a ``constant engineering strain rate''.
The length of the box, $\cor{L^{*}}$, will be modified over time,
\begin{gather}
    \cor{L^{*}}(t) = \cor{L_o^{*}}\qty(1 + \cor{\mathrm{erate}^{*}}~\cor{dt^{*}}),
\end{gather}
where $\cor{L_o^{*}}$ is the original box length and $\cor{\mathrm{erate}^{*}}$ is the shear rate in units of $1/\cor{t^*}$\citep{LAMMPS}.
This deformation is applied in the $xy$ plane, and the change in length is along the $x$ direction.
This set of parameters ensures that the volume does not change during the deformation\citep{LAMMPS}.
In addition to the \verb|erate| style, the \verb|remap| keyword was set to \verb|x| to remap particle positions without affecting their velocities\citep{LAMMPS}.
Setting remap to x causes the atoms to deform using an affine transformation that is identical to the box deformation.  
It is important to note that, while the atoms are effectively ``moving'' with the box over time, this is due to the remapping rather than their velocity, which tracks the box's change.
Finally, the \verb|flip| keyword is used to flip the box when the tilt factors exceed half the distance of the parallel box length to avoid computational inefficiency and errors\citep{LAMMPS}.

%\paragraph{fix stress} 
This section concludes with an explanation of how LAMMPS computes the stress tensor.
The \verb|compute stress/atom| calculates the stress tensor for an atom $I$ using the equation \citep{thompsonGeneralFormulationPressure2009,LAMMPS},
\begin{gather}
    \cor{S_{ab}^{*}} = -\cor{v_a^{*}} \cor{v_b^{*}} - \cor{W_{ab}^{*}}\label{eqn:stressLAMMPS},
\end{gather}
where $a$ and $b$ represent the spatial coordinates $x$, $y$ and $z$, and $\cor{W^*}_{ab}$ is the virial contribution given by
\begin{multline}
    \cor{W_{ab}^{*}} = \frac{1}{2}\sum_{n=1}^{N_p}\qty(\cor{r_{1a}^*} \cor{F_{1b}^{*}} + \cor{r_{2a}^*} \cor{F_{2b}^{*}})
            +\frac{1}{2}\sum_{n=1}^{N_b}\qty(\cor{r_{1a}^*} \cor{F_{1b}^{*}} + \cor{r_{2a}^*} \cor{F_{2b}^{*}}) \\
            +\frac{1}{3}\sum_{n=1}^{N_a}\qty(\cor{r_{1a}^*} \cor{F_{1b}^{*}} + \cor{r_{2a}^*} \cor{F_{2b}^{*}} + \cor{r_{3a}^*} \cor{F_{3b}^{*}})\label{eqn:virialLAMMPS}.
\end{multline}
\cor{The first term accounts for pairwise interactions, the second for bond contributions, and the third for the angle-based interaction.}
%Since we only declare pairwise interactions and bond and angle interactions, the virial contribution simplifies to

\begin{comment}
\begin{multline}
    \cor{W_{ab}^{*}} = \frac{1}{2}\sum_{n=1}^{N_p}\qty(\cor{r_{1a}^*} \cor{F_{1b}^{*}} + \cor{r_{2a}^*} \cor{F_{2b}^{*}})
    +\frac{1}{2}\sum_{n=1}^{N_b}\qty(\cor{r_{1a}^*} \cor{F_{1b}^{*}} + \cor{r_{2a}^*} \cor{F_{2b}^{*}}) \\
            +\frac{1}{3}\sum_{n=1}^{N_a}\qty(\cor{r_{1a}^*} \cor{F_{1b}^{*}} + \cor{r_{2a}^*} \cor{F_{2b}^{*}} + \cor{r_{3a}^*} \cor{F_{3b}^{*}})
            +\frac{1}{4}\sum_{n=1}^{N_d}\qty(\cor{r_{1a}^*} \cor{F_{1b}^{*}} + \cor{r_{2a}^*} \cor{F_{2b}^{*}} + \cor{r_{3a}^*} \cor{F_{3b}^{*}} + \cor{r_{4a}^*} \cor{F_{4b}^{*}}) \\
            +\frac{1}{4}\sum_{n=1}^{N_i}\qty(\cor{r_{1a}^*} \cor{F_{1b}^{*}} + \cor{r_{2a}^*} \cor{F_{2b}^{*}} + \cor{r_{3a}^*} \cor{F_{3b}^{*}} + \cor{r_{4a}^*} \cor{F_{4b}^{*}}) 
            +\sum_{n=1}^{N_f}\cor{r_{ia}^*} \cor{F_{ib}^{*}}.
\end{multline}
\end{comment}


Substituting the virial term~\eqref{eqn:virialLAMMPS} into the per-particle stress tensor~\eqref{eqn:stressLAMMPS} yields a similar expression for the previously derived stress~\eqref{eqn:DerVirTen23}.
The stress tensor in equation~\eqref{eqn:DerVirTen23} is a temporal and spatial average, whereas the stress tensor in~\eqref{eqn:stressLAMMPS} is defined per particle.
\com{What type of average?}
As a result, in order to study the mechanical response of the Cauchy stress tensor, a suitable time average, followed by a spatial average of the per-particle stress tensor along the patchy particle network, including the patches, is required.

\subsection{Assembly simulation and Shear deformation}

Now let's explain the simulation protocols for the assembly of the polymeric network and the applied deformation.
The assembly process starts by introducing the $N_p$ patchy particles at random positions in a box with lengths that match the desired packing fraction.
Then, the temperature is linearly increased from \num{0} to \num{0.05} in \num{500000} time steps to prevent numerical instability.
Thereafter, the temperature was held constant until the total energy of the system stabilized to a minimum.
This process was performed by varying the number of time steps until it was found that the patchy particle network percolated and the energy of the system didn't increase.
\com{Equil? Explain more why stop there.}
The number of time steps required to meet these conditions is \num{8d6}, which follows the previous \num{5d5} time steps.

After the assembly protocol was completed, the final configuration was saved to a file before applying shear deformation.
The patchy particle network was then subjected to a series of five deformations with matching shear rates.
For each shear deformation simulation, the system energy, stress tensor components, temperature, and phase space configurations were saved.
Following 5 sets of deformation with the same shear rate, a new set of deformations was performed with a different shear rate, beginning with the same initial configuration as the previous one.

It is important to mention that the saved measurements are rolling mean temporal averages.
The time average for the assembly protocol is specified in terms of the damp parameter: $100\cdot\cor{\mathrm{dt}^*}\cdot\cor{\mathrm{damp}^*}$, whereas for the shear deformation, the period was set \cor{such that in the time period the strain achieves} $0.05\cor{\zeta^{*}}$.
This was established for the energy, temperature, and stress tensors.
The intervals for saving the system's phase space were calculated based on the simulation time and strain for assembly and deformation.
After saving the metrics from all of the simulations, an average of the experiments with the identical settings was computed.

\section{Results}

Let's begin with the patchy particle configuration that results from the assembling stage. Figure~\ref{fig:assemblyCluster} shows how the color of the patchy particles indicates clusters. The patchy particles initiate at random positions (panel a); then after \num{8d6} time steps, almost every patch is coupled via the patch-patch interaction potential (panel b).
The number of time steps for this stage has been established by monitoring the energy.
The assembly protocol is stopped when more than 90\% of patches are in the same cluster.
Only one assembly protocol was used for each crosslinker concentration.

After the assembly procedure, shear deformation in the $xy$ plane was conducted.
Panel a of Figure~\ref{fig:flipPatches} shows the initial configuration of a shear deformation.
Panel b of the same image pictures the deformation at a strain of $0.5L$.
For computational stability, LAMMPS flips the deformed box to a box with a strain of $-0.5\cdot L$, as illustrated in panel c.
Finally, this process is repeated until the deformation reaches a strain of $15~\mathrm{L}$.

\begin{figure}[ht!]
    \centering
    \includegraphics[width=\textwidth]{figs/ComputaitonalResults/New/assemblyCluster.png}
    \caption{The network generated during the simulation's assembly stage is visualized as clusters.
        The color in each patchy particle represents the cluster's id.
        If the patches are the same color, they share the cluster's id.
        Panel (a) depicts the initial arrangement.
        Panel (b) depicts the assembly simulation's finished configuration.
        Ovito is used for color processing, with a \num{0.6} cut-off between each patch.
    }\label{fig:assemblyCluster}
\end{figure}


\begin{figure}[ht!]
    \centering
    \includegraphics[width=\textwidth]{figs/ComputaitonalResults/New/flipPatches.png}
    \caption{A graphical representation of the shear deformation applied to the patchy particle network in the $xy$ plane.
            Panel (a) displays the initial arrangement.
            Then, in panel (b), the box achieves the maximum tilt to assure numerical stability.
            Panel (c) then displays the flip manipulation, which continues the deformation.
            Finally, panel (d) shows that the box enters a configuration with no tilt but a system with strain $1$.
            Orange spheres are the central particle of monomers. 
            Red spheres represent the central particle of crosslinkers. 
            Blue spheres represent patches.
    }\label{fig:flipPatches}
\end{figure}

Furthermore, Figure~\ref{fig:shearBonds} shows the system's shear deformation after (panels a and b) and before (panels c and d).  
Orange spheres are the core particles of monomers.  
Red spheres indicate the crosslinker's core particles.  
Blue spheres symbolize patches. 
Panels b and d, on the other hand, display the bonds between patches, providing a useful view of the network's topology.
Although difficult to see, panel d shows that the bonds align parallel to the $x$ coordinate after shear.
After briefly discussing the configurational results of the patchy particle network, let's move on to the strain-stress relationships of the deformations.

\begin{figure}[ht!]
    \centering
    \includegraphics[width=\textwidth]{figs/ComputaitonalResults/New/shearBonds.png}
    \caption{Panels (a) and (c) show the system with patchy particles.
        Panels (b) and (d), on the other hand, illustrate the bonds between patches as a simplified representation of the network's structure.
        In addition, panels (a) and (b) are before the shear and (c) and (d) after the shear.
        After and before shear.
    }\label{fig:shearBonds}
\end{figure}

%\begin{figure}[ht!]
%    \centering
%    \includegraphics[width=\textwidth]{figs/ComputaitonalResults/New/flipBonds.png}
%    \caption{Flip Bonds}\label{fig:flipBonds}
%\end{figure}

%\subsection{Mechanical response
%\cor{\Large\citep{argunInterplaySpatialTopological2024}}

Figure~\ref{fig:stres-strainResults} shows the strain-stress curve of the patchy particle network, demonstrating the effect of different shear rates and crosslinker concentrations.
The stress is calculated as the assembly and time average of the stress tensor's $xy$ component.
From a qualitative perspective, the network's stress levels are rapidly increasing.
As a result, the patchy particle systems display two different responses.
In networks with 3\% and 5\% of crosslinkers at lower shear rates, it is evident that at a particular strain, the stress will cease to increase and eventually acquire a constant value.
At higher concnetrations, the stress rises until it reaches its maximum level.
When the maximum stress value is reached, the stress begins to drop until it reaches a stress value.

\begin{figure}[ht!]
    \centering
    \includegraphics[width=\textwidth]{figs/ComputaitonalResults/comp.pdf}
    \caption{Strain-stress relation for a set of 10 different shear rates applied to three different patchy particles systems with crosslinker concentration of 3, 5 and 10 percent.
        Each line represent a 5 experiments average with an assembly average and a moving time average.}\label{fig:stres-strainResults}
    %Computational results at different crosslinker concentrations. Each line reprensent the average of 5 experiments with \num{8000} patchy particles with a packing fraction of \num{0.5}, damp of \num{0.5} and the time average is in intervals of \num{5d-2}$\dot{\gamma}$.
\end{figure}

This behavior has also been documented in the mechanical response of a carbopol microgel system\citep{divouxStressOvershootSimple2011} and in other polymeric systems\citep{osakiStressOvershootPolymer2000a,ravindranathUniversalScalingCharacteristics2008,boukanyUniversalScalingBehavior2009}.
 The subject has been thoroughly researched, as outlined in relevant literature, including molecular dynamics simulations \citep{jeongEffectChainOrientation2017,caoSimulatingStartupShear2015,mohagheghiMolecularlyBasedCriteria2016,baigFlowEffectsMelt2010a} and an analysis in other articles by \citep{wangExploringStressOvershoot2009}.
According to references \citep{jeongEffectChainOrientation2017,janeschitz-krieglPolymerMeltRheology1983,pearsonFlowInducedBirefringenceConcentrated1989,masubuchiPrimitiveChainNetwork2020}, stress overshoot is one of the most significant nonlinear rheological phenomenon exhibited by polymeric liquids undergoing start-up shear above a certain flow strength.
Furthermore, the articles clarify that the primary driver of this phenomenon is the network's chain orientation.
This was proved by analyzing the system's birefringence, as well as the transient behavior of the order tensor of entanglement strands along the chains, as described in the tube theory for entangled polymers.
The reason for this is that the refractive index is determined by the orientation distribution of these bond vectors.
In other words, the better the alignment of the bond vectors, the greater the difference in refractive index across different directions. 
The anisotropy of the bond vectors produces an asymmetric polarizability tensor, which is seen macroscopically as birefringence.

Moving forward, figure~\ref{fig:stress-strainCLResults} illustrates the strain-stress relationship of three systems with varying crosslinker concentrations at \num{1}, \num{5}, and \num{10} milli shear rates.
The goal of this comparative analysis is to highlight the previously stated qualitative distinctions between them.
By focusing on the same shear rate, it is clear that the crosslinker concentration increases both the overshoot response and the constant stress value.
Furthermore, by evaluating the fluctuation in shear rate at a fixed concentration, we can better detect overshoot as the shear rate increases (same color in each panel).
\begin{figure}[ht!]
    \centering
    \includegraphics[width=\textwidth]{figs/ComputaitonalResults/compCl.pdf}
    \caption{Computational results at different crosslinker concentrations with different shear rates. Each line reprensent the average of 5 experiments with \num{8000} patchy particles with a packing fraction of \num{0.5}, damp of \num{0.5} and the time average is in intervals of \num{5d-2}$\dot{\gamma}$.}\label{fig:stress-strainCLResults}
\end{figure}
It is also worth noting that the peak strain value and peak height increase in proportion to the shear rate.
Panel C of figure~\ref{fig:stress-strainCLResults} shows the stress-relaxation response of viscoelastic materials.
Furthermore, we can see how the crosslinker enhances the viscoelastic response by increasing the maximum stress prior to stress relaxation.
However, it should be noted that in figure~\ref{fig:stress-strainCLResults}, the greatest stress levels occur before the strain value of $2$, which differs from the earlier reports \citep{jeongEffectChainOrientation2017,janeschitz-krieglPolymerMeltRheology1983,pearsonFlowInducedBirefringenceConcentrated1989,masubuchiPrimitiveChainNetwork2020}, which reported that the highest stress occurs at strain values of about $2-3$.
Furthermore, the strain shift at the maximum stress is indistinguishable.

Another intriguing finding is the change in the steady-stress regime after the overshoot.
Figure~\ref{fig:yieldStressResults} illustrates the average stress throughout the strain from \num{10} to \num{15} ($\gamma\in[10,15]$) with respect to the shear rate at various crosslinker concentrations.
The dashed lines represent a powerlaw fits $p_1\dot{\gamma}^{p_2}$, where $p_1$ and $p_2$ stands for parameter one and parameter two.
With exponents ($p_2$) $\{0.5901,0.4880,0.4045\}$ for the \num{10}\%, \num{5}\%, and \num{3}\% crosslinker concentrations, respectively.
This finding served as a starting point for further investigation into yield stress events in polymeric networks.
\begin{figure}[ht!]
    \centering
    \includegraphics[width=0.9\textwidth]{figs/ComputaitonalResults/yieldStress-fits.pdf}
    \caption{Time average of steady-stress regime from computational results. Dashed lines represent exponential fits.
    \cor{Quitar los puntos. La linea de ajuste no se usa, quitar. justificar }
    }\label{fig:yieldStressResults}
\end{figure}

The Herschel-Bulkey model can quantitatively describe a ''\textit{yield stress material},'' and the high congruence between the exponential fit and the numerical results indicates that the polymeric network exhibits similar phenomena.
However, the yield stress $\sigma_y$ was not defined using the Tresca or Von Mises criterion.
Instead, it was chosen to compute the temporal average of the final portion of the deformation, assuming that the system achieves plastic deformation after the overshoot.
%\subsection{Network analysis}


\begin{gather}
    f(\zeta^{*}) = c_1 (\zeta^{*})^{c_2}\label{eqn:fit} 
\end{gather}


\begin{table}[ht!]
\centering
\caption{Parameters of the fit model}\label{tab:fitParameters}
\begin{tabularx}{12cm}{|>{\centering\arraybackslash\bfseries}p{2cm}|>{\centering\arraybackslash}X|>{\centering\arraybackslash}X|>{\centering\arraybackslash}X|}
\toprule
\textbf{Cl~\%} & \textbf{$c_1$} & \textbf{$c_2$} \\
\midrule
\num{10}            & \num{7613.758267}  & \num{0.590128} \\
\midrule
\num{5}      & \num{5888.735284}  & \num{0.488094} \\
\midrule
\num{3}         & \num{5576.947312} &  \num{0.404582} \\
\bottomrule
\end{tabularx}
\end{table}



\begin{comment}
In order to describe the network with a quantitative framework, on figure~\ref{fig:network2} we show a set of histograms of the metrics from the gyration tensor for the same system of figure~\ref{fig:network1} along the hole deformation at different shear rates.
This gyration tensor corresponds to the biggest cluster in the simulation.
In panel a it is show the radii of gyration $R_g$, which is the distance of the patchy particle from its center of mass;
in panel b is the shape anisotropy $k^2$, which is \num{0} when all particles are spherically symmetric and \num{1} when all points lie on a line;
in panel c is the asphericity $b$, which has value of zero when the distribution of particles si spherically symmetric or whenever the particle distribution is symmetric with respecto the three coordinate axes, for example any platonic solid;
finally, in panel d is the acylindricity, which is zero when the distribution of particles is cylindrically symmetric or whenever the particle distribution is symmetric with respect to the two coordinate exis, for example, when particles are distributed uniformly on a regular prism.

In general terms, we can see that the increment of crosslinker concentration, the distirbution shifts to the right in the four metrics.
For panels c and d, this means that the cluster starts to deviate from spherical and cylindrical symmetry, meanwhile, for the anisitrpopy parameter, it tells us, that the cluster starts to align into a thin line.
Another important result, is that when the concentration of crosslinker it is increase, the radii of gyration has a more probability to have numeric value around \num{0}, indicating that the morphology of the cluster is spherically symmetric.
\end{comment}

%\begin{figure}[ht!]
%    \centering
%    \includegraphics[width=0.9\textwidth]{figs/ComputaitonalResults/metricsGyrationTensor.pdf}
%    \caption{Gyration tensor metrics for three different shear rates at constant \num{10}\% crosslinker concentration.}\label{fig:network2}
%\end{figure}





      % Chapter 3 Numerical Experiments
\chapter{Conclusion}

We conclude that we have a conclusion in two years.

The patchy particle protocol is a vlid methodology to simulate the mechanical response of polymeric networks.

In order to simulate hydrogels we need no add a FENE potential.

The reverible interactions are more sutiable to model microgel particle response.

For future works we can \dots analyze the following important topological measures associated with the entanglement network: the primitive path countour length, the number of entanglements per chain, the end-to-end length of an entanglement strand and the number of central particles per entanglement strand.



      % Chapter 4 Conclusion

\bibliography{biblio}     % names file phdrefs.bib as my bibliography file



\end{document}
