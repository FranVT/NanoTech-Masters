\chapter{Conclusion}

This study demonstrates that understanding a hydrogel as a polymeric gel formed by interconnected polymer chains that swell upon solvent absorption, without dissolution, facilitates the application of molecular dynamics to examine the material's mechanical reaction.
The utilization of patchy particles to model the monomers of polymers and crosslinkers is essential for simulating the mechanical response.
This work's implementation exclusively represent systems with physical crosslinking processes resulting from the interaction potential between patches.

\cor{Furthermore, percolation of a network with fully reversible interactions among all patches was achieved with a packing fraction of $\phi=\num{0.5}$ and a damping parameter of $\num{0.1}[1/t^*]$ at a temperature of $T^*=\num{0.05}$ for three distinct crosslinker concentrations: 3\%, 5\%, and 10\%.}
\cor{A viscoelastic response has been observed across all systems from the strain-stress relationships.
 The mechanical response was qualitatively aligned with other findings in the literature.
 One distinction is the estimated strain value at which the maximum stress occurs, which is slightly lower than those documented in the literature.}
\cor{On the other hand, the alignment with the Herschel-Bulkley model indicates that the system is more appropriately characterized as a viscoelastic liquid rather than a viscoelastic solid.
Nonetheless, the same fit confirms the shear-thinning behavior of the system, with the exponents being less than 1.}

\cor{The relationship between the mechanical response and the network remains insufficiently explored.
Nevertheless, the literature and analysis of the results suggest that the network begins to flow upon the application of shear deformation, exhibiting minimal resistance due to the unconstrained interaction between patches.
Meanwhile the overshoot may be attributed to the alignment of the irregular particle strands with the direction of deformation.}


Aquí menciona explícitamente que lograste obtener un modelo de Red que que percola y en la que los enlaces son reformas, esto es importante que lo menciones en tus conclusiones también.

En cuanto a perspectivas, pues puedes hablar un poco de qué nos faltan analizar muchas cosas, por ejemplo, 
    los efectos que tienen al cambiar las concentraciones, 
    menciona que la concentración usada fue solo valor elegido para probar que el modelo funcionaba, 
    pero que aún exploraremos otros valores de concentración.
    Más cosas que faltaron caracterizar los poros de la Red, caracterizar cómo cambian los poros durante la deformación.
puedes también hablar de qué queríamos ver más 
    en detalle la respuesta después de varias deformaciones sucesivas. Por ahí lo mencionas, pero ya no le exploramos tanto. 
También otra cosa que se nos quedó pendiente y que es la inspiración, es ver qué pasa cuando le ponemos inclusiones cómo se modifica la respuesta mecánica.
