\chapter{Introduction}\label{ch1:Intro}

%Try to describe the mechanical response of materials from first principles is an intelectual exercise that I found to be interesting enough to dedicate 2 years of my time and maybe more.

% About this thesis
\cor{The aim of this work is to explore the patchy particle model for molecular simulations to approximate the mechanical response of polymeric gels under a shear deformation.}
\cor{This work}, conducted over two years, establishes a basis for further research, including \cor{the specialization of a methodology for hydrogels to aid the design of this material for a desired application with specific mechanical performance requirements.}
The proposed methodology relies on prior research\citep{gnanSilicoSynthesisMicrogel2017} in which the polymeric network of PNIPAM microgels was replicated using patchy particles and molecular dynamics.
\cor{However, the computational approach was was modified to model polymeric gels in accordance with the objectives of this thesis.}

A review of the literature reveals that hydrogels have a wide range of applications.
However, the precise origin of the mechanical response in \cor{polymeric gels} remains a topic of debate.
\cor{The major objective of this thesis is to} investigate a computational methodology \cor{to determine network characteristics that describe the mechanical response of polymeric gels.}
\cor{The network characteristics can be expressed as parameters that can be included into a viscoelastic relationship} in order to predict the mechanical response of polymeric gels.
It is important to clarify that the explicit formulation of such an equation is not within the scope of this thesis. 
%The final goal is to identify a relation between the network characteristics and its mechanical response.
To that end, the following specific objectives have been established:
\begin{itemize}
    \item Replicate the existing numerical protocol using patchy particles proposed in reference~\citep{gnanSilicoSynthesisMicrogel2017}.
    \item Adapt the protocol to create a more general network.
    \item Apply controlled shear deformations.
    \item Visualize the network before and after shear.
    \item Analyze the strain-stress curves of the material under different shear conditions and different network parameters.
\end{itemize}
\cor{The qualitative analysis of the strain-stress curves will serve as a tool for validating the methods used to approximate the mechanical response of polymeric gels.}

The following part explores the uses of hydrogels found in the literature, followed by an analysis of their distinctive mechanical responses. 
The subsequent chapter on the theoretical framework starts by describing the quantification of material attributes and the application of numerical simulations for system analysis. 
The final section of this document discusses the analysis of the numerical results and the conclusions.

\section{About Hydrogels}

\cor{A hydrogel is defined as a material characterized by a three-dimensional crosslinked polymer network.
However the numerical protocol originally designed to simulate PNIPAM microgels, was subsequently adapted to model polymeric gels, becuase microgels are colloidal-scale particles composed of cross-linked polymer networks that can swell and deswell in response to external stimuli~\citep{gnanSilicoSynthesisMicrogel2017}.
This technique is flexible enough to allow for the necessary adjustments to create a polymeric network, rather than a cross-linked polymer network at the colloidal scale. 
This is possible because it uses two types of decorated particles to build the polymeric network. 
Additionally, the assembly environment, including confinement, temperature, packing fraction, and other factors, can be easily controlled.}

\cor{Gels are non-fluid colloidal network or polymeric networks that can expand throught its whole volume by a fluid~\citep{IUPACG026002025}.
Polymeric gels are polymer networks formed through covalent or supramolecular bonds and are swollen in liquid media, including water or organic solvents~\citep{guPolymerNetworksPlastics2020}.
Hydrogels generally demonstrate viscoelastic and viscoplastic mechanical behaviors.
The material's viscoelastic properties enable it appropriate for shock absorption, vibration damping, and mimicking biological tissues~\cor{\citep{xuRapidlyDampingHydrogels2024a}}.
In addition, viscoplasticity allows the material appropriate for energy dissipation.}
The significance of understanding the relationship between the polymeric network and mechanical response lies in the goal of improving the design of hydrogels for specific applications, while also taking into account their performance and durability during manufacturing and processing.

%\cor{How the reological properties are related?}
\cor{For example, in \citep{townsendFlowBehaviorPrior2019}, it is emphasized that enhanced rheological characterization of hydrogel precursor formulations and standardized testing for surgical applications or 3D bioprinting are essential.
The authors of \citep{ghicaFlowThixotropicParameters2016} develop various sodium carboxymethylcellulose hydrogels containing a BCS class II model drug to assess their flow and thixotropic properties.
The researchers in \citep{yanRheologicalPropertiesPeptidebased2010} examine the literature on the rheological characterization of the bulk mechanical properties of peptide and polypeptide hydrogels, as these properties and their molecular mechanisms are crucial in determining the suitability of these biomaterials for biotechnological applications.}

%The following part of the section briefly presents specific application examples and closes with a basic overview of the hydrogel's mechanical response, serving as an introduction to the theoretical framework.

\subsection{Applications}

%Three principal considerations influence the selection of application examples.
The following summary presents applications of hydrogels across three areas.
Given the pressing threat of climate change to the global ecosystem, research into environmentally pertinent applications is essential.
\cor{Secondly, examples that correspond with the institution's strategic research orientation toward biological applications}.
Ultimately, examples from the field of smart materials were incorporated, drawing upon prior academic experience.

%\paragraph{Environmental applications} 
Hydrogels can effectively remove a wide range of toxic compounds, such as heavy metals, organic pollutants, and pathogens from aqueous environments.
The review~\citep{randoFunctionalBioBasedPolymeric2024} discusses an increasing trend of stimuli-responsive smart hydrogels, emphasizing their potential for both adsorption and detection of water contaminants.
One example can be found in reference~\citep{cinfrigniniGoldRushDesigning2024}, where it explores easy-to-make poly(acrylamide-co-acrylic acid) hydrogels as adsorbents for gold recovery from industrial wastewater containing other precious metals.
In addition, reviews~\citep{darbanHydrogelBasedAdsorbentMaterial2022a,songSynthesisHydrogelsTheir2022} explain the synthesis and adsorption mechanisms, crosslinking methods, their corresponding limitations, and outstanding contributions of applications in the fields of removing environmental pollutants of hydrogel-based adsorbent materials.

%\paragraph{Medical applications} 
Moreover, hydrogels have received a lot of attention in the biomedical field because of their high water content, biocompatibility, and adaptability into biological entities.
%They imitate natural tissue conditions, which improves cell survival and function.
For example, reference~\citep{wuAdvancementsHydrogelsCorneal2024} discusses hydrogels and their significance in corneal tissue engineering.  
It also investigates several formats, including stimuli-responsive versions.  
Reference~\citep{kaurHydrogelsPotentialBiomaterial2024} highlights some of the latest fascinating applications of bioactive hydrogels in the realm of antibacterial wound healing, oral drug delivery, cancer immunotherapy, tissue regeneration, and related potential biomedical aspects.  
And, reference~\citep{thummaIntroductionClassificationApplications2025} investigates the manufacture and therapeutic uses of several naturally occurring and synthesized hydrogels for cancer therapy, mostly through 3D modeling and printing.

%\paragraph{Smart materials} 
Finally, stimuli-responsive hydrogels are considered smart materials due to their ability to change chemical and physical characteristics in response to a variety of stimuli such as pH, temperature, chemicals, pressure, electricity, and light.
In~\citep{bishnoiCellulosebasedSmartMaterials2024}, they investigated the role and overview of cellulose-based hydrogels in energy storage systems.
According to~\citep{zhaoIntelligentHydrogelActuators2021}, near infrared laser-driven intelligent hydrogel actuators with a high response rate were manufactured by three-dimensional printing and hydrothermal synthesis.
In~\citep{shomePhotoresponsiveSmartHydrogels2024}, the basic mechanics of photo-responsiveness in hydrogels are reviewed, as well as their possible applications.
And, reference~\citep{duttaSmartMaterialsFlexible2024}, they discuss the state-of-the-art applications of hydrogels in flexible electronics, such as energy storage, touch panels, memristor devices, and sensors like temperature, gas, humidity, chemical, strain, and textile sensors, as well as the latest hydrogel synthesis methods.
To close, figure~\ref{fig:applications}, retrieved from~\citep{petelinsekToughHydrogelsLoadBearing2024}, presents a graphical representation of \cor{some of the given examples} and other applications.

% Clousure/transition paragraph
\begin{figure}[ht!]
    \centering
    \includegraphics[width=12cm]{figs/applications.png}
    \caption{Examples of tough hydrogels applied in the areas of (a) tissue engineering, (b) soft electronics, (c) shape memory materials, (d) 3D printing, (e) biomedical devices, and (f) adhesives. 
    It is worth noting that the majority of the highlighted examples have biomedical applications.\citep{petelinsekToughHydrogelsLoadBearing2024}.}\label{fig:applications}
\end{figure}

\subsection{Mechanical response}

Although hydrogels have been experimentally implemented in a variety of sectors, an explanation of their properties based on first principles is still lacking due to the materials intricate multi-scale nature\citep{senffTemperatureSensitiveMicrogel1999}.
\cor{This explanation is of great relevance; for instance, cells react to the rigidity of their surroundings by mechanotransduction, which requires an ideal elastic modulus (G') to direct stem cell behavior and preserve distinct cell phenotypes. 
An inadequate modulus may lead to erroneous differentiation, cellular apoptosis, or scaffold malfunction\citep{holleMoreFeelingDiscovering2011,seligMechanotransductionStiffnessSensingMechanisms2020}.}

\cor{Yield stress influences printability and the capacity to be injected through a needle without obstruction or harm to encapsulated cells.
The hydrogel must possess a sufficiently high yield stress to preserve the configuration of a printed filament or to retain particles in suspension prior to deposition. 
An inadequate yield stress results in structural failure (if too low) or nozzle obstruction/excessive injection force (if too high)\citep{zhangCelluloseNanocrystalReinforced2020,mouserYieldStressDetermines2016}.}

\cor{Self-healing and thixotropic characteristics allow the hydrogel to establish a functional implant following injection by liquefying under high shear stress for convenient administration, then quickly re-establishing its network at the designated site. 
Without self-healing capabilities, the material would persist in a liquid state and be unable to effectively contain the substance or provide a proper seal\citep{yasuiRateIndependentSelfHealingDouble2022,guCollagenbasedInjectableSelfhealing2023}.}

\cor{Shear-thinning is crucial for enhancing product spreadability and sensory experience.
The hydrogel retains high viscosity when at rest.
Under pressure, it achieves low viscosity, facilitating easy application. 
Upon pressure release, it returns to a denser consistency to maintain its location.  
Inadequate rheological properties may result in a viscous product or a dense paste that is challenging to apply\citep{herrada-manchonEssentialGuideHydrogel2023,guoPhysicallyCrosslinkedAgarose2025}.}

\cor{Fatigue resistance and toughness determines the functional lifetime under cyclic mechanical loading.
In load-bearing applications, the hydrogel must dissipate energy (often through reversible sacrificial bonds) to resist crack propagation. A hydrogel with low toughness or poor fatigue resistance will fracture, delaminate, or fail prematurely under repetitive stress (e.g., joint movement), leading to a short and inefficient service life\citep{petelinsekToughHydrogelsLoadBearing2024}.}

%Having established this context, we now explore the mechanical properties that enable such a wide range of applications.

In a more detailed description, hydrogels consist of heterogeneous, often disordered polymer networks swollen in water, where molecular interactions (covalent bonds, physical crosslinks, entanglements, and solvent–polymer interactions) collectively determine macroscopic elasticity and viscoelasticity response~\citep{sheikoArchitecturalCodeRubber2019,naritaViscoelasticPropertiesPolyvinyl2013,kongEffectCrossLinkHomogeneity2024,varela-feijooMultiscaleInvestigationViscoelastic2023}.
Accurately bridging atomic-scale forces and chemical bond dynamics to bulk mechanical behavior involves coupling nonlinear polymer physics, solvent effects, and dynamic crosslink kinetics. 
Which presents a significant challenge to current theoretical and computational models.
This thesis used patchy particles to mimic crosslink kinetics and nonlinear polymer physics, with Langevin dynamics accounting for solvent effects and the time evolution of the system.

\cor{Generally, the deformation of a hydrogel leads to modifications in the bonds, intermolecular distances, molecular structure, and chain orientation.}
Furthermore, any molecular event that leads to energy dissipation enhances viscoelasticity, due to the direct relationship between viscosity and energy dissipation.
One such factor is the polymer molecular weight, which affects the movement of entangled polymers.
Another factor that promotes molecular mobility is the interconnectivity and cohesion of the network.
These processes enable network flow under an applied force, thereby inducing viscoelasticity.

Viscoelastic materials have a mechanical response that shifts between elastic and \cor{viscous} deformation.
Hydrogels initially have an elastic response to an applied force before experiencing ongoing viscous deformation.
The elastic response arises when a constant force (stress) is applied to a material's surface, resulting in immediate deformation (strain) that remains until the load is withdrawn, at which point the strain reverts to its original value. 
In contrast with a viscous response, the strain is not instantaneous; it is time-dependent and experiences continuous deformation while the force is applied, persisting even after the force is withdrawn.
Therefore, an ideal viscoelastic response results when strain occurs instantaneously, subsequently evolving into a time-dependent relationship; with the removal of force, the strain progressively reverts to its initial value.
As indicated in figure~\ref{fig:mechresponse0-a}.

\begin{figure}[ht!]
    \centering
    \centering
    \includegraphics[width=10cm]{figs/mechResponse/viscoElasticResponse-1.png}
    \caption{Mechanical response that arises from a constant applied force.
            The elastic response is instant, and when the force is removed, the strain returns to its starting value instantly.
            The viscous response is time dependent, and the strain no longer returns to its starting value after the applied force is removed.
            Finally, the ideal viscoelastic response combines both responses: an instant strain response followed by a time-dependent relation and a time-dependent recovery to the strain starting value once the applied force is removed.
        Figure retrieve from\citep{courbotRoleExtracellularMatrix2025}.}\label{fig:mechresponse0-a}
\end{figure}

%An elastic responses occurs when the deformation is instantaneous and reversible in reaction to applied forces, whereas a viscous response occurs when the deformation is time-dependent and irreversible.
%An ideal viscoelastic response happens if the deformation appears immediately and then enters a time-dependent relationship; once the force is removed, the deformation disappears.

The time-dependent strain response is known as ``creep'' and it occurs when a constant stress is applied over a long period or the temperature is sufficiently high below the yield point~\citep{khaliliNanoindentationResponse3D2023,wangHyperelasticSwellingTough2024}.
On the other hand, when subjected to a constant strain, viscoelastic materials experience ``stress relaxation'', whereby stress decreases over time after the yield point~\citep{chaudhuriHydrogelsTunableStress2016}.
\cor{Delving} into more specific viscoelastic properties, figure~\ref{fig:mechresponse0-b} shows the creep and stress relaxation strain-stress relation for viscoelastic response.
It is understood that a more viscoelastic response is related with an increase in the strain value at a constant applied force.
On the other hand, when the strain is applied to the material, the stress relaxation phenomena appears as a viscoelastic response.
That is, that the stress decreases over time as the applied deformation takes place and once the strain it is removed, the stress returns to its starting value.

%\cor{Talk about hysteresis to talk abput viscoelastic materials}

\begin{figure}[ht!]
    \centering
    \centering
    \includegraphics[width=10cm]{figs/mechResponse/viscoElasticResponse-2.png}
    \caption{The time dependent strain response to an applied stress it is also kwnon as \emph{creep}.
        The viscous response can be also consider as \emph{creep} phenomena.
        On the other hand, when the strain it is applied, the stress relaxation phenomena emerges, by a decrease of the stress under a constant deformation.
        Figure retrieve from\citep{courbotRoleExtracellularMatrix2025}.}\label{fig:mechresponse0-b}
\end{figure}

It's essential to remember that elastic and viscoelastic responses are idealized models, and it's very uncommon to explain a material's mechanical reaction using only those responses.
As a result, it is necessary to discuss the viscoplastic response.
This mechanical response is similar to the viscoelastic response, with the main difference being that after the stress is released, the strain gradually decreases to a different value relative to its original.
This is illustrated in figure~\ref{fig:mechresponse0-c}.

\begin{figure}[ht!]
    \centering
    \centering
    \includegraphics[width=8cm]{figs/mechResponse/viscoElasticResponse-3.png}
    \caption{The viscoplastic response also shows the creep phenomenon; however, when the applied force is removed, the strain returns to a different value than when it began.
        Figure retrieve from\citep{courbotRoleExtracellularMatrix2025}.}\label{fig:mechresponse0-c}
\end{figure}

\cor{The crosslinking mechanisms within the network significantly affect the viscoelastic behavior of hydrogels.}
In the case of weak crosslink mechanisms, stress relaxation is observed; however, in strongly crosslinked hydrogels, this process is prevented. 
This is due to the fact that weak bonds permit the release of energy and the separation of elements under force, thereby facilitating network relaxation or creep.
Furthermore, weak bonds are capable of responding to force by dissociating and rebinding, which can result in plastic or permanent deformations.
This is in contrast to strong crosslink mechanisms, which prevent the network from flowing and dissipating energy, thereby promoting the material's predominantly elastic properties.
It is widely accepted that permanent cross-links are responsible for the elastic properties of the material, while dynamic cross-links regulate energy dissipation~\citep{ProgressHydrogelToughening,moEnergydissipativeDualcrosslinkedHydrogels2021,gcohnGettingControlHydrogel2021}.

Finally, the water content has a significant impact on the viscoelasticity of the hydrogels.
Applying pressure to the hydrogels causes water to flow in or out of the network, resulting in a time-dependent response due to changes in volume. This leads to strand extension and therefore firmness enhancement.
This phenomenon is known as poroelasticity. 
It is important to note that these changes result in viscous dissipation, which in turn impacts the viscoelastic properties\citep{sheikoArchitecturalCodeRubber2019,courbotRoleExtracellularMatrix2025}.

In this regard, the thesis's main focus is on investigating how molecular dynamics can be used to evaluate how much of the mechanical response of hydrogels can be explained by the topological properties of polymeric networks.
The mathematical tools used for quantifying this relationship are explained in the following chapter, along with a more in-depth examination of the hydrogel's mechanical response.
Then, a detailed description of how to model hydrogels with molecular dynamics and obtain mechanical responses is provided.
Finally, the numerical results are shown in relation to the mechanical response described in the literature.

