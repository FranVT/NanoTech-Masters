\chapter{Methods and Results}
% Patchy particle scheme for hydrophilic polymeric networks

Now that the theoretical framework is covered, let's delve into the numerical tools that will help us find relationships between the polymeric network and the mechanical response. 
First, the patchy particle scheme for simulating \cor{polymeric} networks is presented, detailing the \cor{modifications of the} numerical simulation protocol \cor{presented in~\cite{gnanSilicoSynthesisMicrogel2017}}.
Also, an introduction to the LAMMPS software and how it was used to simulate this system is covered.
To finish the chapter, the numerical results are analyzed.

\section{Simulation protocol}

\cor{The numercial techinique to create a polymeric network in silico is based on} a flexible numerical method that can imitate \emph{individual microgel particles} made from PNIPAM crosslinked networks~\citep{gnanSilicoSynthesisMicrogel2017}.
The primary benefit of this protocol is that previous numerical efforts in \cor{polymeric networks} modeling have predominantly concentrated on unrealistic networks consisting of chains of equivalent length, frequently establishing cross-linked connections on crystalline lattice regions or where closed polymer networks are assembled by directly integrating randomly dispersed cross-linkers with polymer chains\citep{grasingerPolymerNetworksWhich2023,tobitaDesignControlPolymeric2025}.

\cor{In this work, the numerical method is modified to create a polymeric gel rather than a microgel particle.
To achieve that, the spherical confinement and the temperature-sensitive swelling potential are removed from the original protocol.}
\cor{Consequently, the main approach involves using a binary mixture of patchy particles to construct a disordered polymeric network structure, followed by the application of shear deformation.}
\cor{To decrease the solvent's impact on the mechanical response during shear deformation, both the temperature and the friction coefficient are held at low values throughout the assembly and shear simulations.}

\subsection{Patchy particles representation}

A patchy particle\citep{bianchiPhaseDiagramPatchy2006,bianchiTheoreticalNumericalStudy2008} can be defined as a sphere with radius $r$ containing $n$ spheres of radius $l<r$ on its surface.
The smaller spheres are typically referred to as ``patches,'' and the number of patches is often referred to as ``functionality''.
The center of the patches can be placed \cor{inside the enclosed volume} of the central particle.

\cor{Making use of patchy particles as monomers and crosslinkers is an efficient approach, as it facilitates the incorporation of Langevin dynamics' infinitesimal representation with a particle possessing volume and functionality.
The functionality is crucial since it allows the formation of the polymeric network~\citep{ramadhanEnzymaticallyPreparedDual2019}. 
In more detail, the volume of the particle is modeled with a repulsive pairwise interaction defined between the central particles.
Meanwhile, the formation of the polymeric network is encouraged by a pairwise interaction between patches.
Furthermore, the pairwise interaction between central particles and patches is null.
The subsequent paragraphs give a detailed description of the pairwise interactions.
After that, the selection of the numeric values of the parameters is explained.
}

%The use of patchy particles as monomers and crosslinkers is a highly effective method since it allows for the integration of Langevin dynamics' infinitesimal representation with a particle with volume and functionality.
%The functionality is important because it allows for the representation of the monomer and cross-linker molecules that can form a polymeric network.
%However, it is important to note that the monomers and interaction sites are considered to be spherical.

%\cor{To asign volume to the particles a respulsive interaction is defined betwee the central particles.
%Meanwhile, the formation of the polymeric network is encouraged by a pairwise interaction between patches.
%}



%In contrast, the softness explained by particle interactions is characterized by the form of the repulsive pair potential between two particles.
%Finally, the particle volume fraction contributes to the ability of the particles to deform or compress, in contrast to hard spheres\citep{vlassopoulosTunableRheologyDense2014}.%\footnote{The patchy particles are hard spheres, but the hydrogel network is a soft ``particle''.}

\subsection{Description of the system}

\cor{Let's start with the description of the patchy particles.}
The patches and the central particles are linked by harmonic potentials.
\begin{align}
    E_r &= \cor{K_{br}}\qty(r-\cor{r_{bo}})^2, \\
    E_\theta &= \cor{K_{b\theta}}\qty(\theta-\cor{\theta_{bo}})^2.
\end{align}
Where \cor{$r_{bo}$} and \cor{$\theta_{bo}$} represent the equilibrium bond distance and angle.
Meanwhile, \cor{$K_{br}$} and \cor{$K_{br}$} are equal to $k/2$, where $k$ is the energy of the bond.
\cor{This allow to placed the patches at desire positions.
It is important to recognize that the patchy particles can also be represented through a rigid interaction between the central particle and the patch.
As a result, to rectify the numerical problems, the central-patch interaction was modified using a bond potential, so resolving those concerns.
Now, let us continue to describe the pairwise interactions between patchy particles.
}

%\paragraph{Interaction potentials}
%Let's start by describing the interaction potentials between patchy particles.
The interaction between the central particles is modeled using a Weeks-Chandler-Andersen repulsive potential.
\begin{gather}
    U_{\mathrm{WCA}}(r_{i,j}) =\left\{ 
        \begin{array}{ll}
            4\epsilon_{i,j}\left[\qty(\frac{\cor{D}}{r_{i,j}})^{12}-\qty(\frac{\cor{D}}{r_{i,j}})^6\right]+\epsilon_{i,j}, & r_{i,j}\in[0,2^{1/6}\cor{D}], \\
            0, & r_{i,j}>2^{1/6}\cor{D}
        \end{array}
\right.
    ,\label{eqn:CL-MO_interaction}
\end{gather}
where $r_{i,j}$ is the distance between the center of the central particles, $\cor{D}$ is the diameter of the particles and $\epsilon_{i,j}$ is the energy of the interacton.
It is noteworthy that the WCA potential can be understood as the Lennard-Jones potential with a vertical shift of $\epsilon$ \cor{and a cutoff}, resulting in entirely repulsive interactions.
On the other hand, the patch-patch interaction is modeled with an attractive potential,
\begin{gather}
    U_{\mathrm{patchy}}\qty(r_{\mu\upsilon}) = \left\{
        \begin{array}{ll}
            2\epsilon_{\mu\upsilon}\left(\frac{\cor{D}^4}{2 r_{\mu\upsilon}^4}-1\right)\exp\left[\frac{\cor{D}_p}{\qty(r_{\mu\upsilon}-r_{c})}+2\right], & r_{\mu\upsilon}\in\qty[0,r_c], \\
            0, & r_{\mu,\upsilon}>r_c,
        \end{array}
            \right.\label{eqn:patch-patch_interaction}
\end{gather}
where $r_{\mu\upsilon}$ is the distance between two patches, $\cor{D}_p$ is the diameter of the patches, $r_c$ is the cut distance of interaction set to $1.5\cor{D}_p$ and $\epsilon_{\mu,\upsilon}$ is the interaction energy between the patches.
%This potential can be interpreted as a reversible interaction.

\begin{figure}[ht!]
    \centering
    \includegraphics[width=12cm]{figs/numerical/patchpatch.png}
    \caption{\cor{Comparison between the potentials used in the simulation with standard potentials.
        The orange and green lines are the potentials used in the simulation for the interaction between central particles and patches, respectively.
    The blue and red lines represent the Lennard-Jones and FENE potentials, which are the commonly used to represent interparticle interactions.}
    \cor{For comparison purposes, the parameters are set to \num{1}, $K=1$, $R_o=1$, $\epsilon=1$, $D=1$}
}\label{fig:patchpatchpot}
\end{figure}

\cor{In contrast with the well-known Lennard-Jones potential, the potential well of the patch-patch interaction is at the distance equal to the diameter of the particle, meanwhile the Lennard-Jones potential is $0$ at that distance.
Moreover, equation~\eqref{eqn:patch-patch_interaction} indicates a more strong interaction between the patches than a Lennard-Jones potential.
This is more appropriately depicted in Figure~\label{fig:patchpatchpot}.
}

\cor{In the same figure it is shown the FENE potential}
\begin{equation}
    U_{\mathrm{FENE}}(r) = -\frac{1}{2}KR_o^2\ln\left[1-\left(\frac{r}{R_o}\right)^2\right]+U_{WCA}(r),\label{eqn:FENEpot}
\end{equation}
where $R_o$ is the maximum extent of the bond and $K$ is the energy of the bond and has units of energy over distance squared.
This potential is commonly used to represent covalent bonds between monomers in a polymers\citep{leRheologyStructuralProperties2008}.
\cor{This key distinction enables us to design the patch-patch interaction as a reversible interaction rather than a non-reversible interaction via the FENE potential.}
As a result, the computational methodology simulates a polymeric network with physical crosslinkers.

%Figure~\ref{fig:patchpatchpot} compares the patch-patch interaction potential, the Lennard-Jones potential, and the FENE potential \cor{with in order to explore a qualitative comparisson.}\com{Poner eso en la caption}
%The Lennard-Jones and patch-patch potentials are qualitatively similar in that they both tend to zero after a certain cutoff distance.
%In contrast, the FENE potential goes to infinity if the distance is too short or too wide.

%The patch-patch interaction differs from the Lennard-Jones potential in that the minimum potential is horizontally translated, and the rate of change from the minimum to the cut distance is more apparent.
%Equation

\cor{However, if the simulation of the polymeric network happens just with the WCA and patch-patch potentials, the patches will take part in interactions involving several patches, which is undesired.
Additionally, because of the low temperature of the system, a three-body repulsive potential is introduced to promote network percolation.
}
This potential is defined in terms of~\eqref{eqn:patch-patch_interaction}, which facilitates an efficient bond-swapping mechanism, enabling the system to equilibrate readily even at low temperatures, while simultaneously maintaining the single bond per patch condition~\citep{sciortinoThreebodyPotentialSimulating2017}.
The algebraic expression of the interaction is given by:
\begin{gather}
    U_{\mathrm{swap}}(r_{l,m},r_{l,n}) = w\sum_{l,m,n}\epsilon_{m,n}U_3\qty(r_{l,m})U_3\qty(r_{l,n}),\quad r_{l,n}\in\qty[0,r_c],\label{eqn:swap_interaction}
\end{gather}
where
\begin{gather}
    U_{3}\qty(\cor{\alpha}) = \left\{
        \begin{array}{ll}
            1 & \cor{\alpha}\in\qty[0,\cor{\alpha}_{\min}], \\
            -U_{\mathrm{patchy}}\qty(\cor{\alpha})/\epsilon_{m,n}, & \cor{\alpha}\in\qty[\cor{\alpha}_{\min},\cor{\alpha}_c]
        \end{array}
        \right.\label{eqn:swapmod_interaction}.
\end{gather}
\cor{Where $\alpha$ is the distance $r_{l,m}$ or $r_{l,n}$.}
The sum in~\eqref{eqn:swap_interaction} runs over all triples of bonded patches (patch $l$ bonded both with $m$ and $n$).
$r_{l,m}$ and $r_{l,n}$ are the distances between the reference patch and the other two patches.
The parameter $\epsilon_{m,n}$ is the energy of repulsion, and $w$ is used to tune the swapping ($w=1$) and non-swapping bonds ($w\gg1$). 
The cutoff distance $\alpha_c$ is the same as in the potential of interaction between patches, meanwhile the minimum distance $\alpha_{\min}$ is the distance at the minimum of~\eqref{eqn:patch-patch_interaction}, \textit{i.e.} $\epsilon_{m,n}\equiv\abs{U_{\mathrm{patchy}}(\alpha_{\min})}$.

Figure~\ref{fig:swappot} shows the patch-patch potential \cor{and} the swap potential.
The patch-patch potentia, the blue line in the figure, is the energy of the interaction between patch $i$ and patch $k$.
The swap potential is the energy between patch $i$ and patch $k$, leaving the distance between patch $i$ and patch $j$ fixed.
Taking into account this, when the patches $i$ and $j$ are at the potential well ($r_{ij}=\cor{D}$), the interaction between $i$ and $k$ is null.
When the distance between patches $i$ and $j$ are bigger than the potential well but smaller than the cutoff distance ($r_{\mathrm{cut}}>r_{ij}>\cor{D}$), the interaction between $i-k$ is mildly attractive.
Finally, when the patches $i-j$ are bigger than the cutoff distance ($r_{\mathrm{cut}}<r_{ij}$), the interaction between $i-k$ is repulsive.

\begin{figure}[ht!]
    \centering
    \includegraphics[width=12cm]{figs/numerical/swapPotential.png}
    \caption{Swap potential for patch-patch interaction to ensure single bond per patch condition.
        Blue line represents the patch-patch interaction potential.
        Orange, green and red lines represent the swap potential with a fix distance between patch $i$ with patch $j$ and a free patch $k$.
        When the patches $i$ and $j$ are at the potential well, the interaction between partches $i-k$ and $j-k$ is repulsive (orange line).
        However, when the distance between patches $i$ and $j$ starts to increase, the repulsive interaction with patch $k$ diminishes.
    }\label{fig:swappot}
\end{figure}

\cor{A visual representation of the specified interactions is illustrated in figure~\ref{fig:interactionPatches}.
The irregular particles featuring an orange central particle represent monomers due to the presence of two patches.
The irregular particle featuring a red central component indicates the crosslinker due to its four patches, indicating a valence of 4.
Panels (a) and (b) illustrate the permitted interactions between patches throughout the simulation.
Panels (c) and (d) illustrate the impact of the exchange mechanism.
Rather than establishing a bond among the three patches, the interaction can change to build an interaction with the incoming particle.
}

\begin{figure}[ht!]
    \centering
    \includegraphics[width=8cm]{figs/numerical/interactionPatches.png}
    \caption{Interaction between patchy particles.
    Panels (a) and (b) represent the monomer-monomer and monomer-crosslink interactions, respectively.
    Meanwhile, panels (c) and (d) represent the swap potential interaction when the patches are not at the potential well distance, allowing the swap of bonds.
    }\label{fig:interactionPatches}
\end{figure}

\subsection{\cor{Reduce units}}

\cor{Before detailing the numerical values for the parameters, it is crucial to recognize that the numerical simulations are performed using reduced Lennard-Jones units for computational and physical efficiency.
The Lennard-Jones reduced units are achieved by rescaling the distance, energy, and mass to a characteristic quantity and setting $k_B=1$.
The distance is rescaled by the diameter of the particle, $r^*=r/D$; the energy is rescaled by the potential well of the interaction potentials $\epsilon^{*} = E/\epsilon$; meanwhile, the mass is rescaled to a characteristic quantity of the system $m^* = m/m_{\mathrm{ref}}$.
Finally, by setting $k_B$ to $1$ with the other quantities, the time is rescaled by $t^* = t\sqrt{\epsilon/(m_{\mathrm{ref}}O^2)}$.}

\cor{Although energy is not a fundamental unit, $k_\mathrm{B}$ is selected for rescaling as it is essential for establishing the reduced unit for temperature.
The reduced temperature is rescaled by $T^{*} = k_{\mathrm{B}}T/\epsilon$, or}
\begin{equation}
    \cor{\epsilon T^* = k_{\mathrm{B}}T.}
\end{equation}
\cor{Recognizing that $k_{\mathrm{B}}$ serves as a conversion factor between temperature and energy, this decision facilitates a straightforward comparison of thermal energy with interaction energy.
By establishing $T^*=1$, the thermal energy is equated to the potential well depth $\epsilon$.
Conversely, establishing $T^*=0.5$ represents a system having half the thermal energy relative to the potential well depth.}

\cor{From this moment onward, all quantities are presented in reduced Lennard-Jones units.
For clarity, quantities will be represented with an asterisk in their symbols, for example, $r \to r^{*}$ or $T \to T^{*}$, and so on.}

\subsection{\cor{Parameters}}

%\paragraph{Polymeric network parameters}
Each system had a fixed number of patchy particles, $N_p$, packing fraction, $\phi$, and cross-link concentration $c$.
Based on these quantities, the box's volume and the total number of patchy particles with functionality 2 (PB) and functionality 4 (PA) were calculated.
\cor{For the purpose of this model investigation, the total number of patchy particles is set at $N_p=\num{8000}$.
This quantity was chosen to reduce artifical connections arising from particle interactions with their periodic pictures; also, an increased number of particles enhances averaging and reduces statistical errors.
It is important to mention that the study initiated with a system of \num{10000} patchy particles, which was subsequently lowered for time efficiency; a configuration of \num{8000} patchy particles exhibited comparable observables to the system of \num{10000}.
}

%\paragraph{Number of patchy particles}
\cor{For both patchy particles, the values of $\cor{K_{br}^{*}}$ and $\cor{K_{b\theta}^{*}}$ are established at \num{100}~\cor{$\epsilon^*$} to create a robust bond and inhibit the elongation of monomers and crosslinkers.
\cor{Meanwhile the radius of the central particle is set to \num{0.5}~$r^{*}$ and the radius of the patches is \num{0.2}~$r^*$.}
The distance between the center of the central particle and the patches is set to \num{0.45}$r^*$.
Finally, the value of $\theta_{bo}$ is established at \SI{180}{\degree} for PB particles and \SI{109.4712}{\degree} for PA particles.}
\cor{The inter-center distance was adjusted to ensure that the centers do not coincide with the surface of the central particle.
Meanwhile, the amount of patches and angles corresponds to that described in~\citep{gnanSilicoSynthesisMicrogel2017}.}

%\paragraph{Packing fraction}
%\paragraph{Crosslinker concentrations}
\cor{The packing fraction is set at \num{0.5} to enable the formation of a percolated network; however, it is important to note that polymeric networks are also possible at lower packing percentages~\citep{gallegosLocationGellikeBoundary2021,gimperleinStructuralDynamicalProperties2021,formanekGelFormationReversibly2021b}.
 The concentrations of crosslinkers for testing were established according to their ratio in the hydrogels.
}
The volume of the box was calculated by determining the volume of the patchy particles A and B and then scaling those values by the number of particles and the desired packing fraction.
\begin{align*}
    \cor{V_{\mathrm{box}}^{*}} &= \frac{N_{\mathrm{patchyA}}\cor{V_{\mathrm{patchyA}}^{*}}+N_{\mathrm{patchyB}}\cor{V_{\mathrm{patchyB}}^{*}}}{\phi}
\end{align*}
The number of patchy particles of type A is computed as $N_{\mathrm{patchyA}} = c N_p$, and the number of patchy particles of type B as $N_{\mathrm{patchyB}}= N_p - N_{\mathrm{patchyA}}= N_p(1 - c )$.


\cor{The temperature is established at $\cor{T^*}=\num{0.05}$.
 The thermal energy represents only 5\% of the pairwise interactions between patchy particles.
 Consequently, once the network is formed, it is exceedingly unlikely to reorganize spontaneously due to insufficient energy to disrupt the patch-patch interactions.
 This scenario further reinforces the significance of incorporating the three-body potential for assisting a percolated network.}

%\paragraph{Langeving parameters}

\cor{The $\mathrm{damp}$ parameter was set to a value of \num{0.1}~$[1/t^*]$ as a remnant from a debugging stage in the shear deformation simulation.
Initially, the parameter was established at \num{1} or marginally higher values.
Nevertheless, those values resulted in significant temperature variations, and the adjusted temperature values exceeded the acceptable threshold.
To determine the cause of the problem, the damping parameter was significantly decreased to \num{0.1}, revealing that the three-body potential was incorrectly implemented.
This will be elaborated upon in  section~\ref{subsec:LammpsImplementation}.
After rectifying the implementation, the value was kept low to achieve the appropriate temperature. 
}

\cor{This implies that the system is in a high viscosity solvent and that the system is overcontrolled and that bond-breaking events and structural rearrangements may be artificially rare.
}
\cor{Consequently, to ensure numerical stability, the time step is set at \num{0.001}$t^{*}$. 
The stability criterion is defined as $\mathrm{damp}^{*}dt^{*}<2$~\citep{leimkuhlerContractionConvergenceRates2024}.
Considering that $\mathrm{damp}^{*}dt^{*} = \num{1d-4}$, numerical stability is assured, allowing us to acquire acceptable average observables.}



%\cor{In order to achieve a fast temperature relaxation the $\mathrm{damp}$ parameter is set to \num{0.1}~$[1/t^*]$.
%The reason of this value is due to a long process of debugging the protocol.
%Higher values $\mathrm{1}>1$ 

%By setting the mass at $m^*=1$, the dissipative term can be interpreted as removing 10\% of the momenta each reduce time unit. 
%}

%Finally, 
%in Lennard-Jones units; meanwhile, the damp parameter was set to \cor{$\mathrm{damp}=\num{0.1}$}.
%It is very important to note that the damp controls the viscous response caused by the interaction between the thermal bath and the particles, which symbolizes the interaction of water molecules with the polymer network. 
%The diameter of the central particle is set to $\cor{D^*}=1$ and the diameter of the patches at $\cor{D_p^*}=0.4$.



%However, this represents a reduced quantity of particles in comparison to other simulations~\citep{gnanSilicoSynthesisMicrogel2017}.
%\corConsequently, to offset this, we compute the average of five simulations of the shear deformation.}
%\cor{The time step is established at  to guarantee numerical stability. 

The energy of interaction between central particles is $\cor{\epsilon_{i,j}^*}=1$. 
In contrast, the energy of interaction between patches is defined as follows: 
The interaction of patchy particles B is set to $\cor{\epsilon_{\mu,\upsilon}^*}=1$, whereas the interaction between patchy particles A is set to $\cor{\epsilon_{\mu,\upsilon}^*}=0$, and the interaction between patches of patchy particles A with patchy particles B is set to $\cor{\epsilon_{\mu,\upsilon}^*}=1$. 
This is to allow only crosslinker-monomer and monomer-monomer bonding. 
\cor{Finally, the parameter that controls the swapping mechanism is set to \num{0.75} to allow swapping bonds.}

\cor{The selected deformation for analyzing the mechanical response was shear deformation, as shear testing yields a more uniform stress distribution across the hydrogel sample than tensile testing~\citep{jungEvaluatingMechanicalResponse2022,bernhardt-barryAnalysisStressDistribution2021}.
Moreover, shear rheometry is proficient in delineating the intricate viscoelastic characteristics that determine hydrogel functionality~\citep{insuaRheologicalInsight3D2025}.
Numerous hydrogels demonstrate shear-thinning properties, which are essential for applications such as injection and 3D bioprinting~\citep{tangStimuliresponsivePentapeptideNanofiber2019,fragalMicrobialBiosurfactantHydrogels2025}.
} 

The shear deformation was conducted at a constant shear rate in the $xy$ plane.
The shear rates were in the \num{d-3}\cor{[$1/t^*$]} order of magnitude, and the final strain was \num{15}.
This aims to characterize the mechanical response and to deform beyond the plastic deformation threshold.
The shear rate was varied to examine the viscoelastic response of the material.

%shear forces dominate biological environments where hydrogels are typically deployed.
%\cor{In this work the parameter is set to \num{0.75}.}

%\paragraph{Deformation protocol}
%In rheological measurements using parallel plate or cone-and-plate geometries, the applied shear stress is distributed evenly across the sample, eliminating edge effects and stress concentrations that plague tensile testing.

\subsection{LAMMPS implementation}\label{subsec:LammpsImplementation}

As said before, the LAMMPS software is used to solve the Langevin equation in a many-particle system.
However, it is useful to explain how the simulation is defined in this software.
In this regard, the following paragraphs briefly discuss the damp parameter, the implementation of the swap potential~\eqref{eqn:swapmod_interaction}, the shear deformation, and the calculation of the stress tensor.

%\paragraph{damp}
Comparing equation~\eqref{eqn:MolDylammps1} with equation~\eqref{eqn:MolDylammps2}, the viscosity parameter \cor{$\mu$} of the Langevin equation~\eqref{eqn:BrownianDyn1} does not appear; instead, the $\mathrm{damp}$ parameter appears.
This parameter is specified in time units and determines how rapidly the temperature is relaxed so that it can be more easily used as a thermostat\citep{LAMMPS}.
That is, if damp is set to \num{100}, the temperature will relax in a timespan of roughly \num{100} time units.
By making dimensional analysis, the damp factor can be thought of as inversely related to the viscosity of the solvent.
This tells us that a small damp represents a high-viscosity solvent and vice versa.
%In addition it is important to mention that all the simulations are done with Lennard Jones units.
%Therefore, since the damp is set to \cor{\num{0.1}}, the polymeric network is in a high-viscosity solvent.

%\paragraph{Three-body potential}
In regard to the swap potential, the \verb|threebody/table| pair style command is used to implement generic tabulated three-body interactions.
However, in LAMMPS, the tabulation is done on a three-dimensional plane of the two distances $\cor{r_{ij}^*}$ and $\cor{r_{ik}^*}$ with the angle $\theta_{ijk}$, where the forces on all three particles $I$, $J$, and $K$ lie within the plane defined by the three inter-particle distance vectors $\vec{\cor{r_{IJ}^*}}$, $\vec{\cor{r_{IK}^*}}$, and $\vec{\cor{r_{JK}^*}}$\citep{LAMMPS}.
Allowing the following property to project the forces onto the inter-particle distance vectors,
\begin{equation}
    \begin{pmatrix}\vec{\cor{f_i^*}} \\ \vec{\cor{f_j^*}} \\ \vec{\cor{f_k^*}}\end{pmatrix}
    =
    \begin{pmatrix}\cor{f_{i1}^*} & \cor{f_{i2}^*} & 0 \\ \cor{f_{j1}^*} & 0 & \cor{f_{j2}^*} \\ 0 & \cor{f_{k1}^*} & \cor{f_{k2}^*} \end{pmatrix}
    \begin{pmatrix}\vec{\cor{r_{ij}^*}} \\ \vec{\cor{r_{ik}^*}} \\ \vec{\cor{r_{jk}^*}}\end{pmatrix}.
\end{equation}
And due to symmetry interactions, $\cor{f_{i1}^*}=-\cor{f_{j1}^*}$, $\cor{f_{i2}^*}=-\cor{f_{k1}^*}$, and $\cor{f_{j2}^*}=-\cor{f_{k2}^*}$.

Therefore, to have a correct tabulation, it is necessary to project the force into the inter-particle plane.
Recalling that the force is equivalent to $-\nabla \cor{U^*}(\cor{r^*})$ and the potential has only radial dependence, the force can be expressed as
\begin{gather}
    \vec{\cor{f_n^*}} = -\pdv{\cor{U_{\mathrm{swap}}^{*}}(\cor{r_m^*},\cor{r_l^*})}{\cor{r^*}}\hat{e}_{\cor{r^*}},
\end{gather}
where $n$ represent the particle $i$, $j$ or $k$, while $m$ and $l$ are place holders for distances $ij$, $ik$ and $jk$.
Hence, 
\begin{align}
    \vec{\cor{f_{i}^*}} &= -\pdv{\cor{U_{\mathrm{swap}}^{*}}(\cor{r_{ij}^*},\cor{r_{ik}^*})}{\cor{r^*}}\hat{e}_{\cor{r^*}}\label{eqn:3body1}, \\
    \vec{\cor{f_{j}^*}} &= -\pdv{\cor{U_{\mathrm{swap}}^{*}}(\cor{r_{ji}^*},\cor{r_{jk}^*})}{\cor{r^*}}\hat{e}_{\cor{r^*}}\label{eqn:3body2}, \\
    \vec{\cor{f_{k}^*}} &= -\pdv{\cor{U_{\mathrm{swap}}^{*}}(\cor{r_{ki}^*},\cor{r_{kj}^*})}{\cor{r^*}}\hat{e}_{\cor{r^*}}\label{eqn:3body3}.
\end{align}
The projection of the force into the plane is computed via the dot product between $\hat{e}_{\cor{r^*}}$ and the basis that represents the plane. 
Which can be defined by the following 2-dimensional basis, $\hat{e}_1=\qty[1,0] $ and $\hat{e}_2=\qty[\cos\theta,\sin\theta] $, and $\theta$ is the angle between distances $\vec{\cor{r_{ij}^*}}$ and $\vec{\cor{r_{ik}^*}}$ since the software defines the plane using the distances between the particles.
With this we can compute the following projections:
\begin{align}
    \hat{e}_{\cor{r^*}} \cdot \hat{e}_1 &= 1\\
    \hat{e}_{\cor{r^*}} \cdot \hat{e}_2 &= \cos\theta. 
\end{align}
Also, the following vector can be defined as $\hat{e}_3 = \hat{e}_1 - \hat{e}_2 = \qty[1-\cos\theta,-\sin\theta]$ to represent the $j-k$ distance, and the projection will be
\begin{align}
    \hat{e}_{\cor{r^*}} \cdot \hat{e}_3 &= 1-\cos\theta.
\end{align}

With these projections the forces can be expressed in~\eqref{eqn:3body1},~\eqref{eqn:3body2}, and~\eqref{eqn:3body3} as follows:
\begin{align}
    \cor{f_{i1}^*} &= -\pdv{\cor{U_{\mathrm{swap}}^{*}}(\cor{r_{ij}^*},\cor{r_{ik}^*})}{\cor{r^*}}\label{eqn:3body4a}, \\
    \cor{f_{i2}^*} &= -\pdv{\cor{U_{\mathrm{swap}}^{*}}(\cor{r_{ij}^*},\cor{r_{ik}^*})}{\cor{r^*}}\cos\theta\label{eqn:3body4b}, \\
    \cor{f_{j2}^*} &= -\pdv{\cor{U_{\mathrm{swap}}^{*}}(\cor{r_{ji}^*},\cor{r_{jk}^*})}{\cor{r^*}}\qty(1-\cos\theta)\label{eqn:3body5}.
\end{align}
And due to the symmetry relations, the potential can be tabulated into the LAMMPS software to introduce the one-bond-per-patch condition and mitigate numerical instability during shear deformation.

The \verb|fix deform| command simulates a shear deformation on the $xy$ plane.
It uses the engineering deformation rate (\verb|erate| style) to adjust the box's dimension at a ``constant engineering strain rate''.
The length of the box, $\cor{L^{*}}$, will be modified over time,
\begin{gather}
    \cor{L^{*}}(t) = \cor{L_o^{*}}\qty(1 + \cor{\dot{\gamma}^{*}}~\cor{dt^{*}}),
\end{gather}
where $\cor{L_o^{*}}$ is the original box length and $\cor{\dot{\gamma}^{*}}$ is the shear rate in units of $1/\cor{t^*}$\citep{LAMMPS}.
This deformation is applied in the $xy$ plane, and the change in length is along the $x$ direction.
This set of parameters ensures that the volume does not change during the deformation\citep{LAMMPS}.
In addition to the \verb|erate| style, the \verb|remap| keyword was set to \verb|x| to remap particle positions without affecting their velocities\citep{LAMMPS}.
Setting remap to x causes the atoms to deform using an affine transformation that is identical to the box deformation.  
It is important to note that, while the atoms are effectively ``moving'' with the box over time, this is due to the remapping rather than their velocity, which tracks the box's change.
Finally, the \verb|flip| keyword is used to flip the box when the tilt factors exceed half the distance of the parallel box length to avoid computational inefficiency and errors\citep{LAMMPS}.

%\paragraph{fix stress} 
This section concludes with an explanation of how LAMMPS computes the stress tensor.
The \verb|compute stress/atom| calculates the stress tensor for an atom $I$ using the equation \citep{thompsonGeneralFormulationPressure2009,LAMMPS},
\begin{gather}
    \cor{\sigma_{ab}^{*}} = -\cor{v_a^{*}} \cor{v_b^{*}} - \cor{W_{ab}^{*}}\label{eqn:stressLAMMPS},
\end{gather}
where $a$ and $b$ represent the spatial coordinates $x$, $y$ and $z$, and $\cor{W_{ab}^*}$ is the virial contribution given by
\begin{multline}
    \cor{W_{ab}^{*}} = \frac{1}{2}\sum_{n=1}^{N_p}\qty(\cor{r_{1a}^*} \cor{F_{1b}^{*}} + \cor{r_{2a}^*} \cor{F_{2b}^{*}})
            +\frac{1}{2}\sum_{n=1}^{N_b}\qty(\cor{r_{1a}^*} \cor{F_{1b}^{*}} + \cor{r_{2a}^*} \cor{F_{2b}^{*}}) \\
            +\frac{1}{3}\sum_{n=1}^{N_a}\qty(\cor{r_{1a}^*} \cor{F_{1b}^{*}} + \cor{r_{2a}^*} \cor{F_{2b}^{*}} + \cor{r_{3a}^*} \cor{F_{3b}^{*}})\label{eqn:virialLAMMPS}.
\end{multline}
\cor{The first term accounts for pairwise interactions, the second for bond contributions, and the third for the angle-based interaction.}
%Since we only declare pairwise interactions and bond and angle interactions, the virial contribution simplifies to

\begin{comment}
\begin{multline}
    \cor{W_{ab}^{*}} = \frac{1}{2}\sum_{n=1}^{N_p}\qty(\cor{r_{1a}^*} \cor{F_{1b}^{*}} + \cor{r_{2a}^*} \cor{F_{2b}^{*}})
    +\frac{1}{2}\sum_{n=1}^{N_b}\qty(\cor{r_{1a}^*} \cor{F_{1b}^{*}} + \cor{r_{2a}^*} \cor{F_{2b}^{*}}) \\
            +\frac{1}{3}\sum_{n=1}^{N_a}\qty(\cor{r_{1a}^*} \cor{F_{1b}^{*}} + \cor{r_{2a}^*} \cor{F_{2b}^{*}} + \cor{r_{3a}^*} \cor{F_{3b}^{*}})
            +\frac{1}{4}\sum_{n=1}^{N_d}\qty(\cor{r_{1a}^*} \cor{F_{1b}^{*}} + \cor{r_{2a}^*} \cor{F_{2b}^{*}} + \cor{r_{3a}^*} \cor{F_{3b}^{*}} + \cor{r_{4a}^*} \cor{F_{4b}^{*}}) \\
            +\frac{1}{4}\sum_{n=1}^{N_i}\qty(\cor{r_{1a}^*} \cor{F_{1b}^{*}} + \cor{r_{2a}^*} \cor{F_{2b}^{*}} + \cor{r_{3a}^*} \cor{F_{3b}^{*}} + \cor{r_{4a}^*} \cor{F_{4b}^{*}}) 
            +\sum_{n=1}^{N_f}\cor{r_{ia}^*} \cor{F_{ib}^{*}}.
\end{multline}
\end{comment}


Incorporating the virial term~\eqref{eqn:virialLAMMPS} into the per-particle stress tensor~\eqref{eqn:stressLAMMPS} produces an analogous expression to the previously established stress~\eqref{eqn:DerVirTen23}.
The stress tensor in equation~\eqref{eqn:DerVirTen23} represents a temporal and spatial average, while the stress tensor in~\eqref{eqn:stressLAMMPS} is defined on a per-particle basis.
\cor{The temporal average is a rolling mean, whereas the spatial average is an arithmetic mean throughout the spatial domain.}
Consequently, to analyze the mechanical response of the Cauchy stress tensor, an appropriate temporal average, succeeded by a spatial average of the per-particle stress tensor across the patchy particle network, inclusive of the patches, is necessary.

\subsection{Assembly simulation and Shear deformation}

The assembly method begins by placing the $N_p$ particles at random locations within a box whose dimensions correspond to the chosen packing fraction.
The temperature is incrementally raised from \num{0} to \num{0.05} over \num{500000} time steps to avoid numerical instability.
Thereafter, the temperature was held constant until the total energy of the system stabilized to a minimum.
\cor{After achieving a stable energy value, a percolation test was conducted on the network.
A cluster with a cutoff distance of \num{0.6}~$[r^*]$ was created using OVITO.
Subsequently, the system replicated in the directions $\pm\hat{e}_{x},~\pm\hat{e}_{y},~\pm\hat{e}_{z}$ and verified whether the largest cluster expanded in size by a factor of \num{27}.
Through iteration, we determined that a suitable number of time steps to reliably meet these constraints was \num{8d6} time steps.
The simulation at the LAVIS facilities required a minimum of 8 hours for execution with the specified amount of time steps.
}


Following finishing the network assembly, the final configuration was stored in a file before the application of shear deformation.
 The irregular particle network was subsequently exposed to a sequence of five deformations with corresponding shear rates.
 For each shear deformation simulation, the system's energy, stress tensor components, temperature, and phase space configurations were captured.
 After five sets of deformations at a consistent shear rate, a new series of deformations was conducted at a different shear rate, starting from the identical initial configuration as before.

The observed measurements are rolling mean temporal averages.
The temporal average for the assembly protocol is defined by the damping parameter: $100\cdot\cor{\mathrm{dt}^*}\cdot\cor{\mathrm{damp}^*}$, while for shear deformation, the duration was established to ensure that the strain reaches $0.05\cor{\gamma^{*}}$ within that time frame.
This was established for the energy, temperature, and stress tensors.
\cor{The intervals were derived by adjusting the time window to minimize noise while adequately capturing the physical response. 
A brief time interval yields noisy measurements, while an extended time interval dampens the fundamental dynamics.}
The intervals for preserving the system's phase space were determined based on the simulation duration and strain during assembly and deformation.
After saving the metrics from all simulations, an average of the experiments with equal parameters was calculated.

\section{Results}

Starting with the patchy particle  network formed during the assembly protocol.
Figure~\ref{fig:assemblyCluster} illustrates the cluster analysis of a specific network. 
The color of the particles signifies that they are within the same cluster. 
The patchy particles initiate at random positions (panel a); then after \num{8.5d6} time steps, almost every patch is coupled via the patch-patch interaction potential (panel b).
\cor{The patchy particles that are not within a cluster where leave it alone.
Only a single assembly protocol was carried out for each crosslink concentration.
One for the 3\%, another for the 5\%, and the final one for the 10\%.
}

%The number of time steps for this stage has been established by monitoring the energy.
%The assembly protocol is stopped when more than 90\% of patches are in the same cluster.
%Only one assembly protocol was used for each crosslinker concentration\com{What happen with the free patchy particles?}.

\begin{figure}[ht!]
    \centering
    \includegraphics[width=\textwidth]{figs/ComputaitonalResults/New/assemblyCluster.png}
    \caption{The network generated during the simulation's assembly stage is visualized as clusters.
        The color in each patchy particle represents the cluster's id.
        If the patches are the same color, they share the cluster's id.
        Panel (a) depicts the initial arrangement.
        Panel (b) depicts the assembly simulation's finished configuration.
        Ovito is used for color processing, with a \num{0.6} cut-off between each patch.
    }\label{fig:assemblyCluster}
\end{figure}

\newpage

After the assembly procedure, shear deformation in the $xy$ plane was conducted.
Panel (a) of Figure~\ref{fig:flipPatches} shows the initial configuration of a shear deformation.
Panel (b) of the same image pictures the deformation at a strain of $0.5$.
For computational stability, LAMMPS flips the deformed box to a box with a strain of $0.5$, as illustrated in panel c.
Finally, this process is repeated until the deformation reaches a strain of $15~L^*$.

\begin{figure}[ht!]
    \centering
    \includegraphics[width=\textwidth]{figs/ComputaitonalResults/New/flipPatches.png}
    \caption{A graphical representation of the shear deformation applied to the patchy particle network in the $xy$ plane.
            Panel (a) displays the initial arrangement.
            Then, in panel (b), the box achieves the maximum tilt to assure numerical stability \cor{$(\gamma^{*}=0.5)$}.
            Panel (c) then displays the flip manipulation, which continues the deformation \cor{$(\gamma^{*}=0.5)$}.
            Finally, panel (d) shows that the box enters a configuration with no tilt but a system with strain \cor{$(\gamma^{*}=1)$}.
            Orange spheres are the central particle of monomers. 
            Red spheres represent the central particle of crosslinkers. 
            Blue spheres represent patches.
    }\label{fig:flipPatches}
\end{figure}

\newpage



Furthermore, Figure~\ref{fig:shearBonds} shows the system's shear deformation after (panels (a) and (b)) and before (panels (c) and (d)).  
Orange spheres are the core particles of monomers.  
Red spheres indicate the crosslinker's core particles.  
Blue spheres symbolize patches. 
Panels (b) and (d), on the other hand, display the bonds between patches, providing a useful view of the network's topology.
Although difficult to see, panel d shows that the bonds align parallel to the $x$ coordinate after shear.
After briefly discussing the configurational results of the patchy particle network, let's move on to the strain-stress relationships of the deformations.

\begin{figure}[ht!]
    \centering
    \includegraphics[width=\textwidth]{figs/ComputaitonalResults/New/shearBonds.png}
    \caption{Panels (a) and (c) show the system with patchy particles.
        Panels (b) and (d), on the other hand, illustrate the bonds between patches as a simplified representation of the network's structure.
        In addition, panels (a) and (b) are before the shear and (c) and (d) after the shear.
        After and before shear \cor{with a shear rate of $\dot{\gamma}^{*}=0.005$}.
    }\label{fig:shearBonds}
\end{figure}

%\begin{figure}[ht!]
%    \centering
%    \includegraphics[width=\textwidth]{figs/ComputaitonalResults/New/flipBonds.png}
%    \caption{Flip Bonds}\label{fig:flipBonds}
%\end{figure}

%\subsection{Mechanical response
%\cor{\Large\citep{argunInterplaySpatialTopological2024}}

% divoux => carbopol

\newpage

\subsection{\cor{Mechanical response}}

Figure~\ref{fig:stres-strainResults} shows the strain-stress curve of the patchy particle network, highlighting the influence of varying shear rates and crosslinker concentrations.
From a qualitative perspective, the network's stress levels are escalating significantly.
Moreover, the strain-stress relation exhibit two distinct responses.
In networks with 3\% and 5\% crosslinkers at lower shear rates, it is apparent that at a specific strain, the stress will stop increasing and ultimately reach a constant value.
At elevated concentrations, the stress escalates until it reaches a highest level.
Once reaching the maximum stress value, the stress then decreases until it achieves a lower stress value.
This behavior has also been documented in the mechanical response of \cor{a polymeric gel obtained by using a custom variation of the randim Gaussian network generation algorithm\citep{argunInterplaySpatialTopological2024}} and in other polymeric systems\citep{osakiStressOvershootPolymer2000a,ravindranathUniversalScalingCharacteristics2008,boukanyUniversalScalingBehavior2009}.

\cor{The sudden increase in the stress value is commonly referred to as \emph{stress overshoot}, and} the subject has been thoroughly researched, as outlined in relevant literature, including molecular dynamics simulations \citep{jeongEffectChainOrientation2017,caoSimulatingStartupShear2015,mohagheghiMolecularlyBasedCriteria2016,baigFlowEffectsMelt2010a} and an analysis in other articles by \citep{wangExploringStressOvershoot2009}.
According to references \citep{jeongEffectChainOrientation2017,janeschitz-krieglPolymerMeltRheology1983,pearsonFlowInducedBirefringenceConcentrated1989,masubuchiPrimitiveChainNetwork2020}, stress overshoot is one of the most significant nonlinear rheological phenomenon exhibited by polymeric liquids undergoing start-up shear above a certain flow strength.
Furthermore, the articles clarify that the primary driver of this phenomenon is the network's chain orientation.
This was proved by analyzing the system's birefringence, as well as the transient behavior of the order tensor of entanglement strands along the chains, as described in the tube theory for entangled polymers~\citep{jeongEffectChainOrientation2017}.
The reason for this is that the refractive index is determined by the orientation distribution of these bond vectors.
In other words, the better the alignment of the bond vectors, the greater the difference in refractive index across different directions. 
The anisotropy of the bond vectors produces an asymmetric polarizability tensor, which is seen macroscopically as birefringence.

\cor{While the study~\cite{jeongEffectChainOrientation2017} provided significant insights into the overshoot phenomenon, an analysis of the bond vectors was not conducted, as our primary objective was to determine whether \emph{fully reversible} polymeric gels can exhibit yield stress phenomena.
Nonetheless, given that the parameters of temperature (with thermal energy constituting only 5\% of the pairwise interaction) and damp (a minimal value signifying a highly viscous solvent) complicate the bond-breaking process, it appears that bond alignment serves as a more appropriate explanation than bond breaking.}

\begin{figure}[ht!]
    \centering
    \includegraphics[width=\textwidth]{figs/ComputaitonalResults/comp.pdf}
    \caption{Strain-stress relation for a set of 10 different shear rates applied to three different patchy particles systems with crosslinker concentration of 3, 5 and 10 percent.
    Each line represent a 5 experiments average with an assembly average and a moving time average.\com{Explain what happens in the overshoot.}}\label{fig:stres-strainResults}
    %Computational results at different crosslinker concentrations. Each line reprensent the average of 5 experiments with \num{8000} patchy particles with a packing fraction of \num{0.5}, damp of \num{0.5} and the time average is in intervals of \num{5d-2}$\dot{\gamma}$.
\end{figure}

\newpage

Subsequently, figure~\ref{fig:stress-strainCLResults} shows the strain-stress relationship of three systems with different crosslinker concentrations at \num{1}, \num{5}, and \num{10} milli shear rates.
The objective of this comparative analysis is to highlight the previously mentioned qualitative differences between them.
Keeping the same shear rate reveals that an increase in crosslinker concentration enhances both the overshoot response and the steady-state stress value.
\cor{The increased overshoot signifies a more elastic reaction, accumulating greater energy prior to yielding.
Beyond this point, the elevation in the stress value indicates an increased viscous restriction to flow.
This demonstrates how crosslinker concentration enhances the gel's microstructure, enhancing its solid-like strength and the network's residual resistance during flow.
This is consistent across all reported shear rates.
}

\begin{figure}[ht!]
    \centering
    \includegraphics[width=\textwidth]{figs/ComputaitonalResults/compCl.pdf}
    \caption{Computational results at different crosslinker concentrations with different shear rates. 
    Each line reprensent the average of 5 experiments with \num{8000} patchy particles with a packing fraction of \num{0.5}, damp of \cor{\num{0.1}} and the time average is in intervals of \cor{$0.05/(\dot{\gamma}^{*}dt^{*})$.}}\label{fig:stress-strainCLResults}
\end{figure}

\cor{By maintaining a constant crosslinker concentration while varying the shear rate, it is evident that an increased shear rate enhances both the overshoot and the stable stress value (same color in each panel).
Maintaining a fixed crosslinker concentration while changing the shear rate demonstrates that a higher shear rate amplifies both the overshoot and the stable stress value (shown by the same color in each panel).
 From this point of view, the increase of the steady stress value following the overshoot can be understood as the network's reorganization and accelerated flow.
This interpretation allow us to reconsider the breaking of pairwise interactions as a relevant phenomenon in the shear process.
}

\cor{With a temperature value of $T^{*}=\num{0.05}$ and a damping factor of $\num{0.1}$, the mechanical response is expected to be brittle and elastic rather than exhibiting viscous flow.
The extremely low temperature inhibits spontaneous bond rupture; if a rupture occur, the robust thermostat will removed any heat generated by the shear deformation via thermal fluctuations.}
\cor{Nevertheless, according to the strain-stress relations and prior discussions, the system exhibits a viscoelastic response.
Consequently, this response can be understood by the elevated packing fraction.
When a patch-patch connection is disrupted caused by deformation, it can reorganize since adjacent patches are available.}

It is also worth noting that the peak strain value and peak height increase in proportion to the shear rate.
Panel C of figure~\ref{fig:stress-strainCLResults} shows the stress-relaxation response of viscoelastic materials.
Furthermore, we can see how the crosslinker enhances the viscoelastic response by increasing the maximum stress prior to stress relaxation.
However, it should be noted that in figure~\ref{fig:stress-strainCLResults}, the greatest stress levels occur before the strain value of $2$, which differs from the earlier reports \citep{jeongEffectChainOrientation2017,janeschitz-krieglPolymerMeltRheology1983,pearsonFlowInducedBirefringenceConcentrated1989,masubuchiPrimitiveChainNetwork2020}, which reported that the highest stress occurs at strain values of about $2-3$\com{Posible causa de la discrepancia de sus resultados contra la literatura.}.
Furthermore, the strain shift at the maximum stress is indistinguishable.

Another intriguing finding is the change in the steady-stress regime after the overshoot.
Figure~\ref{fig:yieldStressResults} illustrates the average stress throughout the strain from \num{10} to \num{15} ($\gamma\in[10,15]$) with respect to the shear rate at various crosslinker concentrations.
The dashed lines represent a powerlaw fits $p_1\dot{\gamma}^{p_2}$, where $p_1$ and $p_2$ stands for parameter one and parameter two.
With exponents ($p_2$) $\{0.5901,0.4880,0.4045\}$ for the \num{10}\%, \num{5}\%, and \num{3}\% crosslinker concentrations, respectively.
This finding served as a starting point for further investigation into yield stress events in polymeric networks.
\begin{figure}[ht!]
    \centering
    \includegraphics[width=0.9\textwidth]{figs/ComputaitonalResults/yieldStress.pdf}
    \caption{Time average of steady-stress regime from computational results. Dashed lines represent exponential fits.
    \cor{Quitar los puntos. La linea de ajuste no se usa, quitar. justificar }
    }\label{fig:yieldStressResults}
\end{figure}

The Herschel-Bulkey\com{Is it?/Una mejor discusión de los resultados.} model can quantitatively describe a ''\textit{yield stress material},'' and the high congruence between the exponential fit and the numerical results indicates that the polymeric network exhibits similar phenomena.
However, the yield stress $\sigma_y$ was not defined using the Tresca or Von Mises criterion.
Instead, it was chosen to compute the temporal average of the final portion of the deformation, assuming that the system achieves plastic deformation after the overshoot.
%\subsection{Network analysis}


\begin{gather}
    f(\dot{\gamma}^{*}) = c_1 (\dot{\gamma}^{*})^{c_2}+c_3\label{eqn:fit} 
\end{gather}


\begin{table}[ht!]
\centering
\caption{Parameters of the fit model}\label{tab:fitParameters}
\begin{tabularx}{12cm}{|>{\centering\arraybackslash\bfseries}p{2cm}|>{\centering\arraybackslash}X|>{\centering\arraybackslash}X|>{\centering\arraybackslash}X|}
\toprule
\textbf{Cl~\%} & \textbf{$c_1$} & \textbf{$c_2$} & \textbf{$c_3$} \\
\midrule
\num{10}            & \num{129.185}  & \num{0.590128} & \num{4.358d-6} \\
\midrule
\num{5}      & \num{202.18}  & \num{0.488094} & \num{3.479d-6}\\
\midrule
\num{3}         & \num{340.918} &  \num{0.404582} & \num{8.517d-6}\\
\bottomrule
\end{tabularx}
\end{table}



\begin{comment}
In order to describe the network with a quantitative framework, on figure~\ref{fig:network2} we show a set of histograms of the metrics from the gyration tensor for the same system of figure~\ref{fig:network1} along the hole deformation at different shear rates.
This gyration tensor corresponds to the biggest cluster in the simulation.
In panel a it is show the radii of gyration $R_g$, which is the distance of the patchy particle from its center of mass;
in panel b is the shape anisotropy $k^2$, which is \num{0} when all particles are spherically symmetric and \num{1} when all points lie on a line;
in panel c is the asphericity $b$, which has value of zero when the distribution of particles si spherically symmetric or whenever the particle distribution is symmetric with respecto the three coordinate axes, for example any platonic solid;
finally, in panel d is the acylindricity, which is zero when the distribution of particles is cylindrically symmetric or whenever the particle distribution is symmetric with respect to the two coordinate exis, for example, when particles are distributed uniformly on a regular prism.

In general terms, we can see that the increment of crosslinker concentration, the distirbution shifts to the right in the four metrics.
For panels c and d, this means that the cluster starts to deviate from spherical and cylindrical symmetry, meanwhile, for the anisitrpopy parameter, it tells us, that the cluster starts to align into a thin line.
Another important result, is that when the concentration of crosslinker it is increase, the radii of gyration has a more probability to have numeric value around \num{0}, indicating that the morphology of the cluster is spherically symmetric.
\end{comment}

%\begin{figure}[ht!]
%    \centering
%    \includegraphics[width=0.9\textwidth]{figs/ComputaitonalResults/metricsGyrationTensor.pdf}
%    \caption{Gyration tensor metrics for three different shear rates at constant \num{10}\% crosslinker concentration.}\label{fig:network2}
%\end{figure}





