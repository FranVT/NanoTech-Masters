\documentclass[main.tex]{subfiles}

\begin{document}
	
\section{Structure and elasticity of model disordered, polydisperse, and defect-free polymer networks \citet{colomboStressLocalizationStiffening2014}}

They generate a polydisperse network via the method described in \citet{gnanSilicoSynthesisMicrogel2017}.
This method was originally develop for Molecular Dynamics simulations of microgels but it can be generalized to the case of bulk systems.
This method allows for an efficient generation of fully bonded networks by using a bottom-up self-assembly approach based on bond-swapping potential.

\section{In Silico Syntehsis of Microgel Particles\citet{gnanSilicoSynthesisMicrogel2017}}
The purpose of the parper is to build up a felxible numerical protocl able to design individual microgel particles \textit{in silico} with properties comparable to the experimental ones.
In this work they build the microgel particle and compare their swelling behavior with experimental results.
they focus in the case of very small microgels, or nanogels, whose diameter in the swollen regime us approximately \SI{50}{\nano\meter}, because it is possible to reproduce the network in a monomer-resolved way by using the classic bead-spring model for polymer.

Two body interaction potential between particles $i$ and $j$:
\begin{align}
	V\left(\vec{r}_{i,j},\{\vec{p}_{i}\},\{\vec{p}_j\}\right) &= V_{WCA}\qty(r_{ij})+\sum_{\mu\in\{p_i\}}\sum_{\upsilon\in\{p_j\}}V_{\mathrm{patchy}}\qty(r_{\mu\upsilon}), \\
	V_{WCA}(r) &=\left\{
	\begin{array}{ll}
		4\epsilon\left[\left(\frac{\sigma}{r}\right)^{12}-\left(\frac{\sigma}{r}\right)^{6}\right]+\epsilon, & r\leq2^{1/6}\sigma \\
		0, & r>2^{1/6}\sigma
	\end{array}
	\right.
	\\
	V_{\mathrm{patchy}}(r_{\mu\upsilon}) &= \left\{
	\begin{array}{ll}
		2 \epsilon_{\mu\upsilon}\left(\frac{\sigma_p^4}{2r_{\mu\upsilon}^4}\exp\left[\frac{\sigma_p}{r_{\mu\upsilon-r_c}}+2\right] \right), & r_{\mu\upsilon}\leq r_c \\
		0 & r_{\mu\upsilon}> r_c 
	\end{array}
	\right.
\end{align}

\begin{itemize}
	\item $\vec{r}_{i,j}$ is the vector connecting $i$ and $j$
	\item $r_{ij}$ is its length
	\item $\{p_k\}$ is the set of patches of particle $k$
	\item $r_{\mu\upsilon}$ is the distance between patch $\mu$ on particle $i$ and patch $\upsilon$ on particle $j$.
	\item $\sigma$ is the particle diameter
	\item $\epsilon$ is energy
	\item $\sigma_p$ is the sets of position of the attractive well
	\item $\epsilon_{\mu\upsilon}$ is the depth of the well
	\item $r_c$ is chosen by imposing $V_{\mathrm{patchy}}(r_c)=0$
\end{itemize}


Interaction between polymers
\begin{gather}
	V_{\mathrm{FENE}} = \left\{
	\begin{array}{cc}
		-\epsilon k_F R_0^2\ln\left[1-\left(\frac{r}{R_0\sigma}\right)^2\right], & r<R_0\sigma \\
		0, & r\leq R_0\sigma
	\end{array}
	\right.
\end{gather}

\begin{itemize}
	\item $k_F$ is the spring constant
	\item $R_0$ is the maximum extension of the bond
\end{itemize}

Introduce temperature in the swelling,
\begin{gather}
	V_\alpha = \left\{
	\begin{array}{ll}
		-\epsilon\alpha & r\leq2^{1/6}\sigma \\
		\frac{1}{2}\alpha\epsilon\left[\cos\left(\gamma\left(\frac{r}{\sigma}\right)^2 +\beta\right)-1\right] & r\in\left(2^{1/6}\sigma,R_0\sigma\right] \\
		0 & r>R_o\sigma
	\end{array}
	\right.
\end{gather}

\begin{itemize}
	\item $\alpha$ controls the solvophobicity of the monomers and plays the role of a temperature.
\end{itemize}


%\section{Three-body potential for simulating bond swaps in molecular dynamics\citet{sciortinoThreebodyPotentialSimulating2017}}

\bibliography{Gels.bib}
\end{document}