

%%%%%%%%% BODY TEXT

La detecci\'on y diagnostico de fallas en procesos autom\'aticos es un tema de gran importancia, debido a los altos requisitos de calidad en los productos  que se comercializan hoy en d\'ia, a la seguridad de operaci\'on de las plantas de los diferentes procesos industriales y a la previsi\'on de cat\'astofres a\'ereos, terrestres y espaciales.
Problemas como desgaste de actuadores, fallas en los sensores, mantenimientos inapropiados de los elementos de control y medición son factores que propician fallas frecuentes en los sistemas de control y monitoreo que cada vez se vuelven m\'as  complejos y grandes.La
falta o un mal dise\~no de sistemas de detecci\'on y diagn\'ostico de fallas en los sensores y actuadores han causado inumerables accidentes aer\'eos en diferentes lugares del mundo. Por ejemplo en el \'ambito industrial se puede recordar la c\'atastrofe sufrida en Chern\'obil donde un aumento r\'apido  de potencia en el reactor 4 produjo el sobrecalentamiento del n\'ucleo del reactor nuclear, provocando la explosi\'on del hidr\'ogeno acumulado en su interior. En el \'aera automotriz  diariamente se esta expuesto a fallas en sensores y actuadores provocando un funcionamiento anormal en el veh\'iculo, si la falla no se detecta y diagn\'ostica oportunamente habr\'a mayores gastos en la reparaci\'on  y perdida tiempo del propietario.
Debido a la gran parte de la seguridad que se requiere en diferentes \'areas es de gran inter\'es desarrollar mejores sistemas para una buena detección y diagn\'ostico y es por eso que se han estado desarrollando en estos \'ultimos años diversas propuestas para afrontar las necesidades de cada aplicaci\'on y as\'i tener sistemas autom\'aticos m\'as confiables.
Este documento hace una revisi\'on de diferentes m\'etodos empleados para la detecci\'on y diagn\'ostico de fallas en diversas \'areas.

%En los \'ultimos años la mayoría de los autores han propuesto t\'ecnicas para la detecci\'on de fallas utilizando redundancia anal\'itica  %apoy\'andose  en los ordenadores para procesar los datos de una manera r\'apida y eficiente. La idea es de detectar la diferencia entre la medición  y un valor esperado, con ello poder dar a conocer que existe una falla y a su %vez poder identificarla .En este art\'iculo se revisan diferentes propuestas realizadas en los ultimos a\~nos para la detecci\'on y diagnosticos de %fallas para diferentes sistemas, predominando propuestas de diagnostico en base a modelos anal\'iticos.
