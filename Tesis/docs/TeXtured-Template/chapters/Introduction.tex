\chapternotnumbered{Introduction} \label{ch:Introduction}

Already when I started writing my bachelor thesis \autocite{Dujava2022}, I tweaked a lot the original \LaTeX{} template (which itself was slightly updated since then \autocite{MaresTemplate}) provided for students at MFF CUNI --- Faculty of Mathematics and Physics, Charles University in Prague.

Some design choices originated already during this stage, particularly the significant use of theorem-like and remark-like environments with highly interlinked structure of the text.

I picked up right where I left off when I started writing my master thesis \autocite{TODO}, and I have been refining the template ever since.
Improved understanding of the coding backbone behind \LaTeX{} and its package ecosystem enabled me to customize it even further to my liking, and add even more \enquote{bells and whistles}.

\begin{remark}[Template Purpose]
    While primarily targeting theses, \TeXtured{} can be used for other document types as well.
\end{remark}

To make it user-friendly, I have restructured the preamble into several files, each of which is responsible for a specific aspect of the document.
This way, the user can (and is encouraged to) easily find the relevant part of the code and modify it.

Numerous comments and explanations are provided throughout the code to further aid the user in understanding the template without always having to consult the documentation of packages (which is recommended for more advanced changes).

\begin{remark}[How to Setup]
    To set up \TeXtured{} template for your document, you can use the \texttt{Overleaf} template or clone the repository on \textsf{GitHub} \autocite{TeXtured}.
    Then, you can start modifying the files to suit your needs.

    Also make sure to check the \texttt{README.md} file for more detailed instructions, particularly on various software dependencies.
    If you encounter any issues, please see \href{https://github.com/jdujava/TeXtured/issues/2}{\texttt{jdujava/TeXtured \#2}} and \href{https://github.com/jdujava/TeXtured/issues/5}{\texttt{jdujava/TeXtured \#5}}.
\end{remark}
