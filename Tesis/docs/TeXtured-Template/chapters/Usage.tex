\chapter{Usage of \TeXtured{}} \label{ch:Usage}

To quickly familiarize yourself with the \TeXtured{} template, we will go through the basic structure of the template files and explain how to use them.
First, have a look at \Cref{fig:file-structure} for a visual representation of the file structure.
\begin{figure}[!ht]
    \fcapside[\FBwidth]{%
        \captionsetup{parskip = .5\baselineskip plus 1pt}%
        \caption[\TeXtured{} Template File Structure]{%
            \TeXtured{} template file structure.

            The main file is \path{thesis.tex}.
            It does not contain the actual content of the document, but instead \fakemacro{\include}s the chapters and \emph{front matter} pages from the corresponding directories.

            Make sure to fill all PDF \grayenclose{meta}data --- like title, author, etc.~--- in the \path{preamble/data.tex} file.
            The bibliography/reference data is stored in the \path{preamble/bibliography.bib} file.

            All of the \TeXtured{} tweaks and settings located in files under \path{preamble/...} directories are loaded by the \path{preamble/main.tex} file, which is itself \fakemacro{\input}ed in the \path{thesis.tex} file.

            The \path{.latexmkrc} file contains a configuration for the \textsf{latexmk} tool, which provides a convenient way to compile the document.
        }%
        \label{fig:file-structure}%
    }{%
        \begin{tikzpicture}[dirtree]
            \node[directory] {\TeXtured{} Root Directory/}
            child { node {thesis.tex}}
            child { node {LICENSE}}
            child { node {README.md}}
            child { node[directory] {chapters/...}}
            child { node[directory] {figures/...}}
            child { node[directory] {frontmatter/...}}
            child { node[directory] {preamble/}
                child { node {bibliography.bib}}
                child { node {data.tex}}
                child { node {main.tex}}
                child { node[directory] {...} }
            }
            child [missing] {} child [missing] {} child [missing] {} child [missing] {}
            child { node {.latexmkrc}}
            child { node {.gitignore}}
            child { node {.gitattributes}}
            child { node[directory] {.github/...}}
            ;
        \end{tikzpicture}%
    }
\end{figure}

The usual workflow looks something like this:
\begin{itemize}
    \item \textsf{Metadata.} Fill in the \path{preamble/data.tex} file with the necessary information about the document --- title, author, and other \emph{metadata}.
    \item \textsf{Content.} Write the content of the document in the \path{chapters/} directory.
          If you need more chapters, just create a new file, and \macro{\include} it in the \path{thesis.tex} file at appropriate place.
    \item \textsf{Figures.} To include figures, you can put them in the \path{figures/} directory.
          Since this directory is by default included in \macro{\graphicspath}, there is no need to specify full/relative path, and it is enough to use just the filename in the \macro{\includegraphics} command.
    \item \textsf{Citations.} Using \grayenclose{for example} \macro{\autocite} macro, you can cite in the text any entry added to the \path{preamble/bibliography.bib} file.
\end{itemize}

\begin{remark}[Toggles]
    There are a couple of \emph{toggles} in the \path{thesis.tex} file that can be used to customize style/layout/creation of the document:
    \begin{itemize}
        \item \textsf{Page Layout} --- you can choose between \emph{Single-Side} or \emph{Two-Sided} printing by uncommenting the appropriate \macro{\documentclass} line.
        \item \textsf{Work-In-Progress Version} --- if you want to mark the document as a \emph{Draft}, leave the \macro{\WIPtrue} line uncommented (comment out for the final version).
        \item \textsf{Censored Version} --- if you want to censor chosen parts of the document, include the \macro{\CENSORtrue} line.
        \item \textsf{Fancy Style} --- if the default style is not to your liking, you can disable some of the more \enquote{fancy} stylistic elements by uncommenting the \macro{\FANCYtrue} line.
        \item \textsf{Include Only \ldots{}} --- if you want to compile only a subset of chapters, you can utilize the \custommacro{\includeonlysmart} command. \qedhere*
    \end{itemize}
\end{remark}

\begin{remark}[MFF CUNI Template Compatibility]
    \TeXtured{} can be used out of the box for theses at the Faculty of Mathematics and Physics, Charles University in Prague.
    Just be sure to include all \emph{front matter} pages and fill out necessary data:
    \begin{itemize}
        \item \textsf{Title Page} with the faculty logo (among other things),
        \item \textsf{Declaration},
        \item \textsf{Dedication} (optional),
        \item \textsf{Information Page} including the \textsf{Abstract}.
    \end{itemize}
    This is done by uncommenting the relevant lines in the main \path{thesis.tex} file.

    Layout of these front matter pages is adapted and modified from the original MFF CUNI template \autocite{MaresTemplate}.
    However, always make sure it is compliant with the faculty guidelines, otherwise please raise an issue on \textsf{GitHub} \autocite{TeXtured}.
\end{remark}

\begin{remark}[License]
    If you want to make your document publicly available (together with the source code), you should not forget to include an appropriate license of your choice --- change the \path{LICENSE} file, specifying the \href{https://creativecommons.org/publicdomain/zero/1.0/}{\texttt{CC0 1.0 Universal}} license of \TeXtured{}.
\end{remark}
