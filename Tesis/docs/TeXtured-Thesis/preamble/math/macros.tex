%%% Macros for math symbols

%% Punctuation in math mode
\NewDocumentCommand{\eqend}{}{\,.}
\NewDocumentCommand{\eqcomma}{}{\,,}

%% (Optional) Automatically use \mathcolor in math mode
% \NewCommandCopy{\textcolorOrig}{\textcolor}
% \RenewDocumentCommand{\textcolor}{O{} m m}{%
%     \ifmmode%
%         \let\colorcmd\mathcolor%
%     \else%
%         \let\colorcmd\textcolorOrig%
%     \fi%
%     \IfBlankTF{#1}{\colorcmd{#2}{#3}}{\colorcmd[#1]{#2}{#3}}%
% }

%% Misc math macros
\NewCommandCopy{\sq}{\sqrt}
\DeclareMathOperator{\sgn}{sgn}
\NewDocumentCommand{\Id}{}{\bm{1}}
\DeclareMathOperator{\Span}{Span}
\DeclareMathOperator{\diag}{diag}
\NewDocumentCommand{\bdry}{m}{\partial #1}
\NewDocumentCommand{\disjunion}{}{\sqcup} % TODO: spacing too small sometimes (in subscripts)
\NewDocumentCommand{\inclusion}{}{\hookrightarrow}
\NewDocumentCommand{\surjection}{}{\twoheadrightarrow}

%% Imaginary unit and Euler's number
%% inspired by Niklas Beisert
\NewDocumentCommand{\I}{}{\mathring{\imath}}
\NewDocumentCommand{\E}{s}{\IfBooleanTF{#1}{\mathrm{e}}{\mathinner{\mathrm{e}}}}

%%% Various fields and number sets
\NewDocumentCommand{\R}{}{\mathbb{R}}
\NewDocumentCommand{\Z}{}{\mathbb{Z}}
\NewDocumentCommand{\C}{}{\mathbb{C}}
\NewDocumentCommand{\F}{}{\mathbb{F}}
\NewDocumentCommand{\N}{}{\mathbb{N}}

%% Override `physics` package command `\dd` to optionally use bold `d`
\RenewDocumentCommand{\dd}{s O{ } m}{
    \mathinner{
        \IfBooleanTF{#1}{\mathrm{d}}{\mathbf{d}}^{\mspace{-1mu}#2}#3
    }
}

%% Group/Representation Theory
\DeclareMathOperator{\Hom}{Hom}
\DeclareMathOperator{\Ker}{Ker}
\DeclareMathOperator{\End}{End}
\DeclareMathOperator{\Aut}{Aut}
\DeclareMathOperator{\gl}{\mathfrak{gl}}
\DeclareMathOperator{\GL}{\mathsf{GL}}
\let\sl\relax % override (anyway deprecated) `sl` command
\DeclareMathOperator{\sl}{\mathfrak{sl}}
\DeclareMathOperator{\SL}{\mathsf{SL}}
\DeclareMathOperator{\oo}{\mathfrak{o}}
\DeclareMathOperator{\OO}{\mathsf{O}}
\DeclareMathOperator{\so}{\mathfrak{so}}
\DeclareMathOperator{\SO}{\mathsf{SO}}
\DeclareMathOperator{\spin}{\mathfrak{spin}}
\DeclareMathOperator{\Spin}{\mathsf{Spin}}
\DeclareMathOperator{\U}{\mathsf{U}}
\DeclareMathOperator{\su}{\mathfrak{su}}
\DeclareMathOperator{\SU}{\mathsf{SU}}
\DeclareMathOperator{\Sp}{\mathsf{Sp}}
\NewDocumentCommand{\g}{}{\mathfrak{g}}
\NewDocumentCommand{\rankgroup}{}{\mathsf{r}}

%% Differential Geometry
\DeclareMathOperator{\Sect}{Sect}
\NewDocumentCommand{\Lie}{}{\mathcal{L}}
\DeclarePairedDelimiterX{\Liebracket}[2]{\dblbracketleft}{\dblbracketright}{#1,#2}
% \DeclarePairedDelimiterX{\Liebracket}[2]{\lbrack}{\rbrack}{#1,#2}

\DeclareMathOperator{\Tor}{\mathsf{Tor}}
\NewDocumentCommand{\Riem}{}{\bm{R}}
\NewDocumentCommand{\Ric}{}{\mathbf{Ric}}
\NewDocumentCommand{\Rscalar}{}{\mathcal{R}}

%% Covariant derivative (with optional arguments for space tweaking)
\NewDocumentCommand{\cder}{s O{a} e{_}}{
    \IfBooleanT{#1}{\mspace{-2mu}} % use starred variant when more \cder after each other
    \bm{\nabla}
    \IfValueT{#3}{                  % when subscript index is given
        _{                          % start subscript
            \mspace{-7mu}           % remove extra space after nabla
            \centerhphantom{#3}{#2} % print #3 with the width of #2
        }
    }
}
%% Coordinate derivative
\NewDocumentCommand{\cpartial}{}{\bm{\partial}}



%%% Tweaking of math index positioning, spacing, etc.
%%% (also definitions of new arrows, etc.)

%% TODO: see if \cramped{...} could be used sometimes to make
%%       (inline) math prettier (to avoid weird line spacing weird)

%% Modify size of `\bigwedge`/`\bigodot` and the position of their superscripts
\NewDocumentCommand{\Ext}{e{^}}{
    \scalerel*{\bigwedge}{\xmathstrut[-0.1]{0.1}} % scale down the \bigwedge a bit
    \IfValueT{#1}{^{\mspace{-2mu}#1}} % shift possible superscript little bit to the left
}
\NewDocumentCommand{\Odot}{e{^}}{
    \scalerel*{\bigodot}{\xmathstrut[-0.1]{0.1}} % scale down the \bigodot a bit
    \IfValueT{#1}{^{\mspace{-0mu}#1}} % (do not) shift possible superscript little bit to the left
}

%% Exchange \epsilon and \varepsilon
\NewCommandCopy{\oldepsilon}{\epsilon}
\RenewCommandCopy{\epsilon}{\varepsilon}
\RenewCommandCopy{\varepsilon}{\oldepsilon}

%% Modify subscript position
\NewDocumentCommand{\lowerindex}{O{0pt} O{0pt} e{_^}}{
    _{\IfValueT{#3}{\raisebox{#1}{\(\scriptstyle #3\)}}}
    ^{\IfValueT{#4}{\raisebox{#2}{\(\scriptstyle #4\)}}}
}

%% Modify superscript/subscript positions for some greek letters
%% NOTE: maybe use \mathstrut or \xmathstrut
\NewCommandCopy{\oldchi}{\chi}
\RenewDocumentCommand{\chi}{}{\oldchi\lowerindex[-1.5pt]}
\NewCommandCopy{\olddelta}{\delta}
\RenewDocumentCommand{\delta}{}{\olddelta\lowerindex[1pt][1.0pt]}
\NewDocumentCommand{\bmdelta}{}{\bm{\olddelta}\lowerindex[1pt][1.0pt]}
\NewDocumentCommand{\altdelta}{}{{\olddelta\xmathstrut[-0.2]{0.0}}}
\NewDocumentCommand{\altbmdelta}{}{{\bm{\olddelta}\xmathstrut[-0.2]{0.0}}}

%% Determinants (use normal position of superscript/subscript)
\NewCommandCopy{\olddet}{\det}
\RenewDocumentCommand{\det}{}{\olddet\nolimits}
\DeclareMathOperator{\Det}{Det}

%% Fix spacing of \left( .. \middle| .. \right)
\NewCommandCopy{\originalleft}{\left}
\NewCommandCopy{\originalright}{\right}
\NewCommandCopy{\originalmiddle}{\middle}
\RenewDocumentCommand{\left}{}{\mathopen{}\mathclose\bgroup\originalleft}
\RenewDocumentCommand{\right}{}{\aftergroup\egroup\originalright}
\RenewDocumentCommand{\middle}{m}{\mathrel{}\originalmiddle#1\mathrel{}}

%% Less space around \setminus (\mathord instead of \mathinner)
\DeclareMathSymbol{\setminus}{\mathord}{symbols}{"6E}

%% Redefine `\square` to Young tableaux
\usepackage{ytableau}           % Young Tableaux
\ytableausetup{centertableaux}
\NewCommandCopy{\oldsquare}{\square}
\NewCommandCopy{\oldotimes}{\otimes}
\RenewDocumentCommand{\square}{O{0.55em} O{top} O{}}{%
    \IfBlankTF{#1}{\ytableausetup{boxsize=0.55em}}{\ytableausetup{boxsize=#1}}%
    \ytableausetup{aligntableaux=#2}%
    \operatorname{\ydiagram[#3]{1}}%
}
\NewDocumentCommand{\smallsquare}{O{0.35em}}{\square[#1]}

%% Smaller \in, \notin, \subset, ...
%% https://tex.stackexchange.com/questions/34393/the-mysteries-of-mathpalette
\NewDocumentCommand{\smallerrel}{m}{\mathrel{\mathpalette\smallerrelaux{#1}}}
\NewDocumentCommand{\smallerrelaux}{mm}{\raisebox{.1ex}{\scalebox{.85}{\(#1#2\)}}}
\NewDocumentCommand{\smallin}{}{\smallerrel{\in}}
\NewDocumentCommand{\smallnotin}{}{\smallerrel{\notin}}
\NewDocumentCommand{\smallsubset}{}{\smallerrel{\subset}}
\NewDocumentCommand{\smallotimes}{}{\mspace{+1mu}\raisebox{.275ex}{\scalebox{.5}{\(\oldotimes\)}}\mspace{+1mu}}


%% Sizeable bullet
\NewCommandCopy{\oldbullet}{\bullet}
\NewDocumentCommand{\sbullet}{O{0.5}}{%
    \mathbin{\ThisStyle{\vcenter{\hbox{\scalebox{#1}{\(\SavedStyle\bullet\)}}}}}
}

%% (abstract) index placeholders
\NewDocumentCommand{\aind}{}{\bullet}
\NewDocumentCommand{\ind}{}{\mathcolor{lightgray}{\bullet}}

%% Argument placeholder
\NewDocumentCommand{\argument}{s o}{%
    \def\squaresize{1.0}%                   % default size
    \IfBooleanT{#1}{\def\squaresize{0.7}}%  % if starred, set size corresponding to scriptstyle
    \IfValueT{#2}{\def\squaresize{#2}}%     % if optional argument is passed, set size to custom value
    \scalebox{\squaresize}{%
        \begin{tikzpicture}[baseline=-0.6ex]
            \node(char)[draw, shape=rectangle, dash=on 1.2pt off 1.05pt phase 0.5pt, dash expand off,
                inner ysep=2pt, inner xsep=2pt, minimum size=0.6em, rounded corners=2pt] {};
            %% alternative:
            % \node(char)[draw, shape=rectangle, dash=on 1.1pt off 0.8pt phase 0.5pt, dash expand off,
            %     inner ysep=2pt, inner xsep=2pt, minimum size=0.6em, rounded corners=2pt] {};
        \end{tikzpicture}%
    }%
}

%% Curly arrows
\tikzset{
    leadsto/.style={-{Stealth[length=0.6em,open,round]},decorate,decoration={snake,amplitude=0.20ex,segment length=0.5em,pre length=0.1em,post length=0.6em}},
    correspondsto/.style={{Stealth[length=0.6em,open,round]}-{Stealth[length=0.6em,open,round]},decorate,decoration={snake,amplitude=0.20ex,segment length=0.5em,pre length=0.7em,post length=0.7em}},
}
\NewDocumentCommand{\longleadsto}{O{2.8em}}{
    \tikz[baseline=-0.5ex]{ \draw[line width=0.7pt, leadsto] (0,0) -- (#1,0); }
}
%% https://tex.stackexchange.com/questions/170092/xleftrightarrows-command-in-tikz-with-arrows-matching-the-latex-font?rq=1
\NewDocumentCommand{\correspondsto}{O{}O{}}{%
    \ensuremath{\mathrel{
            \tikz[baseline=-0.5ex, font=\scriptsize]{
                \node[minimum width=3.48em, inner xsep=0.8em, align=center] (a){\hphantom{#1}\\[0ex]\hphantom{#2}};
                \IfBlankF{#1}{\node[inner sep=0.3ex, above=0.3ex] at (a){#1};}
                \IfBlankF{#2}{\node[inner sep=0.3ex, below=0.3ex] at (a){#2};}
                \draw[line width=0.6pt, correspondsto] (a.west) -- (a.east);
            }
        }}%
}

%% Quotient macro
\NewDocumentCommand{\bigslant}{O{0.2}O{1.7}mm}{
    \left. \raisebox{#1em}{\(#3\)}
    \mspace{-#2mu} \originalmiddle/ \mspace{-\fpeval{#2+1}mu}
    \raisebox{-#1em}{\(#4\)} \mspace{-\fpeval{5*#1}mu} \right.
}
% \NewDocumentCommand{\coset}{O{0.05} O{0.1} m m}{\bigslant[#1][#2]{#3}{#4}}
\NewDocumentCommand{\coset}{O{}O{} m m}{#3/#4}
\NewDocumentCommand{\gt}{}{\bigslant{\g}{\mathfrak{t}}}
\NewDocumentCommand{\gts}{}{\bigslant[0.15]{\scriptstyle\g}{\scriptstyle\mathfrak{t}}}
\NewDocumentCommand{\onehalf}{}{\!\bigslant[0.15]{\scriptscriptstyle 1 \mspace{-1.2mu}}{\scriptscriptstyle 2}}

%% Cancel macro
\definecolor{cancelgray}{gray}{0.85}
\tikzset{
    main node/.style={inner sep=0,outer sep=0},
    label node/.style={inner sep=0.3em,font=\tiny,overlay},
    strike out/.style={shorten <=-.2em,shorten >=-.2em,overlay,thick,double distance = 0em,line cap=round},
}
\NewDocumentCommand{\cancel}{O{cancelgray}mo}{%
    \tikz[baseline=(N.base)]{
        \node[main node](N){\(#2\)};
        \IfValueT{#3}{\node[label node, gray, anchor=south] at (N.north){#3};}
        \draw[strike out, #1]  (N.south west) -- (N.north east);
    }
}
