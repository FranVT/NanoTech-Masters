%% TODO: think about what style is appropriate
%% Set the default bold math sans font to be `sbc` (sans bold condensed)
%% https://tex.stackexchange.com/questions/27843/level-of-boldness-changeable
\NewDocumentCommand{\mathsfbfdefault}{}{sbc}
\SetMathAlphabet{\mathsf}{bold}{T1}{\sfdefault}{\mathsfbfdefault}{\updefault}
% \SetMathAlphabet{\mathsf}{bold}{T1}{\sfdefault}{\bfdefault}{\sldefault}

%% Math sans slanted font
\DeclareMathAlphabet{\mathsfit}{T1}{\sfdefault}{\mddefault}{\sldefault}
\SetMathAlphabet{\mathsfit}{bold}{T1}{\sfdefault}{\bfdefault}{\sldefault}

%% When using sans font for surrounding text, also adapt the math font
%%  -> use Computer Modern Bright when `\mathversion{sans}` is active
%% see https://tex.stackexchange.com/q/33165
%%     https://mirrors.nic.cz/tex-archive/fonts/cmbright/cmbright.pdf
\DeclareMathVersion{sans}
\SetSymbolFont{operators}{sans}{OT1}{cmbr}{m}{n}
\SetSymbolFont{letters}{sans}{OML}{cmbrm}{m}{it}
\SetSymbolFont{symbols}{sans}{OMS}{cmbrs}{m}{n}
\SetSymbolFont{largesymbols}{sans}{OMX}{iwona}{m}{n}
\SetMathAlphabet{\mathit}{sans}{OT1}{cmbr}{m}{sl}
\SetMathAlphabet{\mathsf}{sans}{OT1}{lmss}{m}{n} % still use `lmodern` for `\mathsf`
\SetMathAlphabet{\mathbf}{sans}{OT1}{cmbr}{bx}{n}
\SetMathAlphabet{\mathtt}{sans}{OT1}{cmtl}{m}{n}

%% TODO: bold sans math is not optimal...
\DeclareMathVersion{boldsans}
\SetSymbolFont{operators}{boldsans}{OT1}{cmbr}{b}{n}
\SetSymbolFont{letters}{boldsans}{OML}{cmbrm}{b}{it}
\SetSymbolFont{symbols}{boldsans}{OMS}{cmbrs}{b}{n}
\SetSymbolFont{largesymbols}{boldsans}{OMX}{iwona}{bx}{n}
\SetMathAlphabet{\mathit}{boldsans}{OT1}{cmbr}{b}{sl}
\SetMathAlphabet{\mathsf}{boldsans}{OT1}{lmss}{\mathsfbfdefault}{n} % still use `lmodern` for `\mathsf`
\SetMathAlphabet{\mathbf}{boldsans}{OT1}{cmbr}{bx}{n}
\SetMathAlphabet{\mathtt}{boldsans}{OT1}{cmtl}{b}{n}

%%% Automatic switching between math versions
\newif\ifInSansMode % keep track of whether we are in sans mode
\newif\ifInBoldMode % keep track of whether we are in sans mode
%% automatically switch to sans math version in `\sffamily` text
%% automatically switch to normal math version in `\rmfamily` text
\NewCommandCopy{\sffamilyOrig}{\sffamily}
\NewCommandCopy{\rmfamilyOrig}{\rmfamily}
\RenewDocumentCommand{\sffamily}{}{\sffamilyOrig\InSansModetrue%
    \ifInBoldMode\mathversion{boldsans}\else\mathversion{sans}\fi
}
\RenewDocumentCommand{\rmfamily}{}{\rmfamilyOrig\InSansModefalse%
    \ifInBoldMode\mathversion{bold}\else\mathversion{normal}\fi
}
%% automatically switch to bold math version in `\bfseries` text
%% automatically switch to normal math version in `\bfseries` text
\NewCommandCopy{\bfseriesOrig}{\bfseries}
\NewCommandCopy{\mdseriesOrig}{\mdseries}
\RenewDocumentCommand{\bfseries}{}{\bfseriesOrig\InBoldModetrue%
    \ifInSansMode\mathversion{boldsans}\else\mathversion{bold}\fi%
}
\RenewDocumentCommand{\mdseries}{}{\mdseriesOrig\InBoldModefalse%
    \ifInSansMode\mathversion{sans}\else\mathversion{normal}\fi%
}

%% Sans Greek Letters
\DeclareSymbolFont{sfletters}{OML}{cmbrm}{m}{it}
\DeclareSymbolFont{sflettersbold}{OML}{cmbrm}{b}{it}
\DeclareMathSymbol{\sbGamma}{\mathalpha}{sflettersbold}{0}

%% Alternative caligraphic math font `mathpazo`
\DeclareMathAlphabet{\mathpzc}{OT1}{pzc}{m}{it}

%% Load doublebrackets symbols
%% (need to declare `pdfglyphtounicode` translations to satisfy PDF/A -> done in `pdfA-compliance.tex`)
\DeclareSymbolFont{stmry}{U}{stmry}{m}{n}
\DeclareMathDelimiter{\dblbracketleft}{\mathopen}{stmry}{"4A}{stmry}{"71}
\DeclareMathDelimiter{\dblbracketright}{\mathclose}{stmry}{"4B}{stmry}{"79}

%% Shorter minus sign - https://tex.stackexchange.com/a/469724
\DeclareMathSymbol{\shortminus}{\mathbin}{AMSa}{"39}
\NewDocumentCommand{\shortminusone}{}{\mspace{-1.2mu}\shortminus\mspace{-1.5mu} 1}


%% HACK: Reduce verbosity of "No alphabet change ..." messages
%% https://tex.stackexchange.com/a/663843
\makeatletter
\let\old@font@info\@font@info
\def\@font@info#1{%
    \expandafter\ifx\csname\detokenize{#1}\endcsname\relax
        \old@font@info{#1}%
    \fi
    \expandafter\xdef\csname\detokenize{#1}\endcsname{}%
}
\makeatother
